\documentclass[a4paper, 10pt]{thesis}

% {{{ PACKAGES
% {{{ ENCODAGE ET POLICE
\usepackage[utf8]{inputenc}
\usepackage[T1]{fontenc}
\usepackage[french]{babel}
\usepackage{eurosym}
% }}}

% {{{ FIGURES
\usepackage{graphicx}
\usepackage{subfig}
% }}}

% {{{ INSERTION DE CODE
\usepackage{moreverb}
% }}}

% {{{ INSERTION DE PDF
\usepackage{pdfpages}
% }}}

% {{{ RÉDUCTION DES MARGES
%\usepackage{fullpage}
%\usepackage[top=2.5cm, bottom=2.5cm, left=3.5cm, right=3.5cm]{geometry}
% }}}

% {{{ MATHÉMATIQUES
\usepackage{amsmath}
\usepackage{amsthm}
\usepackage{bbm}
\usepackage{amssymb}
\usepackage{stmaryrd}
% }}}

% {{{ GRAPHES
\usepackage{tikz}
\usepackage{pgf}
\usetikzlibrary{arrows,automata,shapes}
% }}}

% {{{ RESEAUX DE pETRI
\usetikzlibrary{petri}
% }}}

% {{{ ARBRES
\usetikzlibrary{trees}
% }}}

% {{{ DIAGRAMME DE GANTT
\usepackage{pgfgantt}
% }}}

% {{{ ALGORITHMES
\usepackage{algorithm}
\usepackage[noend]{algpseudocode}
%\usepackage{algorithmic}
%\usepackage{algorithmicx}
% }}}
% }}}

% {{{ NEWTHEOREM
% {{{ THÉORÈMES, lEMMES -- NUMÉROTATION PERSONNALISÉE
\theoremstyle{plain}
\newtheorem{thrm}{Théorème}[section]
\newtheorem{lemm}{Lemme}[subsection]
\newtheorem{corol}{Corollaire}[section]
% }}}

% {{{ THÉORÈMES, lEMMES -- NUMÉROTATION CLASSIQUE
\newtheorem{nthrm}{Théorème}
\newtheorem{nlemm}{Lemme}
\newtheorem{ncorol}{Corollaire}
\newtheorem{npb}{Problème}
% }}}

% {{{ THÉORÈMES, lEMMES -- NON NUMÉROTÉS
\newtheorem*{thrm*}{Théorème}
\newtheorem*{lemm*}{Lemme}
\newtheorem*{corol*}{Corollaire}
% }}}

% {{{ DÉFINITIONS, pROPRIÉTÉS -- NUMÉROTATION PERSONNALISÉE
\theoremstyle{definition}
\newtheorem{df}{Définition}[subsection]
\newtheorem{prop}{Propriété}[subsection]
\newtheorem{conj}{Conjecture}[subsection]
% }}}

% {{{ DÉFINITIONS, pROPRIÉTÉS -- NUMÉROTATION CLASSIQUE
\newtheorem{ndf}{Définition}
\newtheorem{nprop}{Propriété}
\newtheorem{nconj}{Conjecture}
% }}}

% {{{ DÉFINITIONS, pROPRIÉTÉS -- NON NUMÉROTÉS
\newtheorem*{df*}{Définition}
\newtheorem*{prop*}{Propriété}
\newtheorem*{conj*}{Conjecture}
% }}}

% {{{ AUTRES -- NUMÉROTATION PERSONNALISÉE
\theoremstyle{remark}
\newtheorem{nota}{Note}[subsubsection]
\newtheorem{pr}{Proposition}[subsection]
\newtheorem{rmq}{Remarque}[subsubsection]
% }}}

% {{{ AUTRES -- NUMÉROTATION CLASSIQUE
\newtheorem{nnota}{Note}
\newtheorem{npr}{Proposition}
\newtheorem{nrmq}{Remarque}
% }}}

% {{{ AUTRES -- NON NUMÉROTÉS
\newtheorem*{nota*}{Note}
\newtheorem*{pr*}{Proposition}
\newtheorem*{rmq*}{Remarque}
% }}}

% {{{ EXEMPLES
\newtheorem*{ex}{Exemple}
\newtheorem*{exo}{Exercice}
\newtheorem*{rappel}{Rappel}
% }}}
% }}}

% {{{ OPÉRATEURS MATHÉMATIQUES
\DeclareMathOperator{\poly}{\textbf{poly}}
\DeclareMathOperator{\gap}{\mbox{gap}}
\DeclareMathOperator{\Bg}{\mbox{Big}}
\DeclareMathOperator{\Sml}{\mbox{Sml}}
\DeclareMathOperator{\Diff}{\mbox{Diff}}
\DeclareMathOperator{\card}{\mbox{Card}}
\DeclareMathOperator{\Obj}{\mbox{Obj}}
\DeclareMathOperator{\concat}{\bullet}
\DeclareMathOperator{\sig}{\mbox{Sig}}
\DeclareMathOperator{\ver}{\mbox{Ver}}
\DeclareMathOperator{\dist}{\mbox{dist}}
\DeclareMathOperator{\bw}{\mbox{branchwidth}}
\DeclareMathOperator{\tw}{\mbox{tw}}
\DeclareMathOperator{\lm}{\leq_m}
\DeclareMathOperator{\lc}{\leq_c}
\DeclareMathOperator{\lex}{\mbox{lex}}
\DeclareMathOperator{\ord}{\mbox{ord}}
% }}}

% {{{ MACCROS
% {{{ MACROS DIVERSES
\newcommand{\npc}{$NP-$complet}
\newcommand{\npd}{$NP-$difficile}
\newcommand{\la}{largeur arborescente}
\newcommand{\legendre}[2]{\left ( \frac{#1}{#2} \right )}
\newcommand{\wqo}[0]{\emph{well-quasi-ordered}}
\newcommand{\Tau}[0]{\mathcal{T} = (T, \{X_t : t \in T\})}
\newcommand{\Taug}[0]{\mathcal{T}_G = (T, \{X_t : t \in T\})}
\newcommand{\Taustar}[0]{\mathcal{T}^* = (T, \{X_t : t \in T\})}
\newcommand{\Taugstar}[0]{\mathcal{T}_G^* = (T, \{X_t : t \in T\})}
\newcommand{\pdsol}[1]{(X_{#1}^0, X_{#1}^1, X_{#1}^2, M_{#1})}
% }}}
                
% {{{ NOM DE PROBLEMES
\newcommand{\hcycle}{\textsc{HAMILTONIAN CYCLE} }
\newcommand{\phcycle}[1]{\textsc{HAMILTONIAN CYCLE }paramétré par \textsc{#1} }
\newcommand{\vcover}{\textsc{VERTEX COVER} }
\newcommand{\twidth}{\textsc{TREE WIDTH} }
\newcommand{\wiset}{\textsc{WEIGHTED INDEPENDANT SET} }
\newcommand{\fvset}{\textsc{FEEDBACK VERTEX SET} }
\newcommand{\kvcover}{\textsc{K-Vertex Cover} }
\newcommand{\oct}{\textrm{\textsc{Odd Cycle Transversal}} }
\newcommand{\dlp}{\textrm{\textsc{Discrete Logarithm Problem}} }
\newcommand{\ecdlp}{\textrm{\textsc{Elliptic Curve Discrete Logarithm Problem}} }

% }}}

% {{{ MISE EN PAGE

\newcommand{\dfpbi}[3]{
    \begin{npb}[#1]
        \begin{itemize}
            \itemindent 10mm
            \item[\textbf{Données :}] #2
            \item[\textbf{Problème :}] #3
        \end{itemize}
    \end{npb}
}

\newcommand{\dfpb}[3]{
    \begin{npb}[#1]
        \begin{itemize}
            \itemindent 10mm
            \item[\textbf{Données :}] #2
            \item[\textbf{Question :}] #3
        \end{itemize}
    \end{npb}
}

\newcommand{\dfpbp}[4]{
    \begin{npb}[#1]
        \begin{itemize}
            \itemindent 10mm
            \item[\textbf{Données :}] #2
            \item[\textbf{Paramètre :}] #4
            \item[\textbf{Question :}] #3
        \end{itemize}
    \end{npb}
}
% }}}

% }}}
    
% {{{ TIKZ
%%% {{{ STYLE
\tikzstyle{edge}=[
	-,
	draw=black
]

\tikzstyle{sized}=[
    minimum width=9mm
]

\tikzstyle{tvertex}=[
	node distance=15mm,
    inner sep = 1mm
]

\tikzstyle{tvert}=[
	node distance=5mm,
    inner sep=1mm
]

\tikzstyle{tmvertex}=[
    tvertex,
    sized
]

\tikzstyle{tmvert}=[
    tvert,
    sized
]

\tikzstyle{mvertex}=[
    tmvertex,
	circle,
	draw=black,
	fill=black!50
]

\tikzstyle{mvert}=[
    tmvert,
	circle,
	draw=black,
	fill=black!50
]

\tikzstyle{vertex}=[
    tvertex,
	circle,
	draw=black,
	fill=black!50
]

\tikzstyle{vert}=[
    tvert,
	circle,
	draw=black,
	fill=black!50
]

\tikzstyle{treedec}=[
	vertex,
	node distance=15mm,
	draw=red,
	fill=red!50
]

\tikzstyle{treeedge}=[
	edge,
	draw=red
]

\tikzstyle{giedge}=[
	edge,
	draw=blue
]

\tikzstyle{bounding}=[
	thick, %ultra
	%dash pattern=on 1mm off 1mm,
	densely dotted,
	fill=blue!10,
	draw=blue!70
]

\tikzstyle{patate}=[
    rounded corners=5pt,
    draw=black,
    fill=black!50,
    text width=12mm,
    text centered
]

\tikzstyle{patatoid}=[
    ellipse,
    node distance=10mm,
    draw=black
]
%%% }}}

% {{{ DESSINS
\newcommand{\base}[4]{
    \draw[->] (0,0) to (0, #1);
    \draw[->] (0,0) to (#2, 0);
    
    \node at (0, #1+2) {#3};
    \node at (#2 + 2, -2) {#4};
}
% }}}
%}}}



\begin{document}

\chapter{Attaque diff\'erentielle sur les chiffrements par bloc}

\section{Différence des entrées sorties d'une $S-$boîte}

On considère une substitution $S : \{0,1\}^{l'} \longmapsto \{0,1\}^{l'}$

\begin{df}
    On considère un couple $(u, u') \in \{0, 1\}^2$ et une différence $\Delta \in \{0,1\}^{l'}$.
    \begin{enumerate}
        \item La différence en entrée de $S$ est le ou exclusif $U \bigoplus U'$
        \item La différence en sortie de $S$
        \item Rattraper
    \end{enumerate}
\end{df}

\begin{ex}
    Considérons l'exemple suivant :
    \begin{displaymath}
        \begin{array}{|c|c|c|c|c|}
            \hline
            U & 00 & 01 & 10 & 11 \\ \hline
            S(U) & 10 & 00 & 11 & 01 \\ \hline
        \end{array}
    \end{displaymath}
    Alors pour $U = 10$ et $U' = 11$, la différence en entrée est $U \bigoplus U' = 01$, celle en
    sortie est donnée par $S(U) \bigoplus S(U') = 11 \bigoplus 01 = 10$ et les couples sont donnés
    par : \begin{displaymath}
        C(01) =\{(00, 01), (01, 00) , (10, 11), (11, 10)\}
    \end{displaymath}
\end{ex}

\begin{df}
    Soit $\Delta, \Delta^* \in \{0,1\}^{l'}$, on définit ce qui suit :
    \begin{enumerate}
        \item $\Diff_{\Delta \mapsto \Delta^*}(S) = \{(U, U') \in \{0, 1\}^{l'} \mbox{ et } U
                \bigoplus U' = \Delta \mbox{ et } S(U) \bigoplus S(U') = \Delta^*$
        \item On définit : \begin{displaymath}
                N_{\Delta \mapsto \Delta^*}(S) = \card(\Diff_{\Delta \mapsto \Delta^*}(S))
            \end{displaymath}
        \item On définit : \begin{displaymath}
                P_{\Delta \mapsto \Delta^*} = \frac{N_{\Delta \mapsto \Delta^*}}{2^l}
            \end{displaymath}
    \end{enumerate}
\end{df}

\begin{ex}
    Considérons l'exemple suivant :
    \begin{displaymath}
        \begin{array}{|c|c|c|c|c|c|c|c|c|}\hline
            U & 0000 & 0001 & 0010 & 0011 & 0100 & 0101 & 0110 & 0111 \\ \hline
            U' & 1011 & 1010 & 1001 & 1000 & 1111 & 1110 & 1101 & 1100 \\ \hline
            S(U) & 1110 & 0100 & 1101 & 0001 & 0010 & 1111 & 1011 & 1000\\ \hline
            S(U') & 1100 & 0110 & 1010 & 0011 & 0111 & 0000 & 1001 & 0101\\ \hline
            S(U) \bigoplus S(U') & 0010 & 0010 & 0111 & 0010 & 0101 & 1111 & 0010 & 1101\\ \hline
        \end{array}
    \end{displaymath}
    \begin{displaymath}
        \begin{array}{|c|c|c|c|c|c|c|c|c|}\hline
            U & 1000 & 1001 & 1010 & 1011 & 1100 & 1101 & 1110 & 1111 \\ \hline
            U' & 0011 & 0010 & 0001 & 0000 & 0111 & 0110 & 0101 & 0100 \\ \hline
            S(U) & 0011 & 1010 & 0110 & 1100 & 0101 & 1001 & 0000& 0111 \\ \hline
            S(U') & 0001 & 1101 & 0100 & 1110 & 1000 & 1011 & 1111 & 0010 \\ \hline
            S(U) \bigoplus S(U') & 0010 & 0111 & 0010 & 0010 & 1101 & 0010 & 1111 & 0101 \\ \hline
        \end{array}
    \end{displaymath}
    On obtient les valeurs de $N_{1011 \mapsto \Delta^*}$ pour $\Delta^* \in \{0, 1\}^4$ suivant :
    \begin{displaymath}
        \begin{array}{|c|c|c|c|c|c|c|c|c|}\hline
            \Delta^* & 0000 & 0001 & 0010 & 0011 & 0100 & 0101 & 0110 & 0111 \\ \hline
            N_{1011 \mapsto \Delta^*} & 0 & 0 & 8 & 0 & 0 & 2 & 0 & 2 \\ \hline
        \end{array}
    \end{displaymath}
    \begin{displaymath}
        \begin{array}{|c|c|c|c|c|c|c|c|c|}\hline
            \Delta^* & 1000 & 1001 & 1010 & 1011 & 1100 & 1101 & 1110 & 1111 \\ \hline
            N_{1011 \mapsto \Delta^*} & 0 & 0 & 0 & 0 & 0 & 2 & 0 & 2 \\ \hline
        \end{array}
    \end{displaymath}
\end{ex}

\section{Propagation de différence sur une ronde d'un SPN}

On considère une ronde d'un SPN :

% figure 1

\begin{pr}
    Soit $S_i : \{0, 1\}^{l'} \mapsto \{0, 1\}^{l'}$ pour les $t$ $S-$boîtes et soit $S : \{0, 1\}^l
    \mapsto \{0, 1\}^l$ la concaténée des $S_i$. Soit $\Delta_i \mapsto \Delta_i^*$ pour $i = 1,
    \dots, t$ les propagations de différence pour chaque boîte $S_i$. On définit enfin :
    \begin{displaymath}
        \left \lbrace
        \begin{array}{ccc}
            \Delta & = & [\Delta_1, \Delta_2, \dots, \Delta_t] \\
            \Delta^* & = & [\Delta_1^*, \Delta_2^*, \dots, \Delta_t^*] \\
        \end{array}
        \right .
    \end{displaymath}
    On a alors : \begin{displaymath}
        P_{\Delta \mapsto \Delta^*}(S) = P_{\Delta_1 \mapsto \Delta_1^*} \times P_{\Delta_2 \mapsto
        \Delta_2^*} \times \dots P_{\Delta_t \mapsto \Delta_t^*} 
    \end{displaymath}
\end{pr}

\begin{proof}
    Il suffit juste de remarquer que : \begin{displaymath}
        \Diff_{\Delta \mapsto \Delta^*} (S) = \Diff_{\Delta_1 \mapsto \Delta_1^*} \times
        \Diff_{\Delta_2 \mapsto \Delta_2^*} \times \dots \times \Diff_{\Delta_t \mapsto \Delta_t^*} 
    \end{displaymath}
    Et donc :
    \begin{displaymath}
        \begin{array}{rrcl}
            & N_{\Delta \mapsto \Delta^*}(S) & = & \prod_{i=1}^t N_{\Delta_i \mapsto \Delta_i^*}\\
            \Rightarrow & P_{\Delta \mapsto \Delta^*}(S) & = & \displaystyle \frac{N_{\Delta \mapsto
    \Delta^*}}{2^l} \\
    \Rightarrow &  P_{\Delta \mapsto \Delta^*}(S) & = & \displaystyle \prod_{i=1}^t \frac{N_{\Delta \mapsto
\Delta^*}}{2^l} \\
    \Rightarrow & P_{\Delta \mapsto \Delta^*}(S) & = & P_{\Delta_i \mapsto \Delta_i^*}
        \end{array}
    \end{displaymath}
\end{proof}

\begin{ex}
    On considère la situation suivante :

    %figure 2

    On en déduit donc que : 
    \begin{displaymath}
        P_{\Delta = (\Delta_1, \Delta_2) \mapsto \Delta^* = (\Delta_1^*, \Delta_2^*)} = \frac{1}{64}
    \end{displaymath}
\end{ex}

La proposition indique comment se propage une différence à travers un XOR avec la clef et la
permutation.

\begin{pr}
    Soit $U, U' \in \{0, 1\}^l$ et $\Delta = U \bigoplus U'$.
    \begin{enumerate}
        \item Soit $W = U \bigoplus K$ et $W' = U' \bigoplus K$ alors $W \bigoplus W' =
            \Delta$\footnote{Autrement dit la clé n'a aucune incidence sur la propagation de
            différence.}
        \item Soit $P : \{1, \dots, l\} \mapsto \{1, \dots, l\}$ une permutation, alors si $V =
            P(U)$ et $V' = P(U')$ alors $V \bigoplus V' = p(\Delta)$
    \end{enumerate}
\end{pr}

\begin{proof} Démontrons chacune des parties de la proposition séparément :\begin{enumerate}
    \item La démonstration est directe : 
        \begin{displaymath}
            \begin{array}{rcl}
                W \bigoplus W' & = & (U \bigoplus \not K) \bigoplus (U' \bigoplus \not K) \\
                               & = & U \bigoplus U' \\
                               & = & \Delta
            \end{array}
        \end{displaymath}
    \item $U$ et $U'$ sont de la forme : $U = [u_1, u_2, \dots, u_l]$ et $U' = [u'_1, u'_2, \dots,
        u'_l]$. On peut alors écrire : 
        \begin{displaymath}
            \begin{array}{rrcl}
                & V = P(U) & = & [u_{P(1)}, \dots, u_{P(l)}] \\
                & V' = P(U') & = & [u'_{P(1)}, \dots, u'_{P(l)}] \\
                \Rightarrow & V \bigoplus V' & = & [u_{P(1)} \bigoplus u'_{P(1)}, \dots, u_{P(l)}
            \bigoplus u'_{P(l)}] \\
                & & = & P([u_1 \bigoplus u'_1, \dots, u_l \bigoplus u'_l] \\
                & & = & P(\Delta) \\
            \end{array}
        \end{displaymath}
    \end{enumerate}
\end{proof}

%figure 3

Sur la ronde, on a la propagation : 
\begin{displaymath}
    \Delta \mapsto P(\Delta^*)
\end{displaymath}
et sa probabilité :
\begin{displaymath}
    P_{\Delta \mapsto P(\Delta^*)} = P_{\Delta_1 \mapsto \Delta_1^*} \times \dots P_{\Delta_t
    \mapsto \Delta_t^*}
\end{displaymath}

\section{Propagation de différence sur $r - 1$ rondes}

Pour construire une propagation de différence entre l'entrée de la première ronde et la sortie de la
$r^e$ ronde, il faut trouver $r-1$ propagation de ronde : $\Delta^{(i)} \mapsto \Delta^{*(i)}$ ou
$i$ est le numéro de la ronde tel que $\Delta^{*(i)} = \Delta^{(i+1)}$, pour $i = 1, \dots, r$.

% figure 4

On suppose les rondes \emph{indépendantes}\footnote{Ce n'est pas vrai en pratique.} celà implique
que la probabilité $\Delta^{(1)} \mapsto \Delta^{*(r-1)}$ entre l'entrée de la première ronde et la
sortie de le $(r - 1)^e$ ronde est donnée par : 
\begin{displaymath}
    P_{\Delta^{(1)} \mapsto \Delta^{*(r-1)}} \cong P_{\Delta^{(1)} \mapsto \Delta^{*(1)}} \times
    \dots \times P_{\Delta^{(r-1)} \mapsto \Delta^{*(r-1)}}
\end{displaymath}

On a une propagation de différence :
\begin{displaymath}
    \Delta^{(0)} \longrightarrow^{p} \Delta^{(r-1)}
\end{displaymath}

depuis le message clair, vers le message chiffré de l'avant dernière ronde avec une probabilité $p$.

\begin{df}
    Une $S-$boite $S_i$ est une boite active si la différence en entrée $\Delta_i^{(r-1)}$ est non
    nulle.
\end{df}

Les bits de la clé de la $r^e$ ronde que l'on va attaquer sont ceux correspondants aux bits sortant
des boites actives. En particulier, si l'on note $K' = P^{-1}(K^{(r)})$, alors la connaissance de
$K_i'$ permet, à partir d'un message chiffré $C$, de calculer $C_i^{(r)}$.

\section{Mise en place de l'attaque}

Pour une propagation de différence $\Delta^{(0)} \longrightarrow \Delta^{(r-1)}$, on a un nombre $N$
de couples $(M,C)$ et $(M \bigoplus \Delta^{(0)}, C')$ de messages clair-chiffré.

\begin{algorithm}
    \caption{Mise en place de l'attaque}
    \label{alg_att_diff}
    \begin{algorithmic}[1]
        \For{all $(M, C)$ and $(M', C')$}
            \For{all $\hat{K}'$ bits de $K'$ correspondant aux boites actives}
                \State On calcule les $C_i^{(r-1)}$ et $C_i^{'(r-1)}$ correspondant aux boites
                actives
                \If{$C_i^{(r-1)} \bigoplus C_i^{'(r-1)} = \Delta_i^{(r-1)}$ pour tout $i$ actif}
                    \State On augmente un compteur associé à $\hat{K'}$
                \EndIf
            \EndFor
        \EndFor
    \end{algorithmic}
\end{algorithm}

Si le nombre de bit $\hat{K'}$ vaut $\alpha$, pour une mauvaise valeur de $\hat{K'}$, le fait que
$C_i^{(r-1)} \bigoplus C_i^{'(r-1)} \forall i$ est vérifié avec une probabilité de $\displaystyle
\frac{1}{2^\alpha}$ et donc le compteur associé aura une valeur proche de $\displaystyle
\frac{N}{2^\alpha}$.

Pour une bonne clé, le fait que $C_i^{(r-1)} \bigoplus C_i^{'(r-1)} = \Delta_i^{(r-1)} \forall i$ se
produit avec une probabilité $p$\footnote{On suppose que $p > \frac{1}{2^\alpha}$.} et la valeur du
compteur attendue est de $N \times p$.

Finalement si $p \gg \frac{1}{2^\alpha}$ la valeur du compteur de la bonne clef satisfait $N_p \gg
\frac{N}{2^\alpha}$ donc a la plus grande valeur.

\chapter{Introduction \`a la cryptographie asym\'etrique}

En cryptographie classique, on a besoin :
\begin{itemize}
    \item une clé secrète $K$
    \item une fonction de chiffrement $e_K$
    \item une fonction de déchiffrement $d_K$
\end{itemize}
$d_k$ s'obtient facilement à partir de $e_k$, dans certains cas, les deux fonctions sont égales.
Tout la sécurité du cryptosystème repose sur le secret de la clé $K$. L'inconvénient de ce système
est le problème d'échange de clé secrète $K$.

C'est pour cette raison qu'a été introduite la notion de cryptographie asymétrique, cherchant à
répondre à cette question: \emph{Est-il possible de concevoir un cryptosystème tel que $d_K$ soit
difficile à calculer à partir de $e_k$?}. ON a donc :
\begin{itemize}
    \item une fonction de chiffrement \underline{publique} $e_k$
    \item une fonction de déchiffrement \underline{privée} connue uniquement de Bob $d_k$
\end{itemize}

L'article fondateur du concept est un article de 1976 par Diffie-Hellman, et la première
implémentation du concept est RSA (1977, Rivest-Shamir,Adleman). RSA fonctionne sur le problème de
la factorisation entière.

\begin{rq}
    Un cryptosystème asymétrique n'est jamais inconditionnellement sûr.
\end{rq}

Considérons $x \in X$, un message, $y = e_K(x) \in Y$ le chiffré de $x$. Si quelqu'un intercepte le
message $y$, il a toujours la possibilité de parcourir l'ensemble du domaine des messages et de les
crypter jusqu'à ce qu'il tombe sur $y$ : 
\begin{displaymath}
    \forall x' \in X, \mbox{ calculer }e_K(x')\mbox{ jusqu'à obtenir } e_k(x') = y
\end{displaymath}

Il est donc primordial d'étudier la \emph{sécurité calculatoire} d'un système asymétrique.

\section{Fonction à sens unique avec trappe}

\begin{df}
    Si $e_k$ est injective (si $x \neq y \Rightarrow e_k(x) \neq e_k(y)$), on dit qu'elle est à sens
    unique si $e_k$ est facile à calculer et si $e_k^{-1}$ est difficile sauf pour Bob qui connaît
    une information secrète supplémentaire, appelée la trappe.
\end{df}

\begin{rq}
    On ne sait pas prouver l'existence de telle fonctions.
\end{rq}

\begin{ex}[Fonction supposée à sens unique]
    Soit une fonction : 
    \begin{displaymath}
        f : \left \{ 
        \begin{array}{rcl}
            \frac{\mathbb{Z}}{n\mathbb{Z}} & \longrightarrow & \frac{\mathbb{Z}}{n\mathbb{Z}} \\
            x & \longmapsto & x^a \mod n \\
        \end{array} \right .
    \end{displaymath}
    Ce pour $x \in \mathbb{N}^*$ et $a \in \mathbb{Z}$.
    La fonction servant à inverser est la suivante :
    \begin{displaymath}
        f^{-1} : \left \{ 
        \begin{array}{rcl}
            \frac{\mathbb{Z}}{n\mathbb{Z}} & \longrightarrow & \frac{\mathbb{Z}}{n\mathbb{Z}} \\
            y & \longrightarrow & y^b \\
        \end{array}
        \right .
    \end{displaymath}
\end{ex}

\section{Rappels d'arithmétiques et d'algèbre}
\subsection{Divisibilité et primalité}

\begin{df}
    Soient $a, b \in \mathbb{Z}$, on dit que $b$ divise $a$ noté $b|a$, s'il existe $q$ tel que $a =
    q . b$
\end{df}
\begin{df}
    $p \in \mathbb{Z}$ est dit premier s'il est divisible uniquement par $1$ et par lui même, ($1$
    n'est pas premier par convention).
\end{df}
\begin{df}
    La division euclidienne de $a \in \mathbb{Z}$ par $b \in \mathbb{Z}$ est le couple $(q, r)$ tel
    que $a = b.q + r$ avec $0 \leq r < |b|$. $(q, r)$ est unique.
\end{df}

\subsection{PGCD}

\begin{df}
    Soient $a, b \in \mathbb{Z}$. Le plus grand diviseur commun à $a$ et $b$, noté $pgcd(a,b)$, ou
    $gcd(a,b)$ ou $(a,b)$ ou $a circonflexe b$, est le plus grand entier $d \in \mathbb{N}$ tel que $d|a$ et
    $d|b$.
\end{df}

\begin{df}
    Si $(a,b) = 1$ alors on dit que $a$ et $b$ sont premiers entre eux.
\end{df}

\subsubsection{Algorithme d'Euclide}

\underline{\textbf{Idée :}} $a, b \in \mathbb{Z}$, $a \geq b \geq 0$, $a = bq + r$ alors $(a, b) =
(b, r)$

\begin{proof}
    Soit $d = (a,b)$, $d' = (b, r)$, on veut montrer que $d|d'$ et $d'|d$ et donc $d = d'$.
    \begin{enumerate}
        \item $d|a \  d|b \Rightarrow d | (a - qb) \Rightarrow d|r$, or $d|b \mbox{ et } d|r \Rightarrow d|d'$
        \item $d'|b$ et $d'|r \Rightarrow d'|(bq + r) \Rightarrow d'|a$, or $d'|a et d'|b
            \Rightarrow d'|d$
    \end{enumerate}
\end{proof}

\begin{algorithm}[h]
    \caption{Algorithme d'Euclide}
    \begin{algorithmic}[1]
        \Function{PGCD}{$(a, b)$}
            \State $R_0 \gets a$
            \State $R_1 \gets b$
            \While{$R_1 \neq 0$}
                \State $T \gets R_1$
                \State $R_1 \gets R_0 \mod R_1$
                \State $R_0 \gets T$
            \EndWhile
            \State \Return $R_0$
        \EndFunction
    \end{algorithmic}
\end{algorithm}

\subsection{Relation de Bezout}

$a, b \in \mathbb{Z},\ d = (a,b)$, alors il existe $u, v \in \mathbb{Z}$ tel que $au + bv = d$

\subsubsection{Algorithme d'euclide étendu}

$a \geq b \geq 0,\ (r_i)_{i>0}$ la suite des restes de l'algorithme d'Euclide. $r_0 = a, r_1 = b,
r_{i+1} = r_{i-1} - q_i r_i$. Si $r_l$ est le dernier reste non nul, alors $r_l = u_la + v_l b = (a,
b)$ avec : \begin{displaymath}
    \left \lbrace \begin{array}{rcl}
        u_0 = 1 & u_1 = 0 & u_{i+1} = u_{i-1} - q_i u_i \\
        v_0 = 0 & v_1 = 1 & v_{i+1} = v_{i-1} - q_i v_i \\
    \end{array}
    \right .
\end{displaymath}

\begin{prop}
    Les suites $(u_i)_i$ et $(v_i)_i$ vérifient $u_i a + v_i b = r_i,\quad \forall i > 0$
\end{prop}

\begin{proof}[induction sur $i$]
    Initialisation OK.\\
    On suppose : \begin{displaymath}
        \begin{array}{rcl}
            u_{i-1}a + v_{i-1}b & = & r_{i-1} \\
            u_i a + v_i b & = & r_i \\
        \end{array}
    \end{displaymath}
\end{proof}

\subsection{Les entiers modulo $N$}

\begin{df}
    $n>0, a, b \in \mathbb{Z}$, on dit que $a$ est congrue à $b$ modulo $n$, noté $a \equiv b (mod
    n)$ si $n|(a-b)$
\end{df}

\begin{prop}
    $a \equiv b (mod n) \leftrightarrow a \mbox{ et } b$ ont le même reste dans la division par $n$
\end{prop}

\begin{df}
    $\forall a \in \mathbb{Z}$, la classe de $a$ est l'ensemble des entiers $b$ tels que $a \equiv b
    (mod n)$.
\end{df}

\begin{prop}
    $\forall a \in \mathbb{Z}, \exists ! 0 \leq r < n$ tel que $a \equiv r (mod n)$. On peut utiliser
    $r$ comme représentant de la classe de $a$.
\end{prop}

\begin{df}
    Les entiers modulo $n$, noté $\frac{\mathbb{Z}}{n\mathbb{Z}}$ est l'ensemble des classes
    d'équivalence des entiers $\{0, 1, 2, \dots, n-1\}$.
\end{df}

On munit $\frac{\mathbb{Z}}{n\mathbb{Z}}$ de deux opérations : $+ mod n$, $\times mod n$.

\begin{rq}
    $\mathbb{Z}/n\mathbb{Z}$ a une structure d'anneau.
\end{rq}

\begin{rq}
    $a+b mod n$ peut se calculer sans division, si $a+b < n$ alors $a+b$ sinon $a + b -n$
\end{rq}

\begin{rq}
    Il existe des algorithmes de multiplication modulo $n$ sans division. On note alors la
    complexité de la multiplacation de $a$ par $b$ ($|a| = n, |b| = m$) : $M(n,m)$.
\end{rq}

\begin{df}
    Soit $a \in \mathbb{Z}/n\mathbb{Z}$. L'inverse multiplicatif de $a mod n$ est l'entier $x \in
    \mathbb{Z}$ tel que $a.x \equiv 1 (mod n)$. Si $x$ existe, il est unique. On dit que $a$ est
    inversible (ou $a$ est une unité). On le note $a^{-1}$
\end{df}

\begin{df}
    La division de $a$ par $b$ modulo $n$ est le produit $a.b^{-1} mod n$ si $b^{-1}$ existe.
\end{df}

\begin{prop}
    $a \in \mathbb{Z}/n\mathbb{Z}$, $a$ est inversible ssi $(a,n) = 1$.
\end{prop}

% figure 1

Pour calculer $a^{-1} mod n$, on utilise Euclide étendu. En effet, d'après Bezout : $au + nv =
pgcd(a, n) = 1$, donc $a^{-1} = u$.

\subsection{L'ensemble $(\mathbb{Z}/n\mathbb{Z})^*$}

\begin{df}[Fonction d'Euler]
    Si $n \geq 1$, on note $\Phi(n)$ le nombre d'entier dans $\llbracket 1, n \rrbracket$ premiers
    avec $n$.
\end{df}

\begin{df}
    Le groupe multiplicatif $(\mathbb{Z}/n\mathbb{Z})^* = \{a \in \mathbb{Z}/n\mathbb{Z} : (a,n) =
    1\}$. L'ordre de $(\mathbb{Z}/n\mathbb{Z})^* = \#(\mathbb{Z} / n\mathbb{Z})^*) = |(\mathbb{Z} / n
    \mathbb{Z})^*|$.
\end{df}

\begin{rq}
    $|(\mathbb{Z}/n\mathbb{Z})^*| = \Phi(n)$
\end{rq}

\begin{prop}
    \begin{itemize}
        \item Si $p$ est premier, $\Phi(p) = p-1$
        \item $\Phi(m.n) = \Phi(m) . \Phi(n)$
    \end{itemize}
\end{prop}

\begin{thrm}[Théorème d'Euler]
    Si $a \in (\mathbb{Z}/n\mathbb{Z})^*$, alors $a^{\Phi(n)} \equiv 1 (mod n)$.
\end{thrm}

\begin{corol}[Petit théorème de Fermat]
    Si $(a, p) = 1$, alors $a^{p-1} \equiv 1 (mod n)$
\end{corol}

\begin{df}
    $a \in (\mathbb{Z} / n\mathbb{Z})^*$, l'ordre de $a$ noté $ord(a)$ est le plus petit entier $t$,
    tel que $a^t \equiv 1 (mod n)$.
\end{df}

\begin{prop}
    Si $ord(a) \equiv t$ et si $a^s \equiv 1$ alors $t|s$, en particulier, $t|\Phi(n)$
\end{prop}

\begin{df}
    Si $ord(a) = \Phi(n)$, alors on dit que $a$ est un générateur (ou élément primitif) de
    $(\mathbb{Z}/n\mathbb{Z})^*$
\end{df}

Si $(\mathbb{Z}/ n\mathbb{Z})^*$ admet un générateur alors on dit que $(\mathbb{Z}/n\mathbb{Z})^*$
est cyclique. Si $g$ est un générateur de $(\mathbb{Z}/n\mathbb{Z})^*$, $(\mathbb{Z}/n\mathbb{Z})^*
= \{g^{(i)} mod n : 0 \leq i < \Phi(n)\}$.

\subsubsection{Exponentiation rapide modulo $n$}

On veut calculer : $x^a = x.x.x. \dots . x,\quad a$ fois.

\underline{\textbf{Principe :}} \begin{displaymath}
    a^k = \left \lbrace
    \begin{array}{rl}
        a^{\frac{k}{2}} & \mbox{ si } k \mbox{ pair } \\
        a^{\frac{k}{2}} . a & \mbox{ sinon }
    \end{array}
    \right .
\end{displaymath}

% fig 2

\section{Echange de clé de Diffie-Hellman}

Soit $G = (\mathbb{Z}/p\mathbb{Z})^*$ avec $p$ premier et $g$ un générateur de $G$, qui sont des
données publiques.

Alice choisit en secret $a \in \mathbb{Z}$, calcule $h = g^a mod p$ et l'envoie à Bob. Bob fait de
même avec un aléatoire $b$ et envoie à Alice $k = g^b mod p$.

A la réception, Alice calcule $k^a mod p = (g^b)^a mod p = g^{ba} mod p$. De son côté Bob calcule
$k^b mod p = (g^a)^b mod p = g^(ab) mod p$. Ils ont donc calculé la même chose.

% fig 3

\subsection{Petit rappel Diffie-Hellman (KAS = Key agreement Scheme)}

Soit $(G, .)$ un groupe et $g \in G$ d'ordre $n$, Alice choisit $a$ au hasard ($0\leq a < n$),
calcule $\hat{a} = g^a$ et envoie $\hat{a}$ à Bob. De son côté, Bob choisit $b$ au hasard ($0 \leq b
< n$), calcule $\hat{b} = g^b$ et envoie $\hat{b}$ à Alice.

Une fois qu'ils ont reçu le message, Alice calcule $K = \hat{b}^a$ et Bob calcule $\hat{a}^b$. Les
deux calculs sont égaux. Ils ont alors leur clé secrète.

Le problème est basé sur le problème appelé CDH (Computational Diffie-Hellman).

\subsection{Computational Diffie-Hellman Problem}

Étant donnés, $g, \hat{a} = g^a, \hat{b} = g^b$, calculer $K = g^{ab}$.

\begin{conj}
    Si $n$ est suffisamment grand, CDH est \emph{impossible} à calculer.
\end{conj}

\begin{rmq}
    Le terme \emph{impossible} est à prendre avec des pincettes, il est possible de le calculer mais
    les ressources de calcul nécessaires à la résolution de ce problème sont très largement
    supérieures à ce que l'on est capable de faire aujourd'hui.
\end{rmq}

\begin{rmq}
    DHKAS est vulnérable à une attaque de type \emph{man in the middle}
\end{rmq}

\begin{df}
    Une attaque de type \emph{man in the middle} est une attaque où Eve récupère les messages
    circulant entre Alice et Bob et est capable de d'envoyer un autre message à la place.
\end{df}

\begin{displaymath}
    \begin{array}{ccccc}
        \underline{Alice} & & \underline{Eve} & & \underline{Bob} \\
        \\
        a, \hat{a} = g^a & \longrightarrow & a' : \tilde{a} = g^{a'} & \longrightarrow & \\
                         & \longleftarrow & b' : \tilde{b} = g^{b'} & \longleftarrow & g^b = \hat{b}, b \\
        K_A = g^{b'a} & & K_A = g^{ab'} & & \\
                      & & K_B = g^{a'b} & & K_B = g^{a'b} \\
    \end{array}
\end{displaymath}

On voit ici que Eve possède une clé pour Bob et une clé pour Alice, en interceptant tous les
messages entre Alice et Bob, elle est capable de les déchiffrer avec la clé pour Alice et de les
rechiffrer avec la clé pour Bob.

Une solution pour éviter ce genre d'attaque et de réaliser un échange de clé authentifié.

\section{RSA (Rivest, Shamir, Adleman, 1977)}

\subsection{Historique}

Les trois chercheurs travaillaient dans le même laboratoire, Rivest et Shamir ont passé pas mal de
temps à chercher des fonctions mathématiques mettant en oeuvre le principe de Diffie-Hellman. A
chaque trouvaille, ils allaient voir Adleman pour vérifier si leur fonction était sûre. Après
quelques temps ils ont fini par en trouver une que Adleman n'a pas été capable de casser. Shamir et
Rivest ont écrit leur article en mettant Adleman dans les auteurs, à la première position (par ordre
alphabétique). Ce dernier refusa d'apparaître dans les auteurs, mais suite à la pression de Rivest
et Shamir, il accepta à condition que son nom apparaisse en dernier.

\subsection{Principe}

\subsubsection{Cadre et notation}

\begin{displaymath}
    \begin{array}{ccccc}
        \mathcal{P} & = & \{ \mbox{ plain msg } \} & = & \mathbb{Z}/ n\mathbb{Z} \\
        \mathcal{C} & = & \{ \mbox{ chiffrés } \} & = & \mathbb{Z} / n\mathbb{Z} \\
        \mathbb{K} & = & \{ \mbox{ clés } \} & = & \{ (n, p, q, a, b) : ab \equiv 1 (\mod \Phi(n))\}
    \end{array}
\end{displaymath}

Où $n = p.q$, avec $p, q$ grands premiers. On a alors : 
\begin{displaymath}
    \Phi(n) = \Phi(p) . \Phi(q) = (p-1)(q-1)
\end{displaymath}

\subsection{Chiffrement}

Pour $K = (n, p, q, a, b)$ :
\begin{displaymath}
    \left .
    \begin{array}{ccccc}
        e_K(x) & = & x^b \mod n & = & y \\
        d_K(x) & = & y^a \mod n \\
    \end{array}
    \right \rbrace
    \quad \mbox{ où } (x, y) \in \mathbb{Z}/n\mathbb{Z}
\end{displaymath}

Les clés, $p, q, a$ sont des clés privées et les clés $n, b$ sont publiques.

\begin{thrm}
    Le déchiffrement d'un message chiffré, donne bien le message initial.
\end{thrm}

\begin{proof}
    Séparons deux cas :
    \begin{itemize}
        \item Si $x \in (\mathbb{Z}/n\mathbb{Z})^*$ :
            \begin{displaymath}
                \begin{array}{rcl}
                    ab & \equiv & 1 (\mod \Phi(n)) \\
                    ab & \equiv & t.\Phi(n) + 1 \\
                    \\
                    d_K(x) & \equiv & y^a \\
                           & \equiv & (x^b)^a \\
                           & \equiv & x^{ba} \\
                           & \equiv & x^{t\Phi(n) + 1} (\mod n) \\
                           & \equiv & (x^{\Phi(n)})^t . x (\mod n) \\
                           & \equiv & 1^t x (\mod n) \\
                           & \equiv & x (\mod n)\\
                \end{array}
            \end{displaymath}
        \item Si $x \in (\mathbb{Z}/n\mathbb{Z}) \\ (\mathbb{Z}/n\mathbb{Z})^*$, on a alors : 
            \begin{displaymath}
                \left .
                \begin{array}{rrcll}
                    & x & \equiv & 0 & (\mod p) \\
                    \Rightarrow x^{ab} & \equiv & 0 & (\mod p) \
                \end{array}
                \right \rbrace \Rightarrow
                \left .
                \begin{array}{rcll}
                    x & \equiv & x^{ab} & (\mod p) \\
                    x & \equiv & x^{ab} & (\mod q) \\
                \end{array}
                \right \rbrace
                \Rightarrow x \equiv x^{ab} (\mod pq)
            \end{displaymath}
    \end{itemize}
\end{proof}

\subsubsection{Généralisation}

Pour tout protocole asymétrique, il faut trois fonctions :
\begin{enumerate}
    \item Génération de la clé
    \item Chiffrement
    \item Déchiffrement
\end{enumerate}

% TP Coder RSA

\subsubsection{Génération des clés}

Il y a plusieurs impératifs à respecter :
\begin{itemize}
    \item choisir 2 grands premiers $p, q$ avec $p \neq q$
    \item calcul $n = pq$, $\Phi(n) = (p - 1)(q - 1)$
    \item choisir $b$ \underline{aléatoirement}, $(1 < b < \Phi(n))$ tel que $(b, \Phi(n)) = 1$
    \item calcul $a = b^{-1} \mod n$ à l'aide de l'algorithme d'Euclide étendu
    \item publier $(n, b)$ et garder secret $(p, q, a, \Phi(n))$
\end{itemize}

\subsection{Sécurité de RSA}

On cherche à calculer $a$, pour ce faire, il faut calculer $b^{-1} \mod \Phi(n)$, mais $\Phi(n)$
n'est pas connu. Si je sais factoriser $n = p.q$ alors je sais calculer $\Phi(n)$. RSA repose sur la
difficulté de factoriser $n$.

\begin{conj}
    RSA est polynomialement équivalent à la factorisation.
\end{conj}

\subsection{Attaque de Wiener}

Cette attaque exploite le fait que si $p$ et $q$ sont trop proches et que si l'exposant secret $a$
est trop petit\footnote{PLus précisément, si $3a < n^{1/4}$ et $q < p < 2q$.} alors :
\begin{displaymath}
    \left | \frac{b}{n} - \frac{t}{a} \right | < \frac{1}{3a^2}
\end{displaymath}
où $ab \equiv t \Phi(n) + 1$. On se trouve dans le cadre où $\frac{t}{a}$ est un convergent du
développement en fractions continues de $\frac{b}{n}$.

\section{Sécurité sémantique et modèle de sécurité}

\subsection{Type d'attaque}

Jusqu'à maintenant, lorsque l'on utilisait le terme de \emph{casser} comme équivalent de
\emph{récupérer la clé privée}. Mais il peut y avoir d'autres type d'objectifs moins ambitieux, tels
que : 
\begin{itemize}
    \item Bris total
    \item Inversion du chiffrement : retrouver l'intégralité d'un message clair à partir d'un
        message chiffré
    \item Sécurité sémantique : retrouver 1 bit d'information sur un message clair associé à un
        chiffré donné, c'est une attaque intéressante du point de vue du concepteur de
        cryptosystèmes. (RSA naïf n'est pas sémantiquement sûr : fuit du symbole de Jacobi puisque
        le symbole de Jacobi est identique pour le message clair et pour le message chiffré)
    \item Attaque par distingueur : avec une bonne probabilité ( > 1/2), l'adversaire est capable de
        distinguer d'un côté les chiffrés de deux messages clairs d'une part et d'autre part, les
        chiffrés d'un message clair et d'une chaîne aléatoire.
\end{itemize}

\subsection{Modèles de sécurité}

\begin{itemize}
    \item Attaque à clairs choisis non adaptatif : on choisit une série de messages et une fois
        qu'ils sont choisis on ne peut plus changer
    \item Attaque à clairs choisis adaptatif : on choisit les messages à chiffrer en fonction de la
        sortie des messsages choisis.
\end{itemize}

\section{Les cryptosystèmes basés sur le problème du logarithme discret}

On le note DLP (Discrete Log Problem).

Soit $(G, \times)$ un groupe fini, $\alpha \in G$ d'ordre $n$. Soit :
\begin{displaymath}
    <\alpha> = \{\alpha ^ i : 0 \leq i < n \}
\end{displaymath}
$<\alpha>$ est un sous groupe de $G$ d'ordre $n$, et $<\alpha>$ est cyclique.p

\subsection{Exemples classiques}

$G = (\mathbb{Z}/p\mathbb{Z})^*$ avec $p$ premier (si $p$ est premier, alors $G$ est un corps), on
notera alors $G = \mathbb{F}_p^*$ le sous groupe multiplicatif d'un corps fini $\mathbb{F}_p$, soit
$\alpha$ un générateur modulo $p$ et $n = |<\alpha>| = p-1$ ou : \\
$\alpha$ : élément d'ordre premier $q$ ($q / (p-1)$), on obtien $\alpha$ en calculant $\alpha =
e^{(p-1)/q} \mod p$ avec $e$ un générateur de $\mathbb{F}_p^*$.

% Vérifier sur un petit exemple

Le DLP est alors défini comme suit :
\begin{df}
    Soit $(G, \times)$, $\alpha \in G$ d'ordre $n$, $\beta \in <\alpha>$, trouver l'unique $0 \leq a <
    n$ tel que $\alpha^a = \beta \mod n$.

    On le note $a = \log_{\alpha}\beta$
\end{df}

\begin{rmq}
    DLP est \emph{difficile} alors que l'inverse (exponentiation mod $p$) est facile.
    On en déduit que l'exponentiation est une fonction à sens unique (dans certains groupes bien
    choisis)
\end{rmq}

\begin{rmq}
    $\mathbb{F}_p = \mathbb{Z} / n\mathbb{Z}$, si l'on prend $\mathbb{F}_p^*$ on revient à l'exemple
    présenté, la fonction exponentiation est donc bien à sens unique. Si au lieu de prendre
    $(\mathbb{Z}/n\mathbb{Z}, \times)$, on prend $(\mathbb{Z}/n\mathbb{Z}, +)$, alors le DLP devient
    : trouver $a$ tel que $a \alpha = \beta$, ce qui est trivial.
\end{rmq}

\subsubsection{El Gamal}

Soit $p$ premier tel que DLP dans $\mathbb{F}_p^*$ est difficile. Soit $\alpha \in \mathbb{F}_p^*$,
un générateur de $\mathbb{F}_p^*$. Soient : 
\begin{displaymath}
    \begin{array}{rcl}
        \mathcal{P} & = & \mathbb{F}_p^* \\
        \mathcal{C} & = & \mathbb{F}_p^* \times \mathbb{F}_p^* \\
        \mathbb{K} & = & \{(p, \alpha, a, \beta) : \beta \equiv \alpha^a (\mod p) \} \\
    \end{array}
\end{displaymath}

Pour $K  (p, \alpha, a; \beta)$ et pour $k \in \mathbb{Z}/(p-1)\mathbb{Z}$ choisi aléatoirement et
gardé secret : 
\begin{itemize}
    \item $e_K = (y_1, y_2)$, où :
        \begin{displaymath}
            \left \lbrace 
            \begin{array}{rcl}
                y_1 & = & \alpha^k \mod p \\
                y_2 & = & x \beta^k \mod p \\
            \end{array}
            \right \| 
            (y_1, y_2) \in \mathbb{F}_p^*
        \end{displaymath}
    \item $d_K(y_1, y_2) = y_2 (y_1^a)^{-1} \mod p$
\end{itemize}

\subsubsection*{Principe}

$x$ est masqué avec $\beta ^k$. Bob connaît $a$, il peut donc calculer $\beta^k$ à partir de
$\alpha^k$ (transmis avec le message chiffré) et d'enlever le masque divisant $y_2$ par $\beta^k$.

\begin{prop}
    El Gamal est viable (le message déchiffré est identique au message initial).
\end{prop}

\begin{proof}
    \begin{displaymath}
        \begin{array}{rcll}
            d_K(y_1, y_2) & \equiv & y_2 / (y_1^a) & \mod p \\
                          & \equiv & (x\beta ^k) / (\alpha^k)^a & \mod p \\
                          & \equiv & (x\beta^k) / (\alpha^a)^k & \mod p \\
                          & \equiv & x \beta^k / \beta^k & \mod p \\
                          & \equiv & x & \mod p \\
        \end{array}
    \end{displaymath}
\end{proof}

\subsubsection*{Sécurité}

Si on sait calculer $a = \log_{\alpha} \beta$ alors on peut déchiffrer, alors El Gamal repose sur le
DLP, ce qui implique que DLP dans $\mathbb{F}_p^*$ doit être difficile et donc celà suppose que $p >
2^{1024}$ et $p-1$ n'est pas friable\footnote{Il n'admet pas une multitude de petits facteurs
premiers.}

En pratique, El Gamal repose sur CDH (Computational Diffie Hellman)\footnote{$\alpha, \alpha^b,
\alpha^c \longrightarrow \alpha^{bc}$}, auquel est associé un problème de décision DDH (Decision
Diffie Hellman) : soient $\alpha, \alpha^b, \alpha^c, \alpha^d \rightarrow $ décider si $d \equiv bc
(\mod n)$.

Or il existe une réduction de Turing de DLP vers CDH et de CDH vers DDH. CDH est donc DLP-calculable
ou DLP-récursive.
\begin{displaymath}
    DDH \leq_T CDH \leq_T DLP
\end{displaymath}

\begin{rmq}
    Il faut trouver de bons groupes, c'est à dire qu'ils doivent vérifier les propriétés suivantes :
    \begin{itemize}
        \item avoir une représentation compacte (beaucoup de groupe par bit)
        \item loi de groupe facile à calculer 
        \item DLP difficile
    \end{itemize}
\end{rmq}

\begin{ex}
    On peut par exemple prendre le groupe des points d'une courbe elliptique ou jacobiennes de
    courbes hyperelliptiques de genre 2 (voire 3)
\end{ex}

\section{Signature numérique}

\begin{tabular}{c|c} 
    Signature traditionnelle & Signature numérique \\ \hline
    Fait partie du message & Est indépendante mais doit être liée au message \\
    Vérification par comparaison & Vérifiable par un algo publique \\
    $\Rightarrow$ facile à falsifier & $\Rightarrow$ difficile à falsifier \\
    Copie $\neq$ original & Copie $=$ original \\
                          & $\Rightarrow$ précautions pour éviter la réutilisation \\
\end{tabular}

Un schéma de signature c'est :
\begin{enumerate}
    \item un algo de signature
    \item un algo de vérification
\end{enumerate}

Il s'agit donc d'un quintuplet $(\mathcal{P}, \mathcal{A}, \mathcal{K}, \mathcal{S}, \mathcal{V})$ :
\\


\begin{tabular}{ccl}
    $\mathcal{P}$ & : & ensemble fini de messages possibles \\
    $\mathcal{A}$ & : & ensemble fini de signatures possibles \\
    $\mathcal{K}$ & : & espace des clés \\ \\
\end{tabular}

$\forall k \in \mathcal{K}$, alors il existe un algorithme $\sig_K \in \mathcal{S}$ et un algorithme
$\ver_K \in \mathcal{V}$ tel que :
\begin{displaymath}
    \ver_K(x, y) = \left \lbrace
    \begin{array}{rc}
        \mbox{vrai } & \mbox{ si } y = \sig_K(x) \\
        \mbox{faux } & \mbox{ sinon }
    \end{array}
    \right .
\end{displaymath}

Le couple $(x, y)$ représente le message signé. Les fonctions $\ver_K$ (publique) et $\sig_K$
(privée) sont polynomiales.

$\forall x \in \mathcal{P}$, il doit être impossible de calculer $y$ tel que $\ver_K(x, y) = $ vrai,
pour qui ne connaît pas $\sig_K$.

\begin{rmq}
    Il peut exister plusieurs signatures $y$ telles que $\ver_K(x, y) = $ vrai. Si Eve sait calculer
    $(x, y)$ tel que $\ver_K(x, y) = $ vrai (et que $x$ n'a pas été au préalable signé par Alice),
    alors $y$ est une contrefaçon.
\end{rmq}

\subsection{Signature RSA}

Soient $n = pq$, $\mathcal{P} = \mathcal{A} = \mathbb{Z} / n\mathbb{Z}$, $\mathcal{K} = \{ (n, p, q,
a, b) : ab \equiv 1 (\mod \Phi(n)) \}$. Pour $K \in \mathcal{K}$, on a : 
\begin{displaymath}
    y = \sig_K(x) = x^a \mod n 
\end{displaymath}
et :
\begin{displaymath}
    \ver_K(x, y) = \mbox{ vrai } \Leftrightarrow x \equiv y^b (\mod n)
\end{displaymath}

\begin{rmq}
    N'importe qui peut contrefaire un message à la place d'Alice : si l'on prend un $y$ au hasard,
    on calcule $x = e_K(y)$, alors $y = \sig_K(x)$ est une signature valide pour le message $x$.
    Heureusement, il ne semble pas exister de méhode de contrefaçon permettant de choisir $x$.
    Pour éviter ce genre de désagréments, on utilise une fonction de hachage.
\end{rmq}

\subsection{Sécurité sémantique des schémas de signature}

Il existe différents modèles similaires à ceux vus précédemment :
\begin{itemize}
    \item sans message : Eve n'a que la clé publique $\ver_k$
    \item à messages connus : Eve possède un ensemble de messages $\{(x_i, y_i)\}$ signés par Alice
        avec $y_i = \sig_K(x_i)$
    \item à messages choisis : Eve possède les signatures d'Alice pour un ensemble de $\{x_i\}$
        qu'elle (Eve) a choisi
\end{itemize}

De la même manière il existe polusieurs objectifs :
\begin{itemize}
    \item Bris total : récupérer la clé privée d'Alice
    \item Contrefaçon universelle : Eve peut signer tous les messages (avec une bonne probabilité)
    \item Contrefaçon existentielle : Eve peut créer une paire $(x, y)$ valide \footnote{C'est
            l'attaque la plus simple, puisque pour n'importe quel schéma de signature non
            correctement étudié, il est possible de d'abord générer une clé et ensuite de calculer
            le messages lui correspondant, même si le message n'a probablement aucun sens, il s'agit
        d'un exemple de contrefaçon existentielle.}
\end{itemize}

\subsection{Les fonctions de hachage}

En pratique, on ne signe pas directement un message $x$ mais $h(x)$ où $h$ est une fonction de
hachage cryptographiquement sûre. On peut en citer quelques une : \begin{itemize}
    \item MD4 (cassé)
    \item MD5 (utilisé pour le stockage de mot de passe sous UNIX/LINUX)
    \item SHA0 (cassé)
    \item SHA1 (cassé)
    \item SHA2
    \item SHA3
\end{itemize}

On a donc ce qui suit :
\begin{displaymath}
    \begin{array}{rcccccccl}
        \underline{\mbox{Alice}} & & & & & & & & \underline{\mbox{Bob}} \\
                                 & x & \longrightarrow & h(x) = z & \longrightarrow_{(x,y)} & z & =
        & h(x) & \\
        & y & = & \sig_K(z) & & \ver_K(z,y) & = & \mbox{vrai} & \\
    \end{array}
\end{displaymath}

La fonction de hachage doit vérifier certaines propriétés cryptographiques. Soit $h : X
\longrightarrow Y$, $X$ étant l'ensemble des messages : $X = \{$msg$\}$ et $Y$ l'ensemble des hachés
possibles $Y = \{$hachés$\}$ tels que $|X| > |Y|\quad (|X| > 2|Y|)$, alors $h$ doit vérifier :
\begin{enumerate}
    \item Résistance à la pré-image : étant donné $y \in Y$, il s'agit de trouver $x$ tel que $y =
        h(x)$
    \item Résistance à la seconde pré-image : étant donné $x \in X$, trouver $x' \in X$ tel que
        $h(x') = h(x)$ avec $x \neq x'$
    \item Résitance aux collisions : il s'agit de trouver $x, x', \ (x \neq x')$ tel que $h(x) \neq
        h(x')$
\end{enumerate}

\subsection{Signature de El Gamal}

\subsubsection{Schéma}

Soient $p$ un nombre premier tel que $DLP$ est difficile dans $\mathbb{F}_b^*$, soit $\alpha$ un
générateur de $\mathbb{F}_p^*$, $\mathcal{P} =\mathbb{F}_p^*$, $A = \mathbb{F}_p^* \times
\mathbb{Z}/(p-1)\mathbb{Z}$, $\mathcal{K} = \{(p, \alpha, a, \beta) : \beta^a (\mod p)\}$. Pour $K =
(p, \alpha, a, \beta)$ et $k$ aléatoire (secret) dans $\mathbb{Z}/(p-1)\mathbb{Z}$, on a :
\begin{displaymath}
    \begin{array}{rcl}
        \sig_K(x) & = & (\gamma, \delta) \\
        \gamma & = & \alpha^k \mod p \\
        \delta & = & (x - a \gamma)k^{-1} \mod p-1
    \end{array}
    \quad x, \gamma \in \mathbb{F}_p^*, \ \delta \in \mathbb{Z}/(p-1)\mathbb{Z}
\end{displaymath}

On a alors :
\begin{displaymath}
    \ver_k(x, (\gamma, \delta)) = \mbox{vrai} \Leftrightarrow \beta^{\gamma} \gamma^{\delta} \equiv \alpha^x (\mod p)
\end{displaymath}

\begin{proof}
    Dans le premier sens : 
    \begin{displaymath}
        \begin{array}{rcll}
            \beta^\gamma . \gamma^\delta & \equiv & (\alpha^a)^\gamma (\alpha^k)^\delta & \mod p \\
                                         & \equiv & \alpha^{a\gamma + k \delta} & \mod p \\
                                         & \equiv & \alpha^{a\gamma + k(x - a\gamma)k^{-1}} & \mod p\\
                                         & \equiv & \alpha^x & \mod p
        \end{array}
    \end{displaymath}

    Pour l'autre sens, il faut supposer $\exists \gamma_1, \delta_1$ tels que $\beta^{\gamma_1}.
    \gamma_1^{\delta_1} \equiv \alpha^x (\mod p)$ et prouver que $\gamma_1 = \gamma$ et $\delta_1 =
    \delta$
\end{proof}

\subsubsection{Problèmes de El Gamal}

La signature de El Gamal est très longue, elle est de l'ordre de deux fois la taille du message. Il
existe donc différentes améliorations pour El Gamal.

Parmis elle, l'algorithme de Schnorr qui consiste simplement à altérer le calcul de $\delta$ celui
ci est alors égal à : \begin{displaymath}
    \delta = (x-a\gamma)k^{-1} \mod q \quad \mbox{ avec } q | (p-1)
\end{displaymath}

On peut aussi citer DSA (Digital Signature Algorithm) qui fut le standard américain jusqu'en 2009,
et le nouveau standard ECDSA (Elliptic Curve Digital Signature Algorithm).

\section{Tests de primalité}

La question que l'on se pose est : \emph{Comment trouver de grands nombres premiers?}\footnote{De
l'ordre de 500 bits (pour une clé de 1000 bits).}

% Check algorithme de génération de nombres aléatoires
En pratique, on génère des nombres aléatoires et on teste leur primalité à l'aide d'algorithmes
probabilistes efficaces (polynomiaux). Dans la mesure ou les algorithmes sont probabilistes, il est
possible qu'il réponde $n$ est premier alors qu'il ne l'est pas. Pour minimiser la probabilité
d'erreur, on fait plusieurs tests.

\subsection{Combient doit on faire de tests?}

Le nombre de nombres premiers inférieurs à un nombre $N$ est à peu près égal à $N / \ln N$, on a
donc $p \leq N$ est premier avec une probabilité $1 / \ln N$ ( $2 / \ln N$ si $p$ est impair ).
D'où le fait qu'en faisant $\ln N/2$ tests, on a de bonnes chances de tomber sur un premier.

\begin{ex}
    Pour $2^{511} \leq N < 2^{512}$, $\ln N \approx 340$, il faut donc environ $170$ tests pour
    avoir de bonnes chances d'avoir un nombre premier.
\end{ex}

\subsubsection{Algorithme de type Monte-Carlo}

Il s'agit d'un algorithme probabiliste pour un problème de décision. On considèrera la version
suivante de l'algorithme :
\emph{Yes biaised Monte Carlo} : la réponse \emph{yes} est toujours correcte, mais la
réponse \emph{no} peut être fausse. On dit que ces algorithmes ont une probabilité d'erreur
$\epsilon$ si, pour toute instance vraie (réponse \emph{yes}) l'agorithme répond \emph{non}
avec une probabilité au plus $\epsilon$

\begin{rq}
    En pratique, les tests de primalité coûtent beaucoup trop cher, on s'intéresse donc au primal du
    test de primalité, appelé \emph{Compositness test}.
\end{rq}

Le problème Composite est le suivant : \emph{Etant donné un entier $n > 2$, est ce que $n$ possède
un facteur non trivial?}

\begin{rq}
    Si l'algorithme répond \emph{yes}, on ne demande pas le facteur.
\end{rq}

On introduit maintenant le test de Solovay-Strassen, il s'agit d'un \emph{yes biaised MC} avec
$\epsilon = 1/2$, s'il répond \emph{yes} alors $n$ n'est pas premier, s'il répond \emph{no}, on
n'est pas beaucoup plus avancé.

\begin{df}
    Soit $1 \leq x \leq p -1$, $x$ est un \underline{résidu quadratique modulo $p$} si $\exists y \in
    \mathbb{Z}/(p-1)\mathbb{Z}$ tel que $y^2 \equiv x \mod p$. So $\not \exists y$ tel que $y^2
    \equiv x \mod p$, alors $x$ est un \underline{non résidu quadratique modulo $p$}.
\end{df}

\begin{thrm}
    \label[thrm1]
    Soit $p>2$ premier, $x$ est résidu quadratique modulo $p$ ssi : \begin{displaymath}
        x^{(p-1) / 2} \equiv 1 (\mod p)
    \end{displaymath}
\end{thrm}

\begin{proof}
    Dans un premier temps on a :
    \begin{displaymath}
        \begin{array}{rcll}
            x^{(p-1) / 2} & \equiv & (y^2)^{(p-1) / 2} & \mod p \\
                          & \equiv & y^{p-1} & \mod p \\
                          & \equiv & 1 & \mod p \quad \mbox{d'après le petit théorème de Fermat}\\
        \end{array}
    \end{displaymath}

    Dans l'autre sens, considérons $g$ un générateur de $(\mathbb{Z}/(p-1)\mathbb{Z})^*$, on peut
    alors écrire : \begin{displaymath}
        \exists i \ (0 < i \leq p-1)\ : \quad x \equiv g^i\ (\mod p)
    \end{displaymath}
    On a donc : \begin{displaymath}
        x^{(p-1) / 2} \equiv g^{i(p-1)/2}
    \end{displaymath}
    Or par définition du générateur : \begin{displaymath}
        \begin{array}{rrcl}
            & \ord(g) & = & p-1 \\
            \Rightarrow & (p-1) & | & i(p-1)/2 \\
            \Rightarrow & i & & \mbox{est pair} \\
        \end{array}
    \end{displaymath}
\end{proof}

\begin{df}[Symbole de Legendre]
    Soit $p$ premier : 
    \begin{displaymath}
        \left ( \frac{a}{p} \right ) = \left \lbrace \begin{array}{rcl}
            0 & \mbox{si} & a|p \\
            1 & \mbox{si} & a \mbox{ est un résidu quadratique modulo }p \\
            -1 & \mbox{si} & a \mbox{ est un non résidu quadratique modulo }p \\
        \end{array} \right .
    \end{displaymath}
\end{df}

\begin{thrm}
    Soir $p > 2$ un nombre premier, alors : \begin{displaymath}
        \forall a \legendre{a}{p} = a^{(p-1)/2} \mod p
    \end{displaymath}
\end{thrm}

\begin{proof}
    Si $a | p,\ a^{(p-1) / 2} \equiv 0 (\mod p)$, si $a$ est un résidu quadratique modulo $p$, alors
    d'après le théorème~\ref{thrm1}, $a^{(p-1) / 2} \equiv 1 (\mod p)$.

    Si $a$ est un non résidu quadratique modulo $p$ : \begin{displaymath}
        a^{(p-1)/2} \equiv -1 (\mod p) \Leftarrow a^{p-1} \equiv 1 (\mod p)
    \end{displaymath}
\end{proof}

\begin{df}[Symbole de Jacobi]
    Soit $n$ un entier positif impair :
    \begin{displaymath}
        n = \Pi_{i = 1}^k p_i^{e_i} (p_i \mbox{ premier })
    \end{displaymath}
    Soit $a \geq 0$, alors : \begin{displaymath}
        \legendre{a}{n} = \Pi_{i=1}^k \legendre{a}{p_i}^{e_i}
    \end{displaymath}
\end{df}

\begin{ex}
    Calculer $\legendre{208}{525}$ : \\
    \begin{displaymath}
        525 = 5 . 105 = 5^2 . 21 = 3 . 5^2 . 7 
    \end{displaymath}
    On a donc :\begin{displaymath}
        \legendre{208}{525} = \legendre{208}{3} \legendre{208}{5}^2 \legendre{208}{7} = 
        \legendre{1}{3} \legendre{3}{5}^2 \legendre{5}{7} = 1 . (-1)^2 . (-1) = -1
    \end{displaymath}
\end{ex}

\begin{prop}
    Soit $ > 2$ impair: \begin{itemize}
        \item Si $n$ est premier, alors \begin{displaymath}
                \legendre{a}{n} = a^{(n-1)/2} \mod p \quad \forall a
            \end{displaymath}
        \item Si $n$ n'est pas premier, et si $\exists a$ tels que :
            \begin{displaymath}
                \legendre{a}{n} = a^{(n-1)/2} \mod n
            \end{displaymath}
            $n$ est alors appelé un \underline{pseudo premier d'Euler} en base $a$
    \end{itemize}
\end{prop}

\begin{ex}
    \begin{displaymath}
        \legendre{10}{91} = -1 = 10^{45} \mod 91
    \end{displaymath}
    $91$ est un pseudo premier d'Euler en base $10$
\end{ex}

\begin{rq}
    On peut montrer que $\forall n$ impair, non premier, $n$ est un pseudo premier d'Euler pour au
    plus la moitié des entiers $1 \leq a \leq n-1$
\end{rq}

On peut maintenant définir l'algorithme de Solovay-Strassen:
\begin{algorithm}[h!]
    \caption{Algorithme de Solovay-Strasse}
    \label{solovay-strassen}
    \begin{algorithmic}[1]
        \Function{Test}{$n>2$}
            \State Choisir $1 \leq a \leq n-1$ de manière aléatoire
            \If{$\displaystyle \legendre{a}{n} \equiv a^{(n-1)/2} (\mod n)$}
                \State \Return $n$ n'a pas de facteurs
            \Else
                \State \Return $n$ a un facteur
            \EndIf
        \EndFunction
    \end{algorithmic}
\end{algorithm}
            
\begin{prop}
    L'algorihtme de SS est de type \emph{yes biaised MC} avec une probabilité d'erreur au plus
    $1/2$.
\end{prop}

\subsection{Complexité : Est ce que SS est polynomial?}

On sait que calculer $a^{(p-1) / 2} \mod p$ est polynomial.

\end{document}

