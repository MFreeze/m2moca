% {{{ PACKAGES
% {{{ ENCODAGE ET POLICE
\usepackage[utf8]{inputenc}
\usepackage[T1]{fontenc}
\usepackage[french]{babel}
\usepackage{eurosym}
% }}}

% {{{ FIGURES
\usepackage{graphicx}
\usepackage{subfig}
% }}}

% {{{ INSERTION DE CODE
\usepackage{moreverb}
% }}}

% {{{ INSERTION DE PDF
\usepackage{pdfpages}
% }}}

% {{{ RÉDUCTION DES MARGES
%\usepackage{fullpage}
%\usepackage[top=2.5cm, bottom=2.5cm, left=3.5cm, right=3.5cm]{geometry}
% }}}

% {{{ MATHÉMATIQUES
\usepackage{amsmath}
\usepackage{amsthm}
\usepackage{bbm}
\usepackage{amssymb}
\usepackage{stmaryrd}
% }}}

% {{{ GRAPHES
\usepackage{tikz}
\usepackage{pgf}
\usetikzlibrary{arrows,automata,shapes}
% }}}

% {{{ RESEAUX DE pETRI
\usetikzlibrary{petri}
% }}}

% {{{ ARBRES
\usetikzlibrary{trees}
% }}}

% {{{ DIAGRAMME DE GANTT
\usepackage{pgfgantt}
% }}}

% {{{ ALGORITHMES
\usepackage{algorithm}
\usepackage[noend]{algpseudocode}
%\usepackage{algorithmic}
%\usepackage{algorithmicx}
% }}}
% }}}

% {{{ NEWTHEOREM
% {{{ THÉORÈMES, lEMMES -- NUMÉROTATION PERSONNALISÉE
\theoremstyle{plain}
\newtheorem{thrm}{Théorème}[section]
\newtheorem{lemm}{Lemme}[subsection]
\newtheorem{corol}{Corollaire}[section]
% }}}

% {{{ THÉORÈMES, lEMMES -- NUMÉROTATION CLASSIQUE
\newtheorem{nthrm}{Théorème}
\newtheorem{nlemm}{Lemme}
\newtheorem{ncorol}{Corollaire}
\newtheorem{npb}{Problème}
% }}}

% {{{ THÉORÈMES, lEMMES -- NON NUMÉROTÉS
\newtheorem*{thrm*}{Théorème}
\newtheorem*{lemm*}{Lemme}
\newtheorem*{corol*}{Corollaire}
% }}}

% {{{ DÉFINITIONS, pROPRIÉTÉS -- NUMÉROTATION PERSONNALISÉE
\theoremstyle{definition}
\newtheorem{df}{Définition}[subsection]
\newtheorem{prop}{Propriété}[subsection]
\newtheorem{conj}{Conjecture}[subsection]
% }}}

% {{{ DÉFINITIONS, pROPRIÉTÉS -- NUMÉROTATION CLASSIQUE
\newtheorem{ndf}{Définition}
\newtheorem{nprop}{Propriété}
\newtheorem{nconj}{Conjecture}
% }}}

% {{{ DÉFINITIONS, pROPRIÉTÉS -- NON NUMÉROTÉS
\newtheorem*{df*}{Définition}
\newtheorem*{prop*}{Propriété}
\newtheorem*{conj*}{Conjecture}
% }}}

% {{{ AUTRES -- NUMÉROTATION PERSONNALISÉE
\theoremstyle{remark}
\newtheorem{nota}{Note}[subsubsection]
\newtheorem{pr}{Proposition}[subsection]
\newtheorem{rmq}{Remarque}[subsubsection]
% }}}

% {{{ AUTRES -- NUMÉROTATION CLASSIQUE
\newtheorem{nnota}{Note}
\newtheorem{npr}{Proposition}
\newtheorem{nrmq}{Remarque}
% }}}

% {{{ AUTRES -- NON NUMÉROTÉS
\newtheorem*{nota*}{Note}
\newtheorem*{pr*}{Proposition}
\newtheorem*{rmq*}{Remarque}
% }}}

% {{{ EXEMPLES
\newtheorem*{ex}{Exemple}
\newtheorem*{exo}{Exercice}
\newtheorem*{rappel}{Rappel}
% }}}
% }}}

% {{{ OPÉRATEURS MATHÉMATIQUES
\DeclareMathOperator{\poly}{\textbf{poly}}
\DeclareMathOperator{\gap}{\mbox{gap}}
\DeclareMathOperator{\Bg}{\mbox{Big}}
\DeclareMathOperator{\Sml}{\mbox{Sml}}
\DeclareMathOperator{\Diff}{\mbox{Diff}}
\DeclareMathOperator{\card}{\mbox{Card}}
\DeclareMathOperator{\Obj}{\mbox{Obj}}
\DeclareMathOperator{\concat}{\bullet}
\DeclareMathOperator{\sig}{\mbox{Sig}}
\DeclareMathOperator{\ver}{\mbox{Ver}}
\DeclareMathOperator{\dist}{\mbox{dist}}
\DeclareMathOperator{\bw}{\mbox{branchwidth}}
\DeclareMathOperator{\tw}{\mbox{tw}}
\DeclareMathOperator{\lm}{\leq_m}
\DeclareMathOperator{\lc}{\leq_c}
\DeclareMathOperator{\lex}{\mbox{lex}}
\DeclareMathOperator{\ord}{\mbox{ord}}
% }}}

% {{{ MACCROS
% {{{ MACROS DIVERSES
\newcommand{\npc}{$NP-$complet}
\newcommand{\npd}{$NP-$difficile}
\newcommand{\la}{largeur arborescente}
\newcommand{\legendre}[2]{\left ( \frac{#1}{#2} \right )}
\newcommand{\wqo}[0]{\emph{well-quasi-ordered}}
\newcommand{\Tau}[0]{\mathcal{T} = (T, \{X_t : t \in T\})}
\newcommand{\Taug}[0]{\mathcal{T}_G = (T, \{X_t : t \in T\})}
\newcommand{\Taustar}[0]{\mathcal{T}^* = (T, \{X_t : t \in T\})}
\newcommand{\Taugstar}[0]{\mathcal{T}_G^* = (T, \{X_t : t \in T\})}
\newcommand{\pdsol}[1]{(X_{#1}^0, X_{#1}^1, X_{#1}^2, M_{#1})}
% }}}
                
% {{{ NOM DE PROBLEMES
\newcommand{\hcycle}{\textsc{HAMILTONIAN CYCLE} }
\newcommand{\phcycle}[1]{\textsc{HAMILTONIAN CYCLE }paramétré par \textsc{#1} }
\newcommand{\vcover}{\textsc{VERTEX COVER} }
\newcommand{\twidth}{\textsc{TREE WIDTH} }
\newcommand{\wiset}{\textsc{WEIGHTED INDEPENDANT SET} }
\newcommand{\fvset}{\textsc{FEEDBACK VERTEX SET} }
\newcommand{\kvcover}{\textsc{K-Vertex Cover} }
\newcommand{\oct}{\textrm{\textsc{Odd Cycle Transversal}} }
\newcommand{\dlp}{\textrm{\textsc{Discrete Logarithm Problem}} }
\newcommand{\ecdlp}{\textrm{\textsc{Elliptic Curve Discrete Logarithm Problem}} }

% }}}

% {{{ MISE EN PAGE

\newcommand{\dfpbi}[3]{
    \begin{npb}[#1]
        \begin{itemize}
            \itemindent 10mm
            \item[\textbf{Données :}] #2
            \item[\textbf{Problème :}] #3
        \end{itemize}
    \end{npb}
}

\newcommand{\dfpb}[3]{
    \begin{npb}[#1]
        \begin{itemize}
            \itemindent 10mm
            \item[\textbf{Données :}] #2
            \item[\textbf{Question :}] #3
        \end{itemize}
    \end{npb}
}

\newcommand{\dfpbp}[4]{
    \begin{npb}[#1]
        \begin{itemize}
            \itemindent 10mm
            \item[\textbf{Données :}] #2
            \item[\textbf{Paramètre :}] #4
            \item[\textbf{Question :}] #3
        \end{itemize}
    \end{npb}
}
% }}}

% }}}
    
% {{{ TIKZ
%%% {{{ STYLE
\tikzstyle{edge}=[
	-,
	draw=black
]

\tikzstyle{sized}=[
    minimum width=9mm
]

\tikzstyle{tvertex}=[
	node distance=15mm,
    inner sep = 1mm
]

\tikzstyle{tvert}=[
	node distance=5mm,
    inner sep=1mm
]

\tikzstyle{tmvertex}=[
    tvertex,
    sized
]

\tikzstyle{tmvert}=[
    tvert,
    sized
]

\tikzstyle{mvertex}=[
    tmvertex,
	circle,
	draw=black,
	fill=black!50
]

\tikzstyle{mvert}=[
    tmvert,
	circle,
	draw=black,
	fill=black!50
]

\tikzstyle{vertex}=[
    tvertex,
	circle,
	draw=black,
	fill=black!50
]

\tikzstyle{vert}=[
    tvert,
	circle,
	draw=black,
	fill=black!50
]

\tikzstyle{treedec}=[
	vertex,
	node distance=15mm,
	draw=red,
	fill=red!50
]

\tikzstyle{treeedge}=[
	edge,
	draw=red
]

\tikzstyle{giedge}=[
	edge,
	draw=blue
]

\tikzstyle{bounding}=[
	thick, %ultra
	%dash pattern=on 1mm off 1mm,
	densely dotted,
	fill=blue!10,
	draw=blue!70
]

\tikzstyle{patate}=[
    rounded corners=5pt,
    draw=black,
    fill=black!50,
    text width=12mm,
    text centered
]

\tikzstyle{patatoid}=[
    ellipse,
    node distance=10mm,
    draw=black
]
%%% }}}

% {{{ DESSINS
\newcommand{\base}[4]{
    \draw[->] (0,0) to (0, #1);
    \draw[->] (0,0) to (#2, 0);
    
    \node at (0, #1+2) {#3};
    \node at (#2 + 2, -2) {#4};
}
% }}}
%}}}



% Algo
% fonction revise3(xij : var, cij : constante) : Booleen
% CHANGE <- F
% for each vi \in D(Xi) do
%   v_J <- first(D(xj));
%   while non cij(vi, vj) and vj != nil do
%       vj <- next(vj, D(Xj))
%   if vj = nil then
%       remove vi from D(Xi)
%       CHANGE <- T
% return CHANGE

% Adaptive Consistancy Deckter et Pearl 89
% Article d'origine
% Lire et comprendre l'article

\begin{document}

\section{Introduction}

\subsection{Problème de coloration de cartes}

\underline{Complexité AC3 :}

% A faire

\begin{algorithm}[t]
    \caption{AC 2001}
    \label{algo_ac2001}
    \begin{algorithmic}[1]
        \Function{Revise 2001}{$x_{ij}$ : var, $c_{ij}$ : constant}
            \State CHANGE $\gets F$;
            \For{each $v_i \in D(x_i) \And \mbox{Last}(x_i, c_{ij} \not \in D(x_j)$}
                \State $v_j \gets \mbox{next}(\mbox{Last}(x_i, c_{ij}))$;
                \While{$\neg c_{ij}(v_i, v_j) \And v_j \neq \mbox{nil}$}
                    \State $v_j \gets (v_j, D-x_j))$;
                \EndWhile
                \If{$v_j = \mbox{nil}$}
                    \State remove $v_i$ from $D(x_i)$;
                    \State CHANGE $\gets T$;
                \EndIf
            \EndFor
            \State \Return CHANGE;
        \EndFunction
    \end{algorithmic}
\end{algorithm}

\subsection{K-consistance}

$N = (X, D, C)$ is $k-$consistant iff: \begin{displaymath}
\exists Y \subseteq X, | Y | = k -1, \quad \forall I \mbox{ locally consistant on } Y, \forall
X_{ik} \in X \backslash Y, \exists v_{ik} \in D(X_{ik}),\newline \mbox{ such that } I \cup \{(X_{ik} =
v_{ik})\} \mbox{ is locally consistant. }
\end{displaymath}

% figure 1
% figure 2

\underline{Note :} Si $n$ est binaire et il existe au plus une contrainte par paire de variables,
alors $2-$consistance est équivalent à arc-consistance.

\begin{df}
    $N$ est fortement $k-$consistant ssi il est $j-$consistant $\forall j \in 1, \dots, k$
\end{df}

\subsubsection*{Complexité}

Si on veut l'optimalité en temps on obtient les complexité suivantes :
\begin{itemize}
    \item temps : $O(n^kd^k)$
    \item espace : $O(n^{k+1}d^{k+1})$
\end{itemize}
Si on veut une complexité optimale en espace on obtient les complexités suivantes :
\begin{itemize}
    \item temps : $O(n^{k+1}d^{k+1})$
    \item espace : $O(n^kd^k)$
\end{itemize}

\begin{df}
    Une instance est dite \em{globalement consistante} ssi elle s'étend en une solution. Un réseau est
    dit \em{globalement consistant} ssi toute instance localement consistante est globalement
    consistante.
    \begin{displaymath} I \in sol(N)[Y] \end{displaymath}
\end{df}

\begin{prop}
    Un réseau est globalement consistant ssi il est fortement $n-$consistant.
\end{prop}

\subsection{Théorèmes de Freuder}

\begin{df}
    On appelle \em{graphe ordonné} un graphe dont les sommets sont ordonnés.
\end{df}
\begin{df}
    On appelle \em{largeur d'un sommet} le nombre de prédécesseurs de ce sommets.
\end{df}
\begin{note}
    \begin{displaymath}
        \begin{array}{rcl}
            \Lambda (X) & = & \Lambda^-(X) \cup \Lambda^+(X)\\
            \Lambda^-(X) & = & \{Y | Y < X \mbox{ et } Y \in \Lambda(X)
        \end{array}
    \end{displaymath}
\end{note}
\begin{df}
    On appelle \em{largeur d'un ordre} la largeur maximale des sommets de cet ordre.
\end{df}
\begin{df}
    On appelle \em{largeur d'un graphe} la largeur de tous les ordres.
\end{df}

\subsection{Algorithme de Freuder}

L'algorithme permet de connaître la largeur d'un graphe $G$.

\begin{algorithm}
    \caption{Algorithme de Freuder}
    \label{algo_freuder}
    \begin{algorithmic}[1]
        \State $K \gets 0$;
        \State Supprimer les sommets isolés;
        \While{$G \neq \emptyset$}
            \State $K \gets K + 1$;
            \While{$\exists X \in G, |\Lambda(X)| \leq K$}
                \State Supprimer $X$;
            \EndWhile
        \EndWhile
        \State /* ordre inverse d'élimination = largeur min */
        \State \Return $K$
    \end{algorithmic}
\end{algorithm}

% figure 3

\begin{thrm}[Premier théorème de Freuder]
    Si un réseau de largeur $k$ est fortement $(k+1)-$consistant, alors il existe un ordre
    d'instanciation sans retour arrière.
\end{thrm}

Le problème de ce théorème est que lorsque l'on cherche à rendre le graphe fortement
$(k+1)-$consistant, l'algorithme rajoute des arêtes sur le graphe et change donc la largeur du
graphe. Le théorème est donc inutile sauf pour les graphes de largeur 1.

\begin{corol}
    Si $N$ est de largeur 1\footnote{Il s'agit alors d'un arbre.}, on peut le résoudre en $O(nd^2)$,
    c'est à dire AC + BT.
\end{corol}

\section{Algorithmes de recherche}

Pour améliorer l'algorithme de BackTrack, on paut :
\begin{itemize}
    \item méthodes rétrospectives\footnote{look-back methods}
    \item méthodes de propogation de contraintes
    \item méthodes d'ordonnancement des variables
    \item méthodes d'ordonnancement des valeurs
\end{itemize}

\subsection{Méthodes rétrospectives}

Ces méthodes ne sont plus au goût du jour, nous ne nous intéresserons qu'aux méthodes les plus
abouties. Ces méthodes se basent sur une meilleure sélection des points de retour arrière.

Voir le Backjumping et la mémorisation des "nogoods" (instanciations impossibles).

\subsection{Méthodes prospectives}

Le principe de base est donné par ce qui suit :\\
A chaque niveau de l'arbre de décision de $1, \dots, i, \dots, n$, on instancie $x_i$ et on supprime
des domaines des variables succédant $x_i$ dans l'arbres, les valeurs incompatibles avec $x_i$. Et
si il existe une variable successeure dont le domaine est vide, on réalise alors un backtrack et on
coupe la branche de l'arbre.

Considérons la procédure \em{Forward Checking}\footnote{Harlick Elliott 80}

\begin{algorithm}
    \caption{Forward Checking}
    \label{algo_fc}
    \begin{algorithmic}[1]
        \Function{FC}{$(N, I)$ : Booléen}
            \If{$|I| = n$}
                \State \Return T
            \EndIf
            \State Choisir $X_i \not \in I$;
            \For{ each $V_i \in D(X_i)$}
                \State $D(X_i) \gets \{V_i\}$;
                \State Supprimer tous les $(X_j, V_j)$ inconsistants avec $I \cup \{(x_i = v_i)\}, \forall X_J \not \in I$;
                \If{$\not \exists J | D(X_J) = \emptyset$}
                    \If{$FC(N, I\cup \{(x_i, v_i)\}$}
                        \State \Return T
                    \EndIf
                \EndIf
                \State Restaurer les caleurs des $X_J$ supprimées à cause de $X_i = V_i$;
            \EndFor
            \State \Return False
        \EndFunction
    \end{algorithmic}
\end{algorithm}

\subsection{Algorithme MAC\footnote{Maintaining Arc Consistency}}

\begin{algorithm}
    \caption{Maintaining Arc Consistency}
    \label{alg_mac}
    \begin{algorithmic}[1]
        \Function{MAC}{$N$ : Boolean}
            \If{$AC(N)$}
                \If{$N$ ground}
                    \State \Return $T$
                \EndIf
                \State Choose $X_i$ unfixed in $N$ and $v_i \in D(X_i)$
                \If{$MAC(N \cup \{ X_i = v_i \})$}
                    \State \Return $T$
                \EndIf
                \If{$MAC(N \cup \{x_i \neq v_i\})$}
                    \State \Return $T$
                \EndIf
            \EndIf
            \State \Return $F$
        \EndFunction
    \end{algorithmic}
\end{algorithm}

\subsection{Ordonnancement des variables}

En 1980, Haralick et Elliott ont ennoncé le \emph{fail first principle}\footnote{On cherche à
assigner d'abord les variables étant à même de provoquer le plus rapidement possible des erreurs.}
et ont proposé une heuristique \emph{dom}.

\subsubsection{Heuristiques statiques (Static Variable Orderind)}

\begin{df}
    \emph{deg} \\
    On choisit la variable de degré maximum dans le graphe de contraintes.
\end{df}

C'est une heuristique statique puisque le graphe des contraintes et statique tout au long de la
recherche et donc l'ordre d'évaluation reste le même.

\subsubsection{Heuristiques dynamiques}

\begin{df}
    \emph{dom}\\
    On choisit la variable de plus petit domaine.
\end{df}

C'est une heuristique dynamique ou statique selon l'algorithme utilisé : par exemple avec le
backtrack, c'est une heuristique statique dans le cas du Forward Checking c'est une heuristique
dynamique.

\section{Décomposition de contraintes globales}

\subsection{Décomposabilité opérationnelle}

\begin{df}
    Soit une instance de contrainte globale $N = (X, D_X, \{c\})$ et sa décomposition $M = (X \cup Y, D_X +
    D_Y, \mathcal{C})$. On dit que $M$ préserve l'arcconsistance sur $N$ ssi pour tout $D'_X
    \sqsubseted D_x$, $AC(X, D_X', \{c\}) = AC(X \cup Y, D'_x \cup D_y, \mathcal{C})$.
\end{df}

\begin{df}
    Checker$(C, D) =$ vrai ssi $\exist I$ sur $D^{X(c)} / c(I)$
\end{df}


\end{document}
