\begin{figure}
    \begin{center}
        \begin{tikzpicture}[every transition/.style={minimum width=2mm, minimum height=7mm}, >=latex]
            \node[place,label=$p_1$,tokens=1]                   (p1) {};
            \node[transition, right=of p1, label=$t_1$]         (t1) {}
                edge[pre] (p1);
            \node[place, below right=of t1, label=$p_2$]        (p2) {}
                edge[pre] (t1);
            \node[place, above right=of t1, label=$p_3$]        (p3) {}
                edge[pre] (t1);

            \foreach \x/\y in {2/4, 3/5, 4/6, 5/7}{
                \node[transition, right=of p\x, label=$t_{\x}$] (t\x) {}
                    edge[pre] (p\x);
                \node[place, right=of t\x, label=$p_{\y}$]      (p\y) {}
                    edge[pre] (t\x);
            }

            \node[transition, above right=of p6, label=$t_6$]   (t6) {}
                edge[pre] (p6)
                edge[pre] (p7);    

            \draw[post] (t6) to (11, 0)
                to (11, -3)
                to (-1, -3)
                to (-1, 0)
                to (p1);
        \end{tikzpicture}
    \end{center}
    \caption{Réseau de Petri modélisant le système de commandes}
    \label{pnetexo1}
\end{figure}

\begin{table}
    \begin{center}
        \begin{tabular}{c|c}
            Place & Correspondance \\\hline
            $p_1$ & $ \displaystyle \overline{Ag} \wedge \overline{Rg} \wedge \overline{Ad} \wedge \overline{Rd}$ \\
            $p_2$ & $ Ag \wedge \overline{Rg}$ \\
            $p_3$ & $ Ad \wedge \overline{Rd}$ \\
            $p_4$ & $ \overline{Ad} \wedge Rd$ \\
            $p_5$ & $ \overline{Ag} \wedge Rg$ \\
            $p_6$ & $ \overline{Ag} \wedge \overline{Rg}$ \\
            $p_7$ & $ \overline{Ad} \wedge \overline{Rd}$ \\
        \end{tabular}
    \end{center}
    \caption{Table de correspondance \emph{Places - États du système}}
    \label{tbexo1}
\end{table}

\begin{figure}
    \begin{center}
        \begin{tikzpicture}[>=latex]
            \node (1) {$1000000$};
            \node[below=of 1] (2) {$0110000$}
                edge[pre] node[auto] {$t_1$} (1);
            \node[below left=of 2] (3) {$0011000$}
                edge[pre] node[auto] {$t_2$} (2);
            \node[below right=of 2] (4) {$0100100$}
                edge[pre] node[auto] {$t_3$} (2);
            \node[below left=of 3] (5) {$0010010$}
                edge[pre] node[auto] {$t_4$} (3);
            \node[below right=of 3] (6) {$0001100$}
                edge[pre] node[auto] {$t_3$} (3)
                edge[pre] node[auto] {$t_2$} (4);
            \node[below right=of 4] (7) {$0100001$}
                edge[pre] node[auto] {$t_5$} (4);
            \node[below right=of 5] (8) {$0000110$}
                edge[pre] node[auto] {$t_3$} (5)
                edge[pre] node[auto] {$t_4$} (6);
            \node[below left=of 7] (9) {$0001001$}
                edge[pre] node[auto] {$t_5$} (6)
                edge[pre] node[auto] {$t_2$} (7);
            \node[below right=of 8] (10) {$0000011$}
                edge[pre] node[auto] {$t_5$} (8)
                edge[pre] node[auto] {$t_4$} (9);

            \draw[->] (10) to[out=0, in=270] (7, -4) to[out=90, in=0] (1);
        \end{tikzpicture}
    \caption{Graphe des marquages accessibles}
    \label{gmaexo1}
    \end{center}
\end{figure}

Considérons le réseau de Petri de la figure~\ref{pnetexo1} modélisant le problème. Chaque place
représente un état du système, la table~\ref{tbexo1} liste les correspondances.

On cherche à étudier les propriétés de ce système et s'assurer de sa fiabilité, pour ce faire,
intéressons nous aux propriétés du réseau décrit précédemment, commençons par définir la matrice
d'incidence de ce dernier :
\begin{displaymath}
    C = \left [ 
    \begin{array}{cccccc}
        -1 & 0 & 0 & 0 & 0 & 1 \\
        1 & -1 & 0 & 0 & 0 & 0 \\
        1 & 0 & -1 & 0 & 0 & 0 \\
        0 & 1 & 0 & -1 & 0 & 0 \\
        0 & 0 & 1 & 0 & -1 & 0 \\
        0 & 0 & 0 & 1 & 0 & -1 \\
        0 & 0 & 0 & 0 & 1 & -1 \\
    \end{array} \right ]
\end{displaymath}

\begin{thrm}
    Le réseau de la figure~\ref{pnetexo1} est cohérent.
\end{thrm}

\begin{proof}
    Chaque n\oe ud du \gma ne comporte que des 0 et des 1 indiquant que, quelque soit la place du
    réseau, celle ci ne peut recevoir qu'une seule marque.
\end{proof}

\begin{thrm}
    Pour le système étudié, il y a exclusion mutuelle pour tous couples de places étiquetées par des
    combinaisons booléennes incompatibles de signaux de sortie.
\end{thrm}

\begin{proof}
    Une première preuve consiste à regarder tous les couples en question et à vérifier dans le \gma
    que, quelque soit le marquage, elles sont en exclusion mutuelle. Mais nous utiliserons, pour
    réaliser la preuve, les invariants du réseau et donc le calculs des $P-$flots associés au
    réseau.

    Les $P-$flots sont obtenus en résolvant le système d'équation décrit par la formule suivante : 
    \begin{displaymath}
        ~^tf \times C = 0
    \end{displaymath}
    où $f$ est un vecteur colonne de dimension 7. On obtient alors le systeme suivant :
    \begin{displaymath}
        \left \{ \begin{array}{ccccccccccccccl}
            -f_1 & + &  f_2 & + &  f_3 &   &      &   &      &   &      &   &      & = & 0\\
                 &   & -f_2 &   &      & + &  f_4 &   &      &   &      &   &      & = & 0\\
                 &   &      &   & -f_3 &   &      & + &  f_5 &   &      &   &      & = & 0\\
                 &   &      &   &      &   & -f_4 &   &      & + &  f_6 &   &      & = & 0\\
                 &   &      &   &      &   &      &   & -f_5 &   &      & + &  f_7 & = & 0\\
             f_1 &   &      &   &      &   &      &   &      & - &  f_6 & - &  f_7 & = & 0\\
        \end{array}
        \right .
    \end{displaymath}
    On obtient alors les égalités suivantes :
    \begin{displaymath}
        \left \{
        \begin{array}{rcl}
            f_2 & = & f_6 \\
            f_3 & = & f_7 \\
            f_4 & = & f_6 \\
            f_5 & = & f_7 \\
            f_1 & = & f_6 + f_7\\
        \end{array}
        \right .
    \end{displaymath}
    Ainsi tout vecteur de la forme : $~^t(f_6 + f_7, f_6, f_7, f_6, f_7, f_6, f_7)$ est solution du
    système donné précédemment, on en déduit alors que toute combinaison linéaire des vecteurs : $a
    = ~^t(1, 1, 0, 1, 0, 1, 0)$ et $b = ~^t(1, 0, 1, 0, 1, 0, 1)$ est solution du système. Et donc
    les invariants nous sont donnés par : \begin{displaymath}
        \left \{ \begin{array}{rcl}
            p_1 + p_2 + p_4 + p_6 & = & 1 \\
            p_1 + p_3 + p_5 + p_7 & = & 1 \\
        \end{array}
        \right .
    \end{displaymath}
    Ce système d'invariants nous assure que les couples $(p_1, p_i)\ \forall i \in \{2, \dots, 7\},$
    $(p_2, p_4)$, $(p_2, p_6)$, $(p_3, p_5)$, $(p_3, p_7)$, $(p_4, p_6)$, $(p_5, p_7)$ sont en exclusions
    mutuelle. Or il s'agit là des couples de places étiquetées par des combinaisons booléennes
    incompatibles de signaux de sortie.
\end{proof}

\begin{rmq}
    Les $P-$flots et les $P-$semiflots de ce réseau sont identiques.
\end{rmq}

\begin{proof}
    Ceci est vrai puisque les $P-$flots calculés sont à valeur entières et positives.
\end{proof}

\begin{thrm}
    Le réseau n'admet pas de blocage.
\end{thrm}

\begin{proof}
    Le \gma ne présente aucun n\oe ud dont le degré sortant est nul.
\end{proof}

\begin{thrm}
    Le réseau est un graphe d'événements.
\end{thrm}

\begin{proof}
    Toute place possède une seule transition d'entrée et une seule transition de sortie. Donc par
    définition, il s'agit d'un graphe d'événement dont les circuits élémentaires sont décrits à la
    figure~\ref{ceexo1}.

    \begin{figure}
        \begin{center}
            \begin{tikzpicture}[every transition/.style={minimum width=2mm, minimum height=7mm},
            >=latex]
                \node[place, tokens=1, label=$p_1$]         (p1) {};
                \node[transition, right=of p1, label=$t_1$] (t1) {}
                    edge[pre] (p1);
                \node[place, right=of t1, label=$p_2$]      (p2) {}
                    edge[pre] (t1);
                \node[transition, below=of p2, label=$t_2$] (t2) {}
                    edge[pre] (p2);
                \node[place, below=of t2, label=$p_4$]      (p4) {}
                    edge[pre] (t2);
                \node[transition, left=of p4, label=$t_4$]  (t4) {}
                    edge[pre] (p4);
                \node[place, left=of t4, label=$p_6$]       (p6) {}
                    edge[pre] (t4);
                \node[transition, above=of p6, label=$t_6$]  (t6) {}
                    edge[pre] (p6)
                    edge[post] (p1);

                \node[place, tokens=1, label=$p_1$] at (5, 0) (p1a) {};
                \node[transition, right=of p1a, label=$t_1$](t1a) {}
                    edge[pre] (p1a);
                \node[place, right=of t1a, label=$p_3$]     (p3) {}
                    edge[pre] (t1a);
                \node[transition, below=of p3, label=$t_3$] (t3) {}
                    edge[pre] (p3);
                \node[place, below=of t3, label=$p_5$]      (p5) {}
                    edge[pre] (t3);
                \node[transition, left=of p5, label=$t_5$]  (t5) {}
                    edge[pre] (p5);
                \node[place, left=of t5, label=$p_7$]       (p7) {}
                    edge[pre] (t5);
                \node[transition, above=of p7, label=$t_7$]  (t7) {}
                    edge[pre] (p7)
                    edge[post] (p1a);

            \end{tikzpicture}
        \end{center}
        \caption{Circuits élémentaires du réseau de Petri}
        \label{ceexo1}
    \end{figure}
\end{proof}
            
On peut aussi remarquer que le réseau est vivant et réinitialisable.
