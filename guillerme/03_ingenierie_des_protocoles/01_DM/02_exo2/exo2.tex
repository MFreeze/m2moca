\begin{figure}
    \begin{center}
        \begin{tikzpicture}[every transition/.style={minimum width=2mm, minimum height=7mm},
        >=latex, node distance=18mm]
            \node[place,label=Repos $(p_0)$,tokens=1] (urepos) at (0,0) {};

            \node[transition,below right=of urepos,label=Décroche $(t_0)$] (udec) {}
                edge[pre] (urepos);

            \node[place,label=Attente $(p_1)$,below left=of udec] (uattente) {}
                edge[pre] (udec);

            \node[transition, below left=of urepos, label=Raccroche $(t_1)$] (urac) {}
                edge[pre] (uattente)
                edge[post] (urepos);

            \node[transition, below=of uattente, label=Sélection $(t_2)$] (uselec) {}
                edge[pre] (uattente);

            \node[place, below=of uselec, label=Attente 2 $(p_2)$] (uattente2) {}
                edge[pre] (uselec);

            \node[transition, below right=of uattente2, label=Acceptation $(t_4)$] (ucaccept) {}
                edge[pre] (uattente2);

            \node[place,label=Repos $(p_3)$,tokens=1,above right=of ucaccept] (crepos) {}
                edge[post] (ucaccept);

            \node[transition, below=of ucaccept,label=Raccroche $(t_5)$] (ucrac) {}
                edge[post, out=45, in=0] (urepos);

            \node[place,left=of ucrac,label=Comm $(p_4)$] (ucomm) {}
                edge[pre] (ucaccept)
                edge[post] (ucrac);

            \node[place,right=of ucrac,label=Comm $(p_5)$] (ccomm) {}
                edge[pre] (ucaccept)
                edge[post] (ucrac);

            \node[transition, below=of ucrac,label=Libère $(t_6)$] (uclib) {}
                edge[pre] (ucomm)
                edge[pre] (ccomm);

            \node[transition, below=of uclib,label=Raccroche $(t_7)$] (ucracr2) {};

            \node[place, left=of ucracr2, label=Faux $(p_6)$] (ufaux) {}
                edge[post] (ucracr2)
                edge[pre] (uclib);

            \node[place, right=of ucracr2, label=Faux $(p_7)$] (cfaux) {}
                edge[post] (ucracr2)
                edge[pre] (uclib);

            \node[place, right=of uclib, label=Attente 3 $(p_8)$] (catt) {}
                edge[pre] (ucrac)
                edge[pre] (ucracr2);

            \node[transition, right=of catt, label=Confirm $(t_8)$] (cconf) {}
                edge[pre] (catt)
                edge[post, bend right] (crepos);

            \node[transition, label=Raccroche $(t_3)$] at (-4, -4) (uraccr3) {}
                edge[pre, bend right] (uattente2)
                edge[post, bend left] (urepos);

            \draw[post, ->] (ucracr2) to[out=225, in=270] (-6, -8) to[out=90, in=180] (urepos);

        \end{tikzpicture}
    \end{center}
    \caption{Repérésentation d'un protocole de communication à l'aide d'un réseau de Pétri}
    \label{fig1_ex2}
\end{figure}
        
Considérons le réseau de la figure~\ref{fig1_ex2} représentant le protocole de communication décrit
dans l'énoncé.
\begin{thrm}
     Le réseau considéré vérifie les spécifications du protocole de communication étudié.
\end{thrm}

\begin{proof}
    Lorsque l'utilisateur décroche, la \emph{disparition} du jeton dans la place $p_0$ permet de
    mémoriser l'événement. Il sélectionne ensuite une ligne ($t_2$) et attend que le contrôleur
    accepte la sélection de la ligne $(t_4)$. Il est alors en communication $(p_4)$. Si la ligne est
    libérée sans que l'utilisateur n'ait raccroché, lui et le contrôleur passent en état de faux
    appel ($p_5$ et $p_7$). Dans l'éventualité où l'utilisateur raccroche après la mise en
    communication, le contrôleur passe alors dans un état d'attente de confirmation de raccrochage
    $p_8$, avant de revenir à son état de repos.
    Une fois que l'utilisateur a raccroché, il revient en état de repos et peut donc à nouveau
    décrocher.

    Les spécifications sont donc vérifiées.
\end{proof}

Définissons la matrice d'incidence du réseau de Pétri : \[
    C = \left [
    \begin{array}{ccccccccc}
        -1 &  1 &  0 &  1 &  0 &  1 &  0 &  1 &  0 \\
         1 & -1 & -1 &  0 &  0 &  0 &  0 &  0 &  0 \\
         0 &  0 &  1 & -1 & -1 &  0 &  0 &  0 &  0 \\
         0 &  0 &  0 &  0 & -1 &  0 &  0 &  0 &  1 \\
         0 &  0 &  0 &  0 &  1 & -1 & -1 &  0 &  0 \\
         0 &  0 &  0 &  0 &  1 & -1 & -1 &  0 &  0 \\
         0 &  0 &  0 &  0 &  0 &  0 &  1 & -1 &  0 \\
         0 &  0 &  0 &  0 &  0 &  0 &  1 & -1 &  0 \\
         0 &  0 &  0 &  0 &  0 &  1 &  0 &  1 & -1 \\
    \end{array}
    \right ]
\]

\begin{thrm}
    Les invariants de ce réseaux sont donnés par : \[
        \left \lbrace \begin{array}{rcl}
            p_0 + p_1 + p_2 + p_4 + p_6 & = & 1\\
            p_0 + p_1 + p_2 + p_5 + p_6 & = & 1\\
            p_0 + p_1 + p_2 + p_4 + p_7 & = & 1\\
            p_0 + p_1 + p_2 + p_5 + p_7 & = & 1\\
            p_3 + p_4 + p_6 + p_8       & = & 1\\
            p_3 + p_5 + p_6 + p_8       & = & 1\\
            p_3 + p_4 + p_7 + p_8       & = & 1\\
            p_3 + p_5 + p_7 + p_8       & = & 1\\
        \end{array} \right .
    \]
\end{thrm}

\begin{proof}
    Pour calculer les $P-$flots, il suffit de résoudre le système d'équations donné par : \[
        ^tf.C = 0
    \]
    avec $^tf = (f_1, f_2, f_3, f_4, f_5, f_6, f_7, f_8, f_9)$.

    On obtient alors : \[
        \left \lbrace \begin{array}{rcl}
            f_2 - f_1 & = & 0 \\
            f_1 - f_2 & = & 0 \\
            f_3 - f_2 & = & 0 \\
            f_3 - f_1 & = & 0 \\
            f_5 + f_6 - f_3 - f_4 & = & 0 \\
            f_1 + f_9 - f_5 - f_6 & = & 0 \\
            f_7 + f_8 - f_5 - f_6 & = & 0 \\
            f_1 + f_9 - f_7 - f_8 & = & 0 \\
            f_4 - f_9 & = & 0 \\
        \end{array}
        \right .
    \]
     On obtient après simplification et suppression des équations redondantes : \[
        \left \lbrace \begin{array}{rcl}
            f_2 & = & f_1 \\
            f_3 & = & f_1 \\
            f_9 & = & f_4 \\
            f_5 + f_6 & = & f_1 + f_4 \\
            f_7 + f_8 & = & f_1 + f_4 \\
        \end{array}
        \right .
    \]
    Ainsi, tout vecteur de la forme $^t(f_1, f_1, f_1, f_4, f_5, f_6, f_7, f_8, f_4)$ vérifiant $f_5
    + f_6 = f_7 + f_8 = f_1 + f_4$ est solution du sytème et donc toute combinaison linéaire des
    vecteurs suivants est solution du système : \[
        \left ( \begin{array}{c}
            1 \\ 1 \\ 1 \\ 0 \\ 1 \\ 0 \\ 1 \\ 0 \\ 0
        \end{array} \right ) \ 
        \left ( \begin{array}{c}
            1 \\ 1 \\ 1 \\ 0 \\ 0 \\ 1 \\ 1 \\ 0 \\ 0
        \end{array} \right ) \ 
        \left ( \begin{array}{c}
            1 \\ 1 \\ 1 \\ 0 \\ 1 \\ 0 \\ 0 \\ 1 \\ 0
        \end{array} \right ) \ 
        \left ( \begin{array}{c}
            1 \\ 1 \\ 1 \\ 0 \\ 0 \\ 1 \\ 0 \\ 1 \\ 0
        \end{array} \right ) \ 
        \left ( \begin{array}{c}
            0 \\ 0 \\ 0 \\ 1 \\ 1 \\ 0 \\ 1 \\ 0 \\ 1
        \end{array} \right ) \ 
        \left ( \begin{array}{c}
            0 \\ 0 \\ 0 \\ 1 \\ 1 \\ 0 \\ 0 \\ 1 \\ 1
        \end{array} \right ) \ 
        \left ( \begin{array}{c}
            0 \\ 0 \\ 0 \\ 1 \\ 0 \\ 1 \\ 1 \\ 0 \\ 1
        \end{array} \right ) \ 
        \left ( \begin{array}{c}
            0 \\ 0 \\ 0 \\ 1 \\ 0 \\ 1 \\ 0 \\ 1 \\ 1
        \end{array} \right ) \ 
    \]
    Le marquage initial étant égal à $p_0p_3$, on en déduit les invariants donnés précédemment.
\end{proof}

\begin{thrm}
    Les $P-$semiflots sont identiques aux $P-$flots.
\end{thrm}

\begin{proof}
    Les invariants précédemment calculés présentant des coefficients entiers positifs les
    $P-$semiflots sont donc identiques aux $P-$flots.
\end{proof}
