\begin{figure}
    \begin{center}
        \begin{tikzpicture}[every transition/.style={minimum width=4mm, minimum height=10mm},
        >=latex, every place/.style={minimum width=1cm}, scale=0.7, every node/.style={transform shape}]
            \node[place, label=Inactif] (inactif) at (9, 20) {$u$};
            \node[transition, label=Decroche] (decr) at (9, 18) {};
            \node[transition, label=Raccroche] (r1) at (12, 18) {};
            \node[place, label=Continu] (cont) at (9, 16) {$u$};
            %\node[transition, label=Raccroche] (r2) at (6, 16) {};
            \node[transition, label=Compose] (comp) at (9, 14) {};
            \node[transition, label=Raccroche] (r3) at (3, 12) {};
            \node[transition, label=Raccroche] (r4) at (6, 12) {};
            \node[place, label=Sans Tonalite] (notone) at (9, 12) {$u$};
            \node[place, label=Requete] (req) at (12, 12) {$u \times u$};
            \node[transition, label=Raccroche] (r5) at (15, 12) {};
            \node[transition, label=Est inactif] (ein) at (6, 10) {};
            \node[transition, label=Est actif] (ea) at (12, 10) {};
            \node[place, label=Court] (court) at (15, 10) {$u$};
            \node[place, label=Appel] (appel) at (3, 8) {$u\times u$};
            \node[place, label=Long] (long) at (6, 8) {$u$};
            \node[place, label=Sonne] (sonne) at (9, 8) {$u$};
            \node[transition, label=Réponds] (rep) at (6, 6) {};
            \node[place, label=Connexion] (con1) at (3, 4) {$u$};
            \node[place, label=Connecté] (con2) at (9, 4) {$u\times u$};
            \node[transition, label=$x$ Raccroche] (rx) at (3, 2) {};
            \node[transition, label=$y$ Raccroche] (ry) at (9, 2) {};
            \node[transition, label=Raccroche] (r6) at (12, 2) {};
            \node[place, label=$y$ Attente] (ay) at (3, 0) {$u$};
            \node[place, label=$x$ Attente] (ax) at (9, 0) {$u$};
            \node[transition, label=Raccroche] (r7) at (12, 0) {};

            \draw (inactif) edge[post] node[auto] {$x$} (decr)
                edge[post] node[auto] {$y$} (ein)
                edge[pre, bend left] node[auto] {$x$} (r1)
                %edge[pre, in=70, out=230] node[auto] {$x$} (r2)
                edge[pre, in=90, out=180] node[auto] {$(x,y)$} (r3)
                edge[pre, bend right] node[auto] {$x$} (r4)
                edge[pre, in=90, out=0] node[auto] {$x$} (r5)
                edge[pre, in=20, out=0] node[auto] {$x$} (r6)
                edge[pre, in=0, out=0] node[auto] {$x$} (r7);
            \draw (cont) edge[pre] node[auto] {$x$} (decr)
                edge[post] node[auto] {$x$} (r1)
                edge[post] node[auto] {$x$} (comp);
            \draw (notone) edge[pre] node[auto] {$x$} (comp)
                edge[post] node[auto] {$x$} (r4)
                edge[post] node[auto] {$x$} (ein)
                edge[post] node[auto] {$x$} (ea);
            \draw (req) edge[pre] node[above] {$(x,y)$} (comp)
                edge[post, bend right] node[above left] {$(x,y)$} (r4)
                edge[post] node[auto] {$(x,y)$} (r5);
            \draw (court) edge[pre] node[above] {$x$} (ea)
                edge[post] node[auto] {$x$} (r5);
            \draw (appel) edge[pre] node[auto] {$(x,y)$} (ein)
                edge[post] node[auto] {$(x,y)$} (r3)
                edge[post] node[auto] {$(x,y)$} (rep);
            \draw (long) edge[pre] node[auto] {$x$} (ein)
                edge[post] node[auto] {$x$} (r3)
                edge[post] node[auto] {$x$} (rep);
            \draw (sonne) edge[pre] node[auto] {$y$} (ein)
                edge[post, in=350, out=130] node[above right] {$y$} (r3)
                edge[post] node[auto] {$y$} (rep);
            \draw (con1) edge[pre] node[above] {$x$} (rep)
                edge[post] node[auto] {$x$} (rx)
                edge[post] node[auto] {$y$} (ry);
            \draw (con2) edge[pre] node[auto] {$(x,y)$} (rep)
                edge[post] node[auto] {$(x,y)$} (rx)
                edge[post] node[auto] {$(x,y)$} (ry);
            \draw (ax) edge[pre] node[auto] {$x$} (ry)
                edge[post] node[auto] {$x$} (r7);
            \draw (ay) edge[pre] node[auto] {$x$} (rx)
                edge[post] node[auto] {$x$} (r6);
        \end{tikzpicture}
    \end{center}
    \caption{Représentation du système téléphonique à l'aide d'un réseau de Petri}
    \label{exo9fig1}
\end{figure}

\begin{figure}
    \begin{center}
        \begin{tikzpicture}[every transition/.style={minimum width=4mm, minimum height=10mm},
        >=latex, every place/.style={minimum width=1cm}, scale=0.7, every node/.style={transform shape}]
            \node[place, label=Inactif] (inactif) at (9, 20) {$u$};
            \node[transition, label=Decroche] (decr) at (9, 18) {};
            \node[transition, label=Raccroche] (r1) at (12, 18) {};
            \node[place, label=Continu] (cont) at (9, 16) {$u$};
            %\node[transition, label=Raccroche] (r2) at (6, 16) {};
            \node[transition, label=Compose] (comp) at (9, 14) {};
            \node[transition, label=Raccroche] (r3) at (3, 12) {};
            \node[transition, label=Raccroche] (r4) at (6, 12) {};
            \node[place, label=Sans Tonalite] (notone) at (9, 12) {$u$};
            \node[place, label=Requete] (req) at (12, 12) {$u \times u$};
            \node[transition, label=Raccroche] (r5) at (15, 12) {};
            \node[transition, label=Est inactif] (ein) at (6, 10) {};
            \node[transition, label=Est actif] (ea) at (12, 10) {};
            \node[place, label=Court] (court) at (15, 10) {$u$};
            \node[place, label=Appel] (appel) at (3, 8) {$u\times u$};
            \node[place, label=Long] (long) at (6, 8) {$u$};
            \node[place, label=Sonne] (sonne) at (9, 8) {$u$};
            \node[transition, label=Réponds] (rep) at (6, 6) {};
            \node[place, label=Connexion] (con1) at (3, 4) {$u$};
            \node[place, label=Connecté] (con2) at (9, 4) {$u\times u$};
            \node[transition, label=$x$ Raccroche] (rx) at (3, 2) {};
            \node[transition, label=$y$ Raccroche] (ry) at (9, 2) {};
            \node[transition, label=Raccroche] (r6) at (12, 2) {};
            \node[place, label=$y$ Attente] (ay) at (3, 0) {$u$};
            \node[place, label=$x$ Attente] (ax) at (9, 0) {$u$};
            \node[transition, label=Raccroche] (r7) at (12, 0) {};

            \node[transition, label=Activation, red] (activ) at (6, 22) {};
            \node[transition, label=Désactivation, red] (deactiv) at (12, 22) {};
            \node[place, label=Désactivé, red] (des) at (9, 24) {$u$};

            \draw (inactif) edge[post] node[auto] {$x$} (decr)
                edge[post] node[auto] {$y$} (ein)
                edge[pre, bend left] node[auto] {$x$} (r1)
                %edge[pre, in=70, out=230] node[auto] {$x$} (r2)
                edge[pre, in=90, out=180] node[auto] {$(x,y)$} (r3)
                edge[pre, bend right] node[auto] {$x$} (r4)
                edge[pre, in=90, out=0] node[auto] {$x$} (r5)
                edge[pre, in=20, out=0] node[auto] {$x$} (r6)
                edge[pre, in=0, out=0] node[auto] {$x$} (r7)
                edge[post, red] node[auto, red] {$x$} (deactiv)
                edge[pre, red] node[auto, red] {$x$} (activ);
            \draw (cont) edge[pre] node[auto] {$x$} (decr)
                edge[post] node[auto] {$x$} (r1)
                edge[post] node[auto] {$x$} (comp);
            \draw (notone) edge[pre] node[auto] {$x$} (comp)
                edge[post] node[auto] {$x$} (r4)
                edge[post] node[auto] {$x$} (ein)
                edge[post] node[auto] {$x$} (ea);
            \draw (req) edge[pre] node[above] {$(x,y)$} (comp)
                edge[post, bend right] node[above left] {$(x,y)$} (r4)
                edge[post] node[auto] {$(x,y)$} (r5);
            \draw (court) edge[pre] node[above] {$x$} (ea)
                edge[post] node[auto] {$x$} (r5);
            \draw (appel) edge[pre] node[auto] {$(x,y)$} (ein)
                edge[post] node[auto] {$(x,y)$} (r3)
                edge[post] node[auto] {$(x,y)$} (rep);
            \draw (long) edge[pre] node[auto] {$x$} (ein)
                edge[post] node[auto] {$x$} (r3)
                edge[post] node[auto] {$x$} (rep);
            \draw (sonne) edge[pre] node[auto] {$y$} (ein)
                edge[post, in=350, out=130] node[above right] {$y$} (r3)
                edge[post] node[auto] {$y$} (rep);
            \draw (con1) edge[pre] node[above] {$x$} (rep)
                edge[post] node[auto] {$x$} (rx)
                edge[post] node[auto] {$y$} (ry);
            \draw (con2) edge[pre] node[auto] {$(x,y)$} (rep)
                edge[post] node[auto] {$(x,y)$} (rx)
                edge[post] node[auto] {$(x,y)$} (ry);
            \draw (ax) edge[pre] node[auto] {$x$} (ry)
                edge[post] node[auto] {$x$} (r7);
            \draw (ay) edge[pre] node[auto] {$x$} (rx)
                edge[post] node[auto] {$x$} (r6);
            \draw (des) edge[pre, red] node[auto, red] {$x$} (deactiv)
                edge[post, red] node[auto, red] {$x$} (activ);
        \end{tikzpicture}
    \end{center}
    \caption{Représentation du système téléphonique à l'aide d'un réseau de Petri}
    \label{exo9fig2}
\end{figure}

Le réseau téléphonique peut être représenté par le réseau de Petri coloré de la
figure~\ref{exo9fig1}. $x$ et $y$ représentant tous deux l'ensemble des numéros disponibles. La
représentation fournie permet aux deux correspondants de raccrocher et donc de rompre la
communication. 

Si l'on veut être en mesure de gérer les numéros inactifs, il faut alors ajouter une place
supplémentaire contenant l'ensemble des numéros inactifs et deux transitions permettant l'activation
ou la désactivation du numéro représenté par le jeton de couleur. Cette modification est donnée à la
figure~\ref{exo9fig2}.
