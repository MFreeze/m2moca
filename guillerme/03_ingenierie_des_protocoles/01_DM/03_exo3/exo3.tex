\begin{figure}
    \begin{center}
        \begin{tikzpicture}[every transition/.style={minimum width=2mm, minimum height=7mm},
        >=latex, node distance=18mm, scale=0.9, every place/.style={transform shape}]
            \node[place, label=$d_0$,tokens=1] (d0) at (0,0) {};
            \node[transition, right=of d0, label=$t_0^0$] (t00) {}
                edge[pre] (d0);
            \node[place, right=of t00, label=$p_0$] (p0) {}
                edge[pre] (t00);
            \node[transition, right=of p0, label=$t_0^1$] (t01) {}
                edge[pre] (p0);
            \node[place, right=of t01, label=$p_0'$] (pp0) {}
                edge[pre] (t01);
            \node[transition, right=of pp0, label=$t_0^2$] (t02) {}
                edge[pre] (pp0);
            \node[place, right=of t02, label=$f_0$] (f0) {}
                edge[pre] (t02);
            \node[transition, above=of p0, label=$t_0^3$] (t03) {}
                edge[pre, bend right] (p0)
                edge[post, bend left] (p0);
            \node[place, right=of t03, label=$c_0'$] (cp0) {}
                edge[pre] node[auto] {2} (t03);
            \node[transition, right=of cp0, label=$t_0^4$] (t04) {}
                edge[pre] (cp0)
                edge[pre, bend right] (pp0)
                edge[post, bend left] (pp0);
            \node[place, above=of cp0, label=$c_0$, structured tokens=$n$] (c0) {}
                edge[pre, bend left] (t04)
                edge[post, bend right] (t03);
        \end{tikzpicture}
    \end{center}
    \caption{Un réseau de Petri particulier $N_0^n$}
    \label{exo3fig1}
\end{figure}
        
\begin{figure}
    \begin{center}
        \begin{tikzpicture}[every transition/.style={minimum width=2mm, minimum height=7mm},
        >=latex, scale=0.65, every node/.style={transform shape}]
            \node[place, label=$d_0$] (d0) at (0,0) {};
            \node[transition, right=of d0, label=$t_0^0$] (t00) {}
                edge[pre] (d0);
            \node[place, right=of t00, label=$p_0$] (p0) {}
                edge[pre] (t00);
            \node[transition, right=of p0, label=$t_0^1$] (t01) {}
                edge[pre] (p0);
            \node[place, right=of t01, label=$p_0'$] (pp0) {}
                edge[pre] (t01);
            \node[transition, right=of pp0, label=$t_0^2$] (t02) {}
                edge[pre] (pp0);
            \node[place, right=of t02, label=$f_0$] (f0) {}
                edge[pre] (t02);
            \node[transition, above=of p0, label=$t_0^3$] (t03) {}
                edge[pre, bend right] (p0)
                edge[post, bend left] (p0);
            \node[place, right=of t03, label=$c_0'$] (cp0) {}
                edge[pre] node[auto] {2} (t03);
            \node[transition, right=of cp0, label=$t_0^4$] (t04) {}
                edge[pre] (cp0)
                edge[pre, bend right] (pp0)
                edge[post, bend left] (pp0);
            \node[place, above=of cp0, label=$c_0$] (c0) {}
                edge[pre, bend left] (t04)
                edge[post, bend right] (t03);

            \node[transition, left=of d0, label=$a_1$] (a1) {}
                edge[post] (d0);
            \node[place, left=of a1, label=$d_1$, tokens=1] (d1) {}
                edge[post] (a1);
            \node[transition, above=of d1, label=$r_1$] (r1) {}
                edge[post, bend left] (d1)
                edge[pre, bend right] (d1)
                edge[post, out=0, in=180] (c0);
            \node[place, above=of r1, label=$c_1$, structured tokens=$n$] (c1) {}
                edge[post] (r1);
            \node[place, below=of d1, label=$c_1'$] (cp1) {}
                edge[pre, in=180, out=180] (r1);
            \node[transition, below=of t01, label=$e_1$] (e1) {}
                edge[pre] (cp1)
                edge[post] (d0)
                edge[pre] (f0);
            \node[transition, right=of f0, label=$t_1^0$] (t10) {}
                edge[pre] (f0);
            \node[place, right=of t10, label=$p_1$] (p1) {}
                edge[pre] (t10);
            \node[transition, right=of p1, label=$t_1^1$] (t11) {}
                edge[pre, bend right] (p1)
                edge[post, bend left] (p1)
                edge[pre, in=0, out=150] (c0)
                edge[post, out=90, in=60] (c1);
            \node[place, right=of t11, label=$f_1$] (f1) {}
                edge[pre] (t11);
        \end{tikzpicture}
    \end{center}
    \caption{Un réseau de Petri particulier $N_1^n$}
    \label{exo3fig2}
\end{figure}

Considérons le réseau de Petri de la figure~\ref{exo3fig1} et le réseau de la figure~\ref{exo3fig2}
construit à partir de celui de la figure~\ref{exo3fig1}.

\begin{thrm}
    Le réseau de Petri de la figure~\ref{exo3fig1} présente un invariant donné par :
    \[
        d_0 + p_0 + p_0' + f_0 = 1 
    \]
\end{thrm}

\begin{proof}
    Pour trouver les invariants d'un réseau, il suffit de trouver les solutions au système
    d'équations obtenu à partir de : \[
        ^tfC = 0
    \]
    Avec $C$ la matrice d'incidence du réseau, que l'on décrira ainsi :
    \[
        \left [ \begin{array}{ccccc}
            -1 & 0 & 0 & 0 & 0 \\
            1 & -1 & 0 & 0 & 0 \\
            0 & 1 & -1 & 0 & 0 \\
            0 & 0 & 1 & 0 & 0 \\
            0 & 0 & 0 & -1 & 1 \\
            0 & 0 & 0 & 2 & -1 \\
        \end{array}
        \right ]
    \]

    Le système est alors donné par : \[
        \left \lbrace \begin{array}{rcl}
            -f_1 + f_2 & = & 0 \\
            -f_2 + f_3 & = & 0 \\
            -f_3 + f_4 & = & 0 \\
            -f_5 + 2f_6 & = & 0 \\
            f_5 - f_6 & = & 0 \\
        \end{array}
        \right .
    \]
    \[
        \Rightarrow \left \lbrace \begin{array}{rcl}
            f_2 & = & f_1 \\
            f_3 & = & f_1 \\
            f_4 & = & f_1 \\
            f_5 & = & 0 \\
            f_6 & = & 0 \\
        \end{array}
        \right .
    \]
    Les vecteurs de la forme $^t(f_1, f_1, f_1, f_1, 0, 0)$ sont solutions du système précédent, on
    en déduit donc l'invariant énoncé précédemment. De la même manière que l'exercice 2, le $P-$flot
    étant à valeur entière et positive, il représente aussi un $P-$semiflot.
\end{proof}

\begin{rmq}
    Le $P-$semi flot trouvé à l'aide de l'outil Tina est identique à celui trouvé précédemment.
\end{rmq}

\verbatiminput{03_exo3/N0n-struct.txt}

\begin{thrm}
    Le $P-$semiflot du réseau de la figure~\ref{exo3fig2} est le suivant : \[
        d_0 + p_0 + p_0' + f_0 + p_1 + d_1 = 1
    \]
\end{thrm}

\begin{proof}
    Nous allons calculer les $P-$flots de la même que précédemment. Commençons par la matrice $C'$,
    définie comme suit : \[
        \left [
        \begin{array}{cccccccccc}
            -1 & 0 & 0 & 0 & 0 & 0 & 1 & 0 & 0 & 1\\
            1 & -1 & 0 & 0 & 0 & 0 & 0 & 0 & 0 & 0\\
            0 & 1 & -1 & 0 & 0 & 0 & 0 & 0 & 0 & 0\\
            0 & 0 & 1 & 0 & 0 & 0 & 0 & -1 & 0 & -1\\
            0 & 0 & 0 & -1 & 1 & 1 & 0 & 0 & -1 & 0\\
            0 & 0 & 0 & 2 & -1 & 0 & 0 & 0 & 0 & 0\\
            0 & 0 & 0 & 0 & 0 & -1 & 0 & 0 & 1 & 0\\
            0 & 0 & 0 & 0 & 0 & 1 & -1 & 0 & 0 & -1\\
            0 & 0 & 0 & 0 & 0 & 0 & 0 & 1 & 0 & 0\\
            0 & 0 & 0 & 0 & 0 & 0 & 0 & 1 & 0 & 0\\
            0 & 0 & 0 & 0 & 0 & 0 & -1 & 0 & 0 & 0\\
        \end{array}
        \right ]
    \]
    On remarque que la matrice $C'$ est de la forme : \[
        \left [ \begin{array}{cc}
            C & A \\
            0 & A' \\
        \end{array} \right ]
    \]
    On peut donc d'ores et déjà poser : 
    \[
        \left \lbrace \begin{array}{rcl}
            f_2 & = & f_1 \\
            f_3 & = & f_1 \\
            f_4 & = & f_1 \\
            f_5 & = & 0 \\
            f_6 & = & 0 \\
        \end{array}
        \right .
    \]
    représentant le système d'équation du réseau précédent. Il reste à résoudre le système suivant :
    \[
        \left \lbrace \begin{array}{rcl}
            f_5 - f_7 + f_8 & = & 0 \\
            f_1 - f_8 - f_{11} & = & 0 \\
            f_9 + f_4 & = & 0 \\
            f_7 + f_{10} - f_5 & = & 0 \\
            f_1 - f_4 - f_8 & = & 0
        \end{array} \right .
    \]
    On obtient alors : \[
        \left \lbrace \begin{array}{ccccccccccc}
            f_1 & = & f_2 & = & f_3 & = & f_4 & = & f_9 & = & f_{11} \\
            f_5 & = & f_6 & = & f_7 & = & f_8 & = & f_{10} & = & 0 \\
        \end{array} \right .
    \]
    Le $P-$ semiflot est dont donné par le vecteur : \[
    \left ( \begin{array}{c}
        1 \\ 1 \\ 1 \\ 1 \\ 0 \\ 0 \\ 0 \\ 0 \\ 1 \\ 0 \\ 1
    \end{array} \right )
    \]
    Le marquage initial étant donné par $d_1c_1^n$, l'invariant est donc :\[
        d_0 + p_0 + p_0' + f_0 + p_1 + d_1 = 1
    \]
\end{proof}

\begin{rmq}
    Le $P-$semiflot trouvé à l'aide de l'outil Tina est identique à celui trouvé précédemment.
\end{rmq}

\verbatiminput{03_exo3/N1n-struct.txt}

La séquence maximale de $N_0^n$ telle que $d_0$ est unitaire est la suivante : \[
    t_0^0 t_0^3 t_0^1 t_0^4 t_0^2
\].

La séquence maximale de $N_1^n$ telle que $d_1$ est unitaire est la suivante : \[
    r_1 a_1 t_0^0 t_0^3 t_0^1 t_0^4 t_0^2 t_1^0 t_1^1
\].

Nous ne donnerons l'arborescence de couverture que pour $N_0^1$ pour des raisons de taille de
l'arborescence. Cette dernière est donnée à la figure~\ref{exo3fig3}.

\begin{figure}
    \begin{center}
        \begin{tikzpicture}
            \node (n1) at (0,0) {$(d_0c_0)$};

            \node[below=of n1] (n2) {$(p_0c_0)$}
                edge[pre] node[auto] {$t_0^0$} (n1);

            \node[below left=of n2] (n3) {$(p_0'c_0)$}
                edge[pre] node[auto] {$t_0^1$} (n2);

            \node[below right=of n2] (n4) {$(p_0'c_0^{'2})$}
                edge[pre] node[auto] {$t_0^3$} (n2);

            \node[below=of n3] (n5) {$(f_0c_0)$}
                edge[pre] node[auto] {$t_0^2$} (n3);

            \node[below=of n4] (n6) {$(p'_0c_0^{'2})$}
                edge[pre] node[auto] {$t_0^1$} (n4);

            \node[below left=of n6] (n7) {$(f_0c_0^{'2})$}
                edge[pre] node[auto] {$t_0^2$} (n6);

            \node[below right=of n6] (n8) {$(p'_0c_0c'_0)$}
                edge[pre] node[auto] {$t_0^4$} (n6);

            \node[below left=of n8] (n9) {$(f_0c_0c'_0)$}
                edge[pre] node[auto] {$t_0^2$} (n8);

            \node[below right=of n8] (n10) {$(p'_0c_0^2)$}
                edge[pre] node[auto] {$t_0^4$} (n8);

            \node[below=of n10] (n11) {$(f_0c_0^2)$}
                edge[pre] node[auto] {$t_0^2$} (n10);
        \end{tikzpicture}
    \end{center}
    \caption{Arborescence de couverture du réseau de Petri $N_0^1$}
    \label{exo3fig3}
\end{figure}
