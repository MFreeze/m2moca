\begin{figure}
    \begin{center}
        \begin{tikzpicture}[every transition/.style={minimum width=7mm, minimum height=2mm}, >=latex]
            \node[place, tokens=1, label=Producteur] at (0, 0) (P) {};
            \node[place] at (0, -4) (int1) {};

            \node[place, label=Buffer] at (2, -4) (buff) {};

            \node[place, label=Consommateur, tokens=1] at (4, 0) (C) {};
            \node[place] at (4, -4) (int2) {};

            \node[transition, label=Produire] at (0, -2) (prod) {}
                edge[pre] (P)
                edge[post] (int1);
            \node[transition, label=Envoyer] at (0, -6) (send) {}
                edge[pre] (int1)
                edge[post] (buff);

            \node[transition, label=Recevoir] at (4, -2) (rec) {}
                edge[pre] (buff)
                edge[pre] (C)
                edge[post] (int2);
            \node[transition, label=Consommer] at (4, -6) (cons) {}
                edge[pre] (int2);

            \draw[post] (send) to (-1.5, -6) to (-1.5, 0) to (P);
            \draw[post] (cons) to (5.5, -6) to (5.5, 0) to (C);
        \end{tikzpicture}
    \end{center}
    \caption{Producteur et consommateur}
    \label{pnetpcexo5}
\end{figure}
    
\begin{figure}
    \begin{center}
        \begin{tikzpicture}[every transition/.style={minimum width=7mm, minimum height=2mm}, >=latex]
            \node[place, structured tokens=$p$, label=Producteurs] at (0, 0) (P) {};
            \node[place] at (0, -4) (int1) {};
            \node[place, label=Buffer] at (2, -4) (buff) {};

            \node[place, label=Consommateurs, structured tokens=$c$] at (4, 0) (C) {};
            \node[place] at (4, -4) (int2) {};

            \node[transition, label=Produire] at (0, -2) (prod) {}
                edge[pre] (P)
                edge[post] (int1);
            \node[transition, label=Envoyer] at (0, -6) (send) {}
                edge[pre] (int1)
                edge[post] (buff);

            \node[transition, label=Recevoir] at (4, -2) (rec) {}
                edge[pre] (buff)
                edge[pre] (C)
                edge[post] (int2);
            \node[transition, label=Consommer] at (4, -6) (cons) {}
                edge[pre] (int2);

            \draw[post] (send) to (-1.5, -6) to (-1.5, 0) to (P);
            \draw[post] (cons) to (5.5, -6) to (5.5, 0) to (C);
        \end{tikzpicture}
    \end{center}
    \caption{Producteurs et consommateurs}
    \label{pnetpcsexo5}
\end{figure}
    
\begin{figure}
    \begin{center}
        \begin{tikzpicture}[every transition/.style={minimum width=7mm, minimum height=2mm}, >=latex]
            \node[place, structured tokens=$p$, label=Producteurs] at (0, 0) (P) {};
            \node[place] at (0, -4) (int1) {};

            \node[place, label=Contrôleur, structured tokens=$z$] at (2, -3) (control) {};
            \node[place, label=Buffer] at (2, -5) (buff) {};

            \node[place, label=Consommateurs, structured tokens=$c$] at (4, 0) (C) {};
            \node[place] at (4, -4) (int2) {};

            \node[transition, label=Produire] at (0, -2) (prod) {}
                edge[pre] (P)
                edge[post] (int1);
            \node[transition, label=Envoyer] at (0, -6) (send) {}
                edge[pre] (int1)
                edge[pre] (control)
                edge[post] (buff);

            \node[transition, label=Recevoir] at (4, -2) (rec) {}
                edge[pre] (buff)
                edge[pre] (C)
                edge[post] (control)
                edge[post] (int2);
            \node[transition, label=Consommer] at (4, -6) (cons) {}
                edge[pre] (int2);

            \draw[post] (send) to (-1.5, -6) to (-1.5, 0) to (P);
            \draw[post] (cons) to (5.5, -6) to (5.5, 0) to (C);
        \end{tikzpicture}
    \end{center}
    \caption{Producteurs et consommateurs avec capacité du Buffer limitée}
    \label{pnetpcbexo5}
\end{figure}
    
\begin{figure}
    \begin{center}
        \begin{tikzpicture}[every transition/.style={minimum width=7mm, minimum height=2mm}, >=latex]
            \node[place, structured tokens=$p$, label=Producteurs] at (0, 0) (P) {};
            \node[place] at (0, -4) (int1) {};

            \node[place, label=$B_1$] at (2, -8) (buff1) {};
            \node at (2, -4) (buff) {$\dots$};
            \node[place, label=$B_b$] at (2, 0) (buffb) {};

            \node[place, label=Consommateurs, structured tokens=$c$] at (4, 0) (C) {};
            \node[place] at (4, -4) (int2) {};

            \node[transition, label=Produire] at (0, -2) (prod) {}
                edge[pre] (P)
                edge[post] (int1);
            \node[transition, label=Envoyer] at (0, -6) (send) {}
                edge[pre] (int1)
                edge[pre, bend right, >=o] (buff1)
                edge[post, bend left] (buff1);

            \node[transition] at (2, -6) (b1) {}
                edge[pre] (buff1)
                edge[pre, >=o, bend right] (buff)
                edge[post, bend left] (buff);

            \node[transition] at (2, -2) (bb) {}
                edge[pre] (buff)
                edge[pre, bend left, >=o] (buffb)
                edge[post, bend right] (buffb);

            \node[transition, label=Recevoir] at (4, -2) (rec) {}
                edge[pre] (buffb)
                edge[pre] (C)
                edge[post] (int2);
            \node[transition, label=Consommer] at (4, -6) (cons) {}
                edge[pre] (int2);

            \draw[post] (send) to (-1.5, -6) to (-1.5, 0) to (P);
            \draw[post] (cons) to (5.5, -6) to (5.5, 0) to (C);
        \end{tikzpicture}
    \end{center}
    \caption{Producteurs et consommateurs avec capacité du Buffer limitée et arcs inhibiteurs}
    \label{pnetpcbexo5}
\end{figure}

\begin{figure}
    \begin{center}
        \begin{tikzpicture}[scale=0.9, >=latex, every node/.style={transform shape}, every
        transition/.style={minimum width=7mm, minimum height=2mm, transform shape}]
            \node[place, structured tokens=$p$, label=Producteurs] at (0, 0) (P) {};
            \node[place] at (0, -4) (int1) {};

            \node[place, label=$B_1$] at (1.5, -4) (b1) {};
            \node[place, label=$B_2$] at (4.5, -4) (b2) {};
            \node[place, label=$B_b$] at (10.5,-4) (bb) {};

            \node[place, label=$C_1$, tokens=1] at (1.5, -6) (c1) {};
            \node[place, label=$C_2$, tokens=1] at (4.5, -2) (c2) {};
            \node[place, label=$C_b$, tokens=1] at (10.5,-2) (cb) {};

            \node at (7.5, -4) (d1) {$\dots$};
            \node at (7.5, -2) (d2) {$\dots$};

            \node[place, label=Consommateurs, structured tokens=$c$] at (12, 0) (C) {};
            \node[place] at (12, -4) (int2) {};

            \node[transition, label=Produire] at (0, -2) (prod) {}
                edge[pre] (P)
                edge[post] (int1);
            \node[transition, label=Envoyer] at (0, -6) (send) {}
                edge[pre] (c1)
                edge[pre] (int1)
                edge[post] (b1);

            \node[transition, label=Recevoir] at (12, -2) (rec) {}
                edge[pre] (C)
                edge[pre] (bb)
                edge[post] (cb)
                edge[post] (int2);
            \node[transition, label=Consommer] at (12, -6) (cons) {}
                edge[pre] (int2);

            \tikzset{every transition/.style={minimum width=2mm, minimum height=7mm, transform shape}};
            \node[transition] at (3, -4) (tb1) {}
                edge[pre] (b1)
                edge[pre] (c2)
                edge[post] (c1)
                edge[post] (b2);

            \node[transition] at (6, -4) (tb2) {}
                edge[pre] (b2)
                edge[pre] (d2)
                edge[post] (c2)
                edge[post] (d1);

            \node[transition] at (9, -4) (tbb) {}
                edge[pre] (d1)
                edge[pre] (bb)
                edge[post] (d2)
                edge[post] (cb);

            \draw[post] (send) to (-1.5, -6) to (-1.5, 0) to (P);
            \draw[post] (cons) to (13.5, -6) to (13.5, 0) to (C);
        \end{tikzpicture}
    \end{center}
    \caption{Producteurs et consommateurs avec capacité du Buffer limitée sans arcs inhibiteurs}
    \label{pnetfinexo5}
\end{figure}

    
Considérons le réseau de petri de la figure~\ref{pnetpcexo5} représentant un système de production -
consommation, avec un producteur et un consommateur. On cherche à étendre le système étudié à $p$
producteurs et $p$ consommateurs. Pour ce faire, il suffit de modifier le nombre de jetons présents
dans les places Producteurs et Consommateurs de façon à en avoir $p$ sur la première et $c$ sur la
seconde comme représenté sur la figure~\ref{pnetpcsexo5}.

Si l'on veut limiter le nombre de ressources présentes simultanément dans le Buffer, il faut
représenter ce dernier non plus par une seule place, mais par deux distinctes : une pour accueillir
les ressources produites que nous appellerons toujours le Buffer, la seconde contenant $b$ jetons
Considérons le réseau de petri de la figure~\ref{pnetpcexo5} représentant un système de production -
consommation, avec un producteur et un consommateur. On cherche à étendre le système étudié à $p$
producteurs et $p$ consommateurs. Pour ce faire, il suffit de modifier le nombre de jetons présents
dans les places Producteurs et Consommateurs de façon à en avoir $p$ sur la première et $c$ sur la
seconde comme représenté sur la figure~\ref{pnetpcsexo5}.

Si l'on veut limiter le nombre de ressources présentes simultanément dans le Buffer, il faut
représenter ce dernier non plus par une seule place, mais par deux distinctes : une pour accueillir
les ressources produites que nous appellerons toujours le Buffer, la seconde contenant $b$ jetons
contrôlant l'envoi des ressources, que nous appellerons le Contrôleur, comme représenté
à la figure~\ref{pnetpcbexo5}.

Enfin, on cherche à ordonner la réception des ressources de manière à ce que ces dernières soient
récupérées dans le même ordre qu'elles ont été envoyées. Il existe deux possibilités équivalentes
l'une nécessitant l'introduction d'une nouvelle extension du modèle, à savoir les arcs inhibiteurs,
l'autre demandant un plus grand nombre de places. Nous décrirons ici, les deux possibilités.\\
Les arcs inhibiteurs sont des arcs permettant de modéliser le test à zéro, ce qui permet une plus
grande expressivité. En effet, dans le modèle de base, si une transition est tirable lorsqu'une
place est vide, elle l'est à prioris, aussi lorsqu'elle contient des jetons. Pour empêcher le tirage
d'une transition lorsqu'une place est non vide, on utilise alors ces arcs : 
\begin{center}
    \begin{tikzpicture}[every transition/.style={minimum width=2mm, minimum height=7mm}, >=o]
        \node[place, label=$p$] (p) {};
        \node[transition, label=$t$, right=of p] (t) {}
            edge[pre] (p);
    \end{tikzpicture}
\end{center}
Ici la transition $t$ est tirable tant que $p$ ne contient aucun jeton.

En utilisant cette extension, on obtient alors le réseau de la figure~\ref{pnetpcbexo5}. Le Buffer
est alors représenté par $b$ places et $b-1$ transitions. L'ordre de réception est le même que
l'ordre d'envoi, en effet, il est impossible d'avoir plus d'un jeton par place à l'intérieur du
buffer et il est impossible pour un producteur d'envoyer une ressource dans le Buffer tant que la
première place est non vide, or pour libérer la première place, il faut que la seconde soit vide et
pour vider la seconde, la troisième doit être libre de tous jetons et ainsi de suite. Donc pour
envoyer un jeton dans le buffer il faut faire avancer tous les jetons consécutifs précédents d'une
place et aucun jeton ne peut en dépasser un autre, l'ordre de réception est donc le même que celui
d'envoi. De plus, il y a au maximum un jeton par place dans le Buffer, donc de manière générale, il
y a au plus $b$ ressources dans ce dernier.

Comme annoncé précédemment, il est possible de ne pas avoir recours aux arcs inhibiteurs en
introduisant $b$ places supplémentaires. On découpe alors le buffer en $b$ \emph{chambres}, chacune
de ces chambres étant constituée de $2$ places. Une de ces deux places possède un jeton lorsqu'une
ressource se trouve dans la chambre\footnote{Elle joue alors le même rôle que la place
correspondante dans le réseau avec arcs inhibiteurs.}, l'autre possède un jeton lorsque la chambre
est vide de ressources\footnote{Cette place quant à elle joue le rôle de l'arc inhibiteur du réseau
précédent, en permettant le test à zéro.}. Les deux places de chacune des chambres sont donc en
exclusion mutuelles et permette de recréer le mécanisme mis en place dans le réseau précédent. La
figure~\ref{pnetfinexo5} est une réprésentation du réseau de Petri que nous venons de décrire.
