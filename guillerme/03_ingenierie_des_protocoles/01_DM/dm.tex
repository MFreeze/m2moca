\documentclass[a4paper,11pt]{article}

/stock/16_git_repo/library/library.tex
../../../../00_resources/page_garde.tex

\author{}
\title{}
\location{Montpellier}\blurb{}
\makeatother
\title{Travaux Pratiques}
\author{Guillerme DUVILLIE}
\location{Montpellier}
\blurb{
    Université Montpellier II\\
    Master Informatique Modélisation, Optimisation Combinatoire et Algorithmique \\[1em]
    Matière : Ingénierie des protocoles\
    Encadrant : Rodolphe Giroudeau
}

\newcommand{\gma}{graphe des marquages accesibles}
\newcommand{\Gma}{Graphe des marquages accesibles}

\begin{document}

\maketitle

\section{Exercice 1 - L'utilisation des réseaux de Petri pour le contrôle commande}

\begin{figure}
    \begin{center}
        \begin{tikzpicture}[every transition/.style={minimum width=2mm, minimum height=7mm}, >=latex]
            \node[place,label=$p_1$,tokens=1]                   (p1) {};
            \node[transition, right=of p1, label=$t_1$]         (t1) {}
                edge[pre] (p1);
            \node[place, below right=of t1, label=$p_2$]        (p2) {}
                edge[pre] (t1);
            \node[place, above right=of t1, label=$p_3$]        (p3) {}
                edge[pre] (t1);

            \foreach \x/\y in {2/4, 3/5, 4/6, 5/7}{
                \node[transition, right=of p\x, label=$t_{\x}$] (t\x) {}
                    edge[pre] (p\x);
                \node[place, right=of t\x, label=$p_{\y}$]      (p\y) {}
                    edge[pre] (t\x);
            }

            \node[transition, above right=of p6, label=$t_6$]   (t6) {}
                edge[pre] (p6)
                edge[pre] (p7);    

            \draw[post] (t6) to (11, 0)
                to (11, -3)
                to (-1, -3)
                to (-1, 0)
                to (p1);
        \end{tikzpicture}
    \end{center}
    \caption{Réseau de Petri modélisant le système de commandes}
    \label{pnetexo1}
\end{figure}

\begin{table}
    \begin{center}
        \begin{tabular}{c|c}
            Place & Correspondance \\\hline
            $p_1$ & $ \displaystyle \overline{Ag} \wedge \overline{Rg} \wedge \overline{Ad} \wedge \overline{Rd}$ \\
            $p_2$ & $ Ag \wedge \overline{Rg}$ \\
            $p_3$ & $ Ad \wedge \overline{Rd}$ \\
            $p_4$ & $ \overline{Ad} \wedge Rd$ \\
            $p_5$ & $ \overline{Ag} \wedge Rg$ \\
            $p_6$ & $ \overline{Ag} \wedge \overline{Rg}$ \\
            $p_7$ & $ \overline{Ad} \wedge \overline{Rd}$ \\
        \end{tabular}
    \end{center}
    \caption{Table de correspondance \emph{Places - États du système}}
    \label{tbexo1}
\end{table}

\begin{figure}
    \begin{center}
        \begin{tikzpicture}[>=latex]
            \node (1) {$1000000$};
            \node[below=of 1] (2) {$0110000$}
                edge[pre] node[auto] {$t_1$} (1);
            \node[below left=of 2] (3) {$0011000$}
                edge[pre] node[auto] {$t_2$} (2);
            \node[below right=of 2] (4) {$0100100$}
                edge[pre] node[auto] {$t_3$} (2);
            \node[below left=of 3] (5) {$0010010$}
                edge[pre] node[auto] {$t_4$} (3);
            \node[below right=of 3] (6) {$0001100$}
                edge[pre] node[auto] {$t_3$} (3)
                edge[pre] node[auto] {$t_2$} (4);
            \node[below right=of 4] (7) {$0100001$}
                edge[pre] node[auto] {$t_5$} (4);
            \node[below right=of 5] (8) {$0000110$}
                edge[pre] node[auto] {$t_3$} (5)
                edge[pre] node[auto] {$t_4$} (6);
            \node[below left=of 7] (9) {$0001001$}
                edge[pre] node[auto] {$t_5$} (6)
                edge[pre] node[auto] {$t_2$} (7);
            \node[below right=of 8] (10) {$0000011$}
                edge[pre] node[auto] {$t_5$} (8)
                edge[pre] node[auto] {$t_4$} (9);

            \draw[->] (10) to[out=0, in=270] (7, -4) to[out=90, in=0] (1);
        \end{tikzpicture}
    \caption{Graphe des marquages accessibles}
    \label{gmaexo1}
    \end{center}
\end{figure}

Considérons le réseau de Petri de la figure~\ref{pnetexo1} modélisant le problème. Chaque place
représente un état du système, la table~\ref{tbexo1} liste les correspondances.

On cherche à étudier les propriétés de ce système et s'assurer de sa fiabilité, pour ce faire,
intéressons nous aux propriétés du réseau décrit précédemment, commençons par définir la matrice
d'incidence de ce dernier :
\begin{displaymath}
    C = \left [ 
    \begin{array}{cccccc}
        -1 & 0 & 0 & 0 & 0 & 1 \\
        1 & -1 & 0 & 0 & 0 & 0 \\
        1 & 0 & -1 & 0 & 0 & 0 \\
        0 & 1 & 0 & -1 & 0 & 0 \\
        0 & 0 & 1 & 0 & -1 & 0 \\
        0 & 0 & 0 & 1 & 0 & -1 \\
        0 & 0 & 0 & 0 & 1 & -1 \\
    \end{array} \right ]
\end{displaymath}

\begin{thrm}
    Le réseau de la figure~\ref{pnetexo1} est cohérent.
\end{thrm}

\begin{proof}
    Chaque n\oe ud du \gma ne comporte que des 0 et des 1 indiquant que, quelque soit la place du
    réseau, celle ci ne peut recevoir qu'une seule marque.
\end{proof}

\begin{thrm}
    Pour le système étudié, il y a exclusion mutuelle pour tous couples de places étiquetées par des
    combinaisons booléennes incompatibles de signaux de sortie.
\end{thrm}

\begin{proof}
    Une première preuve consiste à regarder tous les couples en question et à vérifier dans le \gma
    que, quelque soit le marquage, elles sont en exclusion mutuelle. Mais nous utiliserons, pour
    réaliser la preuve, les invariants du réseau et donc le calculs des $P-$flots associés au
    réseau.

    Les $P-$flots sont obtenus en résolvant le système d'équation décrit par la formule suivante : 
    \begin{displaymath}
        ~^tf \times C = 0
    \end{displaymath}
    où $f$ est un vecteur colonne de dimension 7. On obtient alors le système suivant :
    \begin{displaymath}
        \left \{ \begin{array}{ccccccccccccccl}
            -f_1 & + &  f_2 & + &  f_3 &   &      &   &      &   &      &   &      & = & 0\\
                 &   & -f_2 &   &      & + &  f_4 &   &      &   &      &   &      & = & 0\\
                 &   &      &   & -f_3 &   &      & + &  f_5 &   &      &   &      & = & 0\\
                 &   &      &   &      &   & -f_4 &   &      & + &  f_6 &   &      & = & 0\\
                 &   &      &   &      &   &      &   & -f_5 &   &      & + &  f_7 & = & 0\\
             f_1 &   &      &   &      &   &      &   &      & - &  f_6 & - &  f_7 & = & 0\\
        \end{array}
        \right .
    \end{displaymath}
    On obtient alors les égalités suivantes :
    \begin{displaymath}
        \left \{
        \begin{array}{rcl}
            f_2 & = & f_6 \\
            f_3 & = & f_7 \\
            f_4 & = & f_6 \\
            f_5 & = & f_7 \\
            f_1 & = & f_6 + f_7\\
        \end{array}
        \right .
    \end{displaymath}
    Ainsi tout vecteur de la forme : $~^t(f_6 + f_7, f_6, f_7, f_6, f_7, f_6, f_7)$ est solution du
    système donné précédemment, on en déduit alors que toute combinaison linéaire des vecteurs : $a
    = ~^t(1, 1, 0, 1, 0, 1, 0)$ et $b = ~^t(1, 0, 1, 0, 1, 0, 1)$ est solution du système. Et donc
    les invariants nous sont donnés par : \begin{displaymath}
        \left \{ \begin{array}{rcl}
            p_1 + p_2 + p_4 + p_6 & = & 1 \\
            p_1 + p_3 + p_5 + p_7 & = & 1 \\
        \end{array}
        \right .
    \end{displaymath}
    Ce système d'invariants nous assure que les couples $(p_1, p_i)\ \forall i \in \{2, \dots, 7\},$
    $(p_2, p_4)$, $(p_2, p_6)$, $(p_3, p_5)$, $(p_3, p_7)$, $(p_4, p_6)$, $(p_5, p_7)$ sont en exclusions
    mutuelle. Or il s'agit là des couples de places étiquetées par des combinaisons booléennes
    incompatibles de signaux de sortie.
\end{proof}

\begin{rmq}
    Les $P-$flots et les $P-$semiflots de ce réseau sont identiques.
\end{rmq}

\begin{proof}
    Ceci est vrai puisque les $P-$flots calculés sont à valeur entières et positives.
\end{proof}

\begin{thrm}
    Le réseau n'admet pas de blocage.
\end{thrm}

\begin{proof}
    Le \gma ne présente aucun n\oe ud dont le degré sortant est nul.
\end{proof}

\begin{thrm}
    Le réseau est un graphe d'événements.
\end{thrm}

\begin{proof}
    Toute place possède une seule transition d'entrée et une seule transition de sortie. Donc par
    définition, il s'agit d'un graphe d'événement dont les circuits élémentaires sont décrits à la
    figure~\ref{ceexo1}.

    \begin{figure}
        \begin{center}
            \begin{tikzpicture}[every transition/.style={minimum width=2mm, minimum height=7mm},
            >=latex]
                \node[place, tokens=1, label=$p_1$]         (p1) {};
                \node[transition, right=of p1, label=$t_1$] (t1) {}
                    edge[pre] (p1);
                \node[place, right=of t1, label=$p_2$]      (p2) {}
                    edge[pre] (t1);
                \node[transition, below=of p2, label=$t_2$] (t2) {}
                    edge[pre] (p2);
                \node[place, below=of t2, label=$p_4$]      (p4) {}
                    edge[pre] (t2);
                \node[transition, left=of p4, label=$t_4$]  (t4) {}
                    edge[pre] (p4);
                \node[place, left=of t4, label=$p_6$]       (p6) {}
                    edge[pre] (t4);
                \node[transition, above=of p6, label=$t_6$]  (t6) {}
                    edge[pre] (p6)
                    edge[post] (p1);

                \node[place, tokens=1, label=$p_1$] at (5, 0) (p1a) {};
                \node[transition, right=of p1a, label=$t_1$](t1a) {}
                    edge[pre] (p1a);
                \node[place, right=of t1a, label=$p_3$]     (p3) {}
                    edge[pre] (t1a);
                \node[transition, below=of p3, label=$t_3$] (t3) {}
                    edge[pre] (p3);
                \node[place, below=of t3, label=$p_5$]      (p5) {}
                    edge[pre] (t3);
                \node[transition, left=of p5, label=$t_5$]  (t5) {}
                    edge[pre] (p5);
                \node[place, left=of t5, label=$p_7$]       (p7) {}
                    edge[pre] (t5);
                \node[transition, above=of p7, label=$t_7$]  (t7) {}
                    edge[pre] (p7)
                    edge[post] (p1a);

            \end{tikzpicture}
        \end{center}
        \caption{Circuits élémentaires du réseau de Petri}
        \label{ceexo1}
    \end{figure}
\end{proof}
            
On peut aussi remarquer que le réseau est vivant et réinitialisable.


\section{Exercice 2 - Spécification d'un protocole de communication}

\begin{figure}
    \begin{center}
        \begin{tikzpicture}[every transition/.style={minimum width=2mm, minimum height=7mm},
        >=latex, node distance=18mm]
            \node[place,label=Repos $(p_0)$,tokens=1] (urepos) at (0,0) {};

            \node[transition,below right=of urepos,label=Décroche $(t_0)$] (udec) {}
                edge[pre] (urepos);

            \node[place,label=Attente $(p_1)$,below left=of udec] (uattente) {}
                edge[pre] (udec);

            \node[transition, below left=of urepos, label=Raccroche $(t_1)$] (urac) {}
                edge[pre] (uattente)
                edge[post] (urepos);

            \node[transition, below=of uattente, label=Sélection $(t_2)$] (uselec) {}
                edge[pre] (uattente);

            \node[place, below=of uselec, label=Attente 2 $(p_2)$] (uattente2) {}
                edge[pre] (uselec);

            \node[transition, below right=of uattente2, label=Acceptation $(t_4)$] (ucaccept) {}
                edge[pre] (uattente2);

            \node[place,label=Repos $(p_3)$,tokens=1,above right=of ucaccept] (crepos) {}
                edge[post] (ucaccept);

            \node[transition, below=of ucaccept,label=Raccroche $(t_5)$] (ucrac) {}
                edge[post, out=45, in=0] (urepos);

            \node[place,left=of ucrac,label=Comm $(p_4)$] (ucomm) {}
                edge[pre] (ucaccept)
                edge[post] (ucrac);

            \node[place,right=of ucrac,label=Comm $(p_5)$] (ccomm) {}
                edge[pre] (ucaccept)
                edge[post] (ucrac);

            \node[transition, below=of ucrac,label=Libère $(t_6)$] (uclib) {}
                edge[pre] (ucomm)
                edge[pre] (ccomm);

            \node[transition, below=of uclib,label=Raccroche $(t_7)$] (ucracr2) {};

            \node[place, left=of ucracr2, label=Faux $(p_6)$] (ufaux) {}
                edge[post] (ucracr2)
                edge[pre] (uclib);

            \node[place, right=of ucracr2, label=Faux $(p_7)$] (cfaux) {}
                edge[post] (ucracr2)
                edge[pre] (uclib);

            \node[place, right=of uclib, label=Attente 3 $(p_8)$] (catt) {}
                edge[pre] (ucrac)
                edge[pre] (ucracr2);

            \node[transition, right=of catt, label=Confirm $(t_8)$] (cconf) {}
                edge[pre] (catt)
                edge[post, bend right] (crepos);

            \node[transition, label=Raccroche $(t_3)$] at (-4, -4) (uraccr3) {}
                edge[pre, bend right] (uattente2)
                edge[post, bend left] (urepos);

            \draw[post, ->] (ucracr2) to[out=225, in=270] (-6, -8) to[out=90, in=180] (urepos);

        \end{tikzpicture}
    \end{center}
    \caption{Repérésentation d'un protocole de communication à l'aide d'un réseau de Pétri}
    \label{fig1_ex2}
\end{figure}
        
Considérons le réseau de la figure~\ref{fig1_ex2} représentant le protocole de communication décrit
dans l'énoncé.
\begin{thrm}
     Le réseau considéré vérifie les spécifications du protocole de communication étudié.
\end{thrm}

\begin{proof}
    Lorsque l'utilisateur décroche, la \emph{disparition} du jeton dans la place $p_0$ permet de
    mémoriser l'événement. Il sélectionne ensuite une ligne ($t_2$) et attend que le contrôleur
    accepte la sélection de la ligne $(t_4)$. Il est alors en communication $(p_4)$. Si la ligne est
    libérée sans que l'utilisateur n'ait raccroché, lui et le contrôleur passent en état de faux
    appel ($p_5$ et $p_7$). Dans l'éventualité où l'utilisateur raccroche après la mise en
    communication, le contrôleur passe alors dans un état d'attente de confirmation de raccrochage
    $p_8$, avant de revenir à son état de repos.
    Une fois que l'utilisateur a raccroché, il revient en état de repos et peut donc à nouveau
    décrocher.

    Les spécifications sont donc vérifiées.
\end{proof}

Définissons la matrice d'incidence du réseau de Pétri : \[
    C = \left [
    \begin{array}{ccccccccc}
        -1 &  1 &  0 &  1 &  0 &  1 &  0 &  1 &  0 \\
         1 & -1 & -1 &  0 &  0 &  0 &  0 &  0 &  0 \\
         0 &  0 &  1 & -1 & -1 &  0 &  0 &  0 &  0 \\
         0 &  0 &  0 &  0 & -1 &  0 &  0 &  0 &  1 \\
         0 &  0 &  0 &  0 &  1 & -1 & -1 &  0 &  0 \\
         0 &  0 &  0 &  0 &  1 & -1 & -1 &  0 &  0 \\
         0 &  0 &  0 &  0 &  0 &  0 &  1 & -1 &  0 \\
         0 &  0 &  0 &  0 &  0 &  0 &  1 & -1 &  0 \\
         0 &  0 &  0 &  0 &  0 &  1 &  0 &  1 & -1 \\
    \end{array}
    \right ]
\]

\begin{thrm}
    Les invariants de ce réseaux sont donnés par : \[
        \left \lbrace \begin{array}{rcl}
            p_0 + p_1 + p_2 + p_4 + p_6 & = & 1\\
            p_0 + p_1 + p_2 + p_5 + p_6 & = & 1\\
            p_0 + p_1 + p_2 + p_4 + p_7 & = & 1\\
            p_0 + p_1 + p_2 + p_5 + p_7 & = & 1\\
            p_3 + p_4 + p_6 + p_8       & = & 1\\
            p_3 + p_5 + p_6 + p_8       & = & 1\\
            p_3 + p_4 + p_7 + p_8       & = & 1\\
            p_3 + p_5 + p_7 + p_8       & = & 1\\
        \end{array} \right .
    \]
\end{thrm}

\begin{proof}
    Pour calculer les $P-$flots, il suffit de résoudre le système d'équations donné par : \[
        ^tf.C = 0
    \]
    avec $^tf = (f_1, f_2, f_3, f_4, f_5, f_6, f_7, f_8, f_9)$.

    On obtient alors : \[
        \left \lbrace \begin{array}{rcl}
            f_2 - f_1 & = & 0 \\
            f_1 - f_2 & = & 0 \\
            f_3 - f_2 & = & 0 \\
            f_3 - f_1 & = & 0 \\
            f_5 + f_6 - f_3 - f_4 & = & 0 \\
            f_1 + f_9 - f_5 - f_6 & = & 0 \\
            f_7 + f_8 - f_5 - f_6 & = & 0 \\
            f_1 + f_9 - f_7 - f_8 & = & 0 \\
            f_4 - f_9 & = & 0 \\
        \end{array}
        \right .
    \]
     On obtient après simplification et suppression des équations redondantes : \[
        \left \lbrace \begin{array}{rcl}
            f_2 & = & f_1 \\
            f_3 & = & f_1 \\
            f_9 & = & f_4 \\
            f_5 + f_6 & = & f_1 + f_4 \\
            f_7 + f_8 & = & f_1 + f_4 \\
        \end{array}
        \right .
    \]
    Ainsi, tout vecteur de la forme $^t(f_1, f_1, f_1, f_4, f_5, f_6, f_7, f_8, f_4)$ vérifiant $f_5
    + f_6 = f_7 + f_8 = f_1 + f_4$ est solution du sytème et donc toute combinaison linéaire des
    vecteurs suivants est solution du système : \[
        \left ( \begin{array}{c}
            1 \\ 1 \\ 1 \\ 0 \\ 1 \\ 0 \\ 1 \\ 0 \\ 0
        \end{array} \right ) \ 
        \left ( \begin{array}{c}
            1 \\ 1 \\ 1 \\ 0 \\ 0 \\ 1 \\ 1 \\ 0 \\ 0
        \end{array} \right ) \ 
        \left ( \begin{array}{c}
            1 \\ 1 \\ 1 \\ 0 \\ 1 \\ 0 \\ 0 \\ 1 \\ 0
        \end{array} \right ) \ 
        \left ( \begin{array}{c}
            1 \\ 1 \\ 1 \\ 0 \\ 0 \\ 1 \\ 0 \\ 1 \\ 0
        \end{array} \right ) \ 
        \left ( \begin{array}{c}
            0 \\ 0 \\ 0 \\ 1 \\ 1 \\ 0 \\ 1 \\ 0 \\ 1
        \end{array} \right ) \ 
        \left ( \begin{array}{c}
            0 \\ 0 \\ 0 \\ 1 \\ 1 \\ 0 \\ 0 \\ 1 \\ 1
        \end{array} \right ) \ 
        \left ( \begin{array}{c}
            0 \\ 0 \\ 0 \\ 1 \\ 0 \\ 1 \\ 1 \\ 0 \\ 1
        \end{array} \right ) \ 
        \left ( \begin{array}{c}
            0 \\ 0 \\ 0 \\ 1 \\ 0 \\ 1 \\ 0 \\ 1 \\ 1
        \end{array} \right ) \ 
    \]
    Le marquage initial étant égal à $p_0p_3$, on en déduit les invariants donnés précédemment.
\end{proof}

\begin{thrm}
    Les $P-$semiflots sont identiques aux $P-$flots.
\end{thrm}

\begin{proof}
    Les invariants précédemment calculés présentant des coefficients entiers positifs les
    $P-$semiflots sont donc identiques aux $P-$flots.
\end{proof}


\section{Exercice 3 - Étude d'un réseau de Petri particulier}

\begin{figure}
    \begin{center}
        \begin{tikzpicture}[every transition/.style={minimum width=2mm, minimum height=7mm},
        >=latex, node distance=18mm, scale=0.9, every place/.style={transform shape}]
            \node[place, label=$d_0$,tokens=1] (d0) at (0,0) {};
            \node[transition, right=of d0, label=$t_0^0$] (t00) {}
                edge[pre] (d0);
            \node[place, right=of t00, label=$p_0$] (p0) {}
                edge[pre] (t00);
            \node[transition, right=of p0, label=$t_0^1$] (t01) {}
                edge[pre] (p0);
            \node[place, right=of t01, label=$p_0'$] (pp0) {}
                edge[pre] (t01);
            \node[transition, right=of pp0, label=$t_0^2$] (t02) {}
                edge[pre] (pp0);
            \node[place, right=of t02, label=$f_0$] (f0) {}
                edge[pre] (t02);
            \node[transition, above=of p0, label=$t_0^3$] (t03) {}
                edge[pre, bend right] (p0)
                edge[post, bend left] (p0);
            \node[place, right=of t03, label=$c_0'$] (cp0) {}
                edge[pre] node[auto] {2} (t03);
            \node[transition, right=of cp0, label=$t_0^4$] (t04) {}
                edge[pre] (cp0)
                edge[pre, bend right] (pp0)
                edge[post, bend left] (pp0);
            \node[place, above=of cp0, label=$c_0$, structured tokens=$n$] (c0) {}
                edge[pre, bend left] (t04)
                edge[post, bend right] (t03);
        \end{tikzpicture}
    \end{center}
    \caption{Un réseau de Petri particulier $N_0^n$}
    \label{exo3fig1}
\end{figure}
        
\begin{figure}
    \begin{center}
        \begin{tikzpicture}[every transition/.style={minimum width=2mm, minimum height=7mm},
        >=latex, scale=0.65, every node/.style={transform shape}]
            \node[place, label=$d_0$] (d0) at (0,0) {};
            \node[transition, right=of d0, label=$t_0^0$] (t00) {}
                edge[pre] (d0);
            \node[place, right=of t00, label=$p_0$] (p0) {}
                edge[pre] (t00);
            \node[transition, right=of p0, label=$t_0^1$] (t01) {}
                edge[pre] (p0);
            \node[place, right=of t01, label=$p_0'$] (pp0) {}
                edge[pre] (t01);
            \node[transition, right=of pp0, label=$t_0^2$] (t02) {}
                edge[pre] (pp0);
            \node[place, right=of t02, label=$f_0$] (f0) {}
                edge[pre] (t02);
            \node[transition, above=of p0, label=$t_0^3$] (t03) {}
                edge[pre, bend right] (p0)
                edge[post, bend left] (p0);
            \node[place, right=of t03, label=$c_0'$] (cp0) {}
                edge[pre] node[auto] {2} (t03);
            \node[transition, right=of cp0, label=$t_0^4$] (t04) {}
                edge[pre] (cp0)
                edge[pre, bend right] (pp0)
                edge[post, bend left] (pp0);
            \node[place, above=of cp0, label=$c_0$] (c0) {}
                edge[pre, bend left] (t04)
                edge[post, bend right] (t03);

            \node[transition, left=of d0, label=$a_1$] (a1) {}
                edge[post] (d0);
            \node[place, left=of a1, label=$d_1$, tokens=1] (d1) {}
                edge[post] (a1);
            \node[transition, above=of d1, label=$r_1$] (r1) {}
                edge[post, bend left] (d1)
                edge[pre, bend right] (d1)
                edge[post, out=0, in=180] (c0);
            \node[place, above=of r1, label=$c_1$, structured tokens=$n$] (c1) {}
                edge[post] (r1);
            \node[place, below=of d1, label=$c_1'$] (cp1) {}
                edge[pre, in=180, out=180] (r1);
            \node[transition, below=of t01, label=$e_1$] (e1) {}
                edge[pre] (cp1)
                edge[post] (d0)
                edge[pre] (f0);
            \node[transition, right=of f0, label=$t_1^0$] (t10) {}
                edge[pre] (f0);
            \node[place, right=of t10, label=$p_1$] (p1) {}
                edge[pre] (t10);
            \node[transition, right=of p1, label=$t_1^1$] (t11) {}
                edge[pre, bend right] (p1)
                edge[post, bend left] (p1)
                edge[pre, in=0, out=150] (c0)
                edge[post, out=90, in=60] (c1);
            \node[place, right=of t11, label=$f_1$] (f1) {}
                edge[pre] (t11);
        \end{tikzpicture}
    \end{center}
    \caption{Un réseau de Petri particulier $N_1^n$}
    \label{exo3fig2}
\end{figure}

Considérons le réseau de Petri de la figure~\ref{exo3fig1} et le réseau de la figure~\ref{exo3fig2}
construit à partir de celui de la figure~\ref{exo3fig1}.

\begin{thrm}
    Le réseau de Petri de la figure~\ref{exo3fig1} présente un invariant donné par :
    \[
        d_0 + p_0 + p_0' + f_0 = 1 
    \]
\end{thrm}

\begin{proof}
    Pour trouver les invariants d'un réseau, il suffit de trouver les solutions au système
    d'équations obtenu à partir de : \[
        ^tfC = 0
    \]. Avec $C$ la matrice d'incidence du réseau, que l'on décrira ainsi :
    \[
        \left [ \begin{array}{ccccc}
            -1 & 0 & 0 & 0 & 0 \\
            1 & -1 & 0 & 0 & 0 \\
            0 & 1 & -1 & 0 & 0 \\
            0 & 0 & 1 & 0 & 0 \\
            0 & 0 & 0 & -1 & 1 \\
            0 & 0 & 0 & 2 & -1 \\
        \end{array}
        \right ]
    \]

    Le système est alors donné par : \[
        \left \lbrace \begin{array}{rcl}
            -f_1 + f_2 & = & 0 \\
            -f_2 + f_3 & = & 0 \\
            -f_3 + f_4 & = & 0 \\
            -f_5 + 2f_6 & = & 0 \\
            f_5 - f_6 & = & 0 \\
        \end{array}
        \right .
    \]
    \[
        \Rightarrow \left \lbrace \begin{array}{rcl}
            f_2 & = & f_1 \\
            f_3 & = & f_1 \\
            f_4 & = & f_1 \\
            f_5 & = & 0 \\
            f_6 & = & 0 \\
        \end{array}
        \right .
    \]
    Les vecteurs de la forme $^t(f_1, f_1, f_1, f_1, 0, 0)$ sont solutions du système précédent, on
    en déduit donc l'invariant énoncé précédemment. De la même manière que l'exercice 2, le $P-$flot
    étant à valeur entière et positive, il représente aussi un $P-$semiflot.
\end{proof}

\begin{rmq}
    Le $P-$semi flot trouvé à l'aide de l'outil Tina est identique à celui trouvé précédemment.
\end{rmq}

\verbatiminput{03_exo3/N0n-struct.txt}

\begin{thrm}
    Le $P-$semiflot du réseau de la figure~\ref{exo3fig2} est le suivant : \[
        d_0 + p_0 + p_0' + f_0 + p_1 + d_1 = 1
    \]
\end{thrm}

\begin{proof}
    Nous allons calculer les $P-$flots de la même que précédemment. Commençons par la matrice $C'$,
    définie comme suit : \[
        \left [
        \begin{array}{cccccccccc}
            -1 & 0 & 0 & 0 & 0 & 0 & 1 & 0 & 0 & 1\\
            1 & -1 & 0 & 0 & 0 & 0 & 0 & 0 & 0 & 0\\
            0 & 1 & -1 & 0 & 0 & 0 & 0 & 0 & 0 & 0\\
            0 & 0 & 1 & 0 & 0 & 0 & 0 & -1 & 0 & -1\\
            0 & 0 & 0 & -1 & 1 & 1 & 0 & 0 & -1 & 0\\
            0 & 0 & 0 & 2 & -1 & 0 & 0 & 0 & 0 & 0\\
            0 & 0 & 0 & 0 & 0 & -1 & 0 & 0 & 1 & 0\\
            0 & 0 & 0 & 0 & 0 & 1 & -1 & 0 & 0 & -1\\
            0 & 0 & 0 & 0 & 0 & 0 & 0 & 1 & 0 & 0\\
            0 & 0 & 0 & 0 & 0 & 0 & 0 & 1 & 0 & 0\\
            0 & 0 & 0 & 0 & 0 & 0 & -1 & 0 & 0 & 0\\
        \end{array}
        \right ]
    \]
    On remarque que la matrice $C'$ est de la forme : \[
        \left [ \begin{array}{cc}
            C & A \\
            0 & A' \\
        \end{array} \right ]
    \]
    On peut donc d'ores et déjà poser : 
    \[
        \left \lbrace \begin{array}{rcl}
            f_2 & = & f_1 \\
            f_3 & = & f_1 \\
            f_4 & = & f_1 \\
            f_5 & = & 0 \\
            f_6 & = & 0 \\
        \end{array}
        \right .
    \]
    représentant le système d'équation du réseau précédent. Il reste à résoudre le système suivant :
    \[
        \left \lbrace \begin{array}{rcl}
            f_5 - f_7 + f_8 & = & 0 \\
            f_1 - f_8 - f_{11} & = & 0 \\
            f_9 + f_4 & = & 0 \\
            f_7 + f_{10} - f_5 & = & 0 \\
            f_1 - f_4 - f_8 & = & 0
        \end{array} \right .
    \]
    On obtient alors : \[
        \left \lbrace \begin{array}{ccccccccccc}
            f_1 & = & f_2 & = & f_3 & = & f_4 & = & f_9 & = & f_{11} \\
            f_5 & = & f_6 & = & f_7 & = & f_8 & = & f_{10} & = & 0 \\
        \end{array} \right .
    \]
    Le $P-$ semiflot est dont donné par le vecteur : \[
    \left ( \begin{array}{c}
        1 \\ 1 \\ 1 \\ 1 \\ 0 \\ 0 \\ 0 \\ 0 \\ 1 \\ 0 \\ 1
    \end{array} \right )
    \]
    Le marquage initial étant donné par $d_1c_1^n$, l'invariant est donc :\[
        d_0 + p_0 + p_0' + f_0 + p_1 + d_1 = 1
    \]
\end{proof}

\begin{rmq}
    Le $P-$semiflot trouvé à l'aide de l'outil Tina est identique à celui trouvé précédemment.
\end{rmq}

\verbatiminput{03_exo3/N1n-struct.txt}

La séquence maximale de $N_0^n$ telle que $d_0$ est unitaire est la suivante : \[
    t_0^0 t_0^3 t_0^1 t_0^4 t_0^2
\].

La séquence maximale de $N_1^n$ telle que $d_1$ est unitaire est la suivante : \[
    r_1 a_1 t_0^0 t_0^3 t_0^1 t_0^4 t_0^2 t_1^0 t_1^1
\].

Nous ne donnerons l'arborescence de couverture que pour $N_0^1$ pour des raisons de taille de
l'arborescence. Cette dernière est donnée à la figure~\ref{exo3fig3}.

\begin{figure}
    \begin{center}
        \begin{tikzpicture}
            \node (n1) at (0,0) {$(d_0c_0)$};

            \node[below=of n1] (n2) {$(p_0c_0)$}
                edge[pre] node[auto] {$t_0^0$} (n1);

            \node[below left=of n2] (n3) {$(p_0'c_0)$}
                edge[pre] node[auto] {$t_0^1$} (n2);

            \node[below right=of n2] (n4) {$(p_0'c_0^{'2})$}
                edge[pre] node[auto] {$t_0^3$} (n2);

            \node[below=of n3] (n5) {$(f_0c_0)$}
                edge[pre] node[auto] {$t_0^2$} (n3);

            \node[below=of n4] (n6) {$(p'_0c_0^{'2})$}
                edge[pre] node[auto] {$t_0^1$} (n4);

            \node[below left=of n6] (n7) {$(f_0c_0^{'2})$}
                edge[pre] node[auto] {$t_0^2$} (n6);

            \node[below right=of n6] (n8) {$(p'_0c_0c'_0)$}
                edge[pre] node[auto] {$t_0^4$} (n6);

            \node[below left=of n8] (n9) {$(f_0c_0c'_0)$}
                edge[pre] node[auto] {$t_0^2$} (n8);

            \node[below right=of n8] (n10) {$(p'_0c_0^2)$}
                edge[pre] node[auto] {$t_0^4$} (n8);

            \node[below=of n10] (n11) {$(f_0c_0^2)$}
                edge[pre] node[auto] {$t_0^2$} (n10);
        \end{tikzpicture}
    \end{center}
    \caption{Arborescence de couverture du réseau de Petri $N_0^1$}
    \label{exo3fig3}
\end{figure}


\section{Exercice 4 - Spécification d'une construction d'une maison}

On peut représenter de deux manières le diagramme donné. La première d'entre elle est représentée à
la figure~\ref{pnet1exo4} représentant chacune des tâches par une transition, ainsi chaque place
représente l'état intermédiaire entre la fin d'une tâche et le début d'une autre. La seconde est
réprésentée par la figure~\ref{pnet2exo4} représentant chacune des tâches par des places, ainsi
chaque transition marque la fin d'une tâche et le commencement d'une autre tâche.

Il est aisé de remarquer, en observant les réseaux de la figure~\ref{pnetallexo4}, que la seconde
représentation comporte non seulement moins de places et de transitions, mais paraît aussi plus
intuitive. En effet, pour la première représentation, la transition est franchie une fois la tâche
terminée, donc pour connaître la tâche en cours il faut se référer à la transition suivante, alors
que pour la seconde réprésentation, la tâche en cours est immédiatement connue. De plus, la première
représentation, donne une impresssion d'instantanéité des tâches, qui ne se révèle pas dans la
seconde. 

En conclusion, j'aurais tendance à provilégier la représentation des tâches par des places plutôt
que par des transitions.

\begin{figure}
    \begin{center}
        \subfloat[Première représentation du diagramme]{\label{pnet1exo4}
        \begin{tikzpicture}[scale=0.6, every transition/.style={minimum width=7mm, minimum
        height=2mm, transform shape}, every node/.style={transform shape},>=latex]
            \node[place, tokens=1] (p0) {};

            \node[transition, below=of p0, label=Excavation]        (t1) {}
                edge[pre] (p0);
            \node[place, below=of t1]                               (p1) {}
                edge[pre] (t1);

            \node[transition, below=of p1, label=Contreting]        (t2) {}
                edge[pre] (p1);
            \node[place, below=of t2]                               (p2) {}
                edge[pre] (t2);

            \node[transition, below=of p2, label=Bricklaying]       (t3) {}
                edge[pre] (p2);
            \node[place, below right=of t3]                         (p3) {}
                edge[pre] (t3);
            \node[place, below left=of t3]                          (p4) {}
                edge[pre] (t3);

            \node[transition, below=of p3, label=Roof] (t4) {}
                edge[pre] (p3);
            \node[place, below right=of t4]                         (p5) {}
                edge[pre] (t4);
            \node[place, below left=of t4]                          (p6) {}
                edge[pre] (t4);

            \node[transition, below=of p4, label=Windows and Doors]  (t5) {}
                edge[pre] (p4);
            \node[place, below left=of t5]                          (p7) {}
                edge[pre] (t5);

            \node[transition, below=of p5, label=Ceilings]          (t6) {}
                edge[pre] (p5);
            \node[place, below=of t6]                               (p8) {}
                edge[pre] (t6);

            \node[transition, below=of p6, label=Floors]            (t7) {}
                edge[pre] (p6);
            \node[place, below=of t7]                               (p9) {}
                edge[pre] (t7);

            \node[transition, below=of p7, label=Toilet]            (t8) {}
                edge[pre] (p7);
            \node[place, below=of t8]                               (pa) {}
                edge[pre] (t8);

            \node[transition, below=of p9, label=Kitchen]           (t9) {}
                edge[pre] (p9);
            \node[place, below=of t9]                               (pb) {}
                edge[pre] (t9);

            \node[transition, below=of pb, label=Furniture]         (ta) {}
                edge[pre, bend right] (p8)
                edge[pre, bend left] (pa)
                edge[pre] (pb);
            \node[place, below=of ta]                               (pc) {}
                edge[pre] (ta);
            \node at (7, 0) {};
        \end{tikzpicture}}
    \subfloat[Seconde représentation du diagramme]{\label{pnet2exo4} \hfill
        \begin{tikzpicture}[scale=0.6, every transition/.style={minimum width=7mm, minimum
        height=2mm, transform shape}, every node/.style={transform shape},>=latex]
            \node[place, tokens=1, label=Excavation]                (p1) {};

            \node[transition, below=of p1]                          (t1) {}
                edge[pre] (p1);
            \node[place, below=of t1, label=Contreting]             (p2) {}
                edge[pre] (t1);

            \node[transition, below=of p2]                          (t2) {}
                edge[pre] (p2);
            \node[place, below=of t2, label=Bricklaying]            (p3) {}
                edge[pre] (t2);

            \node[transition, below=of p3]                          (t3) {}
                edge[pre] (p3);
            \node[place, below left=of t3, label=Windows and Doors] (p4) {}
                edge[pre] (t3);
            \node[place, below right=of t3, label=Roof](p5) {}
                edge[pre] (t3);

            \node[transition, below=of p4]                          (t4) {}
                edge[pre] (p4);
            \node[place, below left=of t4, label=Toilet]            (p6) {}
                edge[pre](t4);

            \node[transition, below=of p5]                          (t5) {}
                edge[pre] (p5);
            \node[place, below left=of t5, label=Floors]            (p7) {}
                edge[pre] (t5);
            \node[place, below right=of t5, label=Ceilings]         (p8) {}
                edge[pre] (t5);

            \node[transition, below=of p7]                          (t6) {}
                edge[pre] (p7);
            \node[place, below=of t6, label=Kitchen]                (p9) {}
                edge[pre] (t6);

            \node[transition, below=of p9]                          (t7) {}
                edge[pre] (p9)
                edge[pre, bend left] (p6)
                edge[pre, bend right] (p8);
            \node[place, below=of t7, label=Furniture]              (pa) {}
                edge[pre] node[auto] {$3$} (t7);
            \node[transition, below=of pa]                          (t8) {}
                edge[pre] (pa);
            \node[place, below=of t8, label=End]                    (pb) {}
                edge[pre] (t8);
            \node at (-7,0) {};
        \end{tikzpicture}}
    \end{center}
    \caption{Différentes représentations du diagramme}
    \label{pnetallexo4}
\end{figure}


\section{Exercice 5 - Producteurs et consommateurs}

\begin{figure}
    \begin{center}
        \begin{tikzpicture}[every transition/.style={minimum width=7mm, minimum height=2mm}, >=latex]
            \node[place, tokens=1, label=Producteur] at (0, 0) (P) {};
            \node[place] at (0, -4) (int1) {};

            \node[place, label=Buffer] at (2, -4) (buff) {};

            \node[place, label=Consommateur, tokens=1] at (4, 0) (C) {};
            \node[place] at (4, -4) (int2) {};

            \node[transition, label=Produire] at (0, -2) (prod) {}
                edge[pre] (P)
                edge[post] (int1);
            \node[transition, label=Envoyer] at (0, -6) (send) {}
                edge[pre] (int1)
                edge[post] (buff);

            \node[transition, label=Recevoir] at (4, -2) (rec) {}
                edge[pre] (buff)
                edge[pre] (C)
                edge[post] (int2);
            \node[transition, label=Consommer] at (4, -6) (cons) {}
                edge[pre] (int2);

            \draw[post] (send) to (-1.5, -6) to (-1.5, 0) to (P);
            \draw[post] (cons) to (5.5, -6) to (5.5, 0) to (C);
        \end{tikzpicture}
    \end{center}
    \caption{Producteur et consommateur}
    \label{pnetpcexo5}
\end{figure}
    
\begin{figure}
    \begin{center}
        \begin{tikzpicture}[every transition/.style={minimum width=7mm, minimum height=2mm}, >=latex]
            \node[place, structured tokens=$p$, label=Producteurs] at (0, 0) (P) {};
            \node[place] at (0, -4) (int1) {};
            \node[place, label=Buffer] at (2, -4) (buff) {};

            \node[place, label=Consommateurs, structured tokens=$c$] at (4, 0) (C) {};
            \node[place] at (4, -4) (int2) {};

            \node[transition, label=Produire] at (0, -2) (prod) {}
                edge[pre] (P)
                edge[post] (int1);
            \node[transition, label=Envoyer] at (0, -6) (send) {}
                edge[pre] (int1)
                edge[post] (buff);

            \node[transition, label=Recevoir] at (4, -2) (rec) {}
                edge[pre] (buff)
                edge[pre] (C)
                edge[post] (int2);
            \node[transition, label=Consommer] at (4, -6) (cons) {}
                edge[pre] (int2);

            \draw[post] (send) to (-1.5, -6) to (-1.5, 0) to (P);
            \draw[post] (cons) to (5.5, -6) to (5.5, 0) to (C);
        \end{tikzpicture}
    \end{center}
    \caption{Producteurs et consommateurs}
    \label{pnetpcsexo5}
\end{figure}
    
\begin{figure}
    \begin{center}
        \begin{tikzpicture}[every transition/.style={minimum width=7mm, minimum height=2mm}, >=latex]
            \node[place, structured tokens=$p$, label=Producteurs] at (0, 0) (P) {};
            \node[place] at (0, -4) (int1) {};

            \node[place, label=Contrôleur, structured tokens=$z$] at (2, -3) (control) {};
            \node[place, label=Buffer] at (2, -5) (buff) {};

            \node[place, label=Consommateurs, structured tokens=$c$] at (4, 0) (C) {};
            \node[place] at (4, -4) (int2) {};

            \node[transition, label=Produire] at (0, -2) (prod) {}
                edge[pre] (P)
                edge[post] (int1);
            \node[transition, label=Envoyer] at (0, -6) (send) {}
                edge[pre] (int1)
                edge[pre] (control)
                edge[post] (buff);

            \node[transition, label=Recevoir] at (4, -2) (rec) {}
                edge[pre] (buff)
                edge[pre] (C)
                edge[post] (control)
                edge[post] (int2);
            \node[transition, label=Consommer] at (4, -6) (cons) {}
                edge[pre] (int2);

            \draw[post] (send) to (-1.5, -6) to (-1.5, 0) to (P);
            \draw[post] (cons) to (5.5, -6) to (5.5, 0) to (C);
        \end{tikzpicture}
    \end{center}
    \caption{Producteurs et consommateurs avec capacité du Buffer limitée}
    \label{pnetpcbexo5}
\end{figure}
    
\begin{figure}
    \begin{center}
        \begin{tikzpicture}[every transition/.style={minimum width=7mm, minimum height=2mm}, >=latex]
            \node[place, structured tokens=$p$, label=Producteurs] at (0, 0) (P) {};
            \node[place] at (0, -4) (int1) {};

            \node[place, label=$B_1$] at (2, -8) (buff1) {};
            \node at (2, -4) (buff) {$\dots$};
            \node[place, label=$B_b$] at (2, 0) (buffb) {};

            \node[place, label=Consommateurs, structured tokens=$c$] at (4, 0) (C) {};
            \node[place] at (4, -4) (int2) {};

            \node[transition, label=Produire] at (0, -2) (prod) {}
                edge[pre] (P)
                edge[post] (int1);
            \node[transition, label=Envoyer] at (0, -6) (send) {}
                edge[pre] (int1)
                edge[pre, bend right, >=o] (buff1)
                edge[post, bend left] (buff1);

            \node[transition] at (2, -6) (b1) {}
                edge[pre] (buff1)
                edge[pre, >=o, bend right] (buff)
                edge[post, bend left] (buff);

            \node[transition] at (2, -2) (bb) {}
                edge[pre] (buff)
                edge[pre, bend left, >=o] (buffb)
                edge[post, bend right] (buffb);

            \node[transition, label=Recevoir] at (4, -2) (rec) {}
                edge[pre] (buffb)
                edge[pre] (C)
                edge[post] (int2);
            \node[transition, label=Consommer] at (4, -6) (cons) {}
                edge[pre] (int2);

            \draw[post] (send) to (-1.5, -6) to (-1.5, 0) to (P);
            \draw[post] (cons) to (5.5, -6) to (5.5, 0) to (C);
        \end{tikzpicture}
    \end{center}
    \caption{Producteurs et consommateurs avec capacité du Buffer limitée et arcs inhibiteurs}
    \label{pnetpcbexo5}
\end{figure}

\begin{figure}
    \begin{center}
        \begin{tikzpicture}[scale=0.9, >=latex, every node/.style={transform shape}, every
        transition/.style={minimum width=7mm, minimum height=2mm, transform shape}]
            \node[place, structured tokens=$p$, label=Producteurs] at (0, 0) (P) {};
            \node[place] at (0, -4) (int1) {};

            \node[place, label=$B_1$] at (1.5, -4) (b1) {};
            \node[place, label=$B_2$] at (4.5, -4) (b2) {};
            \node[place, label=$B_b$] at (10.5,-4) (bb) {};

            \node[place, label=$C_1$, tokens=1] at (1.5, -6) (c1) {};
            \node[place, label=$C_2$, tokens=1] at (4.5, -2) (c2) {};
            \node[place, label=$C_b$, tokens=1] at (10.5,-2) (cb) {};

            \node at (7.5, -4) (d1) {$\dots$};
            \node at (7.5, -2) (d2) {$\dots$};

            \node[place, label=Consommateurs, structured tokens=$c$] at (12, 0) (C) {};
            \node[place] at (12, -4) (int2) {};

            \node[transition, label=Produire] at (0, -2) (prod) {}
                edge[pre] (P)
                edge[post] (int1);
            \node[transition, label=Envoyer] at (0, -6) (send) {}
                edge[pre] (c1)
                edge[pre] (int1)
                edge[post] (b1);

            \node[transition, label=Recevoir] at (12, -2) (rec) {}
                edge[pre] (C)
                edge[pre] (bb)
                edge[post] (cb)
                edge[post] (int2);
            \node[transition, label=Consommer] at (12, -6) (cons) {}
                edge[pre] (int2);

            \tikzset{every transition/.style={minimum width=2mm, minimum height=7mm, transform shape}};
            \node[transition] at (3, -4) (tb1) {}
                edge[pre] (b1)
                edge[pre] (c2)
                edge[post] (c1)
                edge[post] (b2);

            \node[transition] at (6, -4) (tb2) {}
                edge[pre] (b2)
                edge[pre] (d2)
                edge[post] (c2)
                edge[post] (d1);

            \node[transition] at (9, -4) (tbb) {}
                edge[pre] (d1)
                edge[pre] (bb)
                edge[post] (d2)
                edge[post] (cb);

            \draw[post] (send) to (-1.5, -6) to (-1.5, 0) to (P);
            \draw[post] (cons) to (13.5, -6) to (13.5, 0) to (C);
        \end{tikzpicture}
    \end{center}
    \caption{Producteurs et consommateurs avec capacité du Buffer limitée sans arcs inhibiteurs}
    \label{pnetfinexo5}
\end{figure}

    
Considérons le réseau de petri de la figure~\ref{pnetpcexo5} représentant un système de production -
consommation, avec un producteur et un consommateur. On cherche à étendre le système étudié à $p$
producteurs et $p$ consommateurs. Pour ce faire, il suffit de modifier le nombre de jetons présents
dans les places Producteurs et Consommateurs de façon à en avoir $p$ sur la première et $c$ sur la
seconde comme représenté sur la figure~\ref{pnetpcsexo5}.

Si l'on veut limiter le nombre de ressources présentes simultanément dans le Buffer, il faut
représenter ce dernier non plus par une seule place, mais par deux distinctes : une pour accueillir
les ressources produites que nous appellerons toujours le Buffer, la seconde contenant $b$ jetons
Considérons le réseau de petri de la figure~\ref{pnetpcexo5} représentant un système de production -
consommation, avec un producteur et un consommateur. On cherche à étendre le système étudié à $p$
producteurs et $p$ consommateurs. Pour ce faire, il suffit de modifier le nombre de jetons présents
dans les places Producteurs et Consommateurs de façon à en avoir $p$ sur la première et $c$ sur la
seconde comme représenté sur la figure~\ref{pnetpcsexo5}.

Si l'on veut limiter le nombre de ressources présentes simultanément dans le Buffer, il faut
représenter ce dernier non plus par une seule place, mais par deux distinctes : une pour accueillir
les ressources produites que nous appellerons toujours le Buffer, la seconde contenant $b$ jetons
contrôlant l'envoi des ressources, que nous appellerons le Contrôleur, comme représenté
à la figure~\ref{pnetpcbexo5}.

Enfin, on cherche à ordonner la réception des ressources de manière à ce que ces dernières soient
récupérées dans le même ordre qu'elles ont été envoyées. Il existe deux possibilités équivalentes
l'une nécessitant l'introduction d'une nouvelle extension du modèle, à savoir les arcs inhibiteurs,
l'autre demandant un plus grand nombre de places. Nous décrirons ici, les deux possibilités.\\
Les arcs inhibiteurs sont des arcs permettant de modéliser le test à zéro, ce qui permet une plus
grande expressivité. En effet, dans le modèle de base, si une transition est tirable lorsqu'une
place est vide, elle l'est à prioris, aussi lorsqu'elle contient des jetons. Pour empêcher le tirage
d'une transition lorsqu'une place est non vide, on utilise alors ces arcs : 
\begin{center}
    \begin{tikzpicture}[every transition/.style={minimum width=2mm, minimum height=7mm}, >=o]
        \node[place, label=$p$] (p) {};
        \node[transition, label=$t$, right=of p] (t) {}
            edge[pre] (p);
    \end{tikzpicture}
\end{center}
Ici la transition $t$ est tirable tant que $p$ ne contient aucun jeton.

En utilisant cette extension, on obtient alors le réseau de la figure~\ref{pnetpcbexo5}. Le Buffer
est alors représenté par $b$ places et $b-1$ transitions. L'ordre de réception est le même que
l'ordre d'envoi, en effet, il est impossible d'avoir plus d'un jeton par place à l'intérieur du
buffer et il est impossible pour un producteur d'envoyer une ressource dans le Buffer tant que la
première place est non vide, or pour libérer la première place, il faut que la seconde soit vide et
pour vider la seconde, la troisième doit être libre de tous jetons et ainsi de suite. Donc pour
envoyer un jeton dans le buffer il faut faire avancer tous les jetons consécutifs précédents d'une
place et aucun jeton ne peut en dépasser un autre, l'ordre de réception est donc le même que celui
d'envoi. De plus, il y a au maximum un jeton par place dans le Buffer, donc de manière générale, il
y a au plus $b$ ressources dans ce dernier.

Comme annoncé précédemment, il est possible de ne pas avoir recours aux arcs inhibiteurs en
introduisant $b$ places supplémentaires. On découpe alors le buffer en $b$ \emph{chambres}, chacune
de ces chambres étant constituée de $2$ places. Une de ces deux places possède un jeton lorsqu'une
ressource se trouve dans la chambre\footnote{Elle joue alors le même rôle que la place
correspondante dans le réseau avec arcs inhibiteurs.}, l'autre possède un jeton lorsque la chambre
est vide de ressources\footnote{Cette place quant à elle joue le rôle de l'arc inhibiteur du réseau
précédent, en permettant le test à zéro.}. Les deux places de chacune des chambres sont donc en
exclusion mutuelles et permette de recréer le mécanisme mis en place dans le réseau précédent. La
figure~\ref{pnetfinexo5} est une réprésentation du réseau de Petri que nous venons de décrire.


\section{Exercice 6 - Producteurs et consommateurs bis}

\section{Exercice 7 - Semaphore}

\section{Exercice 8 - L'utilisation des réseaux de Petri pour le contrôle des trains}

\begin{figure}
    \begin{center}
        \begin{tikzpicture}[scale=0.45, >=latex, every place/.style={transform shape}, every
        transition/.style={transform shape, minimum width=2mm, minimum height=7mm}]
        
            \foreach \lab/\Etok/\Oatok/\Obtok in {0/1/0/0, 1/0/1/0, 2/1/0/0, 3/1/0/0, 4/1/0/0, 5/0/0/1,
            6/1/0/0}{
                \node[place, tokens=\Oatok, label=$O_{a\lab}$] at (\lab * 4    ,  4) (Oa\lab) {};
                \node[place, tokens=\Obtok, label=$O_{b\lab}$] at (\lab * 4    , -4) (Ob\lab) {};
                \node[place, tokens=\Etok,  label=$E_{\lab}$ ] at (\lab * 4 + 2,  0) (E\lab)  {};
            }

            \foreach \x/\y/\z in {0/1/2, 1/2/3, 2/3/4, 3/4/5, 4/5/6}{
                \node[transition] at (\x * 4 + 2,  4) (ta\x) {}
                    edge[pre]  (Oa\x)
                    edge[pre]  (E\y)
                    edge[pre, <->]  (E\z)
                    edge[post] (Oa\y)
                    edge[post] (E\x);
                \node[transition] at (\x * 4 + 2, -4) (tb\x) {}
                    edge[pre]  (Ob\x)
                    edge[pre]  (E\y)
                    edge[pre, <->]  (E\z)
                    edge[post] (Ob\y)
                    edge[post] (E\x);
            }

            \node[transition] at (22,  4) (ta5) {}
                edge[pre]  (Oa5)
                edge[pre]  (E6)
                edge[post] (E5)
                edge[post] (Oa6);
            \node[transition] at (22, -4) (tb5) {}
                edge[pre]  (Ob5)
                edge[pre]  (E6)
                edge[post] (E5)
                edge[post] (Ob6);
                
            \node[transition] at (26,  4) (ta6) {}
                edge[pre]  (Oa6)
                edge[post] (E6);
            \node[transition] at (26, -4) (tb6) {}
                edge[pre]  (Ob6)
                edge[post] (E6);

            \draw[pre, <->] (ta5) to (22, 6) to (-1, 6) to (-1, 2) to (E0);
            \draw[pre, <->] (tb5) to (22, -6) to (-1, -6) to (-1, -2) to (E0);

            \draw[post] (ta6) to (26, 6.5) to (-1.5, 6.5) to (-1.5, 4) to (Oa0);
            \draw[pre]  (ta6) to (27, 4) to (27, 7) to (-2, 7) to (-2, 3.5) to (E0);
            \draw[pre, <->]  (ta6) to (27, 4) to (27, 7) to (-2, 7) to (-2, 3.5) to (E1);
            \draw[post] (tb6) to (26,-6.5) to (-1.5,-6.5) to (-1.5,-4) to (Ob0);
            \draw[pre]  (tb6) to (27,-4) to (27,-7) to (-2,-7) to (-2,-3.5) to (E0);
            \draw[pre, <->]  (tb6) to (27,-4) to (27,-7) to (-2,-7) to (-2,-3.5) to (E1);
        \end{tikzpicture}
    \end{center}
    \caption{Représentation du contrôle de train par un réseau de Petri comptant trois places par
    secteur}
    \label{qu1exo8}
\end{figure}
            
\begin{figure}
    \begin{center}
        \begin{tikzpicture}[scale=0.45, >=latex, every place/.style={transform shape}, every
        transition/.style={transform shape, minimum width=2mm, minimum height=7mm}]
        
            \foreach \lab/\Etok/\Otok in {0/green/white, 1/white/red, 2/green/white, 3/green/white,
                4/green/white, 5/white/blue, 6/green/white}{
                \node[place, colored tokens=\Otok, label=$O_{\lab}$] at (\lab * 4    ,  4) (O\lab) {};
                \node[place, colored tokens=\Etok,  label=$E_{\lab}$ ] at (\lab * 4 + 2,  0) (E\lab)  {};
            }

            \foreach \x/\y/\z in {0/1/2, 1/2/3, 2/3/4, 3/4/5, 4/5/6}{
                \node[transition] at (\x * 4 + 2,  4) (t\x) {}
                    edge[pre]  (O\x)
                    edge[pre]  node[auto] {$f$} (E\y)
                    edge[pre, <->]  node[auto] {$f$} (E\z)
                    edge[post] (O\y)
                    edge[post] node[auto] {$f$} (E\x);
            }

            \node[transition] at (22,  4) (t5) {}
                edge[pre]  (O5)
                edge[pre]  (E6)
                edge[post] (E5)
                edge[post] (O6);
            \node[transition] at (26, 4) (t6) {}
                edge[pre]  (O6)
                edge[post] (E6);

            \draw[pre, <->] (t5) to (22, 6) to node[auto] {$f$} (-1, 6) to (-1, 2) to (E0);

            \draw[post] (t6) to (26,6.5) to (-1.5,6.5) to (-1.5,4) to (O0);
            \draw[pre, <->]  (t6) to (27,4) to (27,-1) to node[auto] {$f$} (6,-1) to (E1);
            \draw[pre]  (t6) to (28,4) to (28,-2) to node[auto] {$f$} (2,-2) to (E0);
        \end{tikzpicture}
        \begin{displaymath}
            f:\left \lbrace \begin{array}{rcl}
                f(\textcolor{red}{a}) & = &  \textcolor{green}{e} \\
                f(\textcolor{blue}{b}) & = & \textcolor{green}{e} \\
            \end{array}
            \right .
        \end{displaymath}
    \end{center}
    \caption{Représentation du contrôle de train par un réseau de Petri comptant deux places par
    secteur}
    \label{qu1bexo8}
\end{figure}
            
\begin{figure}
    \begin{center}
        \begin{tikzpicture}[>=latex]
            \foreach \x/\tok [evaluate=\x as \y using (\x * 360)/7 + 90] in {0/green, 1/red,
            2/green, 3/green, 4/green, 5/blue, 6/green}{
                \node[place, colored tokens=\tok, label=$p_{\x}$] at (\y:5cm) (p\x) {};
            }
            \foreach \x/\z/\v [evaluate=\x as \y using ((\x + 0.5) * 360)/7 + 90] in {0/2/1,
                1/3/2, 2/4/3, 3/5/4, 4/6/5, 5/0/6, 6/1/0}{
                \node[transition, label=$t_{\x}$] at (\y:5cm) (t\x) {}
                    edge[pre, bend left] node[auto] {$g$} (p\z)
                    edge[pre, bend left] node[auto] {$f$} (p\x)
                    edge[pre, bend left] node[auto] {$g$} (p\v)
                    edge[post, bend right=90] node[auto] {$g$} (p\z)
                    edge[post, bend right] node[auto] {$f$} (p\v)
                    edge[post, bend right] node[auto] {$g$}(p\x);
            }
        \end{tikzpicture}
        \begin{displaymath}
            f: \left \lbrace \begin{array}{rcl}
                f(\textcolor{red}{a})  & = & \textcolor{red}{a} \\
                f(\textcolor{blue}{b}) & = & \textcolor{blue}{b} 
            \end{array} \right . \qquad \qquad
            g: \left \lbrace \begin{array}{rcl}
                g(\textcolor{red}{a})  & = & \textcolor{green}{e} \\
                g(\textcolor{blue}{b}) & = & \textcolor{green}{e} 
            \end{array} \right .
        \end{displaymath}
    \end{center}
    \caption{Représentation du contrôle de train par un réseau de Petri comptant une place par
    secteur}
    \label{qu2exo8}
\end{figure}

\begin{figure}
    \begin{center}
        \begin{tikzpicture}[>=latex]
            \foreach \x/\tok [evaluate=\x as \y using (\x * 360)/7 + 90] in {0/white, 1/red,
            2/white, 3/white, 4/white, 5/blue, 6/white}{
                \node[place, colored tokens=\tok, label=$p_{\x}$] at (\y:5cm) (p\x) {};
            }
            \foreach \x/\z/\v [evaluate=\x as \y using ((\x + 0.5) * 360)/7 + 90] in {0/2/1,
                1/3/2, 2/4/3, 3/5/4, 4/6/5, 5/0/6, 6/1/0}{
                \node[transition, label=$t_{\x}$] at (\y:5cm) (t\x) {}
                    edge[pre, bend left, o-] (p\z)
                    edge[pre] (p\x)
                    edge[pre, bend right, o-] (p\v)
                    edge[post, bend left] (p\v);
            }
        \end{tikzpicture}
    \end{center}
    \caption{Représentation du contrôle de train par un réseau de Petri comptant une place par
    secteur avec arcs inhibiteurs}
    \label{qu2bexo8}
\end{figure}

\begin{figure}
    \begin{center}
        \begin{tikzpicture}[>=latex, every transition/.style={minimum width=2mm, minimum height=7mm}]
            \node[place, tokens=0, label=$E$, structured tokens={$e_0$, $e_2$, $e_3$, $e_4$, $e_6$}] at (0,0)  (E) {};
            \node[place, tokens=0, label=$O$, structured tokens={$a_1$, $b_5$}] at (10, 0) (O) {};
            \node[transition] at (5, 0) {Move to next sector}
                edge[pre, bend right] (O)
                edge[pre, bend right] node[auto] {$g$} (E)
                edge[post, bend left] node[auto] {$h$} (E)
                edge[post, bend left] node[auto] {$f$} (O); 
        \end{tikzpicture}
        \begin{displaymath}
            f: \left \lbrace
            \begin{array}{rcl}
                f(a_i) & = & a_{(i + 1)} \\
                f(b_i) & = & b_{(i + 1)} 
            \end{array}
            \right . \ 
            g: \left \lbrace
            \begin{array}{rcl}
                g(a_i) & = & e_{(i + 1)} + e_{(i+2)}\\
                g(b_i) & = & e_{(i + 1)} + e_{(i+2)}\\
            \end{array}
            \right . \ 
            h: \left \lbrace
            \begin{array}{rcl}
                h(a_i) & = & e_{i} + e_{(i+2)}\\
                h(b_i) & = & e_{i} + e_{(i+2)}\\
            \end{array}
            \right . 
        \end{displaymath}
    \end{center}
    \caption{Représentation du contrôle de train par un réseau de Petri comprenant deux places et
    une transition}
    \label{qu3exo8}
\end{figure}

\begin{figure}
    \begin{center}
        \begin{tikzpicture}[>=latex]
            \node[place, structured tokens={15}] at (0, 0) (a) {};
            \node[transition, minimum width=2mm, minimum height=7mm] at (3, 0) {}
                edge[pre, bend left] node[below] {$f$} (a)
                edge[post, bend right] node[above] {$g$} (a);
        \end{tikzpicture}
    \end{center}
    \begin{displaymath}
        f : \left \lbrace
        \begin{array}{rcl}
            D_f & \rightarrow & V_f \\
            f((1, x, y)) & = & (1, x, y)\\
            f((2, x, y)) & = & (1, x, y)
        \end{array}
        \right . \qquad
        g : \left \lbrace
        \begin{array}{rcl}
            D_f & \rightarrow & V_f \\
            g((1, x, y)) & = & (1, x + 1, y) \\
            g((2, x, y)) & = & (1, x, y + 1)
        \end{array} 
        \right .
    \end{displaymath}
    \caption{Représentation du contrôle de train par un réseau de Petri comprenant une place et une
    transition}
    \label{bonusexo8}
\end{figure}

Le but de cet exercice est de modéliser un système de contrôle de train par un réseau de Petri. Pour
des raisons de sécurité, pour qu'un train puisse passer au secteur suivant, ce dernier doit être
vide mais son succeseur doit l'être aussi. Une réprésentation de ce système peut être effectuée en
considérant trois places par secteur, la première place $E$ indique si le secteur est vide et la
seconde (respectivement la troisième) indique si le train $a$ (respectivement $b$) se trouve sur le
secteur. Il doit donc y avoir exclusion mutuelle entre chacune des places d'un même secteur. La
figure~\ref{qu1exo8} est une représentation possible de ce système.

On cherche à présent à réduire le réseau étudié, pour ce faire, nous introduisons deux ensembles de
couleurs, un pour les places $O_i$ : $O = \{a, b\}$ et un pour les places $E_i$ : $E = \{e\}$. Un
jeton de couleur $a$ sur une place $O$ indique la présence du train $a$ dans le secteur associé, de
la même manière un jeton de couleur $b$ représente le train $b$ et enfin un jeton de couleur $e$ sur
une place $E$ indique que le secteur associé est libre de tout train. La figure~\ref{qu1bexo8}
illustre une représentation de ce système.

On s'aperçoit rapidement qu'il est possible de fusionner les places $O$ et les places $E$ sans
ajouter de couleurs, puisque ces places sont en exclusion mutuelles. On peut alors représenter un
secteur par une seule place la couleur du jeton indiquant l'absence de train, la présence du train
$a$ ou la présence du train $b$, la figure~\ref{qu2exo8} en est la preuve. Il est même possible de
réduire l'ensemble des couleurs utilisées à $\{a, b\}$ par l'utilisation d'arcs inhibiteurs (cf.
figure~\ref{qu2bexo8}).

Si l'on veut pousser la réduction un peu plus loin, on peut alors fusionner toutes les cellles $E$
de la figure~\ref{qu2exo8} en une seule place et faire de même pour les places $O$. Les cinq jetons
se trouvant dans la place $E$ ont alors une couleur appartenant à l'ensemble suivant : $C_E = \{e_0,
e_1, e_2, e_3, e_4, e_5, e_6\}$, tandis que le jeton $a$ (respectivement $b$) représentant la
position du train $a$ (respectivement $b$) présente une couleur appartenant à l'ensemble $C_a =
\{a_0, a_1, a_2, a_3, a_4, a_5, a_6\}$ (respectivement $C_b = \{b_0, b_1, b_2, b_3, b_4, b_5,
b_6\}$). Une représentation de ce système nous est donné par la figure~\ref{qu3exo8}.

Nous clôturerons cet exercice en poussant la réduction du réseau à son paroxysme en le réduisant à
un réseau de Petri à une place, une transition et un jeton. Définissons pour ce faire un ensemble de
couleurs que nous appellerons $E = \{(w, x, y) / w \in \{1, 2\},\ x,y \in \{0, 1, \dots 6\}\}$. $x$
représente alors la position du train $a$, $y$ celle du train $b$ et $w$ est utilisé pour générer un
ensemble de couleurs supplémentaire pour la fonction $f$ de franchissement de la transition. En
effet, pour pouvoir différencier quel train passe au secteur suivant lors du franchissement de la
transition il faut doubler le nombre de couleurs, ainsi $f((1, x, y))$ fait avancer le train $a$ au
secteur suivant et $f((2, x, y))$ fait avancer le train $b$. Si l'on appelle $E_1$ (respectivement
$E_2$) l'ensemble $\{(1, x, y) / x,y \in \{0, 1, \dots, 6\}\}$ (respectivement $\{(2, x, y) / x,y
\in \{0, 1, \dots, 6\}\}$), il est alors important de noter que le jeton ne prendra comme couleur
que des couleurs appartenant à $E_1$, $E_2$ ne servant que pour étoffer le dommaine de définition de
$f$. Définissons quelques ensembles remarquables :
\begin{itemize}
    \item $E_{w1} = E_{11} \cup E_{21} = \{(1, x, y) / x = y \} \cup \{(2, x, y) / x = y \}$
        est l'ensemble des couleurs représentant des positions de trains telles que le train $a$ et
        le train $b$ se trouvent sur le même tronçon, qui sont des positions interdites et qui
        doivent donc être retirées du domaine de définition et de valeurs de $f$
    \item $E_{w2} = E_{12} \cup E_{22} = \{(1, x, y) / x = y - 1\} \cup \{(2, x, y) / x = y -
        1 \}$ est l'ensemble des couleurs représentant des positions de trains telles que le train
        $a$ se trouve sur le tronçon précédent celui du train $b$, qui sont elles aussi une
        interdites puisque le train $a$ ne peut avancer si les deux secteurs suivants ne sont pas
        vides. $E_{w2}$ est donc à retirer du domaine de définition et de valeurs de $f$
    \item $E_{w3} = E_{13} \cup E_{23} = \{(1, x, y) / x = y + 1\} \cup \{(2, x, y) / x = y +
        1 \}$ est l'ensemble des couleurs représentant des positions de trains telles que le train
        $b$ se trouve sur le tronçon précédent celui du train $a$. Ces positions étant interdites,
        il faut enlever $E_{w3}$ du domaine de définition et de valeurs de $f$
    \item $E_{w4} = E_{14} \cup E_{24} = \{(1, x, y) / x = y - 2\} \cup \{(2, x, y) / x = y - 2 \}$
        est l'ensemble des couleurs représentant des positions de trains telles que le train $a$ se
        trouve sur l'antépénultième secteur avant celui de $b$. Il s'agit d'une position autorisée
        mais pour laquelle le déplacement du train $a$ est interdit, il faut donc retirer $E_{14}$
        du domaine de définition de la fonction $f$ de consommation.
    \item $E_{w5} = E_{15} \cup E_{25} = \{(1, x, y) / x = y + 2\} \cup \{(2, x, y) / x = y + 2 \}$
        est l'ensemble des couleurs représentant des positions de trains telles que le train $b$ se
        trouve sur l'antépénultième secteur avant celui de $a$. Il s'agit d'une position autorisée
        mais pour laquelle le déplacement du train $b$ est interdit, il faut donc retirer $E_{25}$
        du domaine de définition de la fonction $f$ de consommation.
\end{itemize}

On peut alors définir le domaine de définition de f $D_f = \overline{E_{w1} \cup E_{w2} \cup E_{w3}
\cup E_{14} \cup E_{25}}$ ainsi que le domaine des valeurs de g $V_g = \overline{E_{w1} \cup E_{w2}
\cup E_{w3}} \cap E_1$. Le réseau obtenu est représenté sur la figure~\ref{bonusexo8}.


\section{Exercice 9 - L'utilisation des réseaux de Petri pour le contrôle d'un système téléphonique}

\begin{figure}
    \begin{center}
        \begin{tikzpicture}[every transition/.style={minimum width=4mm, minimum height=10mm},
        >=latex, every place/.style={minimum width=1cm}, scale=0.7, every node/.style={transform shape}]
            \node[place, label=Inactif] (inactif) at (9, 20) {$u$};
            \node[transition, label=Decroche] (decr) at (9, 18) {};
            \node[transition, label=Raccroche] (r1) at (12, 18) {};
            \node[place, label=Continu] (cont) at (9, 16) {$u$};
            %\node[transition, label=Raccroche] (r2) at (6, 16) {};
            \node[transition, label=Compose] (comp) at (9, 14) {};
            \node[transition, label=Raccroche] (r3) at (3, 12) {};
            \node[transition, label=Raccroche] (r4) at (6, 12) {};
            \node[place, label=Sans Tonalite] (notone) at (9, 12) {$u$};
            \node[place, label=Requete] (req) at (12, 12) {$u \times u$};
            \node[transition, label=Raccroche] (r5) at (15, 12) {};
            \node[transition, label=Est inactif] (ein) at (6, 10) {};
            \node[transition, label=Est actif] (ea) at (12, 10) {};
            \node[place, label=Court] (court) at (15, 10) {$u$};
            \node[place, label=Appel] (appel) at (3, 8) {$u\times u$};
            \node[place, label=Long] (long) at (6, 8) {$u$};
            \node[place, label=Sonne] (sonne) at (9, 8) {$u$};
            \node[transition, label=Réponds] (rep) at (6, 6) {};
            \node[place, label=Connexion] (con1) at (3, 4) {$u$};
            \node[place, label=Connecté] (con2) at (9, 4) {$u\times u$};
            \node[transition, label=$x$ Raccroche] (rx) at (3, 2) {};
            \node[transition, label=$y$ Raccroche] (ry) at (9, 2) {};
            \node[transition, label=Raccroche] (r6) at (12, 2) {};
            \node[place, label=$y$ Attente] (ay) at (3, 0) {$u$};
            \node[place, label=$x$ Attente] (ax) at (9, 0) {$u$};
            \node[transition, label=Raccroche] (r7) at (12, 0) {};

            \draw (inactif) edge[post] node[auto] {$x$} (decr)
                edge[post] node[auto] {$y$} (ein)
                edge[pre, bend left] node[auto] {$x$} (r1)
                %edge[pre, in=70, out=230] node[auto] {$x$} (r2)
                edge[pre, in=90, out=180] node[auto] {$(x,y)$} (r3)
                edge[pre, bend right] node[auto] {$x$} (r4)
                edge[pre, in=90, out=0] node[auto] {$x$} (r5)
                edge[pre, in=20, out=0] node[auto] {$x$} (r6)
                edge[pre, in=0, out=0] node[auto] {$x$} (r7);
            \draw (cont) edge[pre] node[auto] {$x$} (decr)
                edge[post] node[auto] {$x$} (r1)
                edge[post] node[auto] {$x$} (comp);
            \draw (notone) edge[pre] node[auto] {$x$} (comp)
                edge[post] node[auto] {$x$} (r4)
                edge[post] node[auto] {$x$} (ein)
                edge[post] node[auto] {$x$} (ea);
            \draw (req) edge[pre] node[above] {$(x,y)$} (comp)
                edge[post, bend right] node[above left] {$(x,y)$} (r4)
                edge[post] node[auto] {$(x,y)$} (r5);
            \draw (court) edge[pre] node[above] {$x$} (ea)
                edge[post] node[auto] {$x$} (r5);
            \draw (appel) edge[pre] node[auto] {$(x,y)$} (ein)
                edge[post] node[auto] {$(x,y)$} (r3)
                edge[post] node[auto] {$(x,y)$} (rep);
            \draw (long) edge[pre] node[auto] {$x$} (ein)
                edge[post] node[auto] {$x$} (r3)
                edge[post] node[auto] {$x$} (rep);
            \draw (sonne) edge[pre] node[auto] {$y$} (ein)
                edge[post, in=350, out=130] node[above right] {$y$} (r3)
                edge[post] node[auto] {$y$} (rep);
            \draw (con1) edge[pre] node[above] {$x$} (rep)
                edge[post] node[auto] {$x$} (rx)
                edge[post] node[auto] {$y$} (ry);
            \draw (con2) edge[pre] node[auto] {$(x,y)$} (rep)
                edge[post] node[auto] {$(x,y)$} (rx)
                edge[post] node[auto] {$(x,y)$} (ry);
            \draw (ax) edge[pre] node[auto] {$x$} (ry)
                edge[post] node[auto] {$x$} (r7);
            \draw (ay) edge[pre] node[auto] {$x$} (rx)
                edge[post] node[auto] {$x$} (r6);
        \end{tikzpicture}
    \end{center}
\end{figure}


\end{document}
