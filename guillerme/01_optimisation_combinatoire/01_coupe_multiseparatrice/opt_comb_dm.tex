\documentclass[a4paper, 11pt]{article}

\usepackage[utf8x]{inputenc}
\usepackage[T1]{fontenc}
\usepackage[french]{babel}
%\usepackage[francais]{babel}
\usepackage{graphicx}
\usepackage{moreverb}
\usepackage{subfig}
\usepackage{pdfpages}
\usepackage{eurosym}
\usepackage{fullpage}
\usepackage[fleqn]{amsmath}
\usepackage{bbm}
\usepackage{amssymb}
\usepackage{tikz}
\usepackage{algorithm}
\usepackage{algorithmic}
\usepackage{pgf}

\usetikzlibrary{arrows,automata}

\newtheorem{lemma}{Lemme}[subsection]
\newtheorem{thrm}{Th\'{e}or\`{e}me}[section]
\newtheorem{corol}{Corollaire}[subsection]
\begin{document}

\section*{Exercice d'approximation polynomiale}

\subsection*{Recherche d'une coupe s\'{e}parant $C_j$ des autres terminaux}

Soient un graphe non orient\'{e} $G = (V, E)$, une fonction de poids positive $w$ sur les ar\^{e}tes et un
ensemble $S = \{s_1, s_2, \dots, s_k\}$ de sommets terminaux de $G$. Consid\'{e}rons un sommet terminal
de $G$ appel\'{e} $s_j$ et $G' = (V', E')$ le graphe obtenu par fusion des sommets $s_i \in \{S
\backslash s_j\}$, le sommet r\'{e}sultant sera appel\'{e} $s_r$, on remarque alors que $V' = V
\backslash \{ (uv) \mbox{ tq } (u, v) \in \{S \backslash s_j\}$. Enfin consid\'{e}rons la fonction
de poids $w'$ d\'{e}finie comme suit :$$
\left \{ \begin{array}{rcll}
        V' & \longrightarrow & \mathbb{R} & \\
        w'(v) & = & w(v) & \quad \forall v \in V' \\
\end{array} \right .
$$

\begin{lemma}
Rechercher une coupe $C_j$ de poids minimum dans $G$ revient alors \`{a} chercher une $(s_r, s_j)-$coupe minimum
dans $G'$.
\end{lemma}

\textbf{\underline{Preuve :}} \\

Supposons qu'il existe une coupe $C_j$ de poids minimum s\'{e}parant $s_j$ des autres terminaux dans le
graphe $G$ dont le poids est inf\'{e}rieur \`{a} celui de la coupe minimum $C_{G'}$ trouv\'{e}e dans le $G'$.
Par construction, toute ar\^{e}te appartenant \`{a} $V'$ est pond\'{e}r\'{e}e de la m\^{e}me mani\`{e}re dans $G$ et dans
$G'$, on en d\'{e}duit que pour que $C_j$ soit diff\'{e}rente de $C_{G'}$, alors $C_j \backslash \{V' \cap C_j \}
\not = \emptyset$. Ceci est impossible puisque $C_j$ est une coupe s\'{e}parant $s_j$ de tous les autres
terminaux. Donc $C_{G'}$ est une coupe de poids minimum s\'{e}parant $s_j$ des autres terminaux.

La recherche d'une $(s,t)-$coupe dans un graphe s'apparente \`{a} une recherche de flot maximum dans le
m\^{e}me graphe pour laquelle il existe des algorithmes polynomiaux\footnote{On peut par exemple citer
les algorithmes de pr\'{e}flots, Dinic ou Edmonds-Karp.}. La transformation de $G$ vers $G'$ se faisant
en temps lin\'{e}aire, l'algorithme est un algorithme s'ex\'{e}cutant en temps polynomial.

\subsection*{Calcul des bornes}

\subsubsection*{Somme des poids des ar\^{e}tes}

Consid\'{e}rons une ar\^{e}te $(uv)$ telle que $u \in C_i^*$ et $v \in C_j^*$. Lorsque l'on somme le poids
de l'ensemble des coupes extraites de $C^*$, le poids de l'ar\^{e}te est comptabilis\'{e} dans $w(C_i^*)$ et
dans $w(C_j^*)$, donc le poids de chaque ar\^{e}te est comptabilis\'{e} deux fois. On en d\'{e}duit alors que :
$$
\sum_{i=1}^k w(C_i^*) = 2 w(C^*)
$$

\subsubsection*{Minorant de $w(C_m)$}

Par d\'{e}finition, $C_m$ est la coupe de poids maximum parmis les $j$ coupes s\'{e}paratrices de poids
minimum. On a donc : $$
\begin{array}{rrcll}
    & w(C_i) & \leq & w(C_m) & \quad \forall i \in \{1, \dots, k\} \\
    \Rightarrow & \sum_{i = 1}^k w(C_i) & \leq & k * w(C_m) & \\
    \Rightarrow & w(C_m) & \geq & \frac{1}{k} \sum_{i = 1} w(C_i) & \\
\end{array}
$$

\subsubsection*{Une $2( 1 - \frac{1}{k})-$approximation}

On sait que : $$
\begin{array}{rrcll}
    & w(C) + w(C_m) & \leq & \sum_{i=1}^k w(C_i) & \\
    \Rightarrow & w(C) & \leq & \sum_{i=1}^k w(C_i) - w(C_m) & \\
    \Rightarrow & w(C) & \leq & \sum_{i=1}^k w(C_i) - \frac{1}{k} \sum_{i=1}^k w(C_i) & \\
    \Rightarrow & w(C) & \leq & (1 - \frac{1}{k}) \sum_{i=1}^k w(C_i) & \\
\end{array} $$

Or par d\'{e}finition, $C_i$ est une coupe minimale s\'{e}parant $s_i$ des autres sommets terminaux, on a
donc : $w(C_i) \leq w(C_i^*)$. Ce qui nous donne : $$
w(C) \leq (1 - \frac{1}{k}) \sum_{i=1}^k w(C_i^*) 
$$

Or nous avons vu que $\sum_{i=1}^k w(C_i^*) = 2w(C^*)$, on a alors : $$
w(C) \leq 2(1 - \frac{1}{k}) w(C^*) $$

\subsubsection*{Atteindre la borne}

Consid\'{e}rons le graphe suivant :

\begin{figure}[h!]
    \begin{center}
        \begin{tikzpicture}
            \tikzset{noeud/.style={circle, draw=black, text centered, minimum size=26pt, inner sep=0pt}, fleche/.style={thick}};

            \node[noeud] (s4) at (0, 0) {$s_4$};
            \node[noeud] (s1) at (0, 4) {$s_1$};
            \node[noeud] (s3) at (4, 0) {$s_3$};
            \node[noeud] (s2) at (4, 4) {$s_4$};
            \node[noeud] (a) at (1, 3) {$a$};
            \node[noeud] (b) at (3, 3) {$b$};
            \node[noeud] (c) at (3, 1) {$c$};
            \node[noeud] (d) at (1, 1) {$d$};

            \draw[fleche] (a) --node[above] {$1$} (b);
            \draw[fleche] (b) --node[right] {$1$} (c);
            \draw[fleche] (c) --node[below] {$1$} (d);
            \draw[fleche] (d) --node[left] {$1$} (a);
            \draw[fleche] (a) --node[above right] {$2 - \epsilon$} (s1);
            \draw[fleche] (b) --node[above left] {$2 - \epsilon$} (s2);
            \draw[fleche] (c) --node[below left] {$2 - \epsilon$} (s3);
            \draw[fleche] (d) --node[below right] {$2 - \epsilon$} (s4);
        \end{tikzpicture}
    \end{center}
\end{figure}

En utilisant l'algorithme d'approximation donn\'{e}, il faut calculer pour chaque terminal la coupe
minimale s\'{e}parant le terminal de tous les autres. Les coupes retourn\'{e}es sont donc les suivantes : $$
C_1 = \{(s_1a)\}, C_2 = \{(s_2b)\}, C_3 = \{(s_3c)\} \mbox{ et } C_4 = \{(s_4d)\} $$
Les coupes \'{e}tant \'{e}gales, on choisit de retirer $C_4$, on obtient alors : $$
C = \{(s_1a), (s_2b), (s_3c)\} $$
On a donc $w(C) = 6 - 3 \epsilon$.

Il est facile de voir que la coupe multi s\'{e}paratrice minimale pour le graphe donn\'{e} est donn\'{e}e par :
$$ C^* = \{(ab), (bc), (cd), (ad)\} $$
Ce qui nous donne $w(C^*) = 4$.

Or : $6 = 8 \times (1 - \frac{1}{k})$, donc lorsque $\epsilon$ tend vers $0$, l'instance donn\'{e}e tend
vers la borne pr\'{e}c\'{e}demment calcul\'{e}e.

\end{document}
