\begin{figure}
    \begin{center}
        \begin{tikzpicture}[scale=0.45, >=latex, every place/.style={transform shape}, every
        transition/.style={transform shape, minimum width=2mm, minimum height=7mm}]
        
            \foreach \lab/\Etok/\Oatok/\Obtok in {0/1/0/0, 1/0/1/0, 2/1/0/0, 3/1/0/0, 4/1/0/0, 5/0/0/1,
            6/1/0/0}{
                \node[place, tokens=\Oatok, label=$O_{a\lab}$] at (\lab * 4    ,  4) (Oa\lab) {};
                \node[place, tokens=\Obtok, label=$O_{b\lab}$] at (\lab * 4    , -4) (Ob\lab) {};
                \node[place, tokens=\Etok,  label=$E_{\lab}$ ] at (\lab * 4 + 2,  0) (E\lab)  {};
            }

            \foreach \x/\y/\z in {0/1/2, 1/2/3, 2/3/4, 3/4/5, 4/5/6}{
                \node[transition] at (\x * 4 + 2,  4) (ta\x) {}
                    edge[pre]  (Oa\x)
                    edge[pre]  (E\y)
                    edge[pre, <->]  (E\z)
                    edge[post] (Oa\y)
                    edge[post] (E\x);
                \node[transition] at (\x * 4 + 2, -4) (tb\x) {}
                    edge[pre]  (Ob\x)
                    edge[pre]  (E\y)
                    edge[pre, <->]  (E\z)
                    edge[post] (Ob\y)
                    edge[post] (E\x);
            }

            \node[transition] at (22,  4) (ta5) {}
                edge[pre]  (Oa5)
                edge[pre]  (E6)
                edge[post] (E5)
                edge[post] (Oa6);
            \node[transition] at (22, -4) (tb5) {}
                edge[pre]  (Ob5)
                edge[pre]  (E6)
                edge[post] (E5)
                edge[post] (Ob6);
                
            \node[transition] at (26,  4) (ta6) {}
                edge[pre]  (Oa6)
                edge[post] (E6);
            \node[transition] at (26, -4) (tb6) {}
                edge[pre]  (Ob6)
                edge[post] (E6);

            \draw[pre, <->] (ta5) to (22, 6) to (-1, 6) to (-1, 2) to (E0);
            \draw[pre, <->] (tb5) to (22, -6) to (-1, -6) to (-1, -2) to (E0);

            \draw[post] (ta6) to (26, 6.5) to (-1.5, 6.5) to (-1.5, 4) to (Oa0);
            \draw[pre]  (ta6) to (27, 4) to (27, 7) to (-2, 7) to (-2, 3.5) to (E0);
            \draw[pre, <->]  (ta6) to (27, 4) to (27, 7) to (-2, 7) to (-2, 3.5) to (E1);
            \draw[post] (tb6) to (26,-6.5) to (-1.5,-6.5) to (-1.5,-4) to (Ob0);
            \draw[pre]  (tb6) to (27,-4) to (27,-7) to (-2,-7) to (-2,-3.5) to (E0);
            \draw[pre, <->]  (tb6) to (27,-4) to (27,-7) to (-2,-7) to (-2,-3.5) to (E1);
        \end{tikzpicture}
    \end{center}
    \caption{Représentation du contrôle de trains par un réseau de Petri comptant trois places par
    secteur}
    \label{qu1exo8}
\end{figure}
            
\begin{figure}
    \begin{center}
        \begin{tikzpicture}[scale=0.45, >=latex, every place/.style={transform shape}, every
        transition/.style={transform shape, minimum width=2mm, minimum height=7mm}]
        
            \foreach \lab/\Etok/\Otok in {0/green/white, 1/white/red, 2/green/white, 3/green/white,
                4/green/white, 5/white/blue, 6/green/white}{
                \node[place, colored tokens=\Otok, label=$O_{\lab}$] at (\lab * 4    ,  4) (O\lab) {};
                \node[place, colored tokens=\Etok,  label=$E_{\lab}$ ] at (\lab * 4 + 2,  0) (E\lab)  {};
            }

            \foreach \x/\y/\z in {0/1/2, 1/2/3, 2/3/4, 3/4/5, 4/5/6}{
                \node[transition] at (\x * 4 + 2,  4) (t\x) {}
                    edge[pre]  (O\x)
                    edge[pre]  node[auto] {$f$} (E\y)
                    edge[pre, <->]  node[auto] {$f$} (E\z)
                    edge[post] (O\y)
                    edge[post] node[auto] {$f$} (E\x);
            }

            \node[transition] at (22,  4) (t5) {}
                edge[pre]  (O5)
                edge[pre]  (E6)
                edge[post] (E5)
                edge[post] (O6);
            \node[transition] at (26, 4) (t6) {}
                edge[pre]  (O6)
                edge[post] (E6);

            \draw[pre, <->] (t5) to (22, 6) to node[auto] {$f$} (-1, 6) to (-1, 2) to (E0);

            \draw[post] (t6) to (26,6.5) to (-1.5,6.5) to (-1.5,4) to (O0);
            \draw[pre, <->]  (t6) to (27,4) to (27,-1) to node[auto] {$f$} (6,-1) to (E1);
            \draw[pre]  (t6) to (28,4) to (28,-2) to node[auto] {$f$} (2,-2) to (E0);
        \end{tikzpicture}
        \begin{displaymath}
            f:\left \lbrace \begin{array}{rcl}
                f(\textcolor{red}{a}) & = &  \textcolor{green}{e} \\
                f(\textcolor{blue}{b}) & = & \textcolor{green}{e} \\
            \end{array}
            \right .
        \end{displaymath}
    \end{center}
    \caption{Représentation du contrôle de train par un réseau de Petri comptant deux places par
    secteur}
    \label{qu1bexo8}
\end{figure}
            
\begin{figure}
    \begin{center}
        \begin{tikzpicture}[>=latex]
            \foreach \x/\tok [evaluate=\x as \y using (\x * 360)/7 + 90] in {0/green, 1/red,
            2/green, 3/green, 4/green, 5/blue, 6/green}{
                \node[place, colored tokens=\tok, label=$p_{\x}$] at (\y:5cm) (p\x) {};
            }
            \foreach \x/\z/\v [evaluate=\x as \y using ((\x + 0.5) * 360)/7 + 90] in {0/2/1,
                1/3/2, 2/4/3, 3/5/4, 4/6/5, 5/0/6, 6/1/0}{
                \node[transition, label=$t_{\x}$] at (\y:5cm) (t\x) {}
                    edge[pre, bend left] node[auto] {$g$} (p\z)
                    edge[pre, bend left] node[auto] {$f$} (p\x)
                    edge[pre, bend left] node[auto] {$g$} (p\v)
                    edge[post, bend right=90] node[auto] {$g$} (p\z)
                    edge[post, bend right] node[auto] {$f$} (p\v)
                    edge[post, bend right] node[auto] {$g$}(p\x);
            }
        \end{tikzpicture}
        \begin{displaymath}
            f: \left \lbrace \begin{array}{rcl}
                f(\textcolor{red}{a})  & = & \textcolor{red}{a} \\
                f(\textcolor{blue}{b}) & = & \textcolor{blue}{b} 
            \end{array} \right . \qquad \qquad
            g: \left \lbrace \begin{array}{rcl}
                g(\textcolor{red}{a})  & = & \textcolor{green}{e} \\
                g(\textcolor{blue}{b}) & = & \textcolor{green}{e} 
            \end{array} \right .
        \end{displaymath}
    \end{center}
    \caption{Représentation du contrôle de train par un réseau de Petri comptant une place par
    secteur}
    \label{qu2exo8}
\end{figure}

\begin{figure}
    \begin{center}
        \begin{tikzpicture}[>=latex]
            \foreach \x/\tok [evaluate=\x as \y using (\x * 360)/7 + 90] in {0/white, 1/red,
            2/white, 3/white, 4/white, 5/blue, 6/white}{
                \node[place, colored tokens=\tok, label=$p_{\x}$] at (\y:5cm) (p\x) {};
            }
            \foreach \x/\z/\v [evaluate=\x as \y using ((\x + 0.5) * 360)/7 + 90] in {0/2/1,
                1/3/2, 2/4/3, 3/5/4, 4/6/5, 5/0/6, 6/1/0}{
                \node[transition, label=$t_{\x}$] at (\y:5cm) (t\x) {}
                    edge[pre, bend left, o-] (p\z)
                    edge[pre] (p\x)
                    edge[pre, bend right, o-] (p\v)
                    edge[post, bend left] (p\v);
            }
        \end{tikzpicture}
    \end{center}
    \caption{Représentation du contrôle de train par un réseau de Petri comptant une place par
    secteur avec arcs inhibiteurs}
    \label{qu2bexo8}
\end{figure}

\begin{figure}
    \begin{center}
        \begin{tikzpicture}[>=latex, every transition/.style={minimum width=2mm, minimum height=7mm}]
            \node[place, tokens=0, label=$E$, structured tokens={$e_0$, $e_2$, $e_3$, $e_4$, $e_6$}] at (0,0)  (E) {};
            \node[place, tokens=0, label=$O$, structured tokens={$a_1$, $b_5$}] at (10, 0) (O) {};
            \node[transition] at (5, 0) {Move to next sector}
                edge[pre, bend right] (O)
                edge[pre, bend right] node[auto] {$g$} (E)
                edge[post, bend left] node[auto] {$h$} (E)
                edge[post, bend left] node[auto] {$f$} (O); 
        \end{tikzpicture}
        \begin{displaymath}
            f: \left \lbrace
            \begin{array}{rcl}
                f(a_i) & = & a_{(i + 1)} \\
                f(b_i) & = & b_{(i + 1)} 
            \end{array}
            \right . \ 
            g: \left \lbrace
            \begin{array}{rcl}
                g(a_i) & = & e_{(i + 1)} + e_{(i+2)}\\
                g(b_i) & = & e_{(i + 1)} + e_{(i+2)}\\
            \end{array}
            \right . \ 
            h: \left \lbrace
            \begin{array}{rcl}
                h(a_i) & = & e_{i} + e_{(i+2)}\\
                h(b_i) & = & e_{i} + e_{(i+2)}\\
            \end{array}
            \right . 
        \end{displaymath}
    \end{center}
    \caption{Représentation du contrôle de train par un réseau de Petri comprenant deux places et
    une transition}
    \label{qu3exo8}
\end{figure}

\begin{figure}
    \begin{center}
        \begin{tikzpicture}[>=latex]
            \node[place, structured tokens={15}] at (0, 0) (a) {};
            \node[transition, minimum width=2mm, minimum height=7mm] at (3, 0) {}
                edge[pre, bend left] node[below] {$f$} (a)
                edge[post, bend right] node[above] {$g$} (a);
        \end{tikzpicture}
    \end{center}
    \begin{displaymath}
        f : \left \lbrace
        \begin{array}{rcl}
            D_f & \rightarrow & V_f \\
            f((1, x, y)) & = & (1, x, y)\\
            f((2, x, y)) & = & (1, x, y)
        \end{array}
        \right . \qquad
        g : \left \lbrace
        \begin{array}{rcl}
            D_f & \rightarrow & V_f \\
            g((1, x, y)) & = & (1, x + 1, y) \\
            g((2, x, y)) & = & (1, x, y + 1)
        \end{array} 
        \right .
    \end{displaymath}
    \caption{Représentation du contrôle de train par un réseau de Petri comprenant une place et une
    transition}
    \label{bonusexo8}
\end{figure}

Le but de cet exercice est de modéliser un système de contrôle de train par un réseau de Petri. Pour
des raisons de sécurité, pour qu'un train puisse passer au secteur suivant, ce dernier doit être
vide mais son succeseur doit l'être aussi. Une réprésentation de ce système peut être effectuée en
considérant trois places par secteur, la première place $E$ indique si le secteur est vide et la
seconde (respectivement la troisième) indique si le train $a$ (respectivement $b$) se trouve sur le
secteur. Il doit donc y avoir exclusion mutuelle entre chacune des places d'un même secteur. La
figure~\ref{qu1exo8} est une représentation possible de ce système.

On cherche à présent à réduire le réseau étudié, pour ce faire, nous introduisons deux ensembles de
couleurs, un pour les places $O_i$ : $O = \{a, b\}$ et un pour les places $E_i$ : $E = \{e\}$. Un
jeton de couleur $a$ sur une place $O$ indique la présence du train $a$ dans le secteur associé, de
la même manière un jeton de couleur $b$ représente le train $b$ et enfin un jeton de couleur $e$ sur
une place $E$ indique que le secteur associé est libre de tout train. La figure~\ref{qu1bexo8}
illustre une représentation de ce système.

On s'aperçoit rapidement qu'il est possible de fusionner les places $O$ et les places $E$ sans
ajouter de couleurs, puisque ces places sont en exclusion mutuelles. On peut alors représenter un
secteur par une seule place la couleur du jeton indiquant l'absence de train, la présence du train
$a$ ou la présence du train $b$, la figure~\ref{qu2exo8} en est la preuve. Il est même possible de
réduire l'ensemble des couleurs utilisées à $\{a, b\}$ par l'utilisation d'arcs inhibiteurs (cf.
figure~\ref{qu2bexo8}).

Si l'on veut pousser la réduction un peu plus loin, on peut alors fusionner toutes les cellles $E$
de la figure~\ref{qu2exo8} en une seule place et faire de même pour les places $O$. Les cinq jetons
se trouvant dans la place $E$ ont alors une couleur appartenant à l'ensemble suivant : $C_E = \{e_0,
e_1, e_2, e_3, e_4, e_5, e_6\}$, tandis que le jeton $a$ (respectivement $b$) représentant la
position du train $a$ (respectivement $b$) présente une couleur appartenant à l'ensemble $C_a =
\{a_0, a_1, a_2, a_3, a_4, a_5, a_6\}$ (respectivement $C_b = \{b_0, b_1, b_2, b_3, b_4, b_5,
b_6\}$). Une représentation de ce système nous est donné par la figure~\ref{qu3exo8}.

Nous clôturerons cet exercice en poussant la réduction du réseau à son paroxysme en le réduisant à
un réseau de Petri à une place, une transition et un jeton. Définissons pour ce faire un ensemble de
couleurs que nous appellerons $E = \{(w, x, y) / w \in \{1, 2\},\ x,y \in \{0, 1, \dots 6\}\}$. $x$
représente alors la position du train $a$, $y$ celle du train $b$ et $w$ est utilisé pour générer un
ensemble de couleurs supplémentaire pour la fonction $f$ de franchissement de la transition. En
effet, pour pouvoir différencier quel train passe au secteur suivant lors du franchissement de la
transition il faut doubler le nombre de couleurs, ainsi $f((1, x, y))$ fait avancer le train $a$ au
secteur suivant et $f((2, x, y))$ fait avancer le train $b$. Si l'on appelle $E_1$ (respectivement
$E_2$) l'ensemble $\{(1, x, y) / x,y \in \{0, 1, \dots, 6\}\}$ (respectivement $\{(2, x, y) / x,y
\in \{0, 1, \dots, 6\}\}$), il est alors important de noter que le jeton ne prendra comme couleur
que des couleurs appartenant à $E_1$, $E_2$ ne servant que pour étoffer le dommaine de définition de
$f$. Définissons quelques ensembles remarquables :
\begin{itemize}
    \item $E_{w1} = E_{11} \cup E_{21} = \{(1, x, y) / x = y \} \cup \{(2, x, y) / x = y \}$
        est l'ensemble des couleurs représentant des positions de trains telles que le train $a$ et
        le train $b$ se trouvent sur le même tronçon, qui sont des positions interdites et qui
        doivent donc être retirées du domaine de définition et de valeurs de $f$
    \item $E_{w2} = E_{12} \cup E_{22} = \{(1, x, y) / x = y - 1\} \cup \{(2, x, y) / x = y -
        1 \}$ est l'ensemble des couleurs représentant des positions de trains telles que le train
        $a$ se trouve sur le tronçon précédent celui du train $b$, qui sont elles aussi une
        interdites puisque le train $a$ ne peut avancer si les deux secteurs suivants ne sont pas
        vides. $E_{w2}$ est donc à retirer du domaine de définition et de valeurs de $f$
    \item $E_{w3} = E_{13} \cup E_{23} = \{(1, x, y) / x = y + 1\} \cup \{(2, x, y) / x = y +
        1 \}$ est l'ensemble des couleurs représentant des positions de trains telles que le train
        $b$ se trouve sur le tronçon précédent celui du train $a$. Ces positions étant interdites,
        il faut enlever $E_{w3}$ du domaine de définition et de valeurs de $f$
    \item $E_{w4} = E_{14} \cup E_{24} = \{(1, x, y) / x = y - 2\} \cup \{(2, x, y) / x = y - 2 \}$
        est l'ensemble des couleurs représentant des positions de trains telles que le train $a$ se
        trouve sur l'antépénultième secteur avant celui de $b$. Il s'agit d'une position autorisée
        mais pour laquelle le déplacement du train $a$ est interdit, il faut donc retirer $E_{14}$
        du domaine de définition de la fonction $f$ de consommation.
    \item $E_{w5} = E_{15} \cup E_{25} = \{(1, x, y) / x = y + 2\} \cup \{(2, x, y) / x = y + 2 \}$
        est l'ensemble des couleurs représentant des positions de trains telles que le train $b$ se
        trouve sur l'antépénultième secteur avant celui de $a$. Il s'agit d'une position autorisée
        mais pour laquelle le déplacement du train $b$ est interdit, il faut donc retirer $E_{25}$
        du domaine de définition de la fonction $f$ de consommation.
\end{itemize}

On peut alors définir le domaine de définition de f $D_f = \overline{E_{w1} \cup E_{w2} \cup E_{w3}
\cup E_{14} \cup E_{25}}$ ainsi que le domaine des valeurs de g $V_g = \overline{E_{w1} \cup E_{w2}
\cup E_{w3}} \cap E_1$. Le réseau obtenu est représenté sur la figure~\ref{bonusexo8}.
