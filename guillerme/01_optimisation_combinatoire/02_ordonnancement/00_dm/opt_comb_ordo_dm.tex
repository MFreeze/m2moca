\documentclass[a4paper,12pt]{thesis}

% {{{ PACKAGES
% {{{ ENCODAGE ET POLICE
\usepackage[utf8]{inputenc}
\usepackage[T1]{fontenc}
\usepackage[french]{babel}
\usepackage{eurosym}
% }}}

% {{{ FIGURES
\usepackage{graphicx}
\usepackage{subfig}
% }}}

% {{{ INSERTION DE CODE
\usepackage{moreverb}
% }}}

% {{{ INSERTION DE PDF
\usepackage{pdfpages}
% }}}

% {{{ RÉDUCTION DES MARGES
%\usepackage{fullpage}
%\usepackage[top=2.5cm, bottom=2.5cm, left=3.5cm, right=3.5cm]{geometry}
% }}}

% {{{ MATHÉMATIQUES
\usepackage{amsmath}
\usepackage{amsthm}
\usepackage{bbm}
\usepackage{amssymb}
\usepackage{stmaryrd}
% }}}

% {{{ GRAPHES
\usepackage{tikz}
\usepackage{pgf}
\usetikzlibrary{arrows,automata,shapes}
% }}}

% {{{ RESEAUX DE pETRI
\usetikzlibrary{petri}
% }}}

% {{{ ARBRES
\usetikzlibrary{trees}
% }}}

% {{{ DIAGRAMME DE GANTT
\usepackage{pgfgantt}
% }}}

% {{{ ALGORITHMES
\usepackage{algorithm}
\usepackage[noend]{algpseudocode}
%\usepackage{algorithmic}
%\usepackage{algorithmicx}
% }}}
% }}}

% {{{ NEWTHEOREM
% {{{ THÉORÈMES, lEMMES -- NUMÉROTATION PERSONNALISÉE
\theoremstyle{plain}
\newtheorem{thrm}{Théorème}[section]
\newtheorem{lemm}{Lemme}[subsection]
\newtheorem{corol}{Corollaire}[section]
% }}}

% {{{ THÉORÈMES, lEMMES -- NUMÉROTATION CLASSIQUE
\newtheorem{nthrm}{Théorème}
\newtheorem{nlemm}{Lemme}
\newtheorem{ncorol}{Corollaire}
\newtheorem{npb}{Problème}
% }}}

% {{{ THÉORÈMES, lEMMES -- NON NUMÉROTÉS
\newtheorem*{thrm*}{Théorème}
\newtheorem*{lemm*}{Lemme}
\newtheorem*{corol*}{Corollaire}
% }}}

% {{{ DÉFINITIONS, pROPRIÉTÉS -- NUMÉROTATION PERSONNALISÉE
\theoremstyle{definition}
\newtheorem{df}{Définition}[subsection]
\newtheorem{prop}{Propriété}[subsection]
\newtheorem{conj}{Conjecture}[subsection]
% }}}

% {{{ DÉFINITIONS, pROPRIÉTÉS -- NUMÉROTATION CLASSIQUE
\newtheorem{ndf}{Définition}
\newtheorem{nprop}{Propriété}
\newtheorem{nconj}{Conjecture}
% }}}

% {{{ DÉFINITIONS, pROPRIÉTÉS -- NON NUMÉROTÉS
\newtheorem*{df*}{Définition}
\newtheorem*{prop*}{Propriété}
\newtheorem*{conj*}{Conjecture}
% }}}

% {{{ AUTRES -- NUMÉROTATION PERSONNALISÉE
\theoremstyle{remark}
\newtheorem{nota}{Note}[subsubsection]
\newtheorem{pr}{Proposition}[subsection]
\newtheorem{rmq}{Remarque}[subsubsection]
% }}}

% {{{ AUTRES -- NUMÉROTATION CLASSIQUE
\newtheorem{nnota}{Note}
\newtheorem{npr}{Proposition}
\newtheorem{nrmq}{Remarque}
% }}}

% {{{ AUTRES -- NON NUMÉROTÉS
\newtheorem*{nota*}{Note}
\newtheorem*{pr*}{Proposition}
\newtheorem*{rmq*}{Remarque}
% }}}

% {{{ EXEMPLES
\newtheorem*{ex}{Exemple}
\newtheorem*{exo}{Exercice}
\newtheorem*{rappel}{Rappel}
% }}}
% }}}

% {{{ OPÉRATEURS MATHÉMATIQUES
\DeclareMathOperator{\poly}{\textbf{poly}}
\DeclareMathOperator{\gap}{\mbox{gap}}
\DeclareMathOperator{\Bg}{\mbox{Big}}
\DeclareMathOperator{\Sml}{\mbox{Sml}}
\DeclareMathOperator{\Diff}{\mbox{Diff}}
\DeclareMathOperator{\card}{\mbox{Card}}
\DeclareMathOperator{\Obj}{\mbox{Obj}}
\DeclareMathOperator{\concat}{\bullet}
\DeclareMathOperator{\sig}{\mbox{Sig}}
\DeclareMathOperator{\ver}{\mbox{Ver}}
\DeclareMathOperator{\dist}{\mbox{dist}}
\DeclareMathOperator{\bw}{\mbox{branchwidth}}
\DeclareMathOperator{\tw}{\mbox{tw}}
\DeclareMathOperator{\lm}{\leq_m}
\DeclareMathOperator{\lc}{\leq_c}
\DeclareMathOperator{\lex}{\mbox{lex}}
\DeclareMathOperator{\ord}{\mbox{ord}}
% }}}

% {{{ MACCROS
% {{{ MACROS DIVERSES
\newcommand{\npc}{$NP-$complet}
\newcommand{\npd}{$NP-$difficile}
\newcommand{\la}{largeur arborescente}
\newcommand{\legendre}[2]{\left ( \frac{#1}{#2} \right )}
\newcommand{\wqo}[0]{\emph{well-quasi-ordered}}
\newcommand{\Tau}[0]{\mathcal{T} = (T, \{X_t : t \in T\})}
\newcommand{\Taug}[0]{\mathcal{T}_G = (T, \{X_t : t \in T\})}
\newcommand{\Taustar}[0]{\mathcal{T}^* = (T, \{X_t : t \in T\})}
\newcommand{\Taugstar}[0]{\mathcal{T}_G^* = (T, \{X_t : t \in T\})}
\newcommand{\pdsol}[1]{(X_{#1}^0, X_{#1}^1, X_{#1}^2, M_{#1})}
% }}}
                
% {{{ NOM DE PROBLEMES
\newcommand{\hcycle}{\textsc{HAMILTONIAN CYCLE} }
\newcommand{\phcycle}[1]{\textsc{HAMILTONIAN CYCLE }paramétré par \textsc{#1} }
\newcommand{\vcover}{\textsc{VERTEX COVER} }
\newcommand{\twidth}{\textsc{TREE WIDTH} }
\newcommand{\wiset}{\textsc{WEIGHTED INDEPENDANT SET} }
\newcommand{\fvset}{\textsc{FEEDBACK VERTEX SET} }
\newcommand{\kvcover}{\textsc{K-Vertex Cover} }
\newcommand{\oct}{\textrm{\textsc{Odd Cycle Transversal}} }
\newcommand{\dlp}{\textrm{\textsc{Discrete Logarithm Problem}} }
\newcommand{\ecdlp}{\textrm{\textsc{Elliptic Curve Discrete Logarithm Problem}} }

% }}}

% {{{ MISE EN PAGE

\newcommand{\dfpbi}[3]{
    \begin{npb}[#1]
        \begin{itemize}
            \itemindent 10mm
            \item[\textbf{Données :}] #2
            \item[\textbf{Problème :}] #3
        \end{itemize}
    \end{npb}
}

\newcommand{\dfpb}[3]{
    \begin{npb}[#1]
        \begin{itemize}
            \itemindent 10mm
            \item[\textbf{Données :}] #2
            \item[\textbf{Question :}] #3
        \end{itemize}
    \end{npb}
}

\newcommand{\dfpbp}[4]{
    \begin{npb}[#1]
        \begin{itemize}
            \itemindent 10mm
            \item[\textbf{Données :}] #2
            \item[\textbf{Paramètre :}] #4
            \item[\textbf{Question :}] #3
        \end{itemize}
    \end{npb}
}
% }}}

% }}}
    
% {{{ TIKZ
%%% {{{ STYLE
\tikzstyle{edge}=[
	-,
	draw=black
]

\tikzstyle{sized}=[
    minimum width=9mm
]

\tikzstyle{tvertex}=[
	node distance=15mm,
    inner sep = 1mm
]

\tikzstyle{tvert}=[
	node distance=5mm,
    inner sep=1mm
]

\tikzstyle{tmvertex}=[
    tvertex,
    sized
]

\tikzstyle{tmvert}=[
    tvert,
    sized
]

\tikzstyle{mvertex}=[
    tmvertex,
	circle,
	draw=black,
	fill=black!50
]

\tikzstyle{mvert}=[
    tmvert,
	circle,
	draw=black,
	fill=black!50
]

\tikzstyle{vertex}=[
    tvertex,
	circle,
	draw=black,
	fill=black!50
]

\tikzstyle{vert}=[
    tvert,
	circle,
	draw=black,
	fill=black!50
]

\tikzstyle{treedec}=[
	vertex,
	node distance=15mm,
	draw=red,
	fill=red!50
]

\tikzstyle{treeedge}=[
	edge,
	draw=red
]

\tikzstyle{giedge}=[
	edge,
	draw=blue
]

\tikzstyle{bounding}=[
	thick, %ultra
	%dash pattern=on 1mm off 1mm,
	densely dotted,
	fill=blue!10,
	draw=blue!70
]

\tikzstyle{patate}=[
    rounded corners=5pt,
    draw=black,
    fill=black!50,
    text width=12mm,
    text centered
]

\tikzstyle{patatoid}=[
    ellipse,
    node distance=10mm,
    draw=black
]
%%% }}}

% {{{ DESSINS
\newcommand{\base}[4]{
    \draw[->] (0,0) to (0, #1);
    \draw[->] (0,0) to (#2, 0);
    
    \node at (0, #1+2) {#3};
    \node at (#2 + 2, -2) {#4};
}
% }}}
%}}}


\pagenumbering{roman}
\pagestyle{fancyplain}
\thispagestyle{empty}
\noindent
\begin{center}
\large{\texttt{Académie de Montpellier}}\\
\Large{\texttt{Université Montpellier II}}\\
\large{\texttt{Sciences et Techniques du Languedoc}}\\
\end{center}

\vspace{1cm}

\begin{center}
\Huge{\textbf{\'{E}TUDE BIBLIOGRAPHIQUE DE\\}}
 \vspace{1.0cm}
\Huge{\textbf{MASTER M2}}
\normalsize
\begin{center}
\vspace{1.0cm}

effectué au Laboratoire d'Informatique de Robotique\\
et de Micro-électronique de Montpellier
\end{center}

\vspace{2mm}
%\Large{\textbf{prÈnom NOM}}

\vspace{0.1cm}
\normalsize

\vspace{3mm}

\large{Spécialité} : \textbf{MOCA}\\
%\Large{Formation Doctorale} : \textbf{Informatique}\\
%\large{{\'E}cole Doctorale} : \textbf{Information, Structures, SystËmes}
\vspace{1.0cm}

\LARGE{\textbf{Complexité et algorithmes d'approximation pour des problèmes d'ordonnancement d'intervalles
avec optimisation de la défragmentation}}
\vspace{2mm}

\begin{center}
  par \textbf{Guillerme DUVILLIÉ}
\end{center}

\vspace{2mm}



\vspace{4cm}

Date de rendu : \textbf{}

\vspace{0.5cm}

Sous la direction de \textbf{Marin BOUGERET, Rodolphe GIROUDEAU et Denis TRYSTRAM}

\vspace{5mm}




\end{center}
\newpage
\tableofcontents
\newpage

\pagenumbering{arabic}


\author{}
\title{}
\location{Montpellier}\blurb{}
\makeatother
\title{Devoir à rendre}
\author{Guillerme DUVILLIE}
\location{Montpellier}
\blurb{
    Université Montpellier II\\
    Master Informatique Modélisation, Optimisation Combinatoire et Algorithmique \\[1em]
    Matière : Optimisation Combinatoire\\
    Encadrant : Marin BOUGERET
}

\begin{document}

\maketitle

\section*{$\frac{3}{2}-$dual approximation pour $P||C_{max}$}

L'objectif de cette partie est de présenter une $\frac{3}{2}-$dual approximation pour le problème
d'ordonnancement $P||C_{max}$. Nous introduirons dans un premier temps le problème considéré, puis
présenterons l'algorithme et dans un troisième temps nous nous consacrerons à l'analyse de ce
dernier.

\subsection*{Le problème $P||C_{max}$}

Ce problème est un problème classique de l'ordonnancement. Il cherche à répartir un nombre $n$ de
tâches sur un nombre $m$ de machines identiques. À chaque tâche est associé un temps d'exécution
identique sur toutes les machines, on cherche alors à minimiser le temps total d'exécution.

Définissons $P||C_{max}$ de manière plus formelle.

\textbf{Entrée :} Un ensemble $T$ de $n$ tâches à ordonnancer, un ensemble $M$ de $m$ machines, une
fonction de poids $p$ définie comme suit : \begin{displaymath}
    p : \left \{
        \begin{array}{rcl}
            $E$ & \rightarrow & \mathbb{N} \\
            i & \mapsto & p(i) = p_i
    \end{array}
    \right .
\end{displaymath}

\textbf{Sortie :} Un ordonnancement des $n$ tâches sur les $m$ machines minimisant le temps
d'exécution.

\subsection*{L'algorithme}

Nous étudierons l'algorithme suivant :

\begin{algorithm}[h!]
    \caption{Une $\frac{3}{2}-$ dual approximation}
    \label{dual_approx}
    \begin{algorithmic}[1]
        \Function{DivLS}{$\omega, I, m$}
            \If{$\{ j | p_j > \omega \} \neq \emptyset $}
                \State \Return FAIL
            \EndIf
            \State $\Bg \gets \{ j | p_j  > \frac{\omega}{2} \}$
            \State $\Sml \gets \{ j | p_j \leq \frac{\omega}{2} \}$
            \For{$i \in \{1, \dots, m \}$}
                \State $x \gets \max(\Bg)$
                \State Ordonnancer $x$ sur $i$
                \State $\Bg \gets \Bg \backslash \{x\}$
            \EndFor
            \If{$\Bg \neq \emptyset$}
                \State \Return FAIL
            \EndIf
            \State \Return Greedy$(\Sml)$
        \EndFunction
        \Function{Greedy}{$\Sml$}
            \While{$\Sml \neq \emptyset ~ \&\& ~ \exists$ une machine $l$ finissant avant $\omega$}
                \State Ordonnancer $x$ sur $l$
                \State $\Sml \gets \Sml \backslash \{x\}$
            \EndWhile
            \If{$\Sml \neq \emptyset$}
                \State \Return FAIL
            \EndIf
        \EndFunction
    \end{algorithmic}
\end{algorithm}

Nous allons voir qu'il s'agit bien d'une $\rho-$dual approximation avec $\rho = \frac{3}{2}$.

\subsection*{Analyse}

Voici quelques remarques et propriétés concernant l'algorithme \ref{dual_approx}.

\begin{nlemma}
    \label{ordo}
    Si l'algorithme n'échoue pas, alors il retourne un ordonnancement.
\end{nlemma}

\begin{proof}
    Si l'algorithme n'échoue pas, alors les ensembles $\Bg$ et $\Sml$ sont vides. Par construction,
    on sait que $\Bg \cup \Sml = I$, donc, dans ce cas, toutes les tâches on été affectées lors de
    l'appel à la fonction DivLS. Il s'agit donc bien d'un ordonnancement.
\end{proof}

\begin{nlemma}
    \label{borne_omega}
    Si l'algorithme échoue, alors $\omega < OPT$.
\end{nlemma}

\begin{proof}
    Pour démontrer le lemme \ref{borne_omega}, nous nous intéresserons à tous les passages de
    l'algorithme pouvant conduire à un échec de ce dernier. Ils sont au nombre de trois.
    \begin{enumerate}
        \item ligne $3$ : l'algorithme échoue s'il existe une tâche dont le temps d'exécution est
            supérieur à $\omega$. L'ordonnancement affectant toutes les tâches et les machines étant
            identiques, il va de soit que : \begin{displaymath}
                OPT \geq \max p_j \quad j \in E
            \end{displaymath}
            Donc s'il existe au moins une tâche $j$ telle que $p_j > \omega$, alors $\max p_j >
            \omega \Rightarrow OPT > \omega$
        \item ligne $11$ : l'échec est provoqué par le fait que le cardinal de $\Bg$ est supérieur
            au nombre de machines $m$. Or, par construction, $\Bg$ contient toutes les tâches dont
            le temps d'exécution est supérieur à $\omega / 2$. Donc, si $|\Bg| > m$, il faut alors
            affecter au moins deux tâches de $\Bg$ à au moins une machine. 

            Soient $a = \min(\Bg)$ et $b = \min(\Bg \backslash \{a\})$, on a alors :
            \begin{displaymath}
                OPT \geq a + b \quad \Rightarrow \quad OPT > 2 \left ( \frac{\omega}{2} \right )
                \quad \Rightarrow \quad OPT > \omega
            \end{displaymath}
        \item ligne $18$ : l'algorithme échoue ici lors du traitement des petites tâches\footnote{C'est
            à dire l'ensemble des tâches appartenant à $\Sml$.}. L'algorithme glouton est alors
            incapable de trouver une machine dont le temps de travail est inférieur à $\omega$, ce
            qui nous indique : \begin{displaymath}
                \frac{1}{m} \sum_{i \in E} i > \omega
            \end{displaymath}
            Or on connaît la borne sur $OPT$ suivante :
            \begin{displaymath}
                OPT \geq \frac{1}{m} \sum_{i \in E} i
            \end{displaymath}
            On a donc $OPT > \omega$.
    \end{enumerate}
\end{proof}

\begin{nthrm}
    \label{th}
    L'algorithme \ref{dual_approx} est une $\rho-$dual approximation et $\rho = \frac{3}{2}$.
\end{nthrm}

\begin{proof}
    On s'intéresse à la valeur de la solution retournée par rapport à $\omega$.

    L'algorithme procède en deux étapes, le traitement des grosses tâches\footnote{Il s'agit des
    tâches appartenant à $\Bg$.}, puis le traitement des petites. 
    
    Lors de la première étape, une seule tâche, au plus, a été affectée par machine, le temps
    d'exécution de cette dernière étant majorée par $\omega$\footnote{Puisque le contraire
    validerait la condition de la ligne $2$, forçant l'algorithme à signaler un échec.}. 

    Lors de la seconde étape, des tâches, dont le temps d'exécution est borné par $\omega / 2$ sont
    rajoutées\footnote{Dans la limite des stocks disponibles...} aux machines dont le temps de
    travail est inférieur à $\omega$. Considérons une machine $l$ dont le temps de travail est égal
    à $\omega - 1$ avant ordonnancement d'une tâche $j$ de pois $\omega / 2$. Après ordonnancement
    de $j$, la machine $l$ aura un temps de travail égal à $(3 \omega/2) - 1$, elle ne pourra donc
    plus être candidate à l'ordonnancement d'une nouvelle tâche, ceci étant vrai quelque soit la
    machine considérée, le makespan retourné est inférieur à $3 \omega/2$.

    De plus les lemmes \ref{ordo} et \ref{borne_omega} complètent la preuve du théorème \ref{th}.
\end{proof}

\section*{Non existence d'un \textbf{FPTAS} pour MKP}

L'objectif de cette partie est de démontrer qu'il n'existe aucun \textbf{FPTAS} pour le problème du
sac à dos multiple. Pour ce faire nous démontrerons que ceci est vrai pour deux sacs à dos, et nous
admettrons que ce résultat se généralise à des instances à $m$ sacs à dos, $m \in \mathbb{N}
\backslash \{0, 1\}$.

\subsection*{Le problème du sac à dos multiple}

Étant donné un ensemble $E$ de $n$ d'objets, caractérisés par leur volume et leur utilité, et un
ensemble $S$ de $m$ sacs à dos, caractérisés par leur contenance maximale, ce problème cherche à
déterminer des sous ensembles d'objets associés à chacun des sacs, de manière à ce que le volume
total d'aucun des sous ensembles ne soit supérieur à la contenance maximale du sac qui lui est
associée et de manière à maximiser l'utilité des objets sélectionnés.

Autrement dit, on cherche à prendre les objets les plus utiles possibles tout en ayant encore la
possibilité de fermer tous les sacs.

On peut décrire ce problème de manière plus formelle.

\textbf{Entrée :} un ensemble $E$ de $n$ objets, un ensemble $S$ de $m$ sacs à dos, une fonction de
poids\footnote{On assimilera l'utilité d'un objet à son poids.} définie comme suit : 
\begin{displaymath}
    p : \left \{ \begin{array}{rcl}
        E & \rightarrow & \mathbb{N} \\
        i & \mapsto & p(i) = p_i
    \end{array} \right .
\end{displaymath}

une fonction de volume définie comme suit :
\begin{displaymath}
    v : \left \{ \begin{array}{rcl}
        E & \rightarrow & \mathbb{N} \\
        i & \mapsto & v(i) = v_i
    \end{array} \right .
\end{displaymath}

et une fonction de capacité définie comme suit :
\begin{displaymath}
    c : \left \{ \begin{array}{rcl}
        S & \rightarrow & \mathbb{N} \\
        l & \mapsto & c(l) = c_l
    \end{array} \right .
\end{displaymath}

\textbf{Sortie :} Un ensemble de $m$ sous-ensembles disjoints de $E$\footnote{On notera $E_l$ le
sous ensemble associé au sac à dos $l$.}, tels que : \begin{displaymath}
    \sum_{i \in E_l} v_i \leq c_l \quad \forall l \in S
\end{displaymath}
et maximisant la fonction objectif suivante : \begin{displaymath}
    \sum_{l \in S} \sum_{i \in E_l} p_i
\end{displaymath}

\subsection*{La réduction gap}

Nous utiliserons une réduction gap de MKP vers $2-$partition pour montrer qu'il n'existe aucun
\textbf{FPTAS} pour MKP.

\begin{nthrm}
    \label{no_approx}
    Il n'existe aucun algorithme \textbf{FPTAS} pour MKP sauf si $P = NP$.
\end{nthrm}

\begin{proof}
    Considérons une instance de $2-$partition\footnote{Il sera considéré que le lecteur connaît ce
    problème.} à $2n$ éléments et de fonction de poids $f$.

    Appelons $A$ le poids des sous-ensemble de la $2-$partition recherchée, on a alors :
    \begin{displaymath}
        A = \frac{1}{2} \sum_{i = 1}^{2n} f_i
    \end{displaymath}

    On cherche à construire une instance de MKP avec deux sacs à dos à partir de l'instance définie
    de $2-$partitions. Pour ce faire, on fait correspondre à chaque élément $i$ de l'ensemble de
    départ de $2-$partition un objet $j$ de volume $v_j = f_i$ et de poids $p_j = 1$. De plus on
    fixe la capacité de chaque sac à dos ainsi : $c_1 = c_2 = A$.

    Supposons à présent l'existence d'un algorithme $(1 - \epsilon)-$approché pour MKP s'exécutant en
    temps polynomial. Si l'algorithme réussit à affecter un sac à chacun des objets, l'optimal sera
    alors défini par : \begin{displaymath}
        OPT = \sum_{i = 1}^{2n} p_i = 2n
    \end{displaymath}
    Dans ce cas, l'affectation de chaque objet à un sac définirait une $2-$partition valide.
    Sinon l'optimal sera borné par au plus $2n -1$. En sélectionnant $\epsilon$ inférieur à
    $\frac{1}{2n}$, il sera possible de différencier ces deux cas, et donc dans le cas ou $OPT =
    2n$, on pourrait répondre OUI au problème de $2-$partition associé, sinon, la réponse à
    $2-$partition serait NON.

    \begin{displaymath}
        \begin{array}{rrcl}
            & (1 - \epsilon) OPT & > & 2n - 1 \\
            \Rightarrow & - \epsilon & > & \displaystyle \frac{2n - 1 - OPT}{OPT} \quad \mbox{or} \quad OPT \geq 2n\\
            \Rightarrow & - \epsilon & > & \displaystyle \frac{-1}{2n} \\
            \Rightarrow & \epsilon & < & \displaystyle \frac{1}{2n}
        \end{array}
    \end{displaymath}

    L'existence d'un \textbf{FPTAS} pour MKP permet la résolution de $2-$partition en temps
    polynomial, ce dernier étant $NP-$complet, ceci l'hypothèse de départ n'est vraie que si $P=NP$.
\end{proof}
\end{document}

