\documentclass[a4paper,)11pt]{article}
\usepackage[francais]{babel}
\usepackage[utf8]{inputenc}
\usepackage{algorithm,algorithmic}


\begin{document}

\section*{Algorithmes}

\subsection*{Notations}
\begin{itemize}
	\item $e = $ nombre de contraintes
	\item $r = $ arité max
	\item $d = $ domaine max
\end{itemize}

\subsection*{BT}
\begin{algorithm}[H]
\caption{BT}
\label{BT}
\begin{algorithmic}[1]
\REQUIRE N un réseau, I une instanciation
\IF {$|I|=n$}
	\STATE return true
\ENDIF
\STATE choose a variable $X_i \notin I$
\FOR {each $V_i \in D(X_i)$ }
	\IF {$I \cup \{(X_i=V_i)\}$ locally consistent }
		\IF {BT($N$, $I \cup \{X_i=V_i\}$ }
		\STATE return true
\ENDIF
\ENDIF
\ENDFOR
\STATE return false
\end{algorithmic}
\end{algorithm}

\subsection*{AC3}
\begin{algorithm}[H]
\caption{AC3}
\label{AC3}
\begin{algorithmic}
\REQUIRE Q
\STATE $Q \leftarrow \{(X_i=C_{ij} | C_{ij} \in \cal{C} \}$
\WHILE {$Q \neq \emptyset$}
\STATE select and delete $(X_i,C_{ij})$ from Q
\IF {revise$(X_i,C_{ij})$}
\IF  {D$(X_i) = \emptyset$}
\STATE return false
\ELSE
\STATE $Q \leftarrow Q \cup \{(X_k,C_{ki}) | k \neq j\}$
\ENDIF
\ENDIF
\ENDWHILE
\STATE return true
\end{algorithmic}
\end{algorithm}

\begin{itemize}
	\item Temps: $O(e\times d^{r+1} \times r)$
	\item Espace: $O(e\times r)$
\end{itemize}


\begin{algorithm}[H]
\caption{revise3}
\label{R3}
\begin{algorithmic}
\REQUIRE $X_i$ une variable, $C_{ij}$ contrainte portant sur $X_i,X_j$
\STATE $CHANGE \leftarrow false$
\FOR {each $V_i \in D(X_i)$}
\STATE $V_j \leftarrow first(D(X_j))$
\WHILE {$\neg C_{ij}(V_i,V_j)$ and $V_{ij} \neq nil$}
\STATE $V_j \leftarrow$ next($V_j, D(X_j)$)
\ENDWHILE
\IF {$V_j = nil$}
\STATE remove $V_i$ from $D(X_i)$
\STATE $CHANGE \leftarrow true$
\ENDIF
\ENDFOR
\STATE return CHANGE
\end{algorithmic}
\end{algorithm}


\subsection*{AC2001}

\begin{algorithm}[H]
\caption{revise(AC2001)}
\label{R2001}
\begin{algorithmic}
\REQUIRE $X_i$ une variable, $C_{ij}$ une contrainte portant sur $X_i,X_j$
\STATE $CHANGE \leftarrow false$
\FOR {each $V_i \in D(X_i)$ and last$(V_i,C_{ij}) \notin D(X_j)$}
\STATE $V_j \leftarrow$ next(last($V_i, C_{ij} $),$D(X_j)$)
\WHILE {$\neg C_{ij}(V_i,V_j)$ and $V_{ij} \neq nil$}
\STATE $V_j \leftarrow$ next($V_j, D(X_j)$)
\ENDWHILE
\IF {$V_j = nil$}
\STATE remove $V_i$ from $D(X_i)$
\STATE $CHANGE \leftarrow true$
\ENDIF
\ENDFOR
\STATE return CHANGE
\end{algorithmic}
\end{algorithm}

\begin{itemize}
	\item LIMITE A DES CONTRAINTES D'ARITE 2!
	\item Temps: $O(e\times d^2)$
	\item Espace: $O(e\times d)$
\end{itemize}

\subsection{FC}

\begin{algorithm}[H]
\caption{FC}
\label{FC}
\begin{algorithmic}
\REQUIRE N,I
\IF {$|I|=n$}
\STATE return true
\ELSE
\STATE choose $X_i \notin I$
\FOR {each $V_i \in D(X_i)$}
\STATE $D(X_i) \leftarrow \{V_i\}$
\STATE supprimer tous les $(X_j,V_j)$ qui sont inconsistant ac $I \cup \{X_i=V_i\},\ \forall X_j \notin I$
\IF {$\not\exists j,\ D(X_j)= \emptyset$}
\IF {FC($N,I \cup \{X_i,V_i\}$)}
\STATE return true
\ENDIF
\ENDIF
\STATE restaurer les valeurs des $X_j$ supprimés à cause de l'instanciation $X_i=V_i$
\ENDFOR
\ENDIF
\STATE renvoyer faux
\end{algorithmic}
\end{algorithm}

\begin{algorithm}[H]
\caption{MAC}
\label{MAC}
\begin{algorithmic}
\REQUIRE N
\IF {AC(N)}
\IF {N is ground}
\STATE return true
\ENDIF
\STATE choose $X_i$ non fixé dans N et $V_i \in D(X_i)$
\IF {MAC($N \cup \{X_i=V_i\}$)}
\STATE return true
\ENDIF
\IF {MAC($N \cup \{X_i \neq V_i\}$)}
\STATE return true
\ENDIF
\ENDIF
\STATE return false
\end{algorithmic}
\end{algorithm}

\section*{Borne consistance}

Soit $N=(X;D;\cal{C})$ et $c \in \cal{C}$.\\
Un "support aux bornes" $\tau$ sur $c$ est un tuple qui satisfait $c$ et pour tout $X_i \in X(c)$, $\min_D (X_i) \le \tau [X_i] \le \max_D (X_i)$.\\
Un contrainte $c$ est BC ssi $\forall X_i \in X(c),\ (X_i,\min_D(X_i))$ et $(X_i, \max_D(X_i))$ ont un support aux bornes.\\
N est BC ssi toutes les contraintes sont BC.

\section*{Singleton Arc Consistancy}
Un réseau $N=(X;D;\cal{C})$ et $c \in \cal{C}$ est SAC ssi $\forall X_i \in X,\ \forall V_i \in D(X_i)$, le sous réseau $N_{|X_i=V_i}$ admet une fermeture arc consistante.\\

\section*{max Restricted Path Consistancy}
Une valeur $V_i \in D(X_i)$ est maxRPC ssi $\forall C_{ij} \in \cal{C},\ \exists V_j \in D(X_j)$ tq $\forall X_k \in X,\ \exists V_k \in D(X_k)$ tq $C_{ik}(V_i,V_k)$ et $C_{jk}8V_j,V_k)$.\\
Un réseau est maxRPC ssi $\forall X_i \in X,\ \forall V_I \in D)X_i),\ V_i$ est maxRPC.

\section*{Contraintes globales}

\subsection*{All-diff}
\begin{algorithm}[H]
\caption{AC\_alldiff}
\label{AC_ad}
\begin{algorithmic}
\STATE A partir d'une contrainte \textbf{all-diff}, générer un graphe biparti variables-valeurs
\STATE Calculer un couplage maximum du graphe
\IF {une variable est libre}
\STATE Pas de solution
\ENDIF
\STATE Appliquer une décomposition D\&M à partir du couplage obtenu
\STATE Calculer les composantes fortement connexes de la partie "bien-valuée"
\STATE Éliminer les valeurs des domaines correspondant aux arêtes relient les parties "bien-valuée" et "sur-valuée" et aux arêtes reliant deux composantes fortement connexes
\end{algorithmic}
\end{algorithm}

\subsection*{Modélisation d'une gcc par un flot}

\paragraph{Sommets:}
Les valeurs, les variables, une source $s$ et un puits $p$.

\paragraph{Arcs:}
\begin{itemize}
\item Un arc $(u,v)$ de chaque valeur $u$ vers sa variable $v$; $l(u,v)=0,\ c(u,v)=1$
\item Un arc $(s,u)$ de $s$ vers chaque valeur $u$; $l(s,u)=l_u,\ c(s,u)=c_u$
\item Un arc $(v,t)$ de chaque variable $v$ vers $t$; $l(v,t)=0,\ c(t,v)=1$
\item Un arc $(t,s)$ tel que $l(t,s)=0,\ c(t,s)= \infty$
\end{itemize}

\subsection*{Algorithme d'AC d'une gcc}
\begin{algorithm}[H]
\caption{AC\_gcc}
\label{AC_gcc}
\begin{algorithmic}
\STATE Générer le graphe $G=gcc(C)$ à partir d'une gcc C
\STATE Initialiser chaque arc $(u,v)$ avec un flot $f(u,v)=0$
\STATE Calculer un flot maximum $f$ par une augmentation de chemin dans le graphe résiduel $R(f,G)$
\IF {f est infaisable}
\STATE C n'est pas consistante
\ENDIF
\STATE Calculer les composantes fortement connexes de $R(f,G)$
\STATE Eliminer toute valeur correspondant à un arc $(u,v)$ de $R(f,G)$ tel que:
\begin{itemize}
\item f(u,v)=0
\item $u$ et $v$ n'appartiennent pas à la même composante fortement connexe de $R(f,G)$
\end{itemize}
\end{algorithmic}
\end{algorithm}


\end{document}


