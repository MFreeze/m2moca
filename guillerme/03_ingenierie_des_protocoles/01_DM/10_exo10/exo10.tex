\begin{figure}
    \begin{center}
        \begin{tikzpicture}[every transition/.style={minimum width=10mm, minimum height=5mm},
        >=latex, every place/.style={minimum width=1cm}, scale=0.7, every node/.style={transform shape}]
            \node[place, label=$P_1$, tokens=1] (p1) at (3, 6) {};
            \node[place, label=$P_2$, tokens=2] (p2) at (12, 6) {};
            \node[transition, label=$t_5$] (t5) at (0, 4) {$[0,3]$};
            \node[transition, label=$t_1$] (t1) at (6, 4) {$[4,9]$};
            \node[transition, label=$t_2$] (t2) at (9, 4) {$[0,2]$};
            \node[transition, label=$t_3$] (t3) at (15, 4) {$[1,3]$};
            \node[place, label=$P_3$] (p3) at (3, 2) {};
            \node[place, label=$P_4$] (p4) at (9,2) {};
            \node[place, label=$P_5$] (p5) at (15, 2) {};
            \node[transition, label=$t_4$] (t4) at (3, 0) {$[0,2]$};

            \draw (p1) edge[post] (t5)
                edge[post] (t1);
            \draw (p3) edge[pre] (t1)
                edge[post] (t5)
                edge[pre, bend right] (t4)
                edge[post, bend left] (t4);
            \draw (p2) edge[pre] (t2)
                edge[pre] (t3)
                edge[post, bend right] node[above left] {$2$} (t1);
            \draw (p4) edge[pre, bend left] (t1)
                edge[post] (t2);
            \draw (p5) edge[pre, in=270, out=200] (t1)
                edge[post] (t3);
        \end{tikzpicture}
    \end{center}
    \caption{Un réseau temporel}
    \label{exo10fig1}
\end{figure}

\begin{figure}
    \begin{center}
        \begin{tikzpicture}
            \node (n1) at (3, 8) {$P_1P_2^2$};
            \node (n2) at (3, 6) {$P_3P_4P_5$};
            \node (n3) at (3, 4) {$P_2P_3P_5$};
            \node (n4) at (3, 2) {$P_2^2P_3$};

            \draw (n2) edge[pre] node[right] {$t_1$} (n1)
                edge[post] node[left] {$t_2$} (n3)
                edge[loop right] node[right] {$t_4$} ();
            \draw (n3) edge[post] node[right] {$t_3$} (n4)
                edge[loop right] node[right] {$t_4$} ();
            \draw (n4) edge[loop right] node[right] {$t_4$} ();
        \end{tikzpicture}
    \end{center}
    \caption{Graphe de classes}
    \label{exo10fig2}
\end{figure}

Considérons le graphe temporel de la figure~\ref{exo10fig1}, le graphe de classes est donné par la
figure~\ref{exo10fig2} auquel il faudrait ajouter une descriptions des états à l'aide de systèmes
d'inéquations.
