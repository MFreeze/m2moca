\section{Calcul de $z$}
L'algorithme va parcourir les intervalles de la gauche vers la droite en calculant les écarts entre
les fins et les débuts d'intervalles consécutifs. Pour chaque intervalle, on considère deux
actions ponctuelless : le début et la fin de cet intervalle. Chaque action peut être vue comme un
couple composé de la date $d_a$ de cette action et de son type $t_a$ avec $t_a \in \{$deb, fin$\}$.

L'agolrithme prends donc en entrée une liste d'actions $L$ triées selon leur date d'exécution par ordre
croissant. Il conserve la dernière fin d'intervalle rencontrée\footnote{Nous appellerons cette fin,
la fin courante.}et à chaque nouvelle action, il vérifie le type de cette dernière :
\begin{itemize}
    \item si l'action est un début d'intervalle, l'écart entre le temps d'exécution de l'action et
        la date de la fin courante est calculé, s'il est supérieur à $0$ et inférieur au minimum
        courant, le minimum est mis à jour. 
    \item si l'action est une fin d'intervalle, la fin courante est mise à jour.
\end{itemize}

L'algorithme en temps linéaire en le nombre d'actions, or il y a exactement deux actions par
intervalle, il s'exécute donc en temps linéaire en le nombre d'intervalles.\footnote{Notons que si
    la liste n'est pas triée, le tri préalable se fait en $O(n \log n)$ et donc l'algorithme
s'exécute aussi en $O(n \log n)$.}

\begin{algorithm}
    \caption{Algorithme pour le calcul de $z$}
    \label{algo_calcul_z}
    \begin{algorithmic}[1]
        \Require une liste d'actions $L$
        \State fincourante $\gets$ \texttt{NULL} 
        \State min $\gets \infty$
        \State $a \gets pop(L)$
        \While{$a \neq $\texttt{NULL} $\ \&\ t_a \neq $fin}
            \State $a \gets pop(L)$
        \EndWhile
        \State fincourante $\gets d_a$
        \State $a \gets pop(L)$
        \While{$a \neq $\texttt{NULL} }
            \If{$t_a = $fin}
                \State fincourante $\gets d_a$
            \Else
                \State tmp $\gets d_a - $fincourante
                \If{tmp $> 0$}
                    \State min $\gets \min\{$min, tmp$\}$
                \EndIf
            \EndIf
        \EndWhile
    \end{algorithmic}
\end{algorithm}
% }}}
% }}}
                
\bibliographystyle{plain}
\bibliography{biblio}


