\documentclass[a4paper,11pt]{article}

/stock/16_git_repo/library/library.tex

\begin{document}

\section{Définition des problèmes d'ordonnancement}

\subsection{Description des tâches}

Dans la littérature, en général (nous adopterons cette définition dans la suite), les problèmes
d'ordonnancement se déclinent de la manière suivante\footnote{Nous avons omis l'ensemble des
    ressources à l'exécution d'une tâche pour ne pas alourdir d'une part la présentation, et d'autre
    part dans le reste de ce manuscrit nous considérons des modèles pour lesquels l'utilisation de
ressources supplémentaires n'est pas pris en compte.} : nous disposons de deux ensembles
$\mathcal{T}=\{T_1,\ldots,T_n\}$ de $n$ tâches, $\mathcal{P}=\{\pi_1,\ldots,\pi_m\}$ de $m$
processeurs. L'ordonnancement consiste à affecter l'ensemble $\mathcal{T}$ des tâches, soumises à
des contraintes imposées, à l'ensemble des processeurs $\mathcal{P}$. Il existe en général deux
contraintes en théorie de l'ordonnancement :

\begin{itemize}
    \item chaque processeur est capable d'exécuter au plus une tâche à chaque instant,
    \item chaque tâche ne peut-être exécutée qu'au plus par un seul processeur\footnote{Dans
        certains modèles hiérarchiques du type tâches modelables ou malléables \cite{blayo,decker},
    une tâche peut s'exécuter sur plusieurs processeurs (les communications sont intégrées dans la
durée d'exécution des tâches). Cependant  la durée dépend du nombre de processeurs exécutant cette
tâche.}.% (il est possible selon les spécifications des ressources additionnelles)
\end{itemize}

\subsection{Environnement des processeurs}

Maintenant caractérisons les processeurs. Ils peuvent être soit parallèles, exécutant les mêmes
fonctions, soit dédiés i.e. spécialisés pour l'exécution de certaines tâches. Les trois types de
processeurs de machines parallèles se distinguent par leurs vitesses de traitement des tâches. Si
tous les processeurs de l'ensemble $\mathcal{P}$ admettent la même vitesse de traitement, nous
dirons que les processeurs sont identiques. Si les vitesses de traitements des processeurs
diffèrent, mais chaque vitesse $b_i$ (facteur d'accélération du traitement de la tâche) de chaque
processeur est constant et ne dépend pas de la tâche à exécuter, alors nous dirons que les
processeurs sont uniformes. En dernier lieu, si les vitesses des processeurs dépendent des tâches à
ordonnancer, nous dirons que les processeurs de la machine parallèle sont généraux.


Dans le cas où le les processeurs sont dédiés, nous pouvons présenter trois modèles : \texttt{flow
shop, open shop} et \texttt{job shop}. Nous renvoyons les personnes intéressées pour ces trois
modèles à la lecture des livres \cite{handbook}, \cite{brucker},  \cite{pinedo}. Ces trois modèles
n'étant pas étudiés dans cette thèse, ils ne seront pas décrits.


En général, une tâche $T_j \in \mathcal{T}$ est caractérisée par les données suivantes :

\begin{enumerate}
    \item un vecteur de temps d'exécution $p_j=[p_{1j}, p_{2j},\ldots,p_{mj}]^T$, où $p_{ij}$
        désigne le temps nécessaire au processeur $\pi_i$ d'exécuter $T_j$. Dans le cas où tous les
        processeurs sont identiques, nous avons $p_{ij}=p_j, i=1,\ldots,m$. Si les processeurs de
        $\mathcal{P}$ sont uniformes, nous avons $p_{ij}=p_j/b_i, i=1,\ldots,m$ où $p_j$ représente
        la vitesse d'exécution standard  et $b_i$ facteur d'accélération du traitement de la tâche
        par le processeur $P_i$.
    \item une date de disponibilité, notée $r_j$, qui désigne la date à partir de laquelle la tâche
        $T_j$ est prête à être exécutée. Si les dates de disponibilités sont équivalentes pour
        toutes les tâches de $\mathcal{T}$, alors nous pouvons supposer que $r_j=0, \forall j$.

    \item une date d'échéance $d_j$, qui spécifie la date limite (souhaité) à laquelle la tâche $T_j$ devra être exécutée.
    \item une date de fin impérative $\tilde{d}_j$, date à laquelle la tâche $T_j$ devra avoir fini son exécution.
    \item un poids (priorité) $w_j$, qui représente l'urgence relative de la tâche $T_j$.
\end{enumerate}

Dans toute la suite, nous supposerons que les paramètres liés aux tâches $p_j, ~r_j, \tilde{d}_j,
d_j$ et $w_j$ sont des entiers. %En fait cette hypothèse n'est pas pas trop restrictive

%Dans la suite, nous proposerons les définitions concernant la préemption des tâches, la duplication
%des tâches et sur les contraintes de précédence portant sur les tâches.

Un ordonnancement est dit préemptif si chaque tâche peut-être préemptée (interrompue) à n'importe
quel moment et son exécution peut-être reprise également à n'importe quel moment. Si la préemption
des tâches est interdite, nous dirons que l'ordonnancement est non-préemptif.

%La duplication des tâches peut-être autorisée ou non selon les problèmes étudiés. Dans le cas,

Les tâches de $\mathcal{T}$ à exécuter sur les processeurs de $\mathcal{P}$ peuvent être dupliquées.
En effet, dans le but de réduire l'influence des délais de communication (voir ci-après), dans le
cas où la duplication des tâches est autorisée, des copies de tâches peuvent être produites. Nous
supposerons par la suite, que les tâches originelles et leurs copies  admettent une même date de
début d'exécution.

Sur l'ensemble des tâches $\mathcal{T}$, des contraintes de précédence peut-être définies. $T_i
\prec T_j$ signifie que la tâche $T_j$ ne pourra commencer son exécution que si la tâche $T_i$ a
fini la sienne. En d'autres termes, nous pouvons définir sur l'ensemble $\mathcal{T}$ une relation
de précédence donnée par $\prec$. Les tâches de l'ensemble $\mathcal{T}$ sont dites dépendantes si
au moins deux tâches de $\mathcal{T}$ admettent un relation de précédence. Dans le cas contraire,
nous dirons que les tâches sont indépendantes.
Les tâches soumises à des relations de précédence, seront représentés par un graphe orienté (valué
ou non, selon la présence ou non de délais de communication) où les sommets correspondent aux tâches
et les arcs les contraintes de précédence. Nous supposerons que nous n'avons pas d'arc transitif
dans le graphe de précédence.


Une tâche $T_j$ est dite disponible à l'instant $t$ si $r_j \leq t$ et  si tous ces prédécesseurs
(en respectant les contraintes de précédence) ont fini leur exécution avant l'instant $t$.

\subsection{Critères d'optimalité}

Maintenant, nous donnons les définitions concernant les ordonnancements et les critères
d'optimalité. Un ordonnancement est une affectation des tâches de $\mathcal{T}$ sur les processeurs
de $\mathcal{P}$ telles que les conditions suivantes soient satisfaites :
\begin{itemize}
    \item à chaque instant un processeur exécute au plus une tâche et chaque tâche est exécutée par un seul processeur,

    \item chaque tâche $T_j$ est exécutée durant l'intervalle $[r_j,\infty)$.
    \item toutes les tâches sont ordonnancées.
    \item si deux tâches $T_i$ et $T_j$ admettent une relation de précédence $T_i \prec T_j$,
        alors la tâche $T_j$ ne peut commencer son exécution qu'après la fin d'exécution de
        $T_i$.
    \item dansle cas d'ordonnancements non préemptifs, aucune tâche ne peut-être interrompue.
        Dans le cas contraire, le nombre de préemptions est fini.
    \item dansle cas où la duplication est non autorisée, aucune tâche ne peut-être dupliquée.
        Dans le cas contraire, nous supposerons que le nombre de copies d'une tâche est fini.
\end{itemize}



Pour représenter les ordonnancements, nous utiliserons un diagramme de Gantt. Un exemple
d'utilisation du diagramme de Gantt est donné à la figure \ref{fig:gantt}. Dans cette figure,
nous avons à notre disposition trois processeurs identiques sur lesquels nous devons exécuter un
ensemble de huit tâches, ayant des durées différentes, soumises à des contraintes de précédence
données par lagraphe de précédence de la figure \ref{fig:gantt0}. Dans cette figure les
valuations surun sommet $x$ correspondent à son temps d'exécution sur un processeur. Pour
chaque tâche $T_j, i=1,\ldots n$ exécutée dans un ordonnancement donné, nous pouvons calculer la
valeur de chaque paramètre suivant :


\begin{itemize}
    \item le temps de  complétion noté $C_j=t_j+p_j$ où $t_j$ désigne la date de début d'exécution de la tâche $T_j$,
    \item le temps d'attente $F_j=C_j-(r_j+p_j)$,
    \item le retard algébrique noté $L_j=C_j-d_j$,
    \item le retard absolue $D_j=\max\{C_j-d_j,0\}$,
    \item l'avance de la tâche $E_j=\max\{d_j-C_j,0\}$,
        %\itemet bien sûr les versions pondérées
\end{itemize}

\section{Classification des problèmes d'ordonnancement  déterministes}



\subsection{Présentation de la notation à trois champs}

La grande variété des problèmes d'ordonnancement a motivé l'introduction d'une notation synthétique
pour classifier les schémas. Nous reprenons la notation proposée par Graham et al. \cite{graham} et
par B\l a\. zewicz et al. \cite{blazewicz} et nous proposons une extension.
%Nous allons étendre les notations  à notre modèle.

Pour définir un problème d'ordonnancement, il est commode d'introduire trois paramètres
$\alpha,\beta,\gamma$. Ces trois paramètres permettent de définir tous les ordonnancements
statiques, et ils permettent de synthétiser la formulation des divers problèmes. Détaillons ces
trois paramètres :

\begin{itemize}

    \item Le paramètre $\alpha=\alpha_1(\alpha_2)$\footnote{Il se peut que des combinaisons n'aient
        pas de sens.} définit l'environnement des processeurs (nombre de processeurs, modèle
        hiérarchique, topologie, $\ldots$)

        % la présence ou non d'un modèle hiérarchique, le type de processeur et leur nombre.

        $\ast \alpha_1= \in \{ P, \bar{P}, P_m,\varnothing  \} \mbox{ et } \alpha_2 \in \{ P,
        \bar{P}, P_m, Q,R,O,F,J  \}$

        \begin{itemize}

            \item Si $\alpha_1 \neq \varnothing$ alors nous sommes en présence d'un modèle
                hiérarchique.
            \item Si $\alpha_i=P$ signifie que les processeurs sont identiques et que leur nombre
                $m$ est une entrée du problème.

            \item Si $\alpha_i=\bar{P}$ ou $\alpha_2=\infty$ signifie que les processeurs sont
                identiques, et que leur nombre est suffisant ou non bornée.

            \item Si $\alpha_i=P_{m}, ~i \in \{1,2\}$ signifie que le  nombre de processeurs identiques est fixé.
            \item Si $\alpha_i=Q$, les processeurs sont uniformes,
            \item Si $\alpha_i=R$, les processeurs sont généraux,
            \item Si $\alpha_i=O$, les processeurs sont dédiés (modèle Open-Shop),
            \item Si $\alpha_i=F$, les processeurs sont dédiés (modèle Flow-Shop),
            \item Si $\alpha_i=J$, les processeurs sont dédiés (modèle Job-shop).
        \end{itemize}

    \item Le paramètre $\beta$  définit le type d'application et ces caractéristiques c'est-à-dire
        le graphe de précédence, le coût des communications, les temps d'exécution des tâches,
        l'autorisation ou pas de la duplication, l'autorisation ou pas de la préemption.

        De manière synthétique, on obtient $\beta=\beta_1\beta_2\beta_3\beta_4\beta_5$ où :

        $\ast \beta_1 \in \{ prec,arbre,chaine,\ldots, .\}$

        \begin{itemize}

            \item Si $\beta_1=$prec (le graphe est quelconque).

            \item Si $\beta_1=$arbre (le graphe est un arbre)

            \item $\vdots$

            \item Si $\beta_1=.$ (les t\^aches sont indépendantes)

        \end{itemize}

        $\ast \beta_2 \in \{ com,c_{jk},c ,.\}$

        \begin{itemize}

            \item Si $\beta_2=$com (les temps de communication sont donnés par le graphe).

            \item Si $\beta_2=c_{jk}$ (représente le temps de communication entre la tâche $j$ et la
                tâche $k$).

            \item Si $\beta_2=c$ (le temps de communication est constant et égal entre chaque paire
                de tâches).

            \item $\beta_2=.$ (il n'existe pas de communication).

        \end{itemize}

        $\ast \beta_3 \in \{p_j,.\}$

        \begin{itemize}

            \item Si $\beta_3=p_j=1$ (toutes les t\^aches ont une durée unitaire).

            \item Si $\beta_3=.$ (les durées d'exécution sont définies par le graphe).

        \end{itemize}

        $\ast \beta_4 \in \{dup,.\}$

        \begin{itemize}

            \item Si $\beta_4=dup$ (on autorise la duplication des t\^aches).

            \item Si $\beta_4=.$ (on n'autorise pas la duplication des t\^aches).

        \end{itemize}



        $\ast \beta_5 \in \{pmtp,.\}$

        \begin{itemize}

            \item Si $\beta_5=pmtp$ (on autorise l'interruption d'une tâche).

            \item Si $\beta_5=.$ (on n'autorise pas l'interruption d'une tâche).

        \end{itemize}

    \item Le paramètre $\gamma$ définit la fonction objective ou le critère d'optimalité. Dans toute
        la suite de cette thèse, nous considérons les deux fonctions objectives suivantes :
        \begin{itemize}
            \item la minimisation de la longueur de l'\ordofin, notée $C_{max}=\max_i \{t_i+p_i\}$
                (où $p_j$ est la durée d'exécution de la tâche $j$ et $t_j$ est la date de début
                d'exécution) par la suite, qui est le temps d'achèvement. De plus, $C^{opt}_{max}$
                désignera la longueur optimale de l'ordonnancement.
            \item la minimisation de la somme des temps de complétude noté $\sum_{j} C_j$ avec
                $C_j=t_j+p_j$.
        \end{itemize}


\end{itemize}
\end{document}
