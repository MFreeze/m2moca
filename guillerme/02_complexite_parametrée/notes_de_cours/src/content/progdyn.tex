% {{{ PROG DYN
Les algorithmes polynomiaux utilisant la programmation dynamique pour des graphes présentant une
largeur arborescente bornée suivent en général le même schéma présenté à la
section~\ref{pd_scheme}. Pour la plupart, ils s'appuient sur le principe de décomposition
arborescente simple que nous allons présenter.

%% {{{ DECOMPOSITION ARBORESCENTE SIMPLE
\section{Décomposition arborescente simple}

\begin{ndf}
    Étant donné un graphe $G=(V,E)$, une décomposition arborescente enracinée $\mathcal{T}_G = (T,
    \{X_t : t \in T\})$ de $G$ est dite simple si chacun des sacs $t$ de $T$ appartient à l'un des
    quatre types suivant :
    \begin{enumerate}
        \item Feuille : $t$ n'admet aucun fils et $|X_t| = 1$ (figure~\ref{fig:feuille})
        \item Ajout : $t$ admet un fils unique $t_f$ tel que $X_t = X_{t_f} \cup v$ avec $v \not
            \in X_{t_f}$ (figure~\ref{fig:ajout})
        \item Suppression : $t$ admet un fils unique $t_f$ tel que $X_{t_f} = X_t \cup v$ avec $v
            \not \in X_t$ (figure~\ref{fig:suppression}) 
        \item Fusion : $t$ admet deux fils $t_{f1}$ et $t_{f2}$ tels que $X_t = X_{t_{f1}} =
            X_{t_{f2}}$ (figure~\ref{fig:fusion})
    \end{enumerate}
\end{ndf}

\begin{nthrm}
    \label{pd_red}
    Étant donnée une décomposition arborescente enracinée $\Taug$ d'un graphe $G=(V,E)$ de largeur
    $k$ comportant $c$ sacs, il est possible de se ramener à une décomposition arborescente simple de
    largeur $k$ comportant $k^2.c$ sacs en temps $O(kn)$.
\end{nthrm}

\begin{proof}
    Le principe de la transformation est décrit par Kloks~\cite{klo94}
\end{proof}

Réduire une décomposition arborescente à une décomposition arborescente simple n'apporte pas, en
général, de nouvelles possibilité d'un point de vue algorithmique, mais cela permet de
considérablement simplifier la construction d'algorithmes.

% {{{ FIGURES
\begin{figure}
    \begin{center}
        \hfill
        \subfloat[Feuille]{\label{fig:feuille}
            \begin{tikzpicture}
                \node[patatoid, minimum width=15mm] {$\bullet$};

            \end{tikzpicture}
        }
        \hfill
        \subfloat[Ajout]{\label{fig:ajout}
            \begin{tikzpicture}
                \node[patatoid] (1) {$\bullet$ $\bullet$ $\bullet$ \textcolor{blue}{$\bullet$}};
\node[patatoid, below=of 1] (2) {$\bullet$ $\bullet$ $\bullet$};


\draw[edge] (1) to (2);

            \end{tikzpicture}
        }
        \hfill
        \subfloat[Suppression]{\label{fig:suppression}
            \begin{tikzpicture}
                \node[patatoid] (1) {$\bullet$ $\bullet$ $\bullet$};
\node[patatoid, below=of 1] (2) {$\bullet$ $\bullet$ $\bullet$ \textcolor{blue}{$\bullet$}};


\draw[edge] (1) to (2);

            \end{tikzpicture}
        }
        \hfill
        \subfloat[Fusion]{\label{fig:fusion}
            \begin{tikzpicture}
                \node[patatoid] (1) {$\bullet$ $\bullet$ $\bullet$};
\node[patatoid, below right=of 1] (2) {$\bullet$ $\bullet$ $\bullet$};
\node[patatoid, below left=of 1] (3) {$\bullet$ $\bullet$ $\bullet$};

\draw[edge] (2) to (1) to (3);

            \end{tikzpicture}
        }
        \hfill
        \caption{Les différents types de sac d'une décomposition arborescente simple}
        \label{type_sac}
    \end{center}
\end{figure}
        
% }}}

%% }}}

%% {{{ PRINCIPE
\section{Le principe des algorithmes de programmation dynamique sur une décomposition arborescente}
\label{pd_scheme}

Une grande partie des algorithmes FPT se basant sur la largeur arborescente d'un graphe $G = (V, E)$
fonctionnent de la manière suivante :
\begin{enumerate}
    \item Dans un premier temps, l'algorithme se charge de trouver une décomposition arborescente
        enracinée $\Taug$ de largeur bornée par une constante\footnote{Il n'est pas nécessaire que
        cette valeur soit optimale.}
    \item Une décomposition arborescente simple $\Taugstar$ est calculée à partir de $\mathcal{T}_G$
    \item Puis vient l'algorithme programmation dynamique travaillant sur $\mathcal{T}_G^*$, il
        travaille des feuilles vers la racine. Il calcule pour chaque sac, l'ensemble des solutions
        réalisables à partir des solutions réalisables de son ou ses fils. Ces dernières sont
        stockées dans des tables, chaque sac possédant sa table contenant les solutions partielles
        qui lui sont propres.  Ainsi, la table de la racine de l'arbre contiennent
        les solutions au problème étudié
\end{enumerate}

% TODO : il existe des algorithmes qui calculent en temps poly si une décomposition de taille k
% existe

On se rend compte, alors, de l'intérêt de la décomposition arborescente simple, puisqu'elle limite
le nombre de n\oe uds différents, elle limite donc le nombre de cas à traiter, comme nous allons le
voir.

% TODO: étoffer cette partie
%% }}}
% }}}

