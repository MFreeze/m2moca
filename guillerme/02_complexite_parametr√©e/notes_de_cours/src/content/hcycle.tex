% {{{ RAPPEL SUR LE CYCLE HAMILTONIEN
%% {{{ CYCLE HAMILTONIEN DEFINITION DU PROBLEME
Avant toute chose, définissons de manière formelle le problème sur lequel nous allons travailler.

\dfpb{\hcycle}{$G(V, E)$, un graphe non orienté}{Existe-t-il un chemin $H_G$ de $G$, tel
que tous les sommets de $G$ sont visités par $H_G$ une seule fois et revenant à son point de départ ?}
%% }}}

%% {{{ NP-COMPLETUDE
\section{$NP-$complétude}
\label{hcycle_npc}
\begin{nthrm}
    Le problème \hcycle est \npc
\end{nthrm}

\begin{proof}
    La preuve sera menée en trois parties. Dans un premier temps, nous décrirons une
    réduction de \vcover à ce problème, puis nous démontrerons qu'une solution existe pour \vcover
    si et seulement si le graphe construit admet un cycle Hamiltonien. Enfin nous clôturerons cette
    preuve par la preuve que cette réduction se fait en temps polynomial.

    Définissons dans un premier temps le problème \vcover que nous allons utiliser.

    \dfpb{\vcover}{$G(V, E)$ un graphe non orienté et $k$ un entier}{Existe-t-il un ensemble $VC$ de
        sommets de $V$ de taille $k$ tel que toute arête de $E$ est adjacente à au moins un sommet
    de $VC$ ?}

    Ce problème a été démontré \npc~\cite{karp}, dans ce même article Karp démontre aussi
    la $NP-$complétude de \hcycle. %Décrire rapidement la technique utilisée par Karp
    La technique de réduction que nous utilisons consiste à remplacer les
    arêtes du graphe de départ $G$, pour lequel une couverture est cherchée, par un motif
    particulier dans le graphe d'arrivée $G'$, pour lequel nous cherchons un cycle Hamiltonien.
    Il existe trois façons de parcourir ce motif de façon à ce que tous les sommets ne soient
    visités qu'une seule fois, correspondant à chacune des trois façons de couvrir l'arête $(uv)$
    ainsi remplacée. Donc pour une arête $(uv) \in E$, le chemin
    Hamiltonien décrit à la figure~\ref{motif_u} représente le cas où $u$ appartient au \vcover de
    taille $k$, celui de la figure~\ref{motif_v} représente le cas où $v$ appartient au \vcover de
    taille $k$ et enfin la figure~\ref{motif_uv} représente le cas où $u$ et $v$ appartiennent tous
    deux au \vcover de taille $k$.

    % {{{ MOTIFS DE REMPLACEMENT
    \begin{figure}
        \begin{center}
            \begin{tikzpicture}
                \node[vertex] (u1) {$u_1$};
\node[vertex, below=of u1] (v1) {$v_1$};

\foreach \x [evaluate=\x as \y using (\x - 1)] in {2,...,6} {
    \node[vertex, right=of u\y] (u\x)  {$u_{\x}$};
    \node[vertex, right=of v\y] (v\x)  {$v_{\x}$};
    \draw[edge] (u\y) to (u\x);
    \draw[edge] (v\y) to (v\x);
}

\draw[edge] (u1) to (v3);
\draw[edge] (v1) to (u3);
\draw[edge] (u4) to (v6);
\draw[edge] (v4) to (u6);


            \end{tikzpicture}
            \caption{Motif de remplacement d'une arête}
            \label{motif_npc}
        \end{center}
    \end{figure}

    \begin{figure}
        \begin{center}
            \begin{tikzpicture}
                \node[vertex] (u1) {$u_1$};
\node[vertex, below=of u1] (v1) {$v_1$};

\foreach \x [evaluate=\x as \y using (\x - 1)] in {2,...,6} {
    \node[vertex, right=of u\y] (u\x)  {$u_{\x}$};
    \node[vertex, right=of v\y] (v\x)  {$v_{\x}$};
    \draw[edge] (u\y) to (u\x);
    \draw[edge] (v\y) to (v\x);
}

\draw[edge] (u1) to (v3);
\draw[edge] (v1) to (u3);
\draw[edge] (u4) to (v6);
\draw[edge] (v4) to (u6);



%\foreach \x [evaluate=\x as \y using (\x + 1)] in {1, 2, 4, 5} {
%    \draw[draw=red, very thick] (u\x) to (u\y);
%    \draw[draw=red, very thick] (v\x) to (v\y);
%}

\draw[draw=red, line width=2pt] (u1) to (u2);
\draw[draw=red, line width=2pt] (u2) to (u3);
\draw[draw=red, line width=2pt] (u3) to (v1);
\draw[draw=red, line width=2pt] (v1) to (v2);
\draw[draw=red, line width=2pt] (v2) to (v3);
\draw[draw=red, line width=2pt] (v3) to (v4);
\draw[draw=red, line width=2pt] (v4) to (v5);
\draw[draw=red, line width=2pt] (v5) to (v6);
\draw[draw=red, line width=2pt] (v6) to (u4);
\draw[draw=red, line width=2pt] (u4) to (u5);
\draw[draw=red, line width=2pt] (u5) to (u6);

            \end{tikzpicture}
            \caption{Cas 1 : $u$ appartient au \vcover}
            \label{motif_u}
        \end{center}
    \end{figure}

    \begin{figure}
        \begin{center}
            \begin{tikzpicture}
                \node[vertex] (u1) {$u_1$};
\node[vertex, below=of u1] (v1) {$v_1$};

\foreach \x [evaluate=\x as \y using (\x - 1)] in {2,...,6} {
    \node[vertex, right=of u\y] (u\x)  {$u_{\x}$};
    \node[vertex, right=of v\y] (v\x)  {$v_{\x}$};
    \draw[edge] (u\y) to (u\x);
    \draw[edge] (v\y) to (v\x);
}

\draw[edge] (u1) to (v3);
\draw[edge] (v1) to (u3);
\draw[edge] (u4) to (v6);
\draw[edge] (v4) to (u6);



%\foreach \x [evaluate=\x as \y using (\x + 1)] in {1, 2, 4, 5} {
%    \draw[draw=red, very thick] (u\x) to (u\y);
%    \draw[draw=red, very thick] (v\x) to (v\y);
%}

\draw[draw=red, line width=2pt] (v1) to (v2);
\draw[draw=red, line width=2pt] (v2) to (v3);
\draw[draw=red, line width=2pt] (v3) to (u1);
\draw[draw=red, line width=2pt] (u1) to (u2);
\draw[draw=red, line width=2pt] (u2) to (u3);
\draw[draw=red, line width=2pt] (u3) to (u4);
\draw[draw=red, line width=2pt] (u4) to (u5);
\draw[draw=red, line width=2pt] (u5) to (u6);
\draw[draw=red, line width=2pt] (u6) to (v4);
\draw[draw=red, line width=2pt] (v4) to (v5);
\draw[draw=red, line width=2pt] (v5) to (v6);

            \end{tikzpicture}
            \caption{Cas 2 : $v$ appartient au \vcover}
            \label{motif_v}
        \end{center}
    \end{figure}

    \begin{figure}
        \begin{center}
            \begin{tikzpicture}
                \node[vertex] (u1) {$u_1$};
\node[vertex, below=of u1] (v1) {$v_1$};

\foreach \x [evaluate=\x as \y using (\x - 1)] in {2,...,6} {
    \node[vertex, right=of u\y] (u\x)  {$u_{\x}$};
    \node[vertex, right=of v\y] (v\x)  {$v_{\x}$};
    \draw[edge] (u\y) to (u\x);
    \draw[edge] (v\y) to (v\x);
}

\draw[edge] (u1) to (v3);
\draw[edge] (v1) to (u3);
\draw[edge] (u4) to (v6);
\draw[edge] (v4) to (u6);



%\foreach \x [evaluate=\x as \y using (\x + 1)] in {1, 2, 4, 5} {
%    \draw[draw=red, very thick] (u\x) to (u\y);
%    \draw[draw=red, very thick] (v\x) to (v\y);
%}

\draw[draw=red, line width=2pt] (u1) to (u2);
\draw[draw=red, line width=2pt] (u2) to (u3);
\draw[draw=red, line width=2pt] (u3) to (u4);
\draw[draw=red, line width=2pt] (u4) to (u5);
\draw[draw=red, line width=2pt] (u5) to (u6);

\draw[draw=red, line width=2pt] (v1) to (v2);
\draw[draw=red, line width=2pt] (v2) to (v3);
\draw[draw=red, line width=2pt] (v3) to (v4);
\draw[draw=red, line width=2pt] (v4) to (v5);
\draw[draw=red, line width=2pt] (v5) to (v6);

            \end{tikzpicture}
            \caption{Cas 3 : $u$ et $v$ appartiennent au \vcover}
            \label{motif_uv}
        \end{center}
    \end{figure}
    % }}}

    Il s'agit maintenant de relier les motifs entre eux, pour ce faire considérons deux arêtes
    $(uv)$ et $(uw)$ de $E$ adjacentes toutes deux au sommet $u$. Chacune de ces arêtes est remplacée
    par le motif présenté précédemment, appelons $U_v$ l'ensemble des sommets $\{u_1, u_2, \dots,
    u_6\}$ du motif correspondant à l'arête $(uv)$ et $U_w$ l'ensemble des sommets $\{u_1, u_2,
    \dots, u_6\}$ du motif correspondant à l'arête $(uw)$. Dans la construction finale, $U_v$ et
    $U_w$ sont reliés entre eux par une arête, comme présenté figure~\ref{hcycle_couple}. Et l'on
    fait de même pour tous les sommets et arêtes de $G'$. Chaque sommet $v$ de $G$ est alors représenté
    par une chaîne dans $G'$ composée de $6*d(v)$ sommets.

    Pour illustrer la transformation, considérons le graphe $G = (V, E)$ dont une représentation est
    donnée à la figure~\ref{hcycle_graphe_g}. La figure~\ref{hcycle_graphe_gprime} représente le
    graphe obtenu après transformation des arêtes. On peut ainsi se rendre compte qu'il existe un
    \vcover de taille $k$ si et seulement si il existe $k$ chemins partant du côté gauche et
    arrivant du côté droit passant par l'ensemble des sommets du graphe $G'$.

    Pour arriver à une recherche de cycle Hamiltonien dans le graphe $G'$ correspondant à un \vcover de taille
    $k$, il suffit de rajouter $k$ sommets $s_1, s_2, \dots, s_k$ reliant le côté droit au côté
    gauche, de sorte qu'une fois arrivé au bout de la chaîne représentant un sommet, il est possible
    de retourner du côté gauche sur une nouvelle chaîne représentant un autre sommet.

    La dernière phase de la transformation, pour un \vcover de taille $2$ est détaillée
    figure~\ref{hcycle_last}, les sommets $1_{deb}$ et $1_{fin}$ désignant respectivement le premier
    et le dernier n\oe ud de la chaîne représentant le sommet $1$. 

    La recherche d'un vertex cover de taille $k$ dans $G$ nous amène donc alors à al recherche d'un
    chemin Hamiltonien dans le graphe $G'$ dont une représentation est donnée à la
    figure~\ref{hcycle_finish}.

    % {{{ TRANSFORMATION
    \begin{figure}
        \begin{center}
            \begin{tikzpicture}[scale=0.8, every node/.style={transform shape}]
                \node[tmvert] (u0) {$U$};
\node[tmvert, below=of u0] (tp0) {};
\node[tmvert, below=of tp0] (v0) {$V$};
\node[tmvert, below=of v0] (tp1) {};
\node[tmvert, below=of tp1] (w0) {$W$};

\node[mvert, right=of u0] (u1) {$u_1$};
\node[mvert, right=of v0] (v1) {$v_1$};

\foreach \x [evaluate=\x as \y using (\x - 1)] in {2,...,6} {
    \node[mvert, right=of u\y] (u\x)  {$u_{\x}$};
    \node[mvert, right=of v\y] (v\x)  {$v_{\x}$};
    \draw[edge] (u\y) to (u\x);
    \draw[edge] (v\y) to (v\x);
}

\draw[edge] (u1) to (v3);
\draw[edge] (v1) to (u3);
\draw[edge] (u4) to (v6);
\draw[edge] (v4) to (u6);

\node[mvert, right=of u6] (uw1) {$u'_{1}$};
\node[tmvert, below=of v6] (t2) {};
\node[tmvert, below=of t2] (t3) {};
\node[mvert, right=of t3] (w1) {$w_1$};

\draw[edge] (u6) to (uw1);

\foreach \x [evaluate=\x as \y using (\x - 1)] in {2,...,6} {
    \node[mvert, right=of uw\y] (uw\x)  {$u'_{\x}$};
    \node[mvert, right=of w\y] (w\x)  {$w_{\x}$};
    \draw[edge] (uw\y) to (uw\x);
    \draw[edge] (w\y) to (w\x);
}

\draw[edge] (uw1) to (w3);
\draw[edge] (w1) to (uw3);
\draw[edge] (uw4) to (w6);
\draw[edge] (w4) to (uw6);

\draw[edge] (u0) to (u1);
\draw[edge] (v0) to (v1);
\draw[edge] (w0) to (w1);

            \end{tikzpicture}
            \caption{Liaison de deux motifs représentant deux arêtes adjacentes à un même sommet}
            \label{hcycle_couple}
        \end{center}
    \end{figure}

    \begin{figure}
        \begin{center}
            \begin{tikzpicture}
                \node[vertex] (1) {$1$};
\node[vertex, below=of 1] (2) {$2$};
\node[vertex, below left=of 2] (3) {$3$};
\node[vertex, below right=of 2] (4) {$4$};

\draw[edge, very thick, violet] (1) to (2);
\draw[edge, very thick, green] (2) to (3);
\draw[edge, very thick, orange] (2) to (4);
\draw[edge, very thick, blue] (3) to (4);


            \end{tikzpicture}
            \caption{Exemple de transformation : le graphe $G$}
            \label{hcycle_graphe_g}
        \end{center}
    \end{figure}

    \begin{figure}
        \begin{center}
            \begin{tikzpicture}[scale=0.6, every node/.style={transform shape}, tmp/.style={inner
                    sep=0mm, node distance=3mm}]
                \foreach \x/\y in {2/1, 4/2, 6/3, 8/4}{
    \node[tvert] at (0, \x) (\y) {$\y$};
}

\foreach \w in {0.5}{
    \foreach \x/\y/\col in {1/2/violet, 2/3/green, 3/4/blue, 4/2/orange}{
        \pgfmathparse{int(30*\w + 1)}
        \node[tvert] at (\pgfmathresult, 2*\x) (f\x) {$\x$};
        \draw[edge] (\x) to (f\x);

        \foreach \z in {1, 2, 3, 4, 5, 6}{
            \pgfmathparse{(\x-1)*(7*\w)+1 + (\z*\w)}
            \node[tmp,\col] at (\pgfmathresult, \x*2) (\x\y\z) {$\bullet$};
            \node[tmp,\col] at (\pgfmathresult, \y*2) (\y\x\z) {$\bullet$};
        }

        \foreach \z [evaluate=\z as \a using \z -1] in {2, 3, 4, 5, 6}{
            \draw[edge, \col] (\x\y\a) to (\x\y\z);
            \draw[edge, \col] (\y\x\a) to (\y\x\z);
        }

        \draw[edge,\col] (\x\y1) to (\y\x3);
        \draw[edge,\col] (\x\y3) to (\y\x1);
        \draw[edge,\col] (\x\y4) to (\y\x6);
        \draw[edge,\col] (\x\y6) to (\y\x4);
    }
}

            \end{tikzpicture}
            \caption{Exemple de transformation : le graphe $G'$ obtenu après transformation de $G$}
            \label{hcycle_graphe_gprime}
        \end{center}
    \end{figure}

    \begin{figure}
        \begin{center}
            \begin{tikzpicture}[scale=2, tmp/.style={vertex, minimum size=10mm}]
                \node[tmp] at (3, 3.5) (s1) {$s_1$};
\node[tmp] at (3, 1.5) (s2) {$s_2$};

\foreach \y in {1, 2, 3, 4}{
    \foreach \x/\lab in {0/fin, 6/deb}{
        \node[tmp] at (\x, \y) (\lab\y) {$\y_{\lab}$};
        \draw[edge] (\lab\y) to (s1);
        \draw[edge] (\lab\y) to (s2);
    }
}


            \end{tikzpicture}
            \caption{Exemple de transformation : la dernière phase}
            \label{hcycle_last}
        \end{center}
    \end{figure}

    \begin{figure}
        \begin{center}
            \begin{tikzpicture}[scale=0.6, every node/.style={transform shape}, tmp/.style={inner
                    sep=0mm, node distance=3mm}]
                \foreach \x/\y in {2/1, 4/2, 6/3, 8/4}{
    \node[tvert] at (0, \x) (\y) {$\y$};
}

\foreach \w in {0.5}{
    \pgfmathparse{int(30*\w + 1)}
    \node[vert] at (\pgfmathresult/2, 12) (s1) {$s_1$};
    \node[vert] at (\pgfmathresult/2, -2) (s2) {$s_2$};

    \node[tvert] at (-2, 0) (t1s2) {};
    \node[tvert] at (\pgfmathresult + 2, 0) (t2s2) {};
    \node[tvert] at (-2, 10) (t1s1) {};
    \node[tvert] at (\pgfmathresult + 2, 10) (t2s1) {};

    \draw[edge] (s1) to[out=180, in=90] (t1s1);
    \draw[edge] (s1) to[out=0, in=90] (t2s1);
    \draw[edge] (s2) to[out=180, in=270] (t1s2);
    \draw[edge] (s2) to[out=0, in=270] (t2s2);

    \foreach \x/\y/\col in {1/2/violet, 2/3/green, 3/4/blue, 4/2/orange}{
        \node[tvert] at (\pgfmathresult, 2*\x) (f\x) {$\x$};
        \draw[edge] (\x) to (f\x);

        \foreach \z in {1, 2, 3, 4, 5, 6}{
            \pgfmathparse{(\x-1)*(7*\w)+1 + (\z*\w)}
            \node[tmp,\col] at (\pgfmathresult, \x*2) (\x\y\z) {$\bullet$};
            \node[tmp,\col] at (\pgfmathresult, \y*2) (\y\x\z) {$\bullet$};
        }

        \foreach \z [evaluate=\z as \a using \z -1] in {2, 3, 4, 5, 6}{
            \draw[edge, \col] (\x\y\a) to (\x\y\z);
            \draw[edge, \col] (\y\x\a) to (\y\x\z);
        }

        \draw[edge,\col] (\x\y1) to (\y\x3);
        \draw[edge,\col] (\x\y3) to (\y\x1);
        \draw[edge,\col] (\x\y4) to (\y\x6);
        \draw[edge,\col] (\x\y6) to (\y\x4);

        \draw[edge] (\x) to[out=180, in=270] (t1s1);
        \draw[edge] (\x) to[out=180, in=90] (t1s2);
        \draw[edge] (f\x) to[out=0, in=270] (t2s1);
        \draw[edge] (f\x) to[out=0, in=90] (t2s2);
    }
}


            \end{tikzpicture}
            \caption{Exemple de transformation : le graphe résultant de la transformation de $G$}
            \label{hcycle_finish}
        \end{center}
    \end{figure}

    % }}}

    ~\\
    Supposons à présent que nous ayons un \hcycle $H$ dans le graphe $G'$, nous avons construit le
    motif de façon à ce que, venant d'un sommet $s_i$, la chaîne correspondant au sommet $u$ choisi
    est entièrement traversée jusqu'à un autre sommet $s_j$. Ce procédé est itéré jusqu'à la
    couverture de tous les sommets $s$, sélectionnant ainsi $k$ sommet dans le graphe $G$. Puisque
    tous les sommets de $G'$ sont couverts, toutes les arêtes de $G$ le sont aussi, on obtient alors
    une couverture dans le graphe $G$.

    ~\\
    Dans l'autre sens, supposons que l'on ait une couverture $VC = {v_1, v_2, \dots, v_k}$ de $G$,
    il est alors possible de construire un chemin Hamiltonien dans $G'$ de la manière suivante :
    \begin{enumerate}
        \item partir de $s_1$ et choisir le sommet $v_1$ de la couverture
        \item parcourir le chemin correspondant à $v_1$ et pour chaque arête $(v_1h)$ de $G$, si $h
            \in VC$ alors parcourir seulement le côté $v_1$ de l'arête, sinon parcourir les deux
            côtés simultanément
        \item connecter la chaîne $v_1$ à $s_2$ et recommencer pour tous les sommets de $VC$
    \end{enumerate}
    En connectant la dernière chaîne au sommet $s_1$, on obtient un cycle Hamiltonien.

    ~\\
    Il existe donc un \vcover de taille $k$ dans $G$ si et seulement si il existe un $\hcycle$ dans
    le graphe $G'$ construit, de plus la réduction étant polynomial en temps et en espace, \hcycle
    est donc bien \npc.

\end{proof}
%% }}}

\section{Noyaux}
\label{hcycle_kbounds}

Dans cette partie du cours, nous présentons quelques résultats quant à l'existence d'un noyau pour
le problème \hcycle paramétré par des paramètres non standards. Je n'ai malheureusement pas le temps
de rentrer dans les détails.

\subsection{Paramétrisation par le nombre maximum de feuilles de $G$}

\dfpbp{\phcycle{Max Leaf Number}}{Un graphe $G = (V,E)$ et un entier $l$}{Si $l$ est le nombre
maximal de feuilles d'un arbre couvrant de $G$, déterminer si $G$ possède un cycle
Hamiltonien.}{$l$}

\begin{nthrm}
    Le problème \phcycle{Max Leaf Number} admet un noyau polynomial.
\end{nthrm}

Ceci a été démontré par Fellows, Lokshtanov, Misra, Mnich, Rosamond et Saurabh~\cite{Fell09}.

\subsection{Paramétrisation par la largeur arborescente}

Il s'agit là du problème pour lequel nous allons détaillé un algorithme de programmation dynamique.
Exposons le de manière formelle.

\dfpbp{\phcycle{Treewidth}}{Un graphe $G = (V,E)$ et une décomposition arborescente $\Tau$ et un
entier $k$}{Si $k$ est la largeur arborescente de $\Tau$, déterminer si $G$ possède un cycle
Hamiltonien.}{$k$}

\begin{nthrm}
    Le problème \phcycle{Treewidth} n'admet pas de noyau polynomial sauf si $NP \subseteq Co-NPpoly$
\end{nthrm}

% }}}

