On peut représenter de deux manières le diagramme donné. La première d'entre elle est représentée à
la figure~\ref{pnet1exo4} représentant chacune des tâches par une transition, ainsi chaque place
représente l'état intermédiaire entre la fin d'une tâche et le début d'une autre. La seconde est
réprésentée par la figure~\ref{pnet2exo4} représentant chacune des tâches par des places, ainsi
chaque transition marque la fin d'une tâche et le commencement d'une autre tâche.

Il est aisé de remarquer, en observant les réseaux de la figure~\ref{pnetallexo4}, que la seconde
représentation comporte non seulement moins de places et de transitions, mais paraît aussi plus
intuitive. En effet, pour la première représentation, la transition est franchie une fois la tâche
terminée, donc pour connaître la tâche en cours il faut se référer à la transition suivante, alors
que pour la seconde réprésentation, la tâche en cours est immédiatement connue. De plus, la première
représentation, donne une impresssion d'instantanéité des tâches, qui ne se révèle pas dans la
seconde. 

En conclusion, j'aurais tendance à provilégier la représentation des tâches par des places plutôt
que par des transitions.

\begin{figure}
    \begin{center}
        \subfloat[Première représentation du diagramme]{\label{pnet1exo4}
        \begin{tikzpicture}[scale=0.6, every transition/.style={minimum width=7mm, minimum
        height=2mm, transform shape}, every node/.style={transform shape},>=latex]
            \node[place, tokens=1] (p0) {};

            \node[transition, below=of p0, label=Excavation]        (t1) {}
                edge[pre] (p0);
            \node[place, below=of t1]                               (p1) {}
                edge[pre] (t1);

            \node[transition, below=of p1, label=Contreting]        (t2) {}
                edge[pre] (p1);
            \node[place, below=of t2]                               (p2) {}
                edge[pre] (t2);

            \node[transition, below=of p2, label=Bricklaying]       (t3) {}
                edge[pre] (p2);
            \node[place, below right=of t3]                         (p3) {}
                edge[pre] (t3);
            \node[place, below left=of t3]                          (p4) {}
                edge[pre] (t3);

            \node[transition, below=of p3, label=Roof] (t4) {}
                edge[pre] (p3);
            \node[place, below right=of t4]                         (p5) {}
                edge[pre] (t4);
            \node[place, below left=of t4]                          (p6) {}
                edge[pre] (t4);

            \node[transition, below=of p4, label=Windows and Doors]  (t5) {}
                edge[pre] (p4);
            \node[place, below left=of t5]                          (p7) {}
                edge[pre] (t5);

            \node[transition, below=of p5, label=Ceilings]          (t6) {}
                edge[pre] (p5);
            \node[place, below=of t6]                               (p8) {}
                edge[pre] (t6);

            \node[transition, below=of p6, label=Floors]            (t7) {}
                edge[pre] (p6);
            \node[place, below=of t7]                               (p9) {}
                edge[pre] (t7);

            \node[transition, below=of p7, label=Toilet]            (t8) {}
                edge[pre] (p7);
            \node[place, below=of t8]                               (pa) {}
                edge[pre] (t8);

            \node[transition, below=of p9, label=Kitchen]           (t9) {}
                edge[pre] (p9);
            \node[place, below=of t9]                               (pb) {}
                edge[pre] (t9);

            \node[transition, below=of pb, label=Furniture]         (ta) {}
                edge[pre, bend right] (p8)
                edge[pre, bend left] (pa)
                edge[pre] (pb);
            \node[place, below=of ta]                               (pc) {}
                edge[pre] (ta);
            \node at (7, 0) {};
        \end{tikzpicture}}
    \subfloat[Seconde représentation du diagramme]{\label{pnet2exo4} \hfill
        \begin{tikzpicture}[scale=0.6, every transition/.style={minimum width=7mm, minimum
        height=2mm, transform shape}, every node/.style={transform shape},>=latex]
            \node[place, tokens=1, label=Excavation]                (p1) {};

            \node[transition, below=of p1]                          (t1) {}
                edge[pre] (p1);
            \node[place, below=of t1, label=Contreting]             (p2) {}
                edge[pre] (t1);

            \node[transition, below=of p2]                          (t2) {}
                edge[pre] (p2);
            \node[place, below=of t2, label=Bricklaying]            (p3) {}
                edge[pre] (t2);

            \node[transition, below=of p3]                          (t3) {}
                edge[pre] (p3);
            \node[place, below left=of t3, label=Windows and Doors] (p4) {}
                edge[pre] (t3);
            \node[place, below right=of t3, label=Roof](p5) {}
                edge[pre] (t3);

            \node[transition, below=of p4]                          (t4) {}
                edge[pre] (p4);
            \node[place, below left=of t4, label=Toilet]            (p6) {}
                edge[pre](t4);

            \node[transition, below=of p5]                          (t5) {}
                edge[pre] (p5);
            \node[place, below left=of t5, label=Floors]            (p7) {}
                edge[pre] (t5);
            \node[place, below right=of t5, label=Ceilings]         (p8) {}
                edge[pre] (t5);

            \node[transition, below=of p7]                          (t6) {}
                edge[pre] (p7);
            \node[place, below=of t6, label=Kitchen]                (p9) {}
                edge[pre] (t6);

            \node[transition, below=of p9]                          (t7) {}
                edge[pre] (p9)
                edge[pre, bend left] (p6)
                edge[pre, bend right] (p8);
            \node[place, below=of t7, label=Furniture]              (pa) {}
                edge[pre] node[auto] {$3$} (t7);
            \node[transition, below=of pa]                          (t8) {}
                edge[pre] (pa);
            \node[place, below=of t8, label=End]                    (pb) {}
                edge[pre] (t8);
            \node at (-7,0) {};
        \end{tikzpicture}}
    \end{center}
    \caption{Différentes représentations du diagramme}
    \label{pnetallexo4}
\end{figure}
