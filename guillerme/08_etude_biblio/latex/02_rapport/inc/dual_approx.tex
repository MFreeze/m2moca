\documentclass[a4paper,9pt]{article}

% {{{ PACKAGES
% {{{ ENCODAGE ET POLICE
\usepackage[utf8]{inputenc}
\usepackage[T1]{fontenc}
\usepackage[french]{babel}
\usepackage{eurosym}
% }}}

% {{{ FIGURES
\usepackage{graphicx}
\usepackage{subfig}
% }}}

% {{{ INSERTION DE CODE
\usepackage{moreverb}
% }}}

% {{{ INSERTION DE PDF
\usepackage{pdfpages}
% }}}

% {{{ RÉDUCTION DES MARGES
%\usepackage{fullpage}
%\usepackage[top=2.5cm, bottom=2.5cm, left=3.5cm, right=3.5cm]{geometry}
% }}}

% {{{ MATHÉMATIQUES
\usepackage{amsmath}
\usepackage{amsthm}
\usepackage{bbm}
\usepackage{amssymb}
\usepackage{stmaryrd}
% }}}

% {{{ GRAPHES
\usepackage{tikz}
\usepackage{pgf}
\usetikzlibrary{arrows,automata,shapes}
% }}}

% {{{ RESEAUX DE pETRI
\usetikzlibrary{petri}
% }}}

% {{{ ARBRES
\usetikzlibrary{trees}
% }}}

% {{{ DIAGRAMME DE GANTT
\usepackage{pgfgantt}
% }}}

% {{{ ALGORITHMES
\usepackage{algorithm}
\usepackage[noend]{algpseudocode}
%\usepackage{algorithmic}
%\usepackage{algorithmicx}
% }}}
% }}}

% {{{ NEWTHEOREM
% {{{ THÉORÈMES, lEMMES -- NUMÉROTATION PERSONNALISÉE
\theoremstyle{plain}
\newtheorem{thrm}{Théorème}[section]
\newtheorem{lemm}{Lemme}[subsection]
\newtheorem{corol}{Corollaire}[section]
% }}}

% {{{ THÉORÈMES, lEMMES -- NUMÉROTATION CLASSIQUE
\newtheorem{nthrm}{Théorème}
\newtheorem{nlemm}{Lemme}
\newtheorem{ncorol}{Corollaire}
\newtheorem{npb}{Problème}
% }}}

% {{{ THÉORÈMES, lEMMES -- NON NUMÉROTÉS
\newtheorem*{thrm*}{Théorème}
\newtheorem*{lemm*}{Lemme}
\newtheorem*{corol*}{Corollaire}
% }}}

% {{{ DÉFINITIONS, pROPRIÉTÉS -- NUMÉROTATION PERSONNALISÉE
\theoremstyle{definition}
\newtheorem{df}{Définition}[subsection]
\newtheorem{prop}{Propriété}[subsection]
\newtheorem{conj}{Conjecture}[subsection]
% }}}

% {{{ DÉFINITIONS, pROPRIÉTÉS -- NUMÉROTATION CLASSIQUE
\newtheorem{ndf}{Définition}
\newtheorem{nprop}{Propriété}
\newtheorem{nconj}{Conjecture}
% }}}

% {{{ DÉFINITIONS, pROPRIÉTÉS -- NON NUMÉROTÉS
\newtheorem*{df*}{Définition}
\newtheorem*{prop*}{Propriété}
\newtheorem*{conj*}{Conjecture}
% }}}

% {{{ AUTRES -- NUMÉROTATION PERSONNALISÉE
\theoremstyle{remark}
\newtheorem{nota}{Note}[subsubsection]
\newtheorem{pr}{Proposition}[subsection]
\newtheorem{rmq}{Remarque}[subsubsection]
% }}}

% {{{ AUTRES -- NUMÉROTATION CLASSIQUE
\newtheorem{nnota}{Note}
\newtheorem{npr}{Proposition}
\newtheorem{nrmq}{Remarque}
% }}}

% {{{ AUTRES -- NON NUMÉROTÉS
\newtheorem*{nota*}{Note}
\newtheorem*{pr*}{Proposition}
\newtheorem*{rmq*}{Remarque}
% }}}

% {{{ EXEMPLES
\newtheorem*{ex}{Exemple}
\newtheorem*{exo}{Exercice}
\newtheorem*{rappel}{Rappel}
% }}}
% }}}

% {{{ OPÉRATEURS MATHÉMATIQUES
\DeclareMathOperator{\poly}{\textbf{poly}}
\DeclareMathOperator{\gap}{\mbox{gap}}
\DeclareMathOperator{\Bg}{\mbox{Big}}
\DeclareMathOperator{\Sml}{\mbox{Sml}}
\DeclareMathOperator{\Diff}{\mbox{Diff}}
\DeclareMathOperator{\card}{\mbox{Card}}
\DeclareMathOperator{\Obj}{\mbox{Obj}}
\DeclareMathOperator{\concat}{\bullet}
\DeclareMathOperator{\sig}{\mbox{Sig}}
\DeclareMathOperator{\ver}{\mbox{Ver}}
\DeclareMathOperator{\dist}{\mbox{dist}}
\DeclareMathOperator{\bw}{\mbox{branchwidth}}
\DeclareMathOperator{\tw}{\mbox{tw}}
\DeclareMathOperator{\lm}{\leq_m}
\DeclareMathOperator{\lc}{\leq_c}
\DeclareMathOperator{\lex}{\mbox{lex}}
\DeclareMathOperator{\ord}{\mbox{ord}}
% }}}

% {{{ MACCROS
% {{{ MACROS DIVERSES
\newcommand{\npc}{$NP-$complet}
\newcommand{\npd}{$NP-$difficile}
\newcommand{\la}{largeur arborescente}
\newcommand{\legendre}[2]{\left ( \frac{#1}{#2} \right )}
\newcommand{\wqo}[0]{\emph{well-quasi-ordered}}
\newcommand{\Tau}[0]{\mathcal{T} = (T, \{X_t : t \in T\})}
\newcommand{\Taug}[0]{\mathcal{T}_G = (T, \{X_t : t \in T\})}
\newcommand{\Taustar}[0]{\mathcal{T}^* = (T, \{X_t : t \in T\})}
\newcommand{\Taugstar}[0]{\mathcal{T}_G^* = (T, \{X_t : t \in T\})}
\newcommand{\pdsol}[1]{(X_{#1}^0, X_{#1}^1, X_{#1}^2, M_{#1})}
% }}}
                
% {{{ NOM DE PROBLEMES
\newcommand{\hcycle}{\textsc{HAMILTONIAN CYCLE} }
\newcommand{\phcycle}[1]{\textsc{HAMILTONIAN CYCLE }paramétré par \textsc{#1} }
\newcommand{\vcover}{\textsc{VERTEX COVER} }
\newcommand{\twidth}{\textsc{TREE WIDTH} }
\newcommand{\wiset}{\textsc{WEIGHTED INDEPENDANT SET} }
\newcommand{\fvset}{\textsc{FEEDBACK VERTEX SET} }
\newcommand{\kvcover}{\textsc{K-Vertex Cover} }
\newcommand{\oct}{\textrm{\textsc{Odd Cycle Transversal}} }
\newcommand{\dlp}{\textrm{\textsc{Discrete Logarithm Problem}} }
\newcommand{\ecdlp}{\textrm{\textsc{Elliptic Curve Discrete Logarithm Problem}} }

% }}}

% {{{ MISE EN PAGE

\newcommand{\dfpbi}[3]{
    \begin{npb}[#1]
        \begin{itemize}
            \itemindent 10mm
            \item[\textbf{Données :}] #2
            \item[\textbf{Problème :}] #3
        \end{itemize}
    \end{npb}
}

\newcommand{\dfpb}[3]{
    \begin{npb}[#1]
        \begin{itemize}
            \itemindent 10mm
            \item[\textbf{Données :}] #2
            \item[\textbf{Question :}] #3
        \end{itemize}
    \end{npb}
}

\newcommand{\dfpbp}[4]{
    \begin{npb}[#1]
        \begin{itemize}
            \itemindent 10mm
            \item[\textbf{Données :}] #2
            \item[\textbf{Paramètre :}] #4
            \item[\textbf{Question :}] #3
        \end{itemize}
    \end{npb}
}
% }}}

% }}}
    
% {{{ TIKZ
%%% {{{ STYLE
\tikzstyle{edge}=[
	-,
	draw=black
]

\tikzstyle{sized}=[
    minimum width=9mm
]

\tikzstyle{tvertex}=[
	node distance=15mm,
    inner sep = 1mm
]

\tikzstyle{tvert}=[
	node distance=5mm,
    inner sep=1mm
]

\tikzstyle{tmvertex}=[
    tvertex,
    sized
]

\tikzstyle{tmvert}=[
    tvert,
    sized
]

\tikzstyle{mvertex}=[
    tmvertex,
	circle,
	draw=black,
	fill=black!50
]

\tikzstyle{mvert}=[
    tmvert,
	circle,
	draw=black,
	fill=black!50
]

\tikzstyle{vertex}=[
    tvertex,
	circle,
	draw=black,
	fill=black!50
]

\tikzstyle{vert}=[
    tvert,
	circle,
	draw=black,
	fill=black!50
]

\tikzstyle{treedec}=[
	vertex,
	node distance=15mm,
	draw=red,
	fill=red!50
]

\tikzstyle{treeedge}=[
	edge,
	draw=red
]

\tikzstyle{giedge}=[
	edge,
	draw=blue
]

\tikzstyle{bounding}=[
	thick, %ultra
	%dash pattern=on 1mm off 1mm,
	densely dotted,
	fill=blue!10,
	draw=blue!70
]

\tikzstyle{patate}=[
    rounded corners=5pt,
    draw=black,
    fill=black!50,
    text width=12mm,
    text centered
]

\tikzstyle{patatoid}=[
    ellipse,
    node distance=10mm,
    draw=black
]
%%% }}}

% {{{ DESSINS
\newcommand{\base}[4]{
    \draw[->] (0,0) to (0, #1);
    \draw[->] (0,0) to (#2, 0);
    
    \node at (0, #1+2) {#3};
    \node at (#2 + 2, -2) {#4};
}
% }}}
%}}}



\begin{document}

Au cours de cette section, nous allons définir un algorithme permettant d'obtenir une
$2$-dual-approximation du problème \unitfisched{}. 

\section{Unalgorithme d'approximation}

Avant de présenter le principe général de l'algorithme, nous introduirons quelques notions
nécessaires à la compréhension de ce dernier.

\subsection{Quelques définitions}

\begin{ndf}[Une \tphase{}]
    Une \tphase{} d'un ensemble de tâches est un groupe de tâches telles que pour toutes les tâches
    appartenant à une même \tphase{} l'exécution commence au même instant et telles que deux tâches
    appartiennent à une \tphase{} différente si et seulement si elles ne s'exécutent pas en même
    temps.

    S'il existe une partition de l'ensemble des tâches en \tphase{}s, l'ordonnancement est alors dit
    \emph{par phases}.
\end{ndf}

\begin{nprop}
    Toute instance du problème \unitfisched{} donne lieu à un ordonnancement par phases.
\end{nprop}

% TODO Preuve

\begin{ndf}[Trous à gauche et à droite au temps $i$]
    Soit $M_j \in \mathcal{M}$ une machine quelconque, soit $T_{M_j}$ la dernière tâche ordonnancée
    sur la machine $M_j$ telle que $\st{T_j} <= i$ et soit $R_{M_j}$ la première réservation
    affectéé à la machine $M_j$ telle que $\sres{R_j} >= i$. On définit alors le \emph{trou à gauche
    de $M_j$ au temps $i$} (respectivement \emph{le trou à droite de $M_j$ au temps $i$}), l'écart
    entre la fin (respectivement le début) d'exécution de la tâche $T_{M_j}$ (respectivement de la
    réservation $R_{M_j}$) et le temps $i$ (respectivement $i+1$).

    Un exemple de trou à gauche (respectivement à droite) est donné par le trou $h_1$
    (respectivement $h_2$) de la figure~\ref{fig:tagtad}.
\end{ndf}

\begin{ndf}[Machine \authmach{k} au temps $i$]
    Une machine $M_j$ est dite \authmach{k} au temps $i$ si et seulement si ses trous à gauche et à
    droite ont une longueur au moins égale à $k$.

    Quelques exemples sont donnés à la figure~\ref{fig:kdisp}, au temps $6$ : 
    \begin{itemize}
        \item $M_1$ est \authmach{1}
        \item $M_2$ est \authmach{3}
        \item $M_3$ est \authmach{5}
        \item $M_4$ est \authmach{\infty}
    \end{itemize}
\end{ndf}

\begin{figure}
    \centering
    \begin{ordo}[10]{1}{1}{12}
        \newtask{1}{1}{0}{$T_{M_1}$}
        \newlabeledresa{1}{1}{10}{$R_{M_1}$}
        \newhole{3}{1}{1}{$h_1$}
        \newhole{5}{1}{5}{$h_2$}
    \end{ordo}
    \caption{Trou à gauche et à droite au temps $4$}
    \label{fig:tagtad}
\end{figure}

\begin{figure}
    \centering
    \begin{ordo}[10]{4}{1}{12}
        \newtask{1}{1}{4}{$T_{M_1}$}
        \newlabeledresa{1}{1}{8}{$R_{M_1}$}

        \newhole{1}{1}{5}{$h_1$}
        \newhole{1}{1}{7}{$h_2$}

        \newtask{1}{2}{0}{$T_{M_2}$}
        \newlabeledresa{1}{2}{10}{$R_{M_2}$}

        \newhole{5}{2}{1}{$h_3$}
        \newhole{3}{2}{7}{$h_4$}

        \newlabeledresa{1}{3}{12}{$R_{M_3}$}
        \newbeghole{6}{3}{$h_5$}
        \newhole{5}{3}{7}{$h_6$}

        \newbeghole{6}{4}{$h_7$}
        \newendhole{5}{4}{7}{$h_8$}
    \end{ordo}
    \caption{Illustration des machines \authmach{k}s}
    \label{fig:kdisp}
\end{figure}

\subsection{Principe de l'algorithme}



%Étant donné un entier $\omega$, l'algorithme proposé réalise les étapes suivantes :
%\begin{enumerate}
%    \item l'ensemble de tâches est partitionné en \nbphase{} \tphase{}s,
%    \item les \tphase{}s sont triées par date de début croissant, et parcourues dans cet ordre,
%    \item l'ensemble des machines \authmach{\frac{\omega}{2}} au temps $0$ est calculé et sera
%        maintenu tout au long du parcours. L'ensemble des machines \authmach{\frac{\omega}{2}} au
%        temps $i$ est noté $\mathcal{M}_i$,
%    \item les tâches à ordonnancer au temps $i$ sont ordonnancées sur les machines appartenant à
%        $\mathcal{M}_i$, ayant le plus petit trou à droite.
%\end{enumerate}
%
%L'algorithme s'arrête alors une fois que la dernière tâche a été ordonnancée ou sià un moment donné
%au cours de son exécution, $\mathcal{M}_i = \emptyset$ et le nombre de tâches restant à ordonanncer
%à l'instant $i$ est non nul.
%
\end{document}
