Au cours de cette partie, le principe de décomposition arborescente sera présenté. Puis nous
introduirons la notion de largeur arborescente, qui sera utilisée comme paramètre dans la
section~\ref{hcycle_tw}.

% {{{ TREEWIDTH QUESACO
%% {{{ INTRODUCTION DE LA DECOMPOSITION ARBORESCENTE
%%% {{{ Principe
\section{Principe}

Le principe de décomposition arborescente a été introduit pour la première fois par Seymour et
Robertson~\cite{robalgorithmic} dans le cadre de la théorie des mineurs dans les graphes. Une
décomposition arborescente est la décomposition d'un graphe en séparateurs de ce graphe connectés
dans un arbre. De manière plus formelle, une décomposition arborescente se définit comme suit :

\begin{ndf}[Décomposition arborescente]
    Étant donné un graphe fini\footnote{Nous ne travaillerons que sur des graphes finis et non
        orientés, même si la notion de de décomposition arborescente peut s'étendre aux graphes
    orientés.} $G = (V, E)$, on appelle décomposition arborescente, la paire $(T, \{X_t : t \in T)$,
    avec $T$ un arbre dont les n\oe uds $t$, que nous appellerons des \emph{sacs}, sont des sous
    ensembles de sommets de $G$ $X_t \subset V$ vérifiant : 
    \begin{enumerate}
        \item $\forall i \in V$, $v$ appartient à au moins un sac de $T$
        \item $\forall (uv) \in E$, il existe au moins un sac contenant $u$ \textbf{et} $v$
        \item si deux sacs $k$ et $m$ contiennent le même sommet $u$ de $G$ alors tous les sacs $l$
            de $T$ sur le chemin entre $k$ et $m$ contiennent aussi $u$, $k, l, m \in T$
    \end{enumerate}
\end{ndf}

Il est important de remarquer qu'une décomposition arborescente n'est pas unique, si l'on considère
le graphe $G = (V, E)$ dont une représentation est donnée à la figure~\ref{tw_g}, alors les arbres
présentés à la figure~\ref{tw_trees} sont tous des décompositions arborescentes de $G$.

% {{{ FIGURES
\begin{figure}
    \begin{center}
        \begin{tikzpicture}
            \node[vertex] (1) {$1$};
\node[vertex, below=of 1] (2) {$2$};
\node[vertex, below left=of 2] (3) {$3$};
\node[vertex, below right=of 2] (4) {$4$};
\node[vertex, below right=of 3] (5) {$5$};
\node[vertex, above right=of 4] (6) {$6$};
\node[vertex, below right=of 4] (7) {$7$};

\draw[edge] (1) to (2);
\draw[edge] (2) to (3);
\draw[edge] (2) to (4);
\draw[edge] (3) to (4);
\draw[edge] (3) to (5);
\draw[edge] (4) to (5);

\draw[edge] (4) to (6);
\draw[edge] (4) to (7);
\draw[edge] (6) to (7);

        \end{tikzpicture}
        \caption{Un graphe $G = (V, E)$}
        \label{tw_g}
    \end{center}
\end{figure}

\begin{figure}
    \begin{center}
        \subfloat[$T_1$]{\label{fig:tw_t1}
            \begin{tikzpicture}
                \node[patate] (1) {$1$ $2$ $3$ $4$ $5$ $6$ $7$};

            \end{tikzpicture}
        }
        \subfloat[$T_2$]{\label{fig:tw_t2}
            \begin{tikzpicture}
                \node[patate] (1) {$1$ $2$};
\node[patate, below=of 1] (2) {$2$ $3$ $4$};
\node[patate, below left=of 2] (3) {$3$ $4$ $5$};
\node[patate, below right=of 2] (4) {$4$ $6$ $7$};

\draw[edge] (1) to (2);
\draw[edge] (2) to (3);
\draw[edge] (2) to (4);


            \end{tikzpicture}
        }
        \subfloat[$T_3$]{\label{fig:tw_t3}
            \begin{tikzpicture}
                \node[patate] (1) {$1$ $2$ $5$};
\node[patate, below=of 1] (2) {$2$ $3$ $4$ $5$};
\node[patate, below=of 2] (3) {$3$ $4$ $6$ $7$};

\draw[edge] (1) to (2);
\draw[edge] (2) to (3);

            \end{tikzpicture}
        }
        \caption{Différentes décompositions arborescentes}
        \label{tw_trees}
    \end{center}
\end{figure}

\begin{figure}
    \begin{center}
        \begin{tikzpicture}
            \node[vertex, draw=red, fill=red!50] (1) {$1$};
\node[vertex, draw=red, fill=red!50, below=of 1] (2) {$2$};
\node[vertex, draw=red, fill=red!50, below left=of 2] (3) {$3$};
\node[vertex, below right=of 2, dotted, fill=black!20] (4) {$4$};
\node[vertex, draw=red, fill=red!50, below right=of 3] (5) {$5$};
\node[vertex, draw=blue, fill=blue!50, above right=of 4] (6) {$6$};
\node[vertex, draw=blue, fill=blue!50, below right=of 4] (7) {$7$};

\draw[edge, red] (1) to (2);
\draw[edge, red] (2) to (3);
\draw[edge, red] (3) to (5);
\draw[edge, dotted] (4) to (5);
\draw[edge, dotted] (2) to (4);
\draw[edge, dotted] (3) to (4);

\draw[edge, dotted] (4) to (6);
\draw[edge, dotted] (4) to (7);
\draw[edge, blue] (6) to (7);

        \end{tikzpicture}
        \caption{$\{4\}$ est un séparateur de $G$}
        \label{tw_sep}
    \end{center}
\end{figure}

\begin{figure}
    \begin{center}
        \begin{tikzpicture}
            \node[vertex, draw=red, fill=red!50] (1) {$1$};
\node[vertex, dotted, fill=black!20, below=of 1] (2) {$2$};
\node[vertex, dotted, fill=black!20, below left=of 2] (3) {$3$};
\node[vertex, below right=of 2, dotted, fill=black!20] (4) {$4$};
\node[vertex, draw=green, fill=green!50, below right=of 3] (5) {$5$};
\node[vertex, draw=blue, fill=blue!50, above right=of 4] (6) {$6$};
\node[vertex, draw=blue, fill=blue!50, below right=of 4] (7) {$7$};

\draw[edge, dotted] (1) to (2);
\draw[edge, dotted] (2) to (3);
\draw[edge, dotted] (3) to (5);
\draw[edge, dotted] (4) to (5);
\draw[edge, dotted] (2) to (4);
\draw[edge, dotted] (3) to (4);

\draw[edge, dotted] (4) to (6);
\draw[edge, dotted] (4) to (7);
\draw[edge, blue] (6) to (7);

        \end{tikzpicture}
        \caption{$\{2,\ ,3\ ,4\}$ est un séparateur de $G$}
        \label{tw_bsep}
    \end{center}
\end{figure}

% }}}

En observant l'exemple donné, on peut mettre en évidence la notion de séparateur : toute arête de
$T$ représente un séparateur de $G$. En effet prenons l'arête $e = (A, B),\ A, B \in T$ tels que
$X_A = \{2, 3, 4\}$ et $X_B = \{4, 6, 7\}$ dans l'arbre $T_2$ de la sous-figure~\ref{fig:tw_t2},
appelons $T_A$ et $T_B$ les sous-arbres induits de la suppression de l'arête $e$ dans $T_2$ tels que
$A \in T_A$ et $B \in T_B$, le séparateur associé à $e$ et l'ensemble de sommet de $G$ défini par
$S_e = X_A \cap X_B$, donc $S_e = \{4\}$. En effet, en retirant le sommet $4$ de $G$ on obtient deux
composantes connexes, l'une formée par les sommets contenus dans les sacs de $T_A$ privé de $4$, et
l'autre formée par les sommets contenus dans les sacs de $T_B$ privé de $4$. Ceci est mis en
évidence à la figure~\ref{tw_sep}. De la même manière, tout sac interne de $T$ est un séparateur de
$G$, puisque supprimer un sac revient à supprimer toutes les arêtes adjacentes à ce sac et donc tous
les séparateurs associés à ces arêtes ainsi que le n\oe uds seulement contenus dans le sac supprimé
(cf figure~\ref{tw_bsep}). Nous noterons donc les deux propriétés suivantes :

\begin{nprop}
    \label{esep}
    Étant donné un graphe $G=(V_G,E_G)$ et une de ses décompositions arborescentes $\mathcal{T} =(T, \{X_t
    : t \in T\})$, toute arête $e=(AB) \in T$ définit un séparateur $S_e$ pour $G$ donné par $S_e =
    X_A \cup X_B$.
\end{nprop}

\begin{nprop}
    \label{nsep}
    Étant donné un graphe $g=(V_G, E_G)$ et $\mathcal{T} = (T, \{X_t : t \in T\}$ une décomposition
    arborescente de $G$, tout sac interne de $T$ définit un séparateur de $G$.
\end{nprop}

Le but d'une telle opération est de ramener un graphe quelconque à un arbre dans lequel, comme nous
l'avons vu en section~\ref{tree_why}, de nombreux problèmes sont plus simples à résoudre. De plus,
la notion de décomposition arborescente permet de définir une notion très intéressante que l'on
appelle la largeur arborescente.
%%% }}}
%% }}}
%% {{{ DEFINITION FORMELLE DE LA TREEWIDTH
\section{Largeur arborescente}

La largeur arborescente est une notion introduite par Robertson et Seymour~\cite{robalgorithmic}
descendant directement du principe de décomposition arborescente, en effet, la largeur arborescente
d'une décomposition arborescente est égale à la taille de son plus gros sac moins $1$ et la largeur
arborescente d'un graphe $G=(V,E)$ donné est égale à la plus petite valeur choisie parmi les
largeurs arborescentes de l'ensemble de ses décompositions. Autrement dit :

\begin{ndf}
    Étant donné un graphe $G=(V,E)$ et $\Theta$ l'ensemble de ses décompositions arborescentes, si l'on
    désigne par $s_t$ l'ensemble des sacs de la décomposition arborescente $t \in T$, alors la
    largeur arborescente est définie par : \[
        \tw(G) = \min_{T\in \Theta}(\tw(t)) = \min_{T\in \Theta}(\max_{t \in T}(|X_t| - 1))
    \]
\end{ndf}

On peut alors définir le problème de décision associé :

\dfpb{\twidth}{$G=(V,E)$ un graphe et $k$ un entier}{Est ce que le graphe $G$ possède une
décomposition arborescente $T$ telle que $\tw(t) \leq k$?}

\begin{nrmq}
    Ce problème a été démontré \npc~\cite{Arn87}.
\end{nrmq}

De manière plus intuitive, la largeur arborescente mesure la ressemblance entre le graphe $G$ pour
lequel cette valeur est calculée et un arbre. C'est pourquoi lorsque cette valeur est faible,
certains algorithmes sont efficaces pour résoudre des problèmes \npc s tels que \hcycle comme nous
le verrons plus loin.

\begin{nlemma}
    \label{tw_approx}
    Une $4-$approximation de la largeur arborescente peut être calculée en temps polynomial.
\end{nlemma}

L'algorithme a été présenté par Diestel et al~\cite{Die00}.

\subsection*{Notations}

Dans la suite de ce document, nous utiliserons les notations suivantes. Soit une décomposition
arborescente $\mathcal{T} = (T, \{X_t : t \in T\})$, si l'on enracine $\mathcal{T}$ alors on désignera par :
\begin{itemize}
    \item $V_t$ l'ensemble des sommets présents dans les descendants de $t$
    \item $G_t$ le sous graphe induit par $V_t$ que l'on peut aussi noter $G[V_t]$
\end{itemize}



%% }}}
% }}}

