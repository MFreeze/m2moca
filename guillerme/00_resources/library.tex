% Encodage et police
\usepackage[utf8]{inputenc}
\usepackage[T1]{fontenc}
\usepackage[french]{babel}
\usepackage{eurosym}

% Figures
\usepackage{graphicx}
\usepackage{subfig}

% Insertion de code
\usepackage{moreverb}

% Insertion de pdf
\usepackage{pdfpages}

% Réduction des marges
%\usepackage{fullpage}
\usepackage[top=2.5cm, bottom=2.5cm, left=3.5cm, right=3.5cm]{geometry}

% Mathématiques
\usepackage[fleqn]{amsmath}
\usepackage{amsthm}
\usepackage{bbm}
\usepackage{amssymb}
\usepackage{stmaryrd}

% Graphes
\usepackage{tikz}
\usepackage{pgf}
\usetikzlibrary{arrows,automata}

% Reseaux de Petri
\usetikzlibrary{petri}

% Arbres
\usetikzlibrary{trees}

% Diagramme de gantt
\usepackage{pgfgantt}

% Algorithmes
\usepackage{algorithm}
\usepackage[noend]{algpseudocode}
%\usepackage{algorithmic}
%\usepackage{algorithmicx}

% Théorèmes, Lemmes -- numérotation personnalisée
\theoremstyle{plain}
\newtheorem{thrm}{Théorème}[section]
\newtheorem{lemma}{Lemme}[subsection]
\newtheorem{corol}{Corollaire}[section]

% Théorèmes, Lemmes -- numérotation classique
\newtheorem{nthrm}{Théorème}
\newtheorem{nlemma}{Lemme}
\newtheorem{ncorol}{Corollaire}

% Théorèmes, Lemmes -- non numérotés
\newtheorem*{thrm*}{Théorème}
\newtheorem*{lemma*}{Lemme}
\newtheorem*{corol*}{Corollaire}

% Définitions, Propriétés -- numérotation personnalisée
\theoremstyle{definition}
\newtheorem{df}{Définition}[subsection]
\newtheorem{prop}{Propriété}[subsection]
\newtheorem{conj}{Conjecture}[subsection]

% Définitions, Propriétés -- numérotation classique
\newtheorem{ndf}{Définition}
\newtheorem{nprop}{Propriété}
\newtheorem{nconj}{Conjecture}

% Définitions, Propriétés -- non numérotés
\newtheorem*{df*}{Définition}
\newtheorem*{prop*}{Propriété}
\newtheorem*{conj*}{Conjecture}

% Autres -- numérotation personnalisée
\theoremstyle{remark}
\newtheorem{note}{Note}[subsubsection]
\newtheorem{pr}{Proposition}[subsection]
\newtheorem{rmq}{Remarque}[subsubsection]

% Autres -- numérotation classique
\newtheorem{nnote}{Note}
\newtheorem{npr}{Proposition}
\newtheorem{nrmq}{Remarque}

% Autres -- non numérotés
\newtheorem*{note*}{Note}
\newtheorem*{pr*}{Proposition}
\newtheorem*{rmq*}{Remarque}

% Exemples
\newtheorem*{ex}{Exemple}
\newtheorem*{exo}{Exercice}
\newtheorem*{rappel}{Rappel}

% Opérateurs mathématiques
\DeclareMathOperator{\poly}{\textbf{poly}}
\DeclareMathOperator{\gap}{\mbox{gap}}
\DeclareMathOperator{\Bg}{\mbox{Big}}
\DeclareMathOperator{\Sml}{\mbox{Sml}}
\DeclareMathOperator{\Diff}{\mbox{Diff}}
\DeclareMathOperator{\card}{\mbox{Card}}
\DeclareMathOperator{\Obj}{\mbox{Obj}}
\DeclareMathOperator{\concat}{\bullet}
\DeclareMathOperator{\sig}{\mbox{Sig}}
\DeclareMathOperator{\ver}{\mbox{Ver}}
\DeclareMathOperator{\dist}{\mbox{dist}}
\DeclareMathOperator{\bw}{\mbox{branchwidth}}
\DeclareMathOperator{\tw}{\mbox{treewidth}}
\DeclareMathOperator{\lm}{\leq_m}
\DeclareMathOperator{\lc}{\leq_c}

\newcommand{\wqo}[0]{\emph{well-quasi-ordered}}
\newcommand{\base}[4]{
    \draw[->] (0,0) to (0, #1);
    \draw[->] (0,0) to (#2, 0);
    
    \node at (0, #1+2) {#3};
    \node at (#2 + 2, -2) {#4};
}
                
    
