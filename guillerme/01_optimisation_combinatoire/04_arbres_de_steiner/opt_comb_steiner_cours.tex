\documentclass[a4paper,11pt]{thesis}

/stock/16_git_repo/library/library.tex

\begin{document}

\chapter{Communication de groupes : Problème de Steiner}

\section{D\'{e}finition du probl\`{e}me}

Problèmatique : On dispose d'un émetteur qui veut émettre à un ensemble d'utilisateur d'un réseau.

Il existe deux stratégies utilisées :
\begin{itemize}
    \item innondation : utilisé pour la télévision. Lorsqu'un routeur reçoit un message, il
        retransmet sur toutes ses sorties. Au niveau du récepteur, l'information est détruite si
        elle ne lui est pas destinée. Il existe une sauvegarde pour éviter de retransmettre trop
        d'information
    \item communication paire à paire : il s'agit d'une retransmission non pas en broadcast mais
        à un utilisateur seulement ce qui implique une dupplication à la source.
\end{itemize}

Chacun des deux modèles ont des problèmes : le premier pose le problème de la saturation du réseau
et de l'envoi de données inutiles. Le second pose le problème de la saturation à la source.

Mise en place des réseaux intelligents.

On cherche alors à trouver un sous réseau physique (ou logique) pour diffuser d'un émetteur vers un
groupe $M$ de noeuds (DEERING). Dans cette problèmatique, on ne fait aucune hypothèse sur $M$.

La modélisation naturelle est la suivante.

\begin{df}[Problème de Steiner]
    ~\\
    \textbf{\underline{Données :}} $G = (V, E), M \subseteq V$ et une fonction de coût $E \mapsto
    \mathbb{R}$\\
    \textbf{\underline{Résultat :}} $H = (V', E)$ avec $V' \subset V, E' \subset E, M \subset V'$.
    Il s'agit d'un sous graphe partiel connexe de $G$ qui minimise des contraintes.
\end{df}

\section{Probl\'{e}me de l'arbre de Steiner}

On se restreint au problème le plus simple : les arêtes ont un poids uniforme, ce qui revient à
minimiser la taille de $|E'|$. Celà revient à chercher un arbre de Steiner, puisque si la solution
optimale est un cycle, alors on peut enlever une arête et garder une composante connexe, ce qui
donne une contradiction sur l'optimalité de la solution trouvée. Un cycle ne peut donc pas être
solution optimale.

\begin{df}
    ~\\
    \textbf{\underline{Données :}} $G = (V, E)$, $k \in \mathbb{N},\ M \subseteq V$\\
    \textbf{\underline{Resultat :}} Existe-t-il un arbre couvrant tous les sommets de $M$, $T = (V',
    E')$ avec $|E'| = k$
\end{df}

Il existe des cas faciles : 
\begin{itemize}
    \item $|M| = 2$, on recherche alors un plus court chemin
    \item $|M| = 3$, on cherche la meilleure étoile
    \item $|M| = K,\ K \in \mathbb{Z}$, l'algorithme est polynomial exposant $K$
    \item $M = V$, il s'agit d'un arbre couvrant de poids minimum
\end{itemize}

\begin{thrm}
    Ce problème est $NP-$complet.
\end{thrm}

\begin{proof}
    On va utiliser le problème \emph{Exact 3-Set Cover}:
    \begin{itemize}
        \item[] \textbf{\underline{Données :}} un ensemble de $3n$ variables $X = \{X_1, X_2, \dots,
            X_{3n}\}$, un ensemble de triplets de $X$, $Q = \{q_1, q_2, \dots, q_n\}$
        \item[] \textbf{\underline{Question :}} Existe-t-il $Q'$, $Q' \subset Q,\ |Q'| = n$ tel que
            : $\bigcup_{q \in Q'} q = X$
    \end{itemize}
    % figure A.1
    Dans le graphe ainsi construit, il existe un arbre de Steiner de taille $4n$ si et seulement si
    la réponse est OUI pour l'instance de \emph{Exact 3-Set Cover}.

    \begin{enumerate}
        \item Supposons que la réponse est OUI pour l'instance de \emph{Exact 3-Set Cover} avec $Q'
            = \{q_1', q_2', \dots, q_n'\}$ Prenons l'arbre induit par $X \cup \{s\} \cup Q'$ c'est
            un arbre de Steiner avec $4n + 1$ sommets, donc $4n$ arêtes.
        \item Supposons que l'arbre de Steiner a $4n$ arêtes, donc il a $4n + 1$ sommets. $3n + 1$
            sommets sont les sommets de $M$, donc il y a $n$ de $Q = \{q_1', q_2', \dots, q_n'\}$.
            Comme l'arbre est connexe, il couvre forcément les sommets de $X$. Il s'agit donc bien
            d'une partition.
    \end{enumerate}
\end{proof}

Trouver le meilleur arbre est $NP-$difficile. Il est même APX, mais il n'existe pas de schéma
d'approximation polynomiale. La meilleure borne connueest supérieure à $1.5$.

\subsection{Une heuristique simple}

Cette heuristique fonctionne même si les arête sont valuées.

% Décrire comme algo
\begin{enumerate}
    \item On construit un complet dont les sommets sont les élément de $M$
    \item cout$(\{x_i, x_j\}) = \dist_G(x_i, x_j)$
    \item On calcule dans $K_m$ un arbre couvrant de poids minimum.
    \item A chaque arête de l'ACPM, on associe dans $G$ un chemin de même longueur que son coût
    \item On supprime des arêtes pour arriver à un arbre
\end{enumerate}

\begin{thrm}
    Cette heuristique est $2-$approchée
\end{thrm}

\begin{proof}
    Soit $\omega^*$ le coût de l'arbre de Steiner optimal et $\omega$ le coup de la solution
    obtenue par l'ago. Par définition, on a $\omega^* \leq \omega \leq 2 \omega^*$. Soit $T^*$
    l'arbre optimal de poids $\omega^*$
    %fig A.2
    Le parcours de l'arbre a un poids égal à  deux fois le poids de $T^*$ :
    \begin{displaymath}
        \begin{array}{rcl}
            2 \omega^* & = & \sum_{i=1}^{n-1}d_{T^*}(x_i, x_{i+1}) + d_{T^*}(x_n, x_1) \\
                       & \geq & \sum_{i=1}^{n-1} g_G(x_i, x_{i+1}) \\
                       & \geq & \omega_1 \\
                       & \geq & \omega 
        \end{array}
    \end{displaymath}
    Avec $\omega_1$ le poids de l'ACPM de $K_M$.
\end{proof}

\subsection{Autres algorithmes d'approximation}

\subsubsection{Arbre $k$ restreint (qui connecte $k$ membres du groupe)}

% Décrire comme algorithme
\begin{enumerate}
    \item Prendre le meilleur arbre $k$ restreint $T_1$ (meilleur qui minimise une fonction)
    \item On contracte $T_1$
    \item On continue jusqu'à résoudre de façon triviale, le problème
\end{enumerate}

Une fonction classique : le poids du meilleur ACPM dans $K_M$ qui contient $T_1$.

\section{Variantes avec contraintes de diam\`{e}tre}

\subsection{Pr\'{e}sentation du probl\`{e}me de Steiner-bis}

\subsubsection*{Arbre de Steiner bis}
\begin{itemize}
    \item[] \textbf{\underline{Données :}} Soit $G, s \in V, M \subseteq V$ et $\Delta$ sous les
        hypothèses : \begin{displaymath}
            \forall e \in E, \quad c(e) = \mbox{delai}(e) = 1
        \end{displaymath}
    \item[] \textbf{\underline{Question :}} Trouver un arbre de cout minimum tel que :
        \begin{displaymath}
            \forall v \in M,\quad \dist(s, w) \leq \Delta
        \end{displaymath}
\end{itemize}

\subsubsection*{Set Cover}

\begin{itemize}
    \item[] \textbf{\underline{Données :}} $X = \{x_1, \dots, x_p\}, Y = \{y_1, \dots, y_q\}, y_j
        \subseteq X$
    \item[] \textbf{\underline{Question :}} Trouver $Y' \subseteq Y$ de cardinalité minimale tel que
        :\begin{displaymath}
            \bigcup_{y \in Y'} y = X
        \end{displaymath}
\end{itemize}

Ce problème n'a pas de garantie de performance meilleure que $\theta(\log n)$.

\subsubsection*{R\'{e}duction}

% fig A.3

$T_h$ l'arbre donné par une heuristique : 
\begin{displaymath}
    Y' = \{y_j | y_j \in T_h \}
\end{displaymath}
Alors $Y'$ couvre $X$.

On a : \begin{displaymath}
    c(T_h) = n|Y'| + p
\end{displaymath}

Si $T^*$ est l'arbre optimal, $c(T^*) = n|Y^*| + p$, avec $Y^* = \{y_j | y_j \in T^* \}$ est une
solution optimale pour \emph{Set Cover}.

On a : \begin{displaymath}
    \frac{c(T_h)}{c(T^*)} = \frac{n|Y'| + p}{n|Y^*| + p}
\end{displaymath}

Supposons l'existence d'un algorithme $\rho-$approché, ce qui se traduit par : 
\begin{displaymath}
    \frac{c(T_h)}{c(T^*)} \leq \rho
\end{displaymath}

On a alors : 
\begin{displaymath}
    \begin{array}{rrcl}
        & \displaystyle \frac{n|Y'| + p}{n|Y^*| + p} & \leq & \rho \\
        \Rightarrow & n|Y'|  + p & \leq & \rho(n |Y^*| + p) \\
        \Rightarrow & n|Y'| & \leq & \rho(n |Y^*| + p) - p \\
        \Rightarrow & |Y'| & \leq & \rho |Y^*| + \displaystyle \frac{p(\rho -1)}{n} \\
        \Rightarrow & ...
    \end{array}
\end{displaymath}

On en déduit l'existence d'un algorithme $2\rho-$approché pour \emph{Set Cover}. Ce qui est
impossible si $P \neq NP$.

\subsection{Relax de contraintes}

Soit $T$ un arbre couvrant un groupe $M$, on s'intéresse à 2 paramètres :
\begin{enumerate}
    \item Le nombre d'arêtes $\omega(T)$
    \item Le diamètre $d(T)$
\end{enumerate}

Un algorithme $h$ est $(\rho, \tau)-$approché si :
\begin{displaymath}
    \max_{G, M} \frac{d(T_h)}{d_G(M)} \leq \tau \mbox{ et } \max_{G, m}
    \frac{\omega(T_h)}{\omega(T^{st})} \leq \rho
\end{displaymath}
Avec $T^{st}$ l'arbre de Steiner optimal pour le groupe $M$.

On va montrer que $\forall \tau > 2, \exists \rho \in \mathbb{R}$ tel qu'une heuristique est $(\rho,
\tau)-$approchée.

\subsubsection{Quelques r\'{e}sultats n\'{e}gatifs}

% Susceptible de tomber à l'examen
Il est impossible de garantir un coût optimal et un ration pour le diamètre constant.\\
% Parce qu'il existe des graphes pour lesquels les arbres de Steiner n'ont pas une largeur bornée

Il est aussi impossible de garantir $\tau < 2$ et $\rho$ constant.\\

\begin{thrm}
    $\forall \rho > 1$ il est impossible de garantir $\tau < 1 /(\rho - 1)$
    % Preuve en exercice
\end{thrm}

% Rattraper le cours

\section{Probl\`{e}me de Steiner}

% Chercher la véritable définition de l'arbre de Steiner
\textbf{\underline{Données :}} Un graphe $G = (V, E)$, $\forall e \in E : c(e) > 0$, $\forall v
\in V : c(v) > 0$ et $M \in V$ l'ensemble des sommets à couvrir.
\textbf{\underline{Solution :}} Un arbre, $T = (V_T, E_T)$ tel que $T$ minimise le coût.

\subsection{Probl\`{e}me de Steiner dans l'espace rectilin\'{e}aire}

% fig C1

Les arêtes de l'arbre de Steiner sont uniquement des lignes parallèles à l'axe des $x$ ou à celui
des $y$

Il y a des applications directes à la microélectronique.

\section{Algorithmes exacts pour le probl\`{e}me de Steiner}

\subsection{STEA (Steiner Tree Enumeration Algorithm)}

% fig C2

\subsubsection{Deo-Hakimi}

Le seul moyen de trouver l'arbre de Steiner de coût minimum, il faut énumérer l'ensemble des
arbres de couverture partielle de $G$ et chercher celui de coût minimum.

On appelle $S = V \backslash M$, l'algorithme de Hakimi est le suivant :
\begin{algorithm}
    \caption{Deo-Hakimi}
    \label{deohakimi}
    \begin{algorithmic}[1]
        \State Enumérer tous les sous-ensembles de S
        \For{$s \in S$}
            \State Créer le graphe engendré par $M \cup s$
            \State Calculer le MST
        \EndFor
    \end{algorithmic}
\end{algorithm}

La complexité de cet algorithme est de l'ordre de $2^{|V| - |M|}. |V|^2$.

\subsubsection{Lauler}

L'algorithme de Lauler est le même que précédemment sauf qu'il travaille sur la fermeture métrique
du graphe de départ.

\subsection{Solution exacte avec programmation dynamique}

\subsubsection{Sous optimalit\'{e}}

Supposons un arbre de Steiner $T$ pour un ensemble de sommets $M$, soit $v$ un sommet de $T$ tel que
$v$ est la racine d'un arbre $T_v$. Alors $T_v$ est un arbre de Steiner pour l'ensemble $M_v \subset
M$.

%fig C3
% La preuve se fait par l'absurde.

\subsubsection{Principe}

%fig C4

\subsubsection{Algorithme}

% Ecrire l'algorithme

\subsection{Balabrishnan et Patel}

% fig C5

On ajoute un noeud ficitif $a$ que l'on relie aux noeuds $n \not \in V \\ M$, chaque arête partant
de $a$ a un coût $0$. On choisit arbitrairement un noeud $b$ que l'on relie à $a$.

On cherche alors le MST dans le graphe obtenu.

%fig C6

On obtient un arbre de longueur $6$ mais à partir duquel, en supprimant $(ab)$ on n'obtient pas
d'arbre couvrant les trois sommets.

%fig C7

L'algorithme examine alors les arbres de longueur $8$, aucun ne marche. On examine ensuite ceux de
longueur $9$

% fig C8

\subsection{ILP (Integer Linear Programming)}

Soit $G = (V, E)$, $c(e) > 0,\ \forall e \in E$ et $M \subseteq V$ un ensemble de terminaux à
couvrir. Soient un ensemble de variable $X$ telles que : \begin{displaymath}
    X_e = \left \lbrace
    \begin{array}{rl}
        1 & \mbox{ si } e \in T \\
        0 & \mbox{ sinon } \\
    \end{array}
    \right .
\end{displaymath}
On notera $\delta(W)$ la coupe de $W \subseteq V$, $X(F) = \sum_{e \in F} X_e,\quad F \subseteq E$,
on a donc $X(\delta(W))$ qui est le nombre d'arêtes de la coupe de $W$ prises dans la solution.

On peut alors définir la fonction objectif : 
\begin{displaymath}
    \min \overline{c}^T . \overline{X} = \sum_{e \in E} c(e)X_e
\end{displaymath}

et les contraintes du système :
\begin{displaymath}
    X(\delta(W)) \geq 1, \qquad \forall W \subseteq V tq W \cap V_T \neq \emptyset \mbox{ et } (V \\
    W) \cap V_T \neq \emptyset
\end{displaymath}

\section{Heuristiques}

\subsection{Shortest Path Tree}

A partir d'un sommet, on cherche l'ensemble des plus courts chemins on les superpose et obtient un
graphe. Ce dernier est toujours un arbre si lorsque deux solutions ont même valeur, on fait une
énumération par ordre alphabétique des n\oe uds.

On ne peut pas approximer à l'aide d'un SPT

\subsection{Takahashi Matsuyama}

On partir d'un sommet quelconque de $M$ on cherche à connecter le noeud le plus proche avec un plus
court chemin. On recalcule alors les distances de chacun des sommets. Et on recommence.

Il s'agit du pendant de l'algorithme de Prime pour les arbres de Steiner. De la même manière il est
possible d'avoir une $2-$approximation avec l'équivalent de Kruskal pour les arbres de Steiner.

\subsection{Kruskal}

Kruskal pour un ACPM trie les arrêtes et les ajoute à l'arbre couvrant une par une sans créer de
cycle. On fait pareil pour Steiner, à la place des arêtes, on prends le chemin de coût minimum.

\subsection{Kou-Markovski-Berman}

On calcule la fermeture métrique de $M$ à l'aide des plus court chemin. On cherche alors un arbre
dans cette fermeture, on supprime les cycles et on obtient une $2-$approximation de l'arbre de
Steiner.

\section{Algorithmes dynamiques}

Considérons un groupe $M_1$ pour lequel on calcule un arbre de Steiner $T_{M_1}$ et un groupe $M_2$
différent de $M_1$ mais proche de ce dernier, pour lequel on veut calculer $T_{M_2}$. En théorie, la
meilleure chose à faire est de supprimer $T_{M_1}$ est de recalculer $T_{M_2}$.

Les bons algorithmes dynamiques permettent de trouver un bon $T_{M_2}$ en gardant une partie de
$T_{M_1}$.

\section{Probl\`{e}mes de recouvrement sous contraintes}

Tout au long de cette section, on cherche une structure connexe, plus précisément, on recherche un
sous graphe connexe. On peut utiliser un bon petit nombre de contraintes différentes, en voici
quelques une : \begin{itemize}
    \item sur les degrés
    \item sur un budget (minimiser la longueur en ne dépassant pas un certain coût)
    \item bout à bout
    \item sur la taille de la structure de recouvrement (on peut par exemple limiter le nombre
        maximum de feuilles)
    \item sur le diamètre
    \item ..
\end{itemize}

En règle générale, quelque soit les contraintes appliquées, on a tendance à rechercher un arbre.

\subsection{Contraintes sur le degr\'{e}}

\subsubsection{Probl\`{e}me de recouvrement total}

On le retrouve noté dans la littérature en DCMST (d... constrained minimum spanning tree).
Soit un graphe $G =(V,E),\ c(e) > 0 \forall e \in E,\ B > 0$. On cherche la structure $T$, telle que
$c(T)$ est minimale et $d_T(v) \leq B, \forall v \in V$.

\begin{rq}
    La solution, lorsque l'on cherche un sous graphe, est un arbre.
\end{rq}

\begin{rq}
    La solution n'existe pas toujours.
\end{rq}

\begin{proof}
    Si $B=2$ alors on recherche un chemin hamiltonien, qui n'existe pas toujours.
\end{proof}

Problème défini pour la première fois par Hakimi en 1968. Il l'a fait sous la forme $ILP$.
Soit $D = d_{ij}$ la matrice des distances, soit $X = x_{ij}$ la matrice de connectivité de la
solution : $x_{ij} = 1$ si $i$ et $j$ sont connectés dans la solution.

\ipl{Problème de recouvrement total}{$\min L(X) = \sum_{i=1}^{n-1} \sum_{j=i+1}^{n} d_{ij}x_{ij}$}{\begin{itemize}
    \item $D = d_{ij}$ la matrice des distances
\item $X = x_{ij}$ la matrice de connectivité de la
solution : $x_{ij} = 1$ si $i$ et $j$ sont connectés dans la solution
\end{itemize}}{\begin{enumerate}
    \item contrainte de degré :
        \begin{displaymath}
            \sum_{k = 1}^{i-1} x_{ki} + \sum_{k=i+1}^n x_{ik} \leq B \quad 0 < i < n
        \end{displaymath}
    \item contrainte de couverture :
        \begin{displaymath}
            \sum_{k=1}^{n-1}x_{ki} + \sum_{k=i+1}^n x_{ik} \geq 1
        \end{displaymath}
    \item contrainte pour obtenir un arbre :
        \begin{displaymath}
            \sum_{i=1}^{n-1} \sum_{j=i+1}^n x_{ij} = n-1
        \end{displaymath}
\end{enumerate}}


%On cherche : $\min L(X) = \sum_{i=1}^{n-1} \sum_{j=i+1}^{n} d_{ij}x_{ij}$ sous les contraintes
%suivantes : \begin{enumerate}
%    \item contrainte de degré :
%        \begin{displaymath}
%            \sum_{k = 1}^{i-1} x_{ki} + \sum_{k=i+1}^n x_{ik} \leq B \quad 0 < i < n
%        \end{displaymath}
%    \item contrainte de couverture :
%        \begin{displaymath}
%            \sum_{k=1}^{n-1}x_{ki} + \sum_{k=i+1}^n x_{ik} \geq 1
%        \end{displaymath}
%    \item contrainte pour obtenir un arbre :
%        \begin{displaymath}
%            \sum_{i=1}^{n-1} \sum_{j=i+1}^n x_{ij} = n-1
%        \end{displaymath}
%\end{enumerate}

\begin{prop}
    Ce problème est NP-difficile
\end{prop}

\begin{proof}
    On va réaliser une réduction de ce problème vers le problème du chemin hamiltonien
    % fig 04/12 fig 2
\end{proof}

\subsubsection{Approximabilit\'{e}}

% Article de Ravi dans la bibliographie
Ravi se ramène à un problème bicritère qu'il exprime sous forme d'un triplet de paramètres : $(A, B,
S)$. $A$ représente un budget (degré), $B$ représente le coût et $S$ la solution.

Chaque paramètre peut prendre différentes valeurs :
\begin{itemize}
    \item A : \begin{enumerate}
            \item bornes homogènes
            \item bornes hétérogènes
        \end{enumerate}
    \item B : \begin{enumerate}
            \item coût sur les sommets
            \item coût sur les arêtes
        \end{enumerate}
    \item S : \begin{enumerate}
            \item spaning tree
            \item steiner tree
            \item steiner forest
        \end{enumerate}
\end{itemize}

Il s'agit donc d'un article relativement générique.
Nous nous intéresserons au cas : (U-DEGREE, E-TOTAL-COST, SPANNING TREE).
\begin{thrm}
    (U-DEGREE, E-TOTAL-COST, SPANNING TREE) n'est pas $\rho-$approximable.
\end{thrm}

\begin{proof}
    La preuve se fait à partir du chemin hamiltonien
\end{proof}

\begin{thrm}[Goemans]
    Il existe un algo polynomial $(B+2, 1)-$approximation.
\end{thrm}

\begin{thrm}[Shing-Chi]
    Il existe un algorithme polynomial $(B+1, 1)-$approximation.
\end{thrm}

\begin{prop}
    Il existe une $(1, 1 + \frac{1}{B-1})-$approximation dans les graphes métriques.
\end{prop}

\begin{prop}
    La sous-optimalité n'est pas vraie
\end{prop}
%fig3

\subsubsection{Algorithmes exacts}

En utilisant l'idée de Gabow (utilisée par Bara-Krishna pour l'arbre de Steiner) :
\begin{enumerate}
    \item Calculer MST
    \item Si $d_T(v) \leq B$ alors Stop
    \item Utiliser Gabow pour le spanning tree suivant
    \item goto 2
\end{enumerate}

\subsubsection{Heuristiques de Boldon}

\begin{enumerate}
    \item Calculer MST
    \item Si $d_T(v) \leq Beq$ alors Stop
    \item $\forall v | d(v) > B$, faire une blacklist des arêtes adjacentes et pénaliser les arêtes
    \item goto 1
\end{enumerate}

\subsubsection{Dans le cas de bornes non uniformes}

\begin{thrm}[Cierlik]
    S'il existe une solution alors :
    \begin{displaymath}
        \sum_{v \in V} \beta(v) \geq 2n - 2
    \end{displaymath}
\end{thrm}

\subsubsection{Motivation : Routage multicast avec contraintes sur le degr\'e}

On peut citer deux exemples :
\begin{itemize}
    \item le cas où un message arrivant ne peut être dupliqué que en un nombre limité de copies
    \item dans le cas d'un routeur optique, on ne peut dupliquer la lumière qu'à l'aide de splitter
        qui son terriblement chers, on se retrouve alors dans le cas de contraintes non uniformes
        sur les degrés : $2$ si le routeur optique ne dispose pas de splitter, $>2$ sinon.
\end{itemize}

\subsection{Arbre de recouvrement partiel $(M \subset V)$}

On cherche $\min c(T),\ d_T(v) \leq \beta(v) \quad v \in V_T$. 
Les résultats sont similaires à ceux obtenus pour le recouvrement complet à savoir la non
approximabilité de ce problème sans relaxation de la contrainte sur les degrés.

\subsubsection{Formulation ILP}

% Rattraper la partie 
% Présenter la formulation ILP du problème

\subsection{Routage avec qualit\'e de service}

Supposons un graphe $G=(V,E)$ ou chaque arête possède plusieurs valeurs contenues dans un vecteur
$w$ associé à chaque arêtes.
% fig pouah!!

Il existe plusieurs types de métriques : \begin{itemize}
    \item additives (sur un chemin, cumul lorsque l'on passe sur plusieurs chemins, la distance, la
        longueur, le délai sont des métriques additives)
    \item multiplicatives (les probabilités sont des métriques multiplicatives). Dans le cas de
        métriques multiplicatives, on peut revenir à une métriques additives (à l'aide du logarithme
        par exemple)
    \item concaves (la bande passante est une métrique concave, imaginons un chemin divisé en deux
        tronçons de débit différents, le débit du chemin est donné par le minimum du débit des deux
        tronçons)
\end{itemize}

On ne s'intéressera que aux métriques additives.

Pour notre problème chaque composante de $w$ est une métrique additive, on dispose aussi d'un
vecteur de qualité de service $L$. Alors un chemin est faisable si :
% fig chemin_faisable

On a alors deux problèmes envisageables pour le routage unicast avec qualité de service:
\begin{enumerate}
    \item on cherche un chemin faisable (MCP)
    \item on cherche un chemin faisable de cout minimum (MCOP)
\end{enumerate}

\begin{prop}
    Les deux problèmes sont NP-difficiles
\end{prop}

\subsubsection{SAMCRA}

Il s'agit d'un Dijkstra modifié. Pour prendre des décisions locales, il utilise une longueur
non-linéaire : \begin{displaymath}
    l(P) = \max_i \frac{\w_i(l)}{L_i}
\end{displaymath}
% Regarder SAMCRA
\end{document}
