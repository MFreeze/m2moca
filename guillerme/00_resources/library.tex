% <<< ENCODAGE ET POLICE
\usepackage[utf8]{inputenc}
\usepackage[T1]{fontenc}
\usepackage[french]{babel}
\usepackage{eurosym}
% >>>
% <<< ITEMIZE
\usepackage{enumitem}
% >>>
% <<< FIGURES
\usepackage{graphicx}
\usepackage{subfig}
% >>>
% <<< INSERTION DE CODE
\usepackage{moreverb}
% >>>
% <<< INSERTION DE PDF
\usepackage{pdfpages}
% >>>
% <<< RÉDUCTION DES MARGES
%\usepackage{fullpage}
\usepackage[top=2.5cm, bottom=2.5cm, left=3.5cm, right=3.5cm]{geometry}
\usepackage{fancyhdr}
% >>>
% <<< MATHÉMATIQUES
\usepackage[fleqn]{amsmath}
\usepackage{amsthm}
\usepackage{amssymb}
\usepackage{bbm}
\usepackage{amssymb}
\usepackage{stmaryrd}
% >>>
% <<< GRAPHES
\usepackage{tikz}
\usepackage{pgf}
\usetikzlibrary{arrows,automata,shapes}
% >>>
% <<< RESEAUX DE PETRI
\usetikzlibrary{petri}
% >>>
% <<< ARBRES
\usetikzlibrary{trees}
% >>>
% <<< SCHEDULING
\usepackage{pgfgantt}
\usepackage{scheduling}
% >>>
% <<< ALGORITHMES
\usepackage{algorithm}
\usepackage[noend]{algpseudocode}
\usepackage{ifthen}
%\usepackage{algorithmic}
%\usepackage{algorithmicx}
% >>>
% <<< COULEURS
%\usepackage[table]{xcolor}
% >>>
% <<< APPENDICE
\usepackage{appendix}
% >>>


% <<< THÉORÈMES, lEMMES -- NUMÉROTATION PERSONNALISÉE
\theoremstyle{plain}
\newtheorem{thrm}{Théorème}[section]
\newtheorem{lemma}{Lemme}[subsection]
\newtheorem{corol}{Corollaire}[section]
% >>>
% <<< THÉORÈMES, lEMMES -- NUMÉROTATION CLASSIQUE
\newtheorem{nthrm}{Théorème}
\newtheorem{nlemma}{Lemme}
\newtheorem{ncorol}{Corollaire}
\newtheorem{npb}{Problème}
\newtheorem{npbopt}{Problème (Optimisation)}
\newtheorem{npbdec}{Problème (Décision)}
% >>>
% <<< THÉORÈMES, lEMMES -- NON NUMÉROTÉS
\newtheorem*{thrm*}{Théorème}
\newtheorem*{lemma*}{Lemme}
\newtheorem*{corol*}{Corollaire}
% >>>
% <<< DÉFINITIONS, pROPRIÉTÉS -- NUMÉROTATION PERSONNALISÉE
\theoremstyle{definition}
\newtheorem{df}{Définition}[subsection]
\newtheorem{prop}{Propriété}[subsection]
\newtheorem{conj}{Conjecture}[subsection]
% >>>
% <<< DÉFINITIONS, pROPRIÉTÉS -- NUMÉROTATION CLASSIQUE
\newtheorem{ndf}{Définition}
\newtheorem{nprop}{Propriété}
\newtheorem{nconj}{Conjecture}
% >>>
% <<< DÉFINITIONS, pROPRIÉTÉS -- NON NUMÉROTÉS
\newtheorem*{df*}{Définition}
\newtheorem*{prop*}{Propriété}
\newtheorem*{conj*}{Conjecture}
% >>>
% <<< AUTRES -- NUMÉROTATION PERSONNALISÉE
\theoremstyle{remark}
\newtheorem{note}{Note}[subsubsection]
\newtheorem{pr}{Proposition}[subsection]
\newtheorem{rmq}{Remarque}[subsubsection]
\newtheorem{nott}{Notation}[subsection]
% >>>
% <<< AUTRES -- NUMÉROTATION CLASSIQUE
\newtheorem{nnote}{Note}
\newtheorem{npr}{Proposition}
\newtheorem{nrmq}{Remarque}
\newtheorem{nnot}{Notation}
% >>>
% <<< AUTRES -- NON NUMÉROTÉS
\newtheorem*{note*}{Note}
\newtheorem*{pr*}{Proposition}
\newtheorem*{rmq*}{Remarque}
\newtheorem*{not*}{Notation}
% >>>
% <<< EXEMPLES
\newtheorem*{ex}{Exemple}
\newtheorem*{exo}{Exercice}
\newtheorem*{rappel}{Rappel}
% >>>


% <<< OPÉRATEURS MATHÉMATIQUES
\DeclareMathOperator{\poly}{\textbf{poly}}
\DeclareMathOperator{\gap}{\mbox{gap}}
\DeclareMathOperator{\Bg}{\mbox{Big}}
\DeclareMathOperator{\Sml}{\mbox{Sml}}
\DeclareMathOperator{\Diff}{\mbox{Diff}}
\DeclareMathOperator{\card}{\mbox{Card}}
\DeclareMathOperator{\Obj}{\mbox{Obj}}
\DeclareMathOperator{\concat}{\bullet}
\DeclareMathOperator{\sig}{\mbox{Sig}}
\DeclareMathOperator{\ver}{\mbox{Ver}}
\DeclareMathOperator{\dist}{\mbox{dist}}
\DeclareMathOperator{\bw}{\mbox{branchwidth}}
\DeclareMathOperator{\tw}{\mbox{tw}}
\DeclareMathOperator{\lm}{\leq_m}
\DeclareMathOperator{\lc}{\leq_c}
\DeclareMathOperator{\lex}{\mbox{lex}}
\DeclareMathOperator{\ord}{\mbox{ord}}
% >>>


% <<< MACROS DIVERSES
\newcommand{\npc}{\textbf{NP}-complet}
\newcommand{\npcude}{\textbf{NP}-complétude}
\newcommand{\npd}{\textbf{NP}-difficile}
\newcommand{\la}{largeur arborescente}
\newcommand{\legendre}[2]{\left ( \frac{#1}{#2} \right )}
\newcommand{\wqo}[0]{\emph{well-quasi-ordered}}
\newcommand{\Tau}[0]{\mathcal{T} = (T, \{X_t : t \in T\})}
\newcommand{\Taug}[0]{\mathcal{T}_G = (T, \{X_t : t \in T\})}
\newcommand{\Taustar}[0]{\mathcal{T}^* = (T, \{X_t : t \in T\})}
\newcommand{\Taugstar}[0]{\mathcal{T}_G^* = (T, \{X_t : t \in T\})}
\newcommand{\pdsol}[1]{(X_{#1}^0, X_{#1}^1, X_{#1}^2, M_{#1})}

\newcommand{\authmach}[1]{$#1$-\emph{disponible}}
\newcommand{\tphase}[0]{phase}
\newcommand{\nbphase}[0]{I_{max}}
% >>>
% <<< NOM DE PROBLEMES
\newcommand{\hcycle}{\textsc{HAMILTONIAN CYCLE}}
\newcommand{\phcycle}[1]{\textsc{HAMILTONIAN CYCLE }paramétré par \textsc{#1}}
\newcommand{\vcover}{\textsc{VERTEX COVER}}
\newcommand{\twidth}{\textsc{TREE WIDTH}}
\newcommand{\wiset}{\textsc{WEIGHTED INDEPENDANT SET}}
\newcommand{\fvset}{\textsc{FEEDBACK VERTEX SET}}
\newcommand{\kvcover}{\textsc{K-Vertex Cover}}
\newcommand{\lpath}{\textsc{Longest Path}}
\newcommand{\fisched}{\textsc{Flexible Interval Scheduling}}
\newcommand{\fischedpi}{\textsc{Decision Flexible Interval Scheduling}}
\newcommand{\unitfischedpi}{\textsc{Decision Unit Flexible Interval Scheduling}}
\newcommand{\unitfisched}{\textsc{Unit Flexible Interval Scheduling}}
\newcommand{\precolor}{\textsc{Precoloring Extension}}
\newcommand{\isched}{\textsc{Interval Scheduling}}
\newcommand{\bisched}{\textsc{Basic Interval Scheduling}}
\newcommand{\isma}{\textsc{Interval Scheduling with Machine Availabilities}}
% >>>
% <<< MISE EN PAGE

\newcommand{\dfopt}[4][nolabel]{
\noindent
\begin{minipage}{\linewidth}
    ~\\
    \hrule
    \begin{npbopt}
        \ifthenelse{\equal{#1}{nolabel}}{}{\label{#1}}
        #2
        \begin{itemize}
            \itemindent 15mm
            \item[\textbf{Données :}] #3
            \item[\textbf{Résultat :}] #4
        \end{itemize}
    \end{npbopt}
    \hrule
    ~\\
\end{minipage}
}

\newcommand{\dfdec}[4][nolabel]{
\noindent
\begin{minipage}{\linewidth}
    ~\\
    \hrule
    \begin{npbdec}
        \ifthenelse{\equal{#1}{nolabel}}{}{\label{#1}}
        #2
        \begin{itemize}
            \itemindent 15mm
            \item[\textbf{Données :}] #3
            \item[\textbf{Question :}] #4
        \end{itemize}
    \end{npbdec}
    \hrule
    ~\\
\end{minipage}
}

\newcommand{\dfpb}[4][nolabel]{
\noindent
\begin{minipage}{\linewidth}
    ~\\
    \hrule
    \begin{npb}
        \ifthenelse{\equal{#1}{nolabel}}{}{\label{#1}}
        #2
        \begin{itemize}
            \itemindent 15mm
            \item[\textbf{Données :}] #3
            \item[\textbf{Question :}] #4
        \end{itemize}
    \end{npb}
    \hrule
    ~\\
\end{minipage}
}

\newcommand{\dfpbp}[5][nolabel]{
\noindent
\begin{minipage}{\linewidth}
    ~\\
    \hrule
    \begin{npb}
        \ifthenelse{\equal{#1}{nolabel}}{}{\label{#1}}
        #2
        \begin{itemize}
            \itemindent 15mm
            \item[\textbf{Données :}] #3
            \item[\textbf{Paramètre :}] #5
            \item[\textbf{Question :}] #4
        \end{itemize}
    \end{npb}
    \hrule
    ~\\
\end{minipage}
}
% >>>

    
%%% <<< STYLE
\tikzstyle{edge}=[
	-,
	draw=black
]

\tikzstyle{sized}=[
    minimum width=9mm
]

\tikzstyle{tvertex}=[
	node distance=15mm,
    inner sep = 1mm
]

\tikzstyle{tvert}=[
	node distance=5mm,
    inner sep=1mm
]

\tikzstyle{tmvertex}=[
    tvertex,
    sized
]

\tikzstyle{tmvert}=[
    tvert,
    sized
]

\tikzstyle{mvertex}=[
    tmvertex,
	circle,
	draw=black,
	fill=black!50
]

\tikzstyle{mvert}=[
    tmvert,
	circle,
	draw=black,
	fill=black!50
]

\tikzstyle{vertex}=[
    tvertex,
	circle,
	draw=black,
	fill=black!50
]

\tikzstyle{vert}=[
    tvert,
	circle,
	draw=black,
	fill=black!50
]

\tikzstyle{treedec}=[
	vertex,
	node distance=15mm,
	draw=red,
	fill=red!50
]

\tikzstyle{treeedge}=[
	edge,
	draw=red
]

\tikzstyle{giedge}=[
	edge,
	draw=blue
]

\tikzstyle{bounding}=[
	thick, %ultra
	%dash pattern=on 1mm off 1mm,
	densely dotted,
	fill=blue!10,
	draw=blue!70
]

\tikzstyle{patate}=[
    rounded corners=5pt,
    draw=black,
    fill=black!50,
    text width=12mm,
    text centered
]

\tikzstyle{patatoid}=[
    ellipse,
    node distance=10mm,
    draw=black
]
%%% >>>
% <<< DESSINS
\newcommand{\base}[4]{
    \draw[->] (0,0) to (0, #1);
    \draw[->] (0,0) to (#2, 0);
    
    \node at (0, #1+2) {#3};
    \node at (#2 + 2, -2) {#4};
}
% >>>

% <<< Bullet itemize
\newenvironment{bitemize}{
    \begin{itemize}[label=$\bullet$]
}
{
    \end{itemize}
}
% >>>
% <<< Psubfig
\newcounter{psubfig}
\renewcommand{\thepsubfig}{\alph{psubfig}}

\newcommand{\sfcaption}[1]{
    \addtocounter{figure}{1}
    \addtocounter{psubfig}{1}
    \textsc{Figure} \thefigure.\thepsubfig{} $-$ #1
    \addtocounter{figure}{-1}
}

% >>>
