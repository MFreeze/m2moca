% {{{ INTRO
Tout au long de ces notes de cours, nous nous intéresserons au problème du cycle Hamiltonien. 

%% {{{ PRESENTATION DU PLAN
Au cours de la première partie nous démontrerons la $NP-$complétude de \hcycle puis nous
détaillerons des bornes relatives au noyau de ce problème. Dans une seconde partie, nous développerons le
principe de décomposition arborescente et nous détaillerons la notion de largeur arborescente
associée. Cette valeur possède une grande importance dans le monde de la complexité paramétrée
puisqu'elle a permis d'introduire de nouvelles paramétrisations conduisant à des algorithmes FPT
pour des problèmes tels que \wiset ou \hcycle, basés sur la programmation dynamique. La partie
suivante développera en détail un de ces algorithmes sur le problème \hcycle. Enfin nous
présenterons les idées se trouvant derrière les algorithmes de programmation dynamiques utilisé pour
la résolution des problèmes tels que \wiset ou \fvset. 
%% }}}
% }}}
