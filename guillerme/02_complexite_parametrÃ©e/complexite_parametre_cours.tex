\documentclass[a4paper, 11pt]{thesis}

% {{{ PACKAGES
% {{{ ENCODAGE ET POLICE
\usepackage[utf8]{inputenc}
\usepackage[T1]{fontenc}
\usepackage[french]{babel}
\usepackage{eurosym}
% }}}

% {{{ FIGURES
\usepackage{graphicx}
\usepackage{subfig}
% }}}

% {{{ INSERTION DE CODE
\usepackage{moreverb}
% }}}

% {{{ INSERTION DE PDF
\usepackage{pdfpages}
% }}}

% {{{ RÉDUCTION DES MARGES
%\usepackage{fullpage}
%\usepackage[top=2.5cm, bottom=2.5cm, left=3.5cm, right=3.5cm]{geometry}
% }}}

% {{{ MATHÉMATIQUES
\usepackage{amsmath}
\usepackage{amsthm}
\usepackage{bbm}
\usepackage{amssymb}
\usepackage{stmaryrd}
% }}}

% {{{ GRAPHES
\usepackage{tikz}
\usepackage{pgf}
\usetikzlibrary{arrows,automata,shapes}
% }}}

% {{{ RESEAUX DE pETRI
\usetikzlibrary{petri}
% }}}

% {{{ ARBRES
\usetikzlibrary{trees}
% }}}

% {{{ DIAGRAMME DE GANTT
\usepackage{pgfgantt}
% }}}

% {{{ ALGORITHMES
\usepackage{algorithm}
\usepackage[noend]{algpseudocode}
%\usepackage{algorithmic}
%\usepackage{algorithmicx}
% }}}
% }}}

% {{{ NEWTHEOREM
% {{{ THÉORÈMES, lEMMES -- NUMÉROTATION PERSONNALISÉE
\theoremstyle{plain}
\newtheorem{thrm}{Théorème}[section]
\newtheorem{lemm}{Lemme}[subsection]
\newtheorem{corol}{Corollaire}[section]
% }}}

% {{{ THÉORÈMES, lEMMES -- NUMÉROTATION CLASSIQUE
\newtheorem{nthrm}{Théorème}
\newtheorem{nlemm}{Lemme}
\newtheorem{ncorol}{Corollaire}
\newtheorem{npb}{Problème}
% }}}

% {{{ THÉORÈMES, lEMMES -- NON NUMÉROTÉS
\newtheorem*{thrm*}{Théorème}
\newtheorem*{lemm*}{Lemme}
\newtheorem*{corol*}{Corollaire}
% }}}

% {{{ DÉFINITIONS, pROPRIÉTÉS -- NUMÉROTATION PERSONNALISÉE
\theoremstyle{definition}
\newtheorem{df}{Définition}[subsection]
\newtheorem{prop}{Propriété}[subsection]
\newtheorem{conj}{Conjecture}[subsection]
% }}}

% {{{ DÉFINITIONS, pROPRIÉTÉS -- NUMÉROTATION CLASSIQUE
\newtheorem{ndf}{Définition}
\newtheorem{nprop}{Propriété}
\newtheorem{nconj}{Conjecture}
% }}}

% {{{ DÉFINITIONS, pROPRIÉTÉS -- NON NUMÉROTÉS
\newtheorem*{df*}{Définition}
\newtheorem*{prop*}{Propriété}
\newtheorem*{conj*}{Conjecture}
% }}}

% {{{ AUTRES -- NUMÉROTATION PERSONNALISÉE
\theoremstyle{remark}
\newtheorem{nota}{Note}[subsubsection]
\newtheorem{pr}{Proposition}[subsection]
\newtheorem{rmq}{Remarque}[subsubsection]
% }}}

% {{{ AUTRES -- NUMÉROTATION CLASSIQUE
\newtheorem{nnota}{Note}
\newtheorem{npr}{Proposition}
\newtheorem{nrmq}{Remarque}
% }}}

% {{{ AUTRES -- NON NUMÉROTÉS
\newtheorem*{nota*}{Note}
\newtheorem*{pr*}{Proposition}
\newtheorem*{rmq*}{Remarque}
% }}}

% {{{ EXEMPLES
\newtheorem*{ex}{Exemple}
\newtheorem*{exo}{Exercice}
\newtheorem*{rappel}{Rappel}
% }}}
% }}}

% {{{ OPÉRATEURS MATHÉMATIQUES
\DeclareMathOperator{\poly}{\textbf{poly}}
\DeclareMathOperator{\gap}{\mbox{gap}}
\DeclareMathOperator{\Bg}{\mbox{Big}}
\DeclareMathOperator{\Sml}{\mbox{Sml}}
\DeclareMathOperator{\Diff}{\mbox{Diff}}
\DeclareMathOperator{\card}{\mbox{Card}}
\DeclareMathOperator{\Obj}{\mbox{Obj}}
\DeclareMathOperator{\concat}{\bullet}
\DeclareMathOperator{\sig}{\mbox{Sig}}
\DeclareMathOperator{\ver}{\mbox{Ver}}
\DeclareMathOperator{\dist}{\mbox{dist}}
\DeclareMathOperator{\bw}{\mbox{branchwidth}}
\DeclareMathOperator{\tw}{\mbox{tw}}
\DeclareMathOperator{\lm}{\leq_m}
\DeclareMathOperator{\lc}{\leq_c}
\DeclareMathOperator{\lex}{\mbox{lex}}
\DeclareMathOperator{\ord}{\mbox{ord}}
% }}}

% {{{ MACCROS
% {{{ MACROS DIVERSES
\newcommand{\npc}{$NP-$complet}
\newcommand{\npd}{$NP-$difficile}
\newcommand{\la}{largeur arborescente}
\newcommand{\legendre}[2]{\left ( \frac{#1}{#2} \right )}
\newcommand{\wqo}[0]{\emph{well-quasi-ordered}}
\newcommand{\Tau}[0]{\mathcal{T} = (T, \{X_t : t \in T\})}
\newcommand{\Taug}[0]{\mathcal{T}_G = (T, \{X_t : t \in T\})}
\newcommand{\Taustar}[0]{\mathcal{T}^* = (T, \{X_t : t \in T\})}
\newcommand{\Taugstar}[0]{\mathcal{T}_G^* = (T, \{X_t : t \in T\})}
\newcommand{\pdsol}[1]{(X_{#1}^0, X_{#1}^1, X_{#1}^2, M_{#1})}
% }}}
                
% {{{ NOM DE PROBLEMES
\newcommand{\hcycle}{\textsc{HAMILTONIAN CYCLE} }
\newcommand{\phcycle}[1]{\textsc{HAMILTONIAN CYCLE }paramétré par \textsc{#1} }
\newcommand{\vcover}{\textsc{VERTEX COVER} }
\newcommand{\twidth}{\textsc{TREE WIDTH} }
\newcommand{\wiset}{\textsc{WEIGHTED INDEPENDANT SET} }
\newcommand{\fvset}{\textsc{FEEDBACK VERTEX SET} }
\newcommand{\kvcover}{\textsc{K-Vertex Cover} }
\newcommand{\oct}{\textrm{\textsc{Odd Cycle Transversal}} }
\newcommand{\dlp}{\textrm{\textsc{Discrete Logarithm Problem}} }
\newcommand{\ecdlp}{\textrm{\textsc{Elliptic Curve Discrete Logarithm Problem}} }

% }}}

% {{{ MISE EN PAGE

\newcommand{\dfpbi}[3]{
    \begin{npb}[#1]
        \begin{itemize}
            \itemindent 10mm
            \item[\textbf{Données :}] #2
            \item[\textbf{Problème :}] #3
        \end{itemize}
    \end{npb}
}

\newcommand{\dfpb}[3]{
    \begin{npb}[#1]
        \begin{itemize}
            \itemindent 10mm
            \item[\textbf{Données :}] #2
            \item[\textbf{Question :}] #3
        \end{itemize}
    \end{npb}
}

\newcommand{\dfpbp}[4]{
    \begin{npb}[#1]
        \begin{itemize}
            \itemindent 10mm
            \item[\textbf{Données :}] #2
            \item[\textbf{Paramètre :}] #4
            \item[\textbf{Question :}] #3
        \end{itemize}
    \end{npb}
}
% }}}

% }}}
    
% {{{ TIKZ
%%% {{{ STYLE
\tikzstyle{edge}=[
	-,
	draw=black
]

\tikzstyle{sized}=[
    minimum width=9mm
]

\tikzstyle{tvertex}=[
	node distance=15mm,
    inner sep = 1mm
]

\tikzstyle{tvert}=[
	node distance=5mm,
    inner sep=1mm
]

\tikzstyle{tmvertex}=[
    tvertex,
    sized
]

\tikzstyle{tmvert}=[
    tvert,
    sized
]

\tikzstyle{mvertex}=[
    tmvertex,
	circle,
	draw=black,
	fill=black!50
]

\tikzstyle{mvert}=[
    tmvert,
	circle,
	draw=black,
	fill=black!50
]

\tikzstyle{vertex}=[
    tvertex,
	circle,
	draw=black,
	fill=black!50
]

\tikzstyle{vert}=[
    tvert,
	circle,
	draw=black,
	fill=black!50
]

\tikzstyle{treedec}=[
	vertex,
	node distance=15mm,
	draw=red,
	fill=red!50
]

\tikzstyle{treeedge}=[
	edge,
	draw=red
]

\tikzstyle{giedge}=[
	edge,
	draw=blue
]

\tikzstyle{bounding}=[
	thick, %ultra
	%dash pattern=on 1mm off 1mm,
	densely dotted,
	fill=blue!10,
	draw=blue!70
]

\tikzstyle{patate}=[
    rounded corners=5pt,
    draw=black,
    fill=black!50,
    text width=12mm,
    text centered
]

\tikzstyle{patatoid}=[
    ellipse,
    node distance=10mm,
    draw=black
]
%%% }}}

% {{{ DESSINS
\newcommand{\base}[4]{
    \draw[->] (0,0) to (0, #1);
    \draw[->] (0,0) to (#2, 0);
    
    \node at (0, #1+2) {#3};
    \node at (#2 + 2, -2) {#4};
}
% }}}
%}}}



% Bouquins intéressants :
%   * Parametrized Complexity Theory

\begin{document}

\section{Complexité paramétrée}
\subsection{Recherche de noyaux, bornes supérieures}

Noyau : en temps polynomial, on est capable de trouver une instance équivalente dont la taille ne
dépend que du paramètre.

Le fait que FPT = noyau, impose que la taille de de noyau et exponentielle.

% schéma 1
% Cours sur la complexite paramétrée à trouver sur le net

\subsubsection{Noyaux polynomiaux}

Vertex cover : on cherche un ensemble $S$ de $k$ sommets $d_1, d_2, \dots, d_k$, tels que si on enlève
ces sommets, $V \backslash S$ est indépendant.

\begin{df}
    Le voisinage clos de $x$, noté $N[x]$ est défini comme suit : $$
    N[x] = N(x) \cup \{x\} $$
\end{df}

\underline{Observation :}
\begin{itemize}
    \item Si $x$ est un sommet de degré $d > k$ alors $x \in S$
        $VERTEX COVER(G, k) = VERTEX COVER(G - x, k - 1)$
        % Preuve 1 + rajout des propositions
    \item Si $x$ est un sommet isolé, il n'existe aucune solution optimale contenant $x$
        $VERTEX COVER(G, k) = VERTEX COVER (G-x, k)$
    \item Si $x$ est un sommet de degré $d = 1$ voisin de $y$, alors il existe une solution optimale
        contenant $y$
        $VERTEX COVER(G, k) = VERTEX COVER(G - \{y, x\}, k - 1)$
\end{itemize}

\begin{lemma}
    Si $G$ possède un $VERTEX COVER$ de taille $k$, alors le graphe réduit possède au plus $k^2 + k$
    sommets et au plus $k^2$ arêtes.
\end{lemma}

\textbf{\underline{Preuve :}}

% Schéma 2

% A rédiger (en récupérant le poly)

\begin{df}
    Soit $(Q, \kappa)$ un problème paramétré sur l'alphabet $\Sigma$. Une fonction calculable en
    temps polynomial : $$
    K : \Sigma^* \longrightarrow \Sigma^* $$
    est une kernalisation de $(Q, \kappa)$ s'il existe une fonction : $$
    h : \mathbb{N} \longrightarrow \mathbb{N} $$
    telle que pour 
    % Récupérer le poly pour finir la définition
\end{df}

\begin{thrm}
    % Récupérer le poly
\end{thrm}

\begin{thrm}
    $VERTEX COVER$ ne possède pas de noyau formé par un graphe de taille $1,36 k$ sommets à moins
    que $P = NP$
\end{thrm}

\subsubsection{Autres exemples de noyaux quadratiques}

$3 HITTING SET$ :
% A faire à la maison
\begin{itemize}
    \item \textbf{Entrée :} Ensemble de sommets avec des relations ternaires de taille $3$ sur ces
        sommets
    \item \textbf{Problème :} Peut on trouver $k$ sommets touchant l'ensemble des ensembles.
\end{itemize}

\begin{df}
    Un module est un ensemble de sommet tels que un sommet extérieur se comporte de la même manière
    vis à vis de tous les sommets du module. C'est à dire, un sommet extérieur est soit voisin de
    tous les sommets composant le module, soit n'est voisin d'aucun d'entre eux.
\end{df}

$FAST$

\textbf{Observation :}

Un tournoi possède un circuit si et seulement si il contient un circuit de longueur $3$.

\textbf{Preuve :}

Soit $C$ un circuit de longueur minimum, supposons que $C$ ne soit pas un triangle. Il existe donc
$x, y, z$ consécutifs dans $C$ tels que l'arc entre $x$ et $z$ n'appartient pas à $C$. Mais le
tournoi $T$ contient un arc entre $x$ et $z$. Or $zx ^ \rightarrow \not \in C$, sinon $\{x, y, z\}$
formerait un triangle, donc $xz ^ \rightarrow \in T$ d'où les sommets de $C \backslash \{y\}$
forment un cycle plus petit que $C$, contradiction.

% Ecrire la preuve sur la règle de Sunflower

% Rattraper le cours

Le problème du vertex cover est similaire au problèle suivant : supprimer au plus $k$ arêtes d'un
graphe $G$ pour obtenir un graphe $H$ sans $K_2(g)$ réduit :
\begin{displaymath}
VC \cong \{K_2\}-\mbox{vertex deletion}
\end{displaymath}

\textbf{$\mathcal{F}-$Vertex Deletion}\\

\textbf{Question :} Le noyau sur $d-HS$ implique-t-il un noyau polynomial sur le problème
$\mathcal{F}-$vertex deletion si $\mathcal{F}$ est finie? NON

\begin{lemma}
    Soit $\mathcal{H} = (V, \mathcal{E})$, tel que $\forall E \in \mathcal{E}, |E| \leq d$. Soit $k
    \in \mathbb{N}^*$, si : $|\mathcal{E}| > (k+1)^d d!$ alors $\exists \mathcal{E}' \subset
    \mathcal{E}$ tel que $X$ est une couverture (HS) de taile $k$ de $\mathcal{H}$ si et seulement
    si $X$ est une couverture de $\mathcal{H}' = (V, \mathcal{E}')$.
\end{lemma}

%%%%% GROS GROS GROS Trou à combler

\begin{thrm}[Courcelle]
    Toute propriété de graphe exprimable en MSOL peut être testée en temps $f(k).n$ sur les graphes
    de treewidth $\leq k$
\end{thrm}

\begin{rq}
    Le théorème de Courcelle n'est pas une caractérisation.
\end{rq}

\begin{ex}
    Ce théorème n'explique pas que le cycle Hamiltonien paramétré par treewidth est $FPT$
\end{ex}

\subsection{Les limites du théorème de Courcelle}

\subsubsection{Le théorème est il constructif?}

Il s'agit d'un domaine de recherche très actif en ce moment. On peut le rendre constructif mais
c'est très compliqué et l'on a pas réussi à rendre constructif toutes ses applications.

\subsubsection{La complexité}

La fonction $f(k)$ donnée par le théorème dépend de la formule $\phi$ et la hauteur de la tour
d'exponentiels dépend des alternances : $\forall - \exists$ dans $\phi$ et c'est une borne inf.

Le théorème de Courcelle et un outil de preuve d'existence d'algo FPT.

\begin{exo}
    Sachant que l'existence d'un chemin de longueur $k$ dans un graphe est exprimable en MSOL.
    Montrer que $k-$longest path est FPT.

    \begin{enumerate}
        \item On calcule un arbre $T$ de parcours en profondeur de $G$ %FIG 1A
            \begin{itemize}
                \item Soit il existe un chemin de longueur $\geq k$ entre $r$ et une feuille
                    (réponse en temps polynomiale)
                \item Soit $T$ a une hauteur $< k$
            \end{itemize}
        \item Si $T$ a une hauteur $< k$, on contruit une décomposition arborescente :
            Les sous ensembles sont l'ensemble des branches de l'arbre
            % fig 1 -B
            % Prouver la validité de la décomposition arborescente
            % Ecrire un algo de programmation dynamique
    \end{enumerate}
\end{exo}

% {{{ Logique monadique du second ordre

\section{Logique monadique du second ordre}

\subsection{Structure relationnelle}

Une structure $S$ est définie par :
\begin{itemize}
    \item un vocabulaire $\tau$ ensemble des symboles relationnels d'arité $\geq 1$
    \item un univers $U$
\end{itemize}

\subsubsection{Représentation des graphes}

% Fig ZC

$\tau_1 = \{E_1\}, \quad U_1 = V \quad E_1 \subseteq V^2$ symbole relationnel définissant les
arêtes. Pour le graphe donné, on aura alors : $U = \{a, b, c\}$ et la relation $E_1^G = \{(a,b),
(b,c) (c,a)\}$.
\\

On peut aussi les définir comme suit :\\
$\tau_2 = \{ VERT, EDGE, INC \}, \quad U_2 = V \cup E $. Ce qui nous donne pour le graphe précédent
: $U_2^G = \{1,2,3,a,b,b\}$ et les relations $VERT^G = \{a, b, c\},\ EDGE^G = \{1,2,3\}$ et $(2,b) \in
INC^G$.

\subsubsection{Représentation des chaînes de caractères}

$\tau_c = \{\geq\} \cup \{P_a | a \in \Sigma\}, \quad U = [n]$.

Prenons comme exemple : $\Sigma = \{a, b\}$ et $s = ababb$, on aura alors $U_s = \{1,2,3,4,5\}$, les
relations : $1 \geq 2 \geq 3 \geq 4 \geq 5$, $P_a^s = \{1, 3\}$ et $P_b^s = \{2, 4, 5 \}$.

\subsection{Petit glossaire}

\subsubsection{Variables individuelles}

$x, y, \dots$ : ce sont les variables de l'univers

\subsubsection{Formules atomiques}

$x=y, x\mathcal{R}y$ Relations d'équivalences entre les variables de l'univers

\subsubsection{Formules du premier ordre}

% fig ZZATR

\begin{exo}
    Trouver une formule $MSOL_1$ pour exprimer la connexité d'un graphe.\\

    \begin{displaymath}
        \exists X (\exists Xx \wedge \exists y \neg Xy \wedge (\forall u,v (Xu \wedge \neg Xv \Rightarrow
        \neg E uv ))) 
    \end{displaymath}

    Ce qui nous donne : \\
    $\exists$ un ensemble de sommets $X$ tq $X \neq \emptyset$ et $\overline{X} \neq \emptyset$ et
    $\forall u \in X, v \not \in X \Rightarrow (u,v) \not \in E$.
\end{exo}

\begin{exo}
    Trouver une formule $MSOL_1$ pour exprimer la présence d'un vertex cover de taille $k$.
\end{exo}

% Programme du 14/12 :
% Lemme de Buchi
% Pumping Lemma (Lemme de l'étoile)
% Correspondance automates finis <-> MSOL
% Correspondance automates d'arbres <-> MSOL (Automates d'arbres <=> Prog Dyn)

\subsection{Langages réguliers et MSOL}

Un petit rappel : les chaînes de caractères de longueur $n$ sur $\Sigma$: \[
    \tau_\Sigma < [n],\ \leq,\ \{P_a | a \in \Sigma \} >
\]

La relation $\leq$ est une relation binaire introduisant un ordre sur les caractères, la relation
$[]$ est la relation d'indiçage ($[3]$ renvoie la lettre d'indice 3) et $P_a$ renvoie l'ensemble des
indices des lettres $a$ dans la chaîne.

% {{{ Théorème de Buchi
\begin{thrm}[Buchi]
    Les assertions suivantes sont équivalentes : \begin{enumerate}
        \item $L$ est un langage régulier
        \item $L$ est reconnaissable par un automate fini $\mathcal{A}$
        \item $L$ est définissable en MSOL
    \end{enumerate}
\end{thrm}

\begin{proof}
    Il existe un algorithme étant donné une formule MSOL $\phi$ tel que $L = L(\phi)$ calcule
    $\mathcal{A}$ et vice versa.
    \begin{itemize}
        \item[$2 \Rightarrow 3$] \emph{Intuition :} la formule \emph{devine} les états de l'automate
            et décrit les fonctions de transition, les états initiaux, $\dots$

            Posons l'hypothèse $\mathcal{A}$ tel que $L = L(\mathcal{A})$ un automate à $m$ états.
            \[
                \phi = \underbrace{\exists X_1 \dots \exists X_m}_A
                \left ( \overbrace{\mbox{unique}}^B \wedge
                \underbrace{\mbox{start}}_C \wedge
                \mbox{transition} \wedge \overbrace{\mbox{accept}}^D \wedge
                \underbrace{\mbox{non vide}}_E \right )
            \]
            Avec : \begin{itemize}
                \item[$A$ :] On devine les états
                \item[$B$ :] A chaque position de la chaine de caractère l'automate est dans une position unique
                \item[$C$ :] $\mathcal{A}$ possède un état initial
                \item[$D$ :] états finaux
                \item[$E$ :] La chaine vide n'est pas acceptée
            \end{itemize}

            $X_i$ variable d'ensemble de positions de la chaîne de caractères, elle va contenir
            l'ensemble des positions $j$ de la chaîne $s$, telles que avant de lire le caractère
            $c_j$, $\mathcal{A}$ est dans l'état $e_i$. \\

            \begin{itemize}
                \item[\underline{unique} :] Les $X_i$ forment une partition des positions du mots, ce qui
                    s'écrit : \[
                        \forall x \left ( \bigvee_{i \in [m]} X_i x n ( \bigwedge_{1 \leq i < i' \leq m}
                        (\neg X_i x \vee \neg X_i x )) \right )
                    \]
                \item[\underline{start} :] \[
                        \forall x ( \forall y x \leq y \Rightarrow X_1 x )
                    \]
                \item[\underline{transition} :] \[
                        \forall x \forall y \left ( x \leq y \wedge x \neq y \wedge \forall z ( z
                        \leq x \vee y \leq z)\right ) \Rightarrow \bigvee_{(i, a, i') \in \Delta} (X_i x
                        \wedge P_a x \wedge X_i y )
                    \]
                \item[\underline{accept} :] \[
                        \forall x \left ( \forall z, z \leq x\right ) \Rightarrow \bigvee_{(i, a, i') \in
                        \Delta i' \in F} (X_i x \wedge P_a x)
                    \]
                \item[\underline{non vide} :] \[
                        \exists x x=x
                    \]
            \end{itemize}
            On remarque que la formule est linéaire en la taille de l'automate.
        \item[$2 \Leftarrow 3$] % Transformation 1412trans
    \end{itemize}
\end{proof}
%}}}

\subsubsection{Automates d'arbre}

\begin{df}[Arbre binaire]
    $T = \left < V_T, E_1, E_2, \lambda \right >$, avec $V_T$ l'ensemble des sommets de $T$, $E_0,
    E_1$ les fils droits et gauches et $\lambda$ la fonction d'étiquette : $\lambda : T
    \longrightarrow \Sigma$
\end{df}

% Fig 1412Arbre

\begin{df}[Automate d'arbre binaire]
    $\mathcal{A} = \left < S, \Sigma, \Delta, F \right >$, avec $S$ l'ensemble des états, $\Delta$
    l'ensemble des transitions et $F$ les état finaux.

    $\Delta_i : S \cup \{ \perp\} \times S \cup \{\perp\} \times \Sigma \longrightarrow S$ \\
    $\perp \sim $ état initial qui correspond aux feuilles.
\end{df}

\begin{ex}
    $S = \{S_0, S_1\}$, $\Sigma ) \{0, 1\}$, $F = \{S_0\}$ et : \[
        \Delta : \begin{array}{rcl}
            (\perp, \perp) & \xrightarrow{1} & S_1 \\
            (\perp, \perp) & \xrightarrow{0} & S_0 \\
            (S_i, \perp)   & \xrightarrow{j} & S_{(i + j) \mod 2} \\
            (\perp, S_i)   & \xrightarrow{j} & S_{(i + j) \mod 2} \\
            (S_i, S_j)     & \xrightarrow{k} & S_{(i+j+k) \mod 2} \\
        \end{array}
    \]
    Cet automate reconnaît les arbres binaires étiquetés par un nombre pair de $1$.
\end{ex}

\begin{lemma}[Généralisation de Buchi]
    Un langage d'arbre est reconnu par un automate d'arbre fini si et seulement so il est définit
    par une formule MSOL
\end{lemma}

\begin{corol}
    Si $\phi$ est une formule MSOL sur une structure d'arbre\footnote{Par abus de langage, on
    utilisera arbre}, alors tester si un arbre vérifie $\phi$ est FPT paramétré par la longueur de
    la formule.
\end{corol}

% }}}

% {{{ Lemme de l'étoile
\begin{lemma}[Lemme de l'étoile]
    Si $L$ est un langage régulier, alors il existe un entier $P$ (appelé longueur de pompage) tel
    que si $s \in L$ et $|s| \geq p$ alors $s = xyz$ \begin{enumerate}
        \item $\forall i \geq 0,\ xy^iz \in L$
        \item $|y|  > 0$
        \item $|xy| \leq p$
    \end{enumerate}
\end{lemma}
%}}}

%{{{ Largeur de Branche
\section{Largeur de branches}

\subsection{Définitions et exemples}

\begin{df}
    Une décomposition en branches d'un graphe $G=(V,E)$ est une paire $(T, \mu)$ où :
    \begin{itemize}
        \item $T$ est un arbre où tous les sommets internes ont degré $3$ (ternaire)
        \item $\mu : L(T) \rightarrow E(G)$ est une bijection entre les feuilles de $T$ et les
            arêtes de $G$
    \end{itemize}
\end{df}

\begin{ex}
    % figure 1.1
    \begin{itemize}
        \item Chaque arête $e$ de $T$ définit une partition $\{a_e, B_e\}$ de $E(G)$
        \item Pour toute arête $e \in E(T)$, on définit $\mid(e) = V(A_e) \cap V(B_e)$
        \item La largeur de $(T, \mu)$ est $w(T, \mu) = \max_{e \in E(T)} |\mid(e)|$
        \item On définir la largeur de branches de $G$ : 
            \begin{displaymath}
                \bw = \min w(T, \mu)
            \end{displaymath}
            Par définition, si $|E(G)| \leq 1$, alors $\bw(G) = 0$
    \end{itemize}
\end{ex}

\subsection{Propriétés basiques}

\begin{prop}
    Si $(T, \mu)$ est une décomposition en branches d'un greaphe $G$ et $e \in E(T)$ telle que
    $|\mid(e)| \geq 1$, alors $\mid(e)$ est un séparateur de $G$.
\end{prop}

\begin{proof}
    %fig 1.2
    La preuve se fait par l'absurde.
\end{proof}

\begin{prop}
    Si $H$ est un mineur de $G$ alors $\bw(H) \leq \bw(G)$
\end{prop}

\begin{ex}
    % fig 1.3
\end{ex}

\begin{proof}
    Si $|E(H) \leq 1|$ alors la propriété est satisfairte. On peut donc supposer que $|E(H) \geq
    2|$. Soit $(T, \mu)$ une décomposition en branche de $G$, soit $S$ le sous arbre connexe de $T$
    tel que $\mu^{-1}(e) \in V(S) \forall e \in E(H)$. Soit $T'$ l'arbre obtenu à partir de $S$ en
    supprimant les sommets de degré 2 et soit $\mu'$ la restriction de $\mu$ aux arêtes de $H$.
    Alors $(T', \mu')$ est une décomposition en branches de $H$ de largeur au plus $w(T, \mu)$
\end{proof}

\begin{prop}
    \begin{enumerate}
        \item $\bw(G) = 0$ ssi toute composante connexe de $G$ a au plus une arête
        \item $\bw(G)$ % Rattraper
    \end{enumerate}
\end{prop}

\begin{prop}
    \begin{displaymath}
        \mbox{Si }n \geq 3$, $\bw(K_n) \leq \left \lceil \frac{2n}{3} \right \rceil
    \end{displaymath}
\end{prop}

\begin{proof}
    $V(K_n) = V_1 \cup V_2 \cup V_3$ tels que $|V_i| \leq \lceil n/3 \rceil$.
    % fig 1.4
    On a bien : $E(K_n) = E_{1,3} \cup E_{3,2} \cup E_{2,1}$
    On peut alors dessiner $T$ de la manière suivante :
    % fig 1.5
    Si l'on prend $e_1$, alors $|\mid(e_1)| \leq |V_1| + |V_2| \leq \lceil 2n/3 \rceil$, et ce parce
    que les sommets de $V_3$ ne sont touchés par aucune arête de $E_{2,1}$. On peut étendre le
    raisonnement à $e_2$ et $e_3$. 

    Considérons à présent une arête $e$ inclue dnas $E_{2,1}$
    % La partition définie est un sous ensemble de celle définie par e_1 ==> V_3 n'est toujours pas
    % touché et donc l'inégalité reste valable
\end{proof}

\begin{prop}
    On a aussi :
    \begin{displaymath}
        \bw(K_n) \geq \left \lceil \frac{2n}{3} \right \rceil
    \end{displaymath}
\end{prop}

\subsection{Relation entre treewidth et branchwidth}

$\tw(G) \leq f(\bw(G)) \forall G ?$
On sait que $\tw(K_n) = n - 1$ et $\bw(K_n) = \lceil 2n/3 \rceil$ ce qui implique que $\tw + 1 \leq
3/2 \bw$.

\begin{prop}
    \begin{displaymath}
        \forall G,\ \tw(G) + 1 \leq \frac{3n}{2} \bw(G)
    \end{displaymath}
\end{prop}

\begin{proof}
    Soit $(T, \mu)$ une décomposition en branches de $G$. Pour tout sommet $t \in V(T)$, on définit
    $X_t \subseteq V(G)$ : 
    \begin{itemize}
        \item si $t$ est une feuille, soit $X_t$  les deux sommets appartenant à l'arête $\mu(t)$
        \item sinon, soit $X_t$ l'ensemble des sommets $v$ de $G$ tels que $\exists f, g \in E(G), v
            \in e, f,$ tels que $t$ st dans le chamin dans $T$ entre $\mu^{-1}(f)$ et $\mu^{-1}(g)$
    \end{itemize}
    % fig 1.6
    Alors $(T, \{X_t : t \in V(T) \})$ est une décomposition en arborescente de $G$. On cherche
    alors à borner $w(T, \{X_t\})$ :
    \begin{itemize}
        \item si $t$ est une feuille : $|X_t| - 1 = 1 $
        \item si $t$ est un sommet interne :
            \begin{displaymath}
                % fig 1.6a
                \forall v \in X_t, v \in \mid(e_1) \cap \mid(e_j)
            \end{displaymath}
            avec $i \neq j$ et $i, j \in \{1, 2? 3\}$.
    \end{itemize}
    On a alors : \begin{displaymath}
        \begin{array}{rcll}
            2|X_t| & \leq & |\mid(e_1)| + |\mid(e_2)| + |\mid(e_3)| & \forall t \\
                   & \leq & 3 w(T, \mu) \\
                   & \leq & 3/2 \bw(G) \\
        \end{array}
    \end{displaymath}
    $\tw(G) \leq w(T, \{X\}) = \max_{t \in V(t)} \{|X_t| - 1\} \leq 3/2 \bw -1$
\end{proof}

\begin{prop}
    Pour tout graphe $g$ : 
    \begin{displaymath}
        \bw(G) \leq \tw(G) + 1
    \end{displaymath}
\end{prop}

\begin{prop}
    \begin{displaymath}
        \forall G, \quad \bw \leq \tw + 1 \leq \frac{3}{2} \bw
    \end{displaymath}
\end{prop}

\subsection{Résultat de complexité}

\begin{thrm}[Seymour et Thomas, 1991]
    Calculer la largeur de branches d'un graphe est $NP-$complet.
\end{thrm}

\begin{thrm}[Thilikos et Rodlerender, 1997]
    Soir $k$ un entier et $G$ un graphe avec $n = |V(G)|$. Il existe un algorithme tel que si
    $\bw(G) \leq K$, alors il construit une décomposition en branches de $G$ de largeur $k$ en tmps
    $f(k).n$. Il existe aussi des algorithmes d'approximation.
\end{thrm}

\section{Programmation Dynamique dans les graphes planaires}

\subsection{Graphes plongés dans une surface topologique}

Il s'agit de graphes dessinés sans croisement d'arêtes.

\begin{ex}
    $K_5$ n'est pas planaire (Euler $m \leq 3n - 6$), mais il est possible de le plonger sur un
    tore.
\end{ex}

\begin{df}
    Soit $G$ un graphe plongé dans une surface $\Sigma$. Un \underline{noose} de $G$ est un
    sous-ensemble de $\Sigma$ qui est homéomorphe à un cercle  intersectant $G$ seulement sur des
    sommets.
    % fig 1.7
\end{df}

\subsection{SC-décomposition des graphes planaires}

\begin{df}
    Soit $G$ un graphe plan. Une \underline{sphere-cut décomposition} de $G$ est une décomposition
    en branches $(T, \mu)$ de $G$ avec la propriété supplémentaire suivante : \\

    Pour toute arête $e \in E(T)$, il existe un noose, $O_e$ de $G$ tel que $O_e \cap V(G) =
    \mid(e)$
\end{df}

\subsection{Programmation Dynamique sur une SC-décomposition}

%fig 05121

\subsubsection{Exemple : Le cycle hamiltonien}

% Idée de prog dyn pour la décomposition arborescente fig 05122

On va faire de la programmation dynamique sur une SC-décomposition (on est dans les graphes
\underline{planaire}).

On sait que pour une SC-décomposition $(T, \mu)$, pour toute arête $e \in E(T)$, les sommets du
séparateur $\mid(e)$ sont placés autour d'un noose $O_e$ dans le plan.

% fig 05123

A l'intérieur du noose, on a le graphe induit par $e$ dans la décomposition et à l'extérieur, on a
le reste du graphe $G$. Comme $G$ est planaire, alors $G_e$ est aussi un graphe planaire.

Cette configuration peut être abstraite. % fig 05124

%Le but est de sauvegarder dans les tables les différentes configurations possibles dans la même
%lignée que la tree décomposition : en effet les sommets appartenant au noose forment un séparateur,
%ils peuvent donc chacun appartenir à trois ensembles.

Le nombre de $4-$upplets $(X_e^1, X_e^2, X_e^3, M)$ est borné par $6^k$, il y a $3^k$ $3-$upplets
$(X_e^1, X_e^2, X_e^3)$ et $2^k$, d'où le $6^k$.

Lorsque l'on remonte dans la décomposition, on va chercher à fusionner les différentes tables,
offrant $36^k$ entrées possibles dans la table, seulement on vient de montrer que le nombre de
configuration maximale acceptables est $6^k$ on va donc nettoyer la table et recommencer.

% fig 05125

\subsection{Généralisation de cette technique}

\subsubsection{Problèmes \emph{faciles}}

On parle de problèmes faciles par rapport à la programmation dynamique : 
\begin{itemize}
    \item ensemble indépendant
    \item vertex cover
    \item dominating set
\end{itemize}

On les appelle \emph{faciles}, puisque la vérification des propriétés est locale :
pour l'ensemble indépendant, vérifier qu'ils sont indépendants se limite à vérifier pour chacun des
sommets contenus dans l'ensemble que l'on doit vérifier.

\subsubsection{Problèmes \emph{difficiles}}

Quelques problèmes difficiles :
\begin{itemize}
    \item cycle hamiltonien
    \item longest path
    \item couverture connexe par sommets
\end{itemize}

Le problème hamiltonien est plus difficile puisqu'il est soumis à une contrainte globale
(connexité), on ne peut pas tester de manière locale si un ensemble de sommets est connexe. Ce sont
des problèmes pour lesquels la SC-décomposition dans les graphes permet d'obtenir des algos en
exponentiel simple.

\subsubsection{Couverture connexe par sommets}

On doit stocker : $(X_e^0, X_e^1, P)$, $P$ représente un ensemble de taille au plus $k$ définissant
une composante connexe. Ce nombre de composantes connexes et borné par le nombre de partitions non
croisées à savoir le nombre catalan d'un ensemble de taille $k$:\begin{displaymath}
    |P| \leq CN(k) \sim 4^k
\end{displaymath}

\section{Bidimensionnalité}

\subsection{Introduction}

Théorié introduite par Fomin, Demaine, Hajiaghayi, Thilikos (2004).
L'objectif est d'obtenir des algorithmes paramétrés en temps
$2^{o(k)}.n^{O(1)}$\footnote{Typiquement $2^{O(\sqrt{k})}$}, il s'agit d'un
temps d'exécution sous exponentiel.

\begin{conj}[ETH : Exponential Time Hypothesis]
    On ne peut pas résoudre SAT en temps $2^{o(n)}$, $n$ étant le nombre de variables.
\end{conj}

\begin{rappel}
    % fig 05126
\end{rappel}

\begin{df}[$H \leq_c G$]
    $H$ est une contraction e $G$ si $H$ peut être obtenu à partir de $G$ en contractant des arêtes.
\end{df}

\begin{df}[$H \leq_m G$]
    $H$ est un mineur de $G$ si $H$ peut être obtenu à partir d'un sous graphe de $G$ en contractant
    des arêtes.
\end{df}

\begin{df}
    Une classe de graphe $\mathcal{G}$ est fermée par mineur si : \begin{displaymath}
        \left . \begin{array}{rcl}
            G & \in & \mathcal{G} \\
            G & \leq_m & G \\
        \end{array} \right \rbrace \Rightarrow H \in \mathcal{G}
    \end{displaymath}
\end{df}

Quelques classes fermées par mineur :
\begin{itemize}
    \item les graphes planaires sont fermés par mineur
    \item les graphes de $\bw \leq K$ ou $\tw \leq k$
    \item les forêts
    \item les ensembles indépendants 
    \item $\dots$
\end{itemize}

Quelques classes non fermées par mineur :
\begin{itemize}
    \item les forêts (la suppression d'un sommet donne une forêt)
    \item les cliques
    \item $\dots$
\end{itemize}

\begin{df}[well-quasi-ordered]
    Toute séquence infine de graphe $G_1, G_2, \dots$ contient 2 graphes $G_i, G_j$ tels que $G_i
    \lm G_j$
\end{df}

\begin{thrm}[Robertson et Seymour, 1986 - 2004]
    Les graphes finis sont \emph{well quasi ordered} par la relation $\leq_m$.
\end{thrm}

\begin{thrm}[Kruskal, 1960]
    Les arbres finis sont \wqo par la relation $\lm$
\end{thrm}

Ceci n'est pas vrai pour tous les graphes connexes avec la relation $\lc$ :
% fig 05127
% Faire la preuve sur la famille de graphe fournie (tips : la famille ne contient que des carrés, la
% contraction d'une arête donne un triangle, puis trois possibilités pour supprimer le triangle,
% aucune n'est concluante)

\begin{corol}[Robertson et Seymour]
    Toute classe de graphe fermée par mineurs a un ensemble \fbox{fini} de mineurs exclus minimaux.
\end{corol}

Un mineur exclu minimal est le plus petit mineur tel qu'il n'appartient plus à la la classe.
Un cas particulier nous est donné par la classe des graphes planaires : en effet, $G$ est planaire
si et seulement si $K_{3,3} \not \lm G$ et $K_5 \not \lm G$.

\begin{proof}
    On liste tous les graphes qui n'appartiennent pas à la classe. On a alors deux cas :
    \begin{itemize}
        \item si la liste est finie, il n'y a rien à prouver
        \item si la liste est infinie, d'après Robertson et Seymour, on peut trouver deux graphes
            $G_i, G_j$ tels que $G_i \lm G_j$, on garde alors $G_i$ et on recommence.
    \end{itemize}
\end{proof}

\begin{df}
    Étant donné un paramètre $P : \mathcal{G} \longrightarrow \mathbb{N}$, le problème associé est
    le suivant : \\
    Etant donné un graphe $G \in \mathcal{G}$, un entier fixé $k$, on se pose la question :
    \begin{displaymath}
        P(G) \left . \begin{array}{c}
            = \\ \leq \\ \geq \\
        \end{array}
        \right .
        K ?
    \end{displaymath}
\end{df}

\begin{df}
    Un paramètre $P$ est fermé par mineur (resp. par contraction) si :
    \begin{displaymath}
        \forall G, H \quad H \lm G\ (\mbox{resp. } H \lc G) \Rightarrow P(H) \leq P(G)
    \end{displaymath}
\end{df}

%fig 05128

% Taille de max independant set, taille min feedback vertex set, taille min connected dominating set
% sont ils fermés par contraction, mineur, les deux ?

Supposons que l'on veuille répondre à la question $P(G) \leq K$ pour un graphe $G \in \mathcal{G}$.
Supposons :
\begin{enumerate}[A]
    \item $\forall G \in \mathcal{G}, \bw(G) \leq \alpha \sqrt{P(G)}$ pour une constante $\alpha >
        0$
    \item $\forall G \in \mathcal{G}$ et étant donné une décomposition en branche optimale $(T,
        \mu)$ de $G$ on peut calculer $P(G)$ en temps $2^{O(\omega(T,\mu))} n^{O(1)}$ (temps
        simplement exponentiel)
\end{enumerate}

\begin{thrm}
    Si on a $(A) + (B)$, alors on peut décider si $P(G) \leq K$ en temps $2^{O(\sqrt{k})} n^{O(1)}$
    pour $G \in \mathcal{G}$ avec $n = |V(G)|$
\end{thrm}

\begin{proof}
    Supposons que l'on ait une décomposition en branche $(T, \mu)$ optimal de $G$. On distingue deux
    cas : \begin{enumerate}
        \item Si $\bw(G) > \alpha \sqrt{k}$, alors d'après $(A)$ : \[
                \alpha \sqrt{k} < \bw(G) \leq \alpha \sqrt{P(G)} \Rightarrow k < P(G)
            \]
        \item  Si $\bw(G) \leq \alpha \sqrt{k}$, d'après $(B)$, je peux calculer $P(G)$ en temps :
            \[
                2^{O(\omega(T, \mu))} n^{O(1)} = 2^{O(\sqrt{k})} . n^{O(1)}
            \]
    \end{enumerate}
\end{proof}

\subsubsection{Comment provuer la partie $(A)$?}

\begin{thrm}[Robertson, Seymour, Thomas, 1994]
    Soit $l \geq 1$, un entier, si $G$ est un graphe planaire, tel que $\bw(G) \geq l$, alors : \[
        H_{\frac{l}{4} \times \frac{l}{4} \geq_m G
    \]
    Avec $H_{a \times a}$ la grille à $a \times a$ sommets.
\end{thrm}

La borne a été améliorée depuis : \[
        H_{\frac{l}{3} \times \frac{l}{3} \geq_m G
    \]

Si $G$ est planaire : \[
    \bw(G) \geq 3l \Rightarrow H_{l \times l} \geq_m G
\]
Par contre si $G$ est quelconque : \[
    \bw(G) \geq 20^{2^{l^5}} \Rightarrow H_{l \times l} \geq_m G
\]

Mais la borne n'est pas optimale.

\begin{ex}[\kvcover]
\end{ex}

\begin{df} 
    Un paramètre $P$ est \emph{mineur bidimmensionnel} avec densité $\delta$ si : \begin{enumerate}
        \item $P$ est fermé par mineurs
        \item $P(H_{r \times r} \geq (\delta r)^2$
    \end{enumerate}
\end{df}

\begin{lemma}
    Si $P$ est un paramètre mineur bidimmensionnel, alors $P$ satisfait la propriété $(A)$ dans les
    graphes planaires.
\end{lemma}

\begin{ex}
    \begin{itemize}
        \item[\vcover] \[
                vc(H_{r \times r}) \geq \frac{r^2}{2} \rightarrow \delta \leq \frac{1}{\sqrt{2}}
            \]
        \item[\fvset] est bien fermé par mineur :
            % fig 211212
            \[
                fvs(H_{r \times r}) \geq \frac{r^2}{4} \rightarrow \delta = \frac{1}{2}
            \]
        \item[\lpath] est bien fermé par mineur :
            \[
                lp(H_{r \times r}) = r^2 \rightarrow \delta = 1
            \]
        \item[\textsc{Minimum Maximal Matching}] consiste à rechercher un couplage
            maximal\foontote{Que l'on ne peut plus étendre} de taille minimum.
            % A FAIRE CHEZ SOI
    \end{itemize}
\end{ex}
%}}}

\end{document}

