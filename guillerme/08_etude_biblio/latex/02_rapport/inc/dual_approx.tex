\documentclass[a4paper,9pt]{article}

/stock/16_git_repo/library/library.tex

\begin{document}

Au cours de cette section, nous allons définir un algorithme permettant d'obtenir une
$2$-dual-approximation\footnote{La définition d'une $\rho$-dual-approximation sera donnée dans le
préambule du chapitre sur les algos d'approximation} du problème \unitfisched{}. 

\section{Un algorithme d'approximation}

Avant de présenter le principe général de l'algorithme, nous introduirons quelques notions
nécessaires à la compréhension de ce dernier.

\subsection{Quelques définitions}

\begin{ndf}[Phases]
    Une \tphase{} d'un ensemble de tâches est un groupe de tâches telles que pour toutes les tâches
    appartenant à une même \tphase{} l'exécution commence au même instant et telles que deux tâches
    appartiennent à une \tphase{} différente si et seulement si elles ne s'exécutent pas en même
    temps.

    S'il existe une partition de l'ensemble des tâches en \tphase{}s, l'ordonnancement est alors dit
    \emph{par phases}.
\end{ndf}

\begin{nprop}
    Toute instance du problème \unitfisched{} donne lieu à un ordonnancement par phases.
\end{nprop}

% TODO Preuve

\begin{ndf}[Trous à gauche et à droite à l'instant $i$]
    Soit $M_j \in \mathcal{M}$ une machine quelconque, soit $T_{M_j}$ la dernière tâche ordonnancée
    sur la machine $M_j$ telle que $\st{T_j} \leq i$ et soit $R_{M_j}$ la première réservation
    affectéé à la machine $M_j$ telle que $\sres{R_j} \geq i$. On définit alors le \emph{trou à gauche
    de $M_j$ à l'instant $i$} (respectivement \emph{le trou à droite de $M_j$ à l'instant $i$}), l'écart
    entre la fin (respectivement le début) d'exécution de la tâche $T_{M_j}$ (respectivement de la
    réservation $R_{M_j}$) et le temps $i$ (respectivement $i+1$).

    Un exemple de trou à gauche (respectivement à droite) est donné par le trou $h_1$
    (respectivement $h_2$) de la figure~\ref{fig:tagtad}.
\end{ndf}

\begin{ndf}[Machine \authmach{k} à l'instant $i$]
    Une machine $M_j$ est dite \authmach{k} à l'instant $i$ si et seulement si ses trous à gauche et à
    droite ont une longueur au moins égale à $k$.

    Quelques exemples sont donnés à la figure~\ref{fig:kdisp}, à l'instant $6$ : 
    \begin{itemize}
        \item $M_1$ est \authmach{1}
        \item $M_2$ est \authmach{3}
        \item $M_3$ est \authmach{5}
        \item $M_4$ est \authmach{\infty}
    \end{itemize}
\end{ndf}

Finissons enfin par quelques notations que nous utiliserons à la
sous-section~\ref{subsec:pres_algo}.

\begin{nnot}
    Le nombre de \tphase{}s d'un ordonnancement par phases est noté $\nbphase{}$.
\end{nnot}

\begin{nnot}
    La date d'exécution de la $i^e$ \tphase{} est notée $t_i$.
\end{nnot}

\begin{nnot}
    La date d'exécution de la $i^e$ \tphase{} est notée $t_i$.
\end{nnot}

\begin{nnot}
    Le nombre de tâches composant la $i^e$ \tphase{} est noté $n_i$.
\end{nnot}

\begin{nnot}
    L'ensemble des machines \authmach{k} à l'instant $t_i$ est notée $\mathcal{M}^{k}_{t_i}$. Au cours
    de la sous-section~\ref{subsec:pres_algo}, nous utiliserons fréquemment la notation
    $\mathcal{M}_{t_i}$ en lieu et place de $\mathcal{M}_{t_i}^{\frac{\omega}{2}}$.
\end{nnot}

\begin{figure}
    \centering
    \begin{ordo}[10]{1}{1}{12}
        \newtask[$T_{M_1}$]{1}{1}{0}
        \newresa[$R_{M_1}$]{1}{1}{10}
        \newhole{3}{1}{1}
        \newhole{5}{1}{5}
    \end{ordo}
    \caption{Trou à gauche et à droite à l'instant $4$}
    \label{fig:tagtad}
\end{figure}

\begin{figure}
    \centering
    \begin{ordo}[10]{4}{1}{12}
        \newtask[$T_{M_1}$]{1}{1}{4}
        \newresa[$R_{M_1}$]{1}{1}{8}

        \newhole{1}{1}{5}
        \newhole{1}{1}{7}

        \newtask[$T_{M_2}$]{1}{2}{0}
        \newresa[$R_{M_2}$]{1}{2}{10}

        \newhole{5}{2}{1}
        \newhole{3}{2}{7}

        \newresa[$R_{M_3}$]{1}{3}{12}
        \newbeghole{6}{3}
        \newhole{5}{3}{7}

        \newbeghole{6}{4}
        \newendhole{5}{4}{7}
    \end{ordo}
    \caption{Illustration des machines \authmach{k}s}
    \label{fig:kdisp}
\end{figure}

\subsection{Principe de l'algorithme}
\label{subsec:pres_algo}

%Étant donné un entier $\omega$, l'algorithme proposé réalise les étapes suivantes :
%\begin{enumerate}
%    \item l'ensemble de tâches est partitionné en \nbphase{} \tphase{}s,
%    \item les \tphase{}s sont triées par date de début croissant, et parcourues dans cet ordre,
%    \item l'ensemble des machines \authmach{\frac{\omega}{2}} à l'instant $0$ est calculé et sera
%        maintenu tout au long du parcours. L'ensemble des machines \authmach{\frac{\omega}{2}} au
%        temps $i$ est noté $\mathcal{M}_i$,
%    \item les tâches à ordonnancer à l'instant $i$ sont ordonnancées sur les machines appartenant à
%        $\mathcal{M}_i$, ayant le plus petit trou à droite.
%\end{enumerate}
%
%L'algorithme s'arrête alors une fois que la dernière tâche a été ordonnancée ou sià un moment donné
%au cours de son exécution, $\mathcal{M}_i = \emptyset$ et le nombre de tâches restant à ordonanncer
%à l'instant $i$ est non nul.
%
Nous supposerons tout au long de cette section que les tâches à ordonnancer sont partionnées en
$\nbphase{}$ \tphase{}s. Dans le cas contraire, il suffit de trier les tâches par dates d'exécution
croissantes, toutes les tâches commençant à la même date définissent alors une \tphase{}.

L'algorithme se base sur le principe que, étant donnée une longueur de trou dite longueur
\emph{seuil}, il est préférable d'utiliser les machines ayant un trou à droite de longueur proche et
supérieure à cette valeur seuil. Autrement dit, on préfère créer deux trous de taille moyenne plutôt
que d'attendre et de créer un gros trou à gauche et un plus petit à droite. Ce principe est
illustré à la figure~\ref{fig:principeordo}. Ici les tâches sont partitionnées en trois
\tphase{}s : $\Theta_1 = \left\{ T_1 \right\}$, $\Theta_2 = \left\{ T_2 \right\}$ et $\Theta_3 =
\left\{ T_3 \right\}$. à l'instant $t_3 = 6$, l'ensemble des machines
\authmach{\frac{\omega}{2}}, avec $\omega = 6$ est :  $\mathcal{M_{t_i}} = \left\{ M_1, M_2
\right\}$. L'algorithme ordonnance alors la tâche $T_3$ sur la machine $M_2$ car son trou à droite
$h_3$ a une longueur inférieure au trou à droite de $M_2$.

\begin{figure}
    \centering
    \begin{minipage}[t]{6.5cm}
        \begin{ordo}[5.2]{2}{2}{12}
            \newtask{1}{2}{0}
            \newtask{1}{1}{2}

            \newtask{1}{3}{6}
            \newresa{1}{1}{12}
            \newresa{1}{2}{10}

            \newhole{5}{2}{1}
            \newhole{3}{1}{3}
            \newhole{3}{2}{7}
            \newhole{5}{1}{7}

        \end{ordo}
        \sfcaption{Tâche d'une \tphase{} à ordonnancer}
        \label{fig:ordo_phase}
    \end{minipage}
    \hfill
    \begin{minipage}[t]{6.5cm}
        \begin{ordo}[5.2]{2}{2}{12}
            \newtask{1}{2}{0}
            \newtask{1}{1}{2}

            \newtask{1}{2}{6}
            \newresa{1}{1}{12}
            \newresa{1}{2}{10}

            \newhole{5}{2}{1}
            \newhole{3}{1}{3}
            \newhole{3}{2}{7}
            \newhole{5}{1}{7}

            \adjust{1}{3}
        \end{ordo}
        \sfcaption{Choix de l'algorithme}
        \label{fig:ordo_choix}
    \end{minipage}
    \caption{Illustration de l'algorithme}
    \label{fig:principeordo}
\end{figure}

La $2$-dual-approximation est donnée par l'algorithme~\ref{alg:dual_approx}.

\begin{algorithm}
    \caption{\unitfisched{} $2$-dual-approximation}
    \label{alg:dual_approx}
    \begin{algorithmic}[1]
        \Require{$O$ une instance de \unitfisched{}, $P$ : une partition en phases de
        $\mathcal{T}$, $\omega$ : un entier}
        \For{$i$ de $1$ à $\nbphase{}$}
            \State Calcul de $\mathcal{M}_{t_i}$
            \If{$n_i < |\mathcal{M}_{t_i}|$}
                \State \Return Fail
            \EndIf
            \State Ordonnancer les tâches sur les machines de $\mathcal{M}_{t_i}$ ayant les trous à
            droite les plus petits
        \EndFor
        \Return Ordonnancement produit
    \end{algorithmic}
\end{algorithm}

\begin{nthrm}
    Si l'algorithme~\ref{alg:dual_approx} n'échoue pas, il fournit un ordonnancement valide tel que
    la taille du trou minimum est au moins égale à $\omega / 2$.
\end{nthrm}

\begin{proof}
    Nous supposons à présent que l'algorithme n'échoue pas. Alors :
    \begin{bitemize}
        \item\textbf{l'ordonnancement produit est valide} : l'algorithme a parcouru toutes les
            \tphase{}s de la partition $P$ et donc par définition l'ensemble des tâches de
            $\mathcal{T}$. Par hypothèse l'algorithme n'échoue pas, donc pour toutes les
            \tphase{}s de $P$, $|\mathcal{M}_{t_i}| \geq n_i$, il est donc possible de d'affecter
            chaque tâche à au moins une machine.
        \item\textbf{la taille du trou est au moins égale à $\omega / 2$} : les tâches ne sont ordonnancées
            que sur des machines appartenant à $\mathcal{M}_{t_i}$. Par définition, une machine
            appartient à $\mathcal{M}_{t_i}$ si et seulement si elle est
            \authmach{\frac{\omega}{2}} à l'instant $t_i$. Donc les trous générés à droite et à gauche
            par l'ordonnancement d'une tâche sur une de ces machines ont une taille au moins égale à
            $\omega / 2$.
    \end{bitemize}
\end{proof}

\begin{nthrm}
    Si l'algorithme échoue, alors il n'existe aucun ordonnancement valide ayant un trou minimum de
    taille au moins égale à $\omega$.
\end{nthrm}

\begin{figure}[]
    \centering
    \begin{ordo}[11]{4}{1}{8}
        
        \newtask{1}{5}{4}
        \newtask{1}{6}{4}
        \newtask{1}{7}{4}

        \foreach \x in {1,...,4}{
            \newbeghole[$h > \omega / 2$]{4}{\x}
        }

        \newresa{1}{1}{6}
        \newresa{1}{2}{6}
        \newresa{1}{4}{8}

        \newhole[$h < \omega / 2$]   {1}{1}{5}
        \newhole[$h < \omega / 2$]   {1}{2}{5}
        \newhole[$h \geq \omega / 2$]{3}{4}{5}
        \newendhole[$h > \omega / 2$]{3}{3}{5}

    \end{ordo}
    \caption{Cas 1 : $\overline{\mathcal{M}_{t_i}^G} = \emptyset$}
    \label{fig:cas1}
\end{figure}
\begin{figure}[]
    \centering
    \begin{ordo}[11]{4}{1}{8}
        \newtask{1}{4}{0}
        \newtask{1}{3}{2}
        \newtask{1}{2}{2}
        
        \newtask{1}{5}{6}
        \newtask{1}{6}{6}
        \newtask{1}{7}{6}

        \newhole[$h \geq \omega / 2$]{5}{4}{1}
        \newhole[$h < \omega / 2$]{3}{3}{3}
        \newhole[$h < \omega / 2$]{3}{2}{3}
        \newbeghole[$h > \omega / 2$]{6}{1}

        \foreach \x in {1,...,4}{
            \newendhole[$h > \omega / 2$]{1}{\x}{7}
        }
    \end{ordo}
    \caption{Cas 2 : $\overline{\mathcal{M}_{t_i}^D} = \emptyset$}
    \label{fig:cas2}
\end{figure}
\begin{figure}[]
    \centering
    \begin{ordo}[11]{4}{1}{8}
        \newtask[$T_1$]{1}{3}{0}
        \newresa[$R_1$]{1}{3}{6}
        \newtask[$T_2$]{1}{1}{2}
        
        \newtask[$T_3$]{1}{5}{4}
        \newtask[$T_4$]{1}{6}{4}
        \newtask[$T_5$]{1}{7}{4}

        \newhole[$h < \omega / 2$]{1}{1}{3}
        \newhole[$h \geq \omega / 2$]{3}{3}{1}
        \newhole[$h < \omega / 2$]{1}{3}{5}
        \newendhole[$h > \omega / 2$]{3}{1}{5}

        \foreach \x in {2,4}{
            \newbeghole{4}{\x}{$h > \omega / 2$}
            \newendhole[$h > \omega / 2$]{3}{\x}{5}
        }
    \end{ordo}
    \caption{Cas 3 : $\overline{\mathcal{M}_{t_i}^D} \neq \emptyset$ et $\overline{\mathcal{M}_{t_i}^D} \neq \emptyset$}
    \label{fig:cas3}
\end{figure}
\begin{figure}[]
    \centering
    \begin{ordo}[11]{4}{1}{8}
        \newtask[$T_1$]{1}{1}{0}
        \newresa[$R_1$]{1}{3}{6}
        \newtask[$T_2$]{1}{3}{2}
        
        \newtask[$T_3$]{1}{1}{4}
        \newtask[$T_4$]{1}{2}{4}
        \newtask[$T_5$]{1}{4}{4}

        \newhole[$h > \omega / 2$]{3}{1}{1}
        \newbeghole[$h > \omega$]{2}{3}
        \newhole[$h < \omega $]{3}{3}{3}
        \newendhole[$h > \omega$]{3}{1}{5}

        \foreach \x in {2,4}{
            \newbeghole{4}{\x}{$h > \omega$}
            \newendhole[$h > \omega$]{3}{\x}{5}
        }
    \end{ordo}
    \caption{Cas 3 : Optimal}
    \label{fig:cas3opt}
\end{figure}
\begin{figure}[]
    \centering
    \begin{ordo}[11]{4}{1}{8}
        \newtask[$T_1$]{1}{1}{0}
        \newtask[$T_2$]{1}{2}{2}
        \newtask[$T_3$]{1}{3}{2}
        \newresa[$R_1$]{1}{1}{6}
        \newresa[$R_2$]{1}{2}{6}
        \newresa[$R_3$]{1}{3}{8}
        
        \newtask[$T_4$]{1}{5}{4}
        \newtask[$T_5$]{1}{6}{4}

        \newhole[$h \geq \omega / 2$]{3}{1}{1}
        \newhole[$ h < \omega / 2$]{1}{2}{3}
        \newhole[$ h < \omega / 2$]{1}{3}{3}
        \newbeghole[$h > \omega / 2$]{4}{4}

        \newhole[$h < \omega / 2$]{1}{1}{5}
        \newhole[$h < \omega / 2$]{1}{2}{5}
        \newhole[$h \geq \omega / 2$]{3}{3}{5}
        \newendhole[$h > \omega / 2$]{3}{4}{5}

    \end{ordo}
    \caption{$\overline{M_{t_i}} \neq \emptyset$}
    \label{fig:cas4}
\end{figure}

\begin{proof}
    Supposons à présent que l'algorihtme échoue. On a donc $|\mathcal{M}_{t_i}| < n_i$, or si l'on
    appelle : 
    \begin{itemize}
        \item $\overline{\mathcal{M}_{t_i}^G}$, l'ensemble des machines ayant, à l'instant $t_i$ un
            trou à gauche de taille inférieure à $\omega / 2$,
        \item $\overline{\mathcal{M}_{t_i}^D}$, l'ensemble des machines ayant, à l'instant $t_i$ un
            trou à droite de taille inférieure à $\omega / 2$,
        \item et $\overline{\mathcal{M}_{t_i}} = \overline{\mathcal{M}_{t_i}^G} \cap
            \overline{\mathcal{M}_{t_i}^D}$,
    \end{itemize}
    on peut alors écrire : \begin{displaymath}
        |\mathcal{M}_{t_i}| = m - |\overline{\mathcal{M}_{t_i}^G}| -
        |\overline{\mathcal{M}_{t_i}^D}| +
        |\overline{\mathcal{M}_{t_i}}|
    \end{displaymath}
    Raisonnons sur $\overline{\mathcal{M}_{t_i}}$ :
    \begin{bitemize}
        \item si $\overline{\mathcal{M}_{t_i}} = \emptyset$ alors nous pouvons distinguer trois cas
            :
            \begin{enumerate}
                \item $\overline{\mathcal{M}_{t_i}^G} = \emptyset$, représenté à la
                    figure~\ref{fig:cas1}. Dans ce cas de figure les machines sont rendues
                    indisponibles uniquement à cause des réservations données en paramètre du
                    problème. Dans une telle situation on a :
                    \begin{displaymath}
                        \begin{array}{rrcl}
                            & |\mathcal{M}_{t_i}| & < & n_i \\
                            \Rightarrow &  m - |\overline{\mathcal{M}_{t_i}^G}| -
                            |\overline{\mathcal{M}_{t_i}^D}| +
                            |\overline{\mathcal{M}_{t_i}}| & < & n_i \\
                            \Rightarrow & m - |\overline{\mathcal{M}_{t_i}^D}| & < & n_i
                        \end{array}
                    \end{displaymath}
                    Supposons un ordonnancement optimal $O^*$ et appelons
                    $\mathcal{M}_{O^*}^i$ l'ensemble des machines auxquelles les tâches de la
                    \tphase{} $i$ sont affectées et $\overline{\mathcal{M}_{O^*}^{iD}}$ les machines
                    \authmach{\omega / 2} à droite à l'instant $t_i$. On a alors :
                    \begin{displaymath}
                        |\mathcal{M}_{O^*}^i| = n_i \quad \mbox{et} \quad
                        \overline{\mathcal{M}_{O^*}^{iD}} = \overline{\mathcal{M}_{t_i}^D}
                    \end{displaymath}
                    On peut alors écrire :
                    \begin{displaymath}
                        \begin{array}{rrcl}
                            & m - |\overline{\mathcal{M}_{t_i}^D}| & < & |\mathcal{M}_{O^*}^i| \\
                            \Rightarrow & |\mathcal{M}| & < & |\mathcal{M}_{O^*}^i| +
                            |\overline{\mathcal{M}_{t_i}^D}| \\
                            \Rightarrow & |\mathcal{M}| & < & |\mathcal{M}_{O^*}^i| +
                            |\overline{\mathcal{M}_{O^*}^{iD}}| \\
                            \Rightarrow & |\mathcal{M}_{O^*}^i \cap
                            \overline{\mathcal{M}_{O^*}^{iD}}| & > & 0 \\
                        \end{array}
                    \end{displaymath}
                    Donc l'ordonnancement optimal $O^*$ utilise une machine non disponible et donc
                    ayant un trou à droite à l'instant $t_i$ de taille inférieure à $\omega / 2$ et
                    donc la taille maximale du trou minimimum est inférieure à
                    $\omega/2$.
                \item $\overline{\mathcal{M}_{t_i}^D} = \emptyset$, représenté à la
                    figure~\ref{fig:cas2}. De la même manière que précédemment, nous prouvons que
                    l'algorithme optimal utilise une machine indisponible et donc que le taille
                    optimale du trou minimum est inférieure à $\omega / 2$.
                \item $\overline{\mathcal{M}_{t_i}^D} \neq \emptyset$ et
                    $\overline{\mathcal{M}_{t_i}^G} \neq \emptyset$, illustré à la
                    figure~\ref{fig:cas3}. 
                    Supposons un ordonnancement optimal $O^*$ tel que la taille du trou minimum est
                    au moins égale à $\omega$, et appelons
                    $\overline{\mathcal{M}_{O^*}^{iG}}$ l'ensemble des machines \authmach{\omega/2}
                    à gauche à l'instant $t_i$, on peut alors écrire
                    \begin{displaymath}
                        |\mathcal{M}_{O^*}^i| = n_i \quad \mbox{et} \quad
                        |\overline{\mathcal{M}_{O^*}^{iG}}| = |\overline{\mathcal{M}_{t_i}^{G}}|
                    \end{displaymath} 
                    Comme l'algorithme a échoué, nous pouvons écrire :
                    \begin{displaymath}
                        \begin{array}{rrcl}
                            & m - |\overline{\mathcal{M}_{t_i}^D}| -
                            |\overline{\mathcal{M}_{t_i}^G}| + |\overline{\mathcal{M}_{t_i}}| & < &
                            |\mathcal{M}_{O^*}^i| \\
                            \Rightarrow & |\mathcal{M}| - |\overline{\mathcal{M}_{O^*}^{iD}}| -
                            |\overline{\mathcal{M}_{O^*}^{iG}}| + |\overline{\mathcal{M}_{t_i}}| & <
                            & |\mathcal{M}_{O^*}^i| \\
                        \end{array}
                    \end{displaymath}
                    Or l'ordonnancement optimal est tel que la taille du trou minimum est au moins
                    égal à $\omega$, on peut donc écrire :
                    \begin{displaymath}
                        |\mathcal{M}| - |\overline{\mathcal{M}_{O^*}^{iD}}| -
                        |\overline{\mathcal{M}_{O^*}^{iG}}| + |\overline{\mathcal{M}_{O^*}^{iD}}
                        \cap \overline{\mathcal{M}_{O^*}^{iG}}|\geq |\mathcal{M}_{O^*}^i|
                    \end{displaymath}
                    Nous pouvons déduire des deux inéquations précédentes ce qui suit :
                    \begin{displaymath}
                        |\overline{\mathcal{M}_{O^*}^{iD}} \cap \overline{\mathcal{M}_{O^*}^{iG}}|
                        \geq |\overline{\mathcal{M}_{t_i}}|
                    \end{displaymath}
                    Ce qui implique $\overline{\mathcal{M}_{O^*}^{iD}} \cap
                    \overline{\mathcal{M}_{O^*}^{iG}} \neq 0$. Autrement dit, $O^*$ affecte à au
                    moins une machine, ayant un trou à droite inférieur à $\omega / 2$ à l'instant
                    $t_i$, une tâche générant un trou à gauche de taille inférieure à $\omega / 2$
                    (comme le décrit la figure\ref{fig:cas3opt})
                    et donc la taille du trou entre cette réservation et cette tâche est inférieure à
                    $\omega$.
            \end{enumerate}
        \item si $\overline{\mathcal{M}_{t_i}} \neq \emptyset$, ce cas est illustré par la
            figure~\ref{fig:cas4}. Supposons à nouveau pour ce cas, qu'il existe un ordonnancement
            $O^*$ tel que la taille du trou minimum est supérieure ou égale à $\omega$. Donc au
            temps $t_i$, en reprenant les notations précédentes, on a :
            \begin{displaymath}
                |\overline{\mathcal{M}_{O^*}^{iD}} \cap \overline{\mathcal{M}_{O^*}^{iG}}| = 0
            \end{displaymath}
            Si tel n'est pas le cas, nous avons vu au cours de l'étude du cas $3$ que la taille
            du trou minimum ne peut être au moins égale à $\omega$. Or de telles contraintes nous
            permettent d'écrire
            \begin{displaymath}
                \begin{array}{rrcl}
                    & m - |\overline{\mathcal{M}_{O^*}^{iD}}| -
                    |\overline{\mathcal{M}_{O^*}^{iG}}| & \geq & n_i \\
                    \Rightarrow & m - |\overline{\mathcal{M}_{t_i}^D}| -
                    |\overline{\mathcal{M}_{t_i}^G}| & \geq & n_i \\
                \end{array}
            \end{displaymath}
            Or ceci est impossible puisque l'algorithme d'approximation a échoué.
    \end{bitemize}

    L'échec de l'algorithme implique donc que la taille optimale du trou minimum est inférieure au
    $\omega$ donné en entrée.
\end{proof}

\end{document}
