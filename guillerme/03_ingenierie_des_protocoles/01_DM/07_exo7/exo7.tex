\begin{figure}
    \begin{center}
        \begin{tikzpicture}[every transition/.style={minimum width=4mm, minimum height=10mm},
        >=latex, every place/.style={minimum width=1cm}, scale=0.7, every node/.style={transform shape}]
            \node[place] (compteur) at (3, 0) {};
            \node[transition, label=$V$] (V) at (0, 0) {};
            \node[transition, label=$P$] (P) at (6, 0) {};

            \draw (compteur) edge[pre] (V)
                edge[post] (P);
        \end{tikzpicture}
    \end{center}
    \caption{Représentation d'un sémaphore à l'aide d'un réseau de Petri}
    \label{exo7fig1}
\end{figure}

    \begin{figure}
\begin{minipage}[c]{0.6\linewidth}
            \begin{tikzpicture}[every transition/.style={minimum width=4mm, minimum height=10mm},
            >=latex, every place/.style={minimum width=1cm}, scale=0.7, every node/.style={transform shape}]
                \node[place, structured tokens=0] (compteur) at (3, 0) {$u$};
                \node[transition, label=$V$] (V) at (0, 0) {};
                \node[transition, label=$P$] (P) at (6, 0) {};

                \draw (compteur) edge[post, bend left] node[auto] {$f$} (V)
                    edge[pre, bend right] node[auto] {$g$} (V)
                    edge[post, bend left] node[auto] {$f$} (P)
                    edge[pre, bend right] node[auto] {$h$} (P);
            \end{tikzpicture}
\end{minipage}
\hfill
\begin{minipage}[c]{0.3\linewidth}
    \[
        \begin{array}{rcl}
            f(x) & = & x \\
            g(x) & = & x+1 \\
            h(x) & = & x-1 \\
        \end{array}
    \]
\end{minipage}
        \caption{Représentation d'un sémaphore à l'aide d'un réseau de Petri coloré}
        \label{exo7fig2}
\end{figure}

Dans le cas du réseau de Petri non coloré représenté à la figure~\ref{exo7fig1},
les opérations $P$ et $V$ peuvent être simultanées à condition que la place contienne d'ores et
déjà un jeton. Si ce n'est pas le cas, alors seule la transition $V$ est tirable et donc $P$ et $V$
ne peuvent être tirées en même temps. De plus l'opération $V$, peut être concurrente à elle même
alors que l'opération $P$ ne peut être concurrent à elle même que si le nombre de jetons présents dans
la place est suffisant.

Dans le cas du réseau de Petri coloré représenté à la figure~\ref{exo7fig2}, toutes les opérations
sont séquentielles et ne peuvent être exécutées simultanément.
